\documentclass[a4paper,10pt]{report}
\usepackage[
    type={CC},
    modifier={by-nc-sa},
    version={3.0},
]{doclicense}


% Chargement d'extensions
\usepackage[utf8]{inputenc}  
\usepackage[T1]{fontenc}
\usepackage[french]{babel}

\usepackage[landscape,left=1cm,top=1.5cm,right=1cm,bottom=1.5cm]{geometry}

\usepackage{fancyhdr}
\usepackage{multicol}
\usepackage{hyperref}
\usepackage{eurosym}
\usepackage{graphicx}
\usepackage[dvipsnames]{xcolor}

\usepackage{minitoc}

\pagestyle{fancy}

\usepackage{titlesec}
\titleformat{\chapter}[display]
{\normalfont\huge\bfseries}{}{0pt}{\Huge}
\titlespacing*{\chapter} {0pt}{20pt}{40pt}

% Informations le titre, le(s) auteur(s), la date
\title{Tentative de reproduction et de corrections des simulations du gouvernement Philippe}
\author{Bruno Scherrer\footnote{bruno.scherrer@inria.fr}}
\date{\today}



\begin{document}


\makeatletter
\renewcommand\tableofcontents{%
    \@starttoc{toc}%
}
\makeatother

\dominitoc


\maketitle

\doclicenseThis

%{\small Bruno Scherrer, \tt{bruno.scherrer@inria.fr}}

%\section*{Simulation de carrières et retraites à points pour différents métiers}

Ce document présente des simulations (sous forme de tableaux détaillés et de graphes) de l'évolution de la carrière et des possibles retraites pour plusieurs métiers. Pour le passé (avant 2019), on s'appuie sur les données macro-économiques (inflation, croissance, point d'indice) réelles. Pour le futur, on se base sur les hypothèses macro-économiques du gouvernement Philippe (croissance des salaires du privé de 1.3\% par an, croissance du point d'indice de 0\% par an, augmentation des primes de 0.23 points par an) et sur son système de retraites par points.
En regardant comment les rémunérations et les pensions évoluent par rapport au SMIC et au salaire moyen, on observe en particulier, dans le temps, une dévalorisation significative de l'ensemble des métiers de la fonction publique.

\setcounter{tocdepth}{0}
\begingroup
\let\clearpage\relax
\tableofcontents
\endgroup



 \begin{center}\includegraphics[width=1\textwidth]{fig/comparaison_modeles.pdf}\end{center} 

\newpage 
 
\chapter{Infirmière en soins généraux (CN, CS, puis HC)} 

\begin{minipage}{0.55\linewidth}\includegraphics[width=0.7\textwidth]{fig/grille_Infirmier.pdf}\end{minipage} 
\begin{minipage}{0.3\linewidth} 
 \begin{center} 

\begin{tabular}[htb]{|c|c|} 
\hline 
 Indice majoré &  Durée (années) \\ 
\hline \hline 
 390 &  2.00 \\ 
\hline 
 404 &  3.00 \\ 
\hline 
 422 &  3.00 \\ 
\hline 
 446 &  3.00 \\ 
\hline 
 469 &  3.00 \\ 
\hline 
 501 &  3.00 \\ 
\hline 
 520 &  4.00 \\ 
\hline 
 544 &  4.00 \\ 
\hline 
 571 &  4.00 \\ 
\hline 
 594 &  4.00 \\ 
\hline 
 627 &   \\ 
\hline 
\hline 
\end{tabular} 
\end{center} 
 \end{minipage} 


 \addto{\captionsenglish}{ \renewcommand{\mtctitle}{}} \setcounter{minitocdepth}{2} 
 \minitoc \newpage 

\section{Début de carrière à 22 ans} 

\subsection{Génération 1975 (début en 1997)} 

\paragraph{Retraites possibles et ratios Revenu/SMIC à 70, 75, 80, 85, 90 ans avec le modèle \emph{Gouvernement truqué (âge-pivot bloqué à 65 ans)}}  
 
{ \scriptsize \begin{center} 
\begin{tabular}[htb]{|c|c||c|c||c|c||c||c|c|c|c|c|c|} 
\hline 
 Retraite en &  Âge &  Âge pivot &  Décote/Surcote &  Retraite (\euro{} 2019) &  Tx Rempl(\%) &  SMIC (\euro{} 2019) &  Retraite/SMIC &  Rev70/SMIC &  Rev75/SMIC &  Rev80/SMIC &  Rev85/SMIC &  Rev90/SMIC \\ 
\hline \hline 
 2037 &  62 &  64 ans 10 mois &  -14.17\% &  1636.88 &  {\bf 42.14} &  2143.00 &  {\bf {\color{red} 0.76}} &  {\bf {\color{red} 0.69}} &  {\bf {\color{red} 0.65}} &  {\bf {\color{red} 0.61}} &  {\bf {\color{red} 0.57}} &  {\bf {\color{red} 0.53}} \\ 
\hline 
 2038 &  63 &  64 ans 11 mois &  -9.58\% &  1786.41 &  {\bf 45.91} &  2170.86 &  {\bf {\color{red} 0.82}} &  {\bf {\color{red} 0.75}} &  {\bf {\color{red} 0.70}} &  {\bf {\color{red} 0.66}} &  {\bf {\color{red} 0.62}} &  {\bf {\color{red} 0.58}} \\ 
\hline 
 2039 &  64 &  65 ans 0 mois &  -5.00\% &  1944.26 &  {\bf 49.88} &  2199.08 &  {\bf {\color{red} 0.88}} &  {\bf {\color{red} 0.82}} &  {\bf {\color{red} 0.77}} &  {\bf {\color{red} 0.72}} &  {\bf {\color{red} 0.67}} &  {\bf {\color{red} 0.63}} \\ 
\hline 
 2040 &  65 &  65 ans 0 mois &  0.00\% &  2119.69 &  {\bf 54.28} &  2227.67 &  {\bf {\color{red} 0.95}} &  {\bf {\color{red} 0.89}} &  {\bf {\color{red} 0.84}} &  {\bf {\color{red} 0.78}} &  {\bf {\color{red} 0.73}} &  {\bf {\color{red} 0.69}} \\ 
\hline 
 2041 &  66 &  65 ans 0 mois &  5.00\% &  2304.94 &  {\bf 58.92} &  2256.63 &  {\bf 1.02} &  {\bf {\color{red} 0.97}} &  {\bf {\color{red} 0.91}} &  {\bf {\color{red} 0.85}} &  {\bf {\color{red} 0.80}} &  {\bf {\color{red} 0.75}} \\ 
\hline 
 2042 &  67 &  65 ans 0 mois &  10.00\% &  2500.52 &  {\bf 63.81} &  2285.97 &  {\bf 1.09} &  {\bf 1.05} &  {\bf {\color{red} 0.99}} &  {\bf {\color{red} 0.92}} &  {\bf {\color{red} 0.87}} &  {\bf {\color{red} 0.81}} \\ 
\hline 
\hline 
\end{tabular} 
\end{center} } 
\paragraph{Retraites possibles et ratios Revenu/SMIC à 70, 75, 80, 85, 90 ans avec le modèle \emph{Gouvernement corrigé (âge-pivot glissant)}}  
 
{ \scriptsize \begin{center} 
\begin{tabular}[htb]{|c|c||c|c||c|c||c||c|c|c|c|c|c|} 
\hline 
 Retraite en &  Âge &  Âge pivot &  Décote/Surcote &  Retraite (\euro{} 2019) &  Tx Rempl(\%) &  SMIC (\euro{} 2019) &  Retraite/SMIC &  Rev70/SMIC &  Rev75/SMIC &  Rev80/SMIC &  Rev85/SMIC &  Rev90/SMIC \\ 
\hline \hline 
 2037 &  62 &  64 ans 10 mois &  -14.17\% &  1636.88 &  {\bf 42.14} &  2143.00 &  {\bf {\color{red} 0.76}} &  {\bf {\color{red} 0.69}} &  {\bf {\color{red} 0.65}} &  {\bf {\color{red} 0.61}} &  {\bf {\color{red} 0.57}} &  {\bf {\color{red} 0.53}} \\ 
\hline 
 2038 &  63 &  64 ans 11 mois &  -9.58\% &  1786.41 &  {\bf 45.91} &  2170.86 &  {\bf {\color{red} 0.82}} &  {\bf {\color{red} 0.75}} &  {\bf {\color{red} 0.70}} &  {\bf {\color{red} 0.66}} &  {\bf {\color{red} 0.62}} &  {\bf {\color{red} 0.58}} \\ 
\hline 
 2039 &  64 &  65 ans 0 mois &  -5.00\% &  1944.26 &  {\bf 49.88} &  2199.08 &  {\bf {\color{red} 0.88}} &  {\bf {\color{red} 0.82}} &  {\bf {\color{red} 0.77}} &  {\bf {\color{red} 0.72}} &  {\bf {\color{red} 0.67}} &  {\bf {\color{red} 0.63}} \\ 
\hline 
 2040 &  65 &  65 ans 1 mois &  -0.42\% &  2110.86 &  {\bf 54.06} &  2227.67 &  {\bf {\color{red} 0.95}} &  {\bf {\color{red} 0.89}} &  {\bf {\color{red} 0.83}} &  {\bf {\color{red} 0.78}} &  {\bf {\color{red} 0.73}} &  {\bf {\color{red} 0.69}} \\ 
\hline 
 2041 &  66 &  65 ans 2 mois &  4.17\% &  2286.65 &  {\bf 58.45} &  2256.63 &  {\bf 1.01} &  {\bf {\color{red} 0.96}} &  {\bf {\color{red} 0.90}} &  {\bf {\color{red} 0.85}} &  {\bf {\color{red} 0.79}} &  {\bf {\color{red} 0.74}} \\ 
\hline 
 2042 &  67 &  65 ans 3 mois &  8.75\% &  2472.11 &  {\bf 63.08} &  2285.97 &  {\bf 1.08} &  {\bf 1.04} &  {\bf {\color{red} 0.98}} &  {\bf {\color{red} 0.91}} &  {\bf {\color{red} 0.86}} &  {\bf {\color{red} 0.80}} \\ 
\hline 
\hline 
\end{tabular} 
\end{center} } 
\paragraph{Retraites possibles et ratios Revenu/SMIC à 70, 75, 80, 85, 90 ans avec le modèle \emph{Destinie2 (revalorisation de la fonction publique)}}  
 
{ \scriptsize \begin{center} 
\begin{tabular}[htb]{|c|c||c|c||c|c||c||c|c|c|c|c|c|} 
\hline 
 Retraite en &  Âge &  Âge pivot &  Décote/Surcote &  Retraite (\euro{} 2019) &  Tx Rempl(\%) &  SMIC (\euro{} 2019) &  Retraite/SMIC &  Rev70/SMIC &  Rev75/SMIC &  Rev80/SMIC &  Rev85/SMIC &  Rev90/SMIC \\ 
\hline \hline 
 2037 &  62 &  64 ans 10 mois &  -14.17\% &  1802.13 &  {\bf 37.22} &  2014.82 &  {\bf {\color{red} 0.89}} &  {\bf {\color{red} 0.81}} &  {\bf {\color{red} 0.76}} &  {\bf {\color{red} 0.71}} &  {\bf {\color{red} 0.66}} &  {\bf {\color{red} 0.62}} \\ 
\hline 
 2038 &  63 &  64 ans 11 mois &  -9.58\% &  1974.56 &  {\bf 40.26} &  2041.01 &  {\bf {\color{red} 0.97}} &  {\bf {\color{red} 0.88}} &  {\bf {\color{red} 0.83}} &  {\bf {\color{red} 0.78}} &  {\bf {\color{red} 0.73}} &  {\bf {\color{red} 0.68}} \\ 
\hline 
 2039 &  64 &  65 ans 0 mois &  -5.00\% &  2157.77 &  {\bf 43.43} &  2067.55 &  {\bf 1.04} &  {\bf {\color{red} 0.97}} &  {\bf {\color{red} 0.91}} &  {\bf {\color{red} 0.85}} &  {\bf {\color{red} 0.80}} &  {\bf {\color{red} 0.75}} \\ 
\hline 
 2040 &  65 &  65 ans 1 mois &  -0.42\% &  2352.36 &  {\bf 46.74} &  2094.43 &  {\bf 1.12} &  {\bf 1.05} &  {\bf {\color{red} 0.99}} &  {\bf {\color{red} 0.93}} &  {\bf {\color{red} 0.87}} &  {\bf {\color{red} 0.81}} \\ 
\hline 
 2041 &  66 &  65 ans 2 mois &  4.17\% &  2558.98 &  {\bf 50.19} &  2121.65 &  {\bf 1.21} &  {\bf 1.15} &  {\bf 1.07} &  {\bf 1.01} &  {\bf {\color{red} 0.94}} &  {\bf {\color{red} 0.88}} \\ 
\hline 
 2042 &  67 &  65 ans 3 mois &  8.75\% &  2778.33 &  {\bf 53.79} &  2149.23 &  {\bf 1.29} &  {\bf 1.24} &  {\bf 1.17} &  {\bf 1.09} &  {\bf 1.02} &  {\bf {\color{red} 0.96}} \\ 
\hline 
\hline 
\end{tabular} 
\end{center} } 

 \begin{center}\includegraphics[width=0.9\textwidth]{fig/Infirmier_1975_22_dest_retraite.pdf}\end{center} \label{fig/Infirmier_1975_22_dest_retraite.pdf} 

\newpage 
 
\subsection{Génération 1980 (début en 2002)} 

\paragraph{Retraites possibles et ratios Revenu/SMIC à 70, 75, 80, 85, 90 ans avec le modèle \emph{Gouvernement truqué (âge-pivot bloqué à 65 ans)}}  
 
{ \scriptsize \begin{center} 
\begin{tabular}[htb]{|c|c||c|c||c|c||c||c|c|c|c|c|c|} 
\hline 
 Retraite en &  Âge &  Âge pivot &  Décote/Surcote &  Retraite (\euro{} 2019) &  Tx Rempl(\%) &  SMIC (\euro{} 2019) &  Retraite/SMIC &  Rev70/SMIC &  Rev75/SMIC &  Rev80/SMIC &  Rev85/SMIC &  Rev90/SMIC \\ 
\hline \hline 
 2042 &  62 &  65 ans 0 mois &  -15.00\% &  1654.36 &  {\bf 42.59} &  2285.97 &  {\bf {\color{red} 0.72}} &  {\bf {\color{red} 0.65}} &  {\bf {\color{red} 0.61}} &  {\bf {\color{red} 0.57}} &  {\bf {\color{red} 0.54}} &  {\bf {\color{red} 0.50}} \\ 
\hline 
 2043 &  63 &  65 ans 0 mois &  -10.00\% &  1820.50 &  {\bf 46.79} &  2315.68 &  {\bf {\color{red} 0.79}} &  {\bf {\color{red} 0.72}} &  {\bf {\color{red} 0.67}} &  {\bf {\color{red} 0.63}} &  {\bf {\color{red} 0.59}} &  {\bf {\color{red} 0.55}} \\ 
\hline 
 2044 &  64 &  65 ans 0 mois &  -5.00\% &  1996.69 &  {\bf 51.22} &  2345.79 &  {\bf {\color{red} 0.85}} &  {\bf {\color{red} 0.79}} &  {\bf {\color{red} 0.74}} &  {\bf {\color{red} 0.69}} &  {\bf {\color{red} 0.65}} &  {\bf {\color{red} 0.61}} \\ 
\hline 
 2045 &  65 &  65 ans 0 mois &  0.00\% &  2183.46 &  {\bf 55.92} &  2376.28 &  {\bf {\color{red} 0.92}} &  {\bf {\color{red} 0.86}} &  {\bf {\color{red} 0.81}} &  {\bf {\color{red} 0.76}} &  {\bf {\color{red} 0.71}} &  {\bf {\color{red} 0.67}} \\ 
\hline 
 2046 &  66 &  65 ans 0 mois &  5.00\% &  2379.61 &  {\bf 60.83} &  2407.18 &  {\bf {\color{red} 0.99}} &  {\bf {\color{red} 0.94}} &  {\bf {\color{red} 0.88}} &  {\bf {\color{red} 0.83}} &  {\bf {\color{red} 0.77}} &  {\bf {\color{red} 0.73}} \\ 
\hline 
 2047 &  67 &  65 ans 0 mois &  10.00\% &  2585.33 &  {\bf 65.97} &  2438.47 &  {\bf 1.06} &  {\bf 1.02} &  {\bf {\color{red} 0.96}} &  {\bf {\color{red} 0.90}} &  {\bf {\color{red} 0.84}} &  {\bf {\color{red} 0.79}} \\ 
\hline 
\hline 
\end{tabular} 
\end{center} } 
\paragraph{Retraites possibles et ratios Revenu/SMIC à 70, 75, 80, 85, 90 ans avec le modèle \emph{Gouvernement corrigé (âge-pivot glissant)}}  
 
{ \scriptsize \begin{center} 
\begin{tabular}[htb]{|c|c||c|c||c|c||c||c|c|c|c|c|c|} 
\hline 
 Retraite en &  Âge &  Âge pivot &  Décote/Surcote &  Retraite (\euro{} 2019) &  Tx Rempl(\%) &  SMIC (\euro{} 2019) &  Retraite/SMIC &  Rev70/SMIC &  Rev75/SMIC &  Rev80/SMIC &  Rev85/SMIC &  Rev90/SMIC \\ 
\hline \hline 
 2042 &  62 &  65 ans 3 mois &  -16.25\% &  1630.03 &  {\bf 41.97} &  2285.97 &  {\bf {\color{red} 0.71}} &  {\bf {\color{red} 0.64}} &  {\bf {\color{red} 0.60}} &  {\bf {\color{red} 0.57}} &  {\bf {\color{red} 0.53}} &  {\bf {\color{red} 0.50}} \\ 
\hline 
 2043 &  63 &  65 ans 4 mois &  -11.67\% &  1786.79 &  {\bf 45.92} &  2315.68 &  {\bf {\color{red} 0.77}} &  {\bf {\color{red} 0.70}} &  {\bf {\color{red} 0.66}} &  {\bf {\color{red} 0.62}} &  {\bf {\color{red} 0.58}} &  {\bf {\color{red} 0.54}} \\ 
\hline 
 2044 &  64 &  65 ans 5 mois &  -7.08\% &  1952.90 &  {\bf 50.10} &  2345.79 &  {\bf {\color{red} 0.83}} &  {\bf {\color{red} 0.77}} &  {\bf {\color{red} 0.72}} &  {\bf {\color{red} 0.68}} &  {\bf {\color{red} 0.63}} &  {\bf {\color{red} 0.60}} \\ 
\hline 
 2045 &  65 &  65 ans 6 mois &  -2.50\% &  2128.87 &  {\bf 54.52} &  2376.28 &  {\bf {\color{red} 0.90}} &  {\bf {\color{red} 0.84}} &  {\bf {\color{red} 0.79}} &  {\bf {\color{red} 0.74}} &  {\bf {\color{red} 0.69}} &  {\bf {\color{red} 0.65}} \\ 
\hline 
 2046 &  66 &  65 ans 7 mois &  2.08\% &  2313.50 &  {\bf 59.14} &  2407.18 &  {\bf {\color{red} 0.96}} &  {\bf {\color{red} 0.91}} &  {\bf {\color{red} 0.86}} &  {\bf {\color{red} 0.80}} &  {\bf {\color{red} 0.75}} &  {\bf {\color{red} 0.70}} \\ 
\hline 
 2047 &  67 &  65 ans 8 mois &  6.67\% &  2506.99 &  {\bf 63.97} &  2438.47 &  {\bf 1.03} &  {\bf {\color{red} 0.99}} &  {\bf {\color{red} 0.93}} &  {\bf {\color{red} 0.87}} &  {\bf {\color{red} 0.81}} &  {\bf {\color{red} 0.76}} \\ 
\hline 
\hline 
\end{tabular} 
\end{center} } 
\paragraph{Retraites possibles et ratios Revenu/SMIC à 70, 75, 80, 85, 90 ans avec le modèle \emph{Destinie2 (revalorisation de la fonction publique)}}  
 
{ \scriptsize \begin{center} 
\begin{tabular}[htb]{|c|c||c|c||c|c||c||c|c|c|c|c|c|} 
\hline 
 Retraite en &  Âge &  Âge pivot &  Décote/Surcote &  Retraite (\euro{} 2019) &  Tx Rempl(\%) &  SMIC (\euro{} 2019) &  Retraite/SMIC &  Rev70/SMIC &  Rev75/SMIC &  Rev80/SMIC &  Rev85/SMIC &  Rev90/SMIC \\ 
\hline \hline 
 2042 &  62 &  65 ans 3 mois &  -16.25\% &  1858.99 &  {\bf 35.99} &  2149.23 &  {\bf {\color{red} 0.86}} &  {\bf {\color{red} 0.78}} &  {\bf {\color{red} 0.73}} &  {\bf {\color{red} 0.69}} &  {\bf {\color{red} 0.64}} &  {\bf {\color{red} 0.60}} \\ 
\hline 
 2043 &  63 &  65 ans 4 mois &  -11.67\% &  2047.53 &  {\bf 39.13} &  2177.17 &  {\bf {\color{red} 0.94}} &  {\bf {\color{red} 0.86}} &  {\bf {\color{red} 0.81}} &  {\bf {\color{red} 0.76}} &  {\bf {\color{red} 0.71}} &  {\bf {\color{red} 0.66}} \\ 
\hline 
 2044 &  64 &  65 ans 5 mois &  -7.08\% &  2248.66 &  {\bf 42.43} &  2205.48 &  {\bf 1.02} &  {\bf {\color{red} 0.94}} &  {\bf {\color{red} 0.88}} &  {\bf {\color{red} 0.83}} &  {\bf {\color{red} 0.78}} &  {\bf {\color{red} 0.73}} \\ 
\hline 
 2045 &  65 &  65 ans 6 mois &  -2.50\% &  2463.12 &  {\bf 45.88} &  2234.15 &  {\bf 1.10} &  {\bf 1.03} &  {\bf {\color{red} 0.97}} &  {\bf {\color{red} 0.91}} &  {\bf {\color{red} 0.85}} &  {\bf {\color{red} 0.80}} \\ 
\hline 
 2046 &  66 &  65 ans 7 mois &  2.08\% &  2689.72 &  {\bf 49.45} &  2263.19 &  {\bf 1.19} &  {\bf 1.13} &  {\bf 1.06} &  {\bf {\color{red} 0.99}} &  {\bf {\color{red} 0.93}} &  {\bf {\color{red} 0.87}} \\ 
\hline 
 2047 &  67 &  65 ans 8 mois &  6.67\% &  2928.82 &  {\bf 53.16} &  2292.61 &  {\bf 1.28} &  {\bf 1.23} &  {\bf 1.15} &  {\bf 1.08} &  {\bf 1.01} &  {\bf {\color{red} 0.95}} \\ 
\hline 
\hline 
\end{tabular} 
\end{center} } 

 \begin{center}\includegraphics[width=0.9\textwidth]{fig/Infirmier_1980_22_dest_retraite.pdf}\end{center} \label{fig/Infirmier_1980_22_dest_retraite.pdf} 

\newpage 
 
\subsection{Génération 1990 (début en 2012)} 

\paragraph{Retraites possibles et ratios Revenu/SMIC à 70, 75, 80, 85, 90 ans avec le modèle \emph{Gouvernement truqué (âge-pivot bloqué à 65 ans)}}  
 
{ \scriptsize \begin{center} 
\begin{tabular}[htb]{|c|c||c|c||c|c||c||c|c|c|c|c|c|} 
\hline 
 Retraite en &  Âge &  Âge pivot &  Décote/Surcote &  Retraite (\euro{} 2019) &  Tx Rempl(\%) &  SMIC (\euro{} 2019) &  Retraite/SMIC &  Rev70/SMIC &  Rev75/SMIC &  Rev80/SMIC &  Rev85/SMIC &  Rev90/SMIC \\ 
\hline \hline 
 2052 &  62 &  65 ans 0 mois &  -15.00\% &  1773.04 &  {\bf 45.65} &  2601.14 &  {\bf {\color{red} 0.68}} &  {\bf {\color{red} 0.61}} &  {\bf {\color{red} 0.58}} &  {\bf {\color{red} 0.54}} &  {\bf {\color{red} 0.51}} &  {\bf {\color{red} 0.47}} \\ 
\hline 
 2053 &  63 &  65 ans 0 mois &  -10.00\% &  1950.49 &  {\bf 50.13} &  2634.96 &  {\bf {\color{red} 0.74}} &  {\bf {\color{red} 0.68}} &  {\bf {\color{red} 0.63}} &  {\bf {\color{red} 0.59}} &  {\bf {\color{red} 0.56}} &  {\bf {\color{red} 0.52}} \\ 
\hline 
 2054 &  64 &  65 ans 0 mois &  -5.00\% &  2137.16 &  {\bf 54.83} &  2669.21 &  {\bf {\color{red} 0.80}} &  {\bf {\color{red} 0.74}} &  {\bf {\color{red} 0.69}} &  {\bf {\color{red} 0.65}} &  {\bf {\color{red} 0.61}} &  {\bf {\color{red} 0.57}} \\ 
\hline 
 2055 &  65 &  65 ans 0 mois &  0.00\% &  2333.24 &  {\bf 59.75} &  2703.91 &  {\bf {\color{red} 0.86}} &  {\bf {\color{red} 0.81}} &  {\bf {\color{red} 0.76}} &  {\bf {\color{red} 0.71}} &  {\bf {\color{red} 0.67}} &  {\bf {\color{red} 0.62}} \\ 
\hline 
 2056 &  66 &  65 ans 0 mois &  5.00\% &  2538.93 &  {\bf 64.90} &  2739.06 &  {\bf {\color{red} 0.93}} &  {\bf {\color{red} 0.88}} &  {\bf {\color{red} 0.83}} &  {\bf {\color{red} 0.77}} &  {\bf {\color{red} 0.73}} &  {\bf {\color{red} 0.68}} \\ 
\hline 
 2057 &  67 &  65 ans 0 mois &  10.00\% &  2754.41 &  {\bf 70.29} &  2774.67 &  {\bf {\color{red} 0.99}} &  {\bf {\color{red} 0.95}} &  {\bf {\color{red} 0.90}} &  {\bf {\color{red} 0.84}} &  {\bf {\color{red} 0.79}} &  {\bf {\color{red} 0.74}} \\ 
\hline 
\hline 
\end{tabular} 
\end{center} } 
\paragraph{Retraites possibles et ratios Revenu/SMIC à 70, 75, 80, 85, 90 ans avec le modèle \emph{Gouvernement corrigé (âge-pivot glissant)}}  
 
{ \scriptsize \begin{center} 
\begin{tabular}[htb]{|c|c||c|c||c|c||c||c|c|c|c|c|c|} 
\hline 
 Retraite en &  Âge &  Âge pivot &  Décote/Surcote &  Retraite (\euro{} 2019) &  Tx Rempl(\%) &  SMIC (\euro{} 2019) &  Retraite/SMIC &  Rev70/SMIC &  Rev75/SMIC &  Rev80/SMIC &  Rev85/SMIC &  Rev90/SMIC \\ 
\hline \hline 
 2052 &  62 &  66 ans 1 mois &  -20.42\% &  1660.06 &  {\bf 42.74} &  2601.14 &  {\bf {\color{red} 0.64}} &  {\bf {\color{red} 0.58}} &  {\bf {\color{red} 0.54}} &  {\bf {\color{red} 0.51}} &  {\bf {\color{red} 0.47}} &  {\bf {\color{red} 0.44}} \\ 
\hline 
 2053 &  63 &  66 ans 2 mois &  -15.83\% &  1824.07 &  {\bf 46.88} &  2634.96 &  {\bf {\color{red} 0.69}} &  {\bf {\color{red} 0.63}} &  {\bf {\color{red} 0.59}} &  {\bf {\color{red} 0.56}} &  {\bf {\color{red} 0.52}} &  {\bf {\color{red} 0.49}} \\ 
\hline 
 2054 &  64 &  66 ans 3 mois &  -11.25\% &  1996.56 &  {\bf 51.22} &  2669.21 &  {\bf {\color{red} 0.75}} &  {\bf {\color{red} 0.69}} &  {\bf {\color{red} 0.65}} &  {\bf {\color{red} 0.61}} &  {\bf {\color{red} 0.57}} &  {\bf {\color{red} 0.53}} \\ 
\hline 
 2055 &  65 &  66 ans 4 mois &  -6.67\% &  2298.33 &  {\bf 58.86} &  2703.91 &  {\bf {\color{red} 0.85}} &  {\bf {\color{red} 0.80}} &  {\bf {\color{red} 0.75}} &  {\bf {\color{red} 0.70}} &  {\bf {\color{red} 0.66}} &  {\bf {\color{red} 0.62}} \\ 
\hline 
 2056 &  66 &  66 ans 5 mois &  -2.08\% &  2367.65 &  {\bf 60.53} &  2739.06 &  {\bf {\color{red} 0.86}} &  {\bf {\color{red} 0.82}} &  {\bf {\color{red} 0.77}} &  {\bf {\color{red} 0.72}} &  {\bf {\color{red} 0.68}} &  {\bf {\color{red} 0.63}} \\ 
\hline 
 2057 &  67 &  66 ans 6 mois &  2.50\% &  2566.61 &  {\bf 65.50} &  2774.67 &  {\bf {\color{red} 0.93}} &  {\bf {\color{red} 0.89}} &  {\bf {\color{red} 0.83}} &  {\bf {\color{red} 0.78}} &  {\bf {\color{red} 0.73}} &  {\bf {\color{red} 0.69}} \\ 
\hline 
\hline 
\end{tabular} 
\end{center} } 
\paragraph{Retraites possibles et ratios Revenu/SMIC à 70, 75, 80, 85, 90 ans avec le modèle \emph{Destinie2 (revalorisation de la fonction publique)}}  
 
{ \scriptsize \begin{center} 
\begin{tabular}[htb]{|c|c||c|c||c|c||c||c|c|c|c|c|c|} 
\hline 
 Retraite en &  Âge &  Âge pivot &  Décote/Surcote &  Retraite (\euro{} 2019) &  Tx Rempl(\%) &  SMIC (\euro{} 2019) &  Retraite/SMIC &  Rev70/SMIC &  Rev75/SMIC &  Rev80/SMIC &  Rev85/SMIC &  Rev90/SMIC \\ 
\hline \hline 
 2052 &  62 &  66 ans 1 mois &  -20.42\% &  2076.85 &  {\bf 35.34} &  2445.56 &  {\bf {\color{red} 0.85}} &  {\bf {\color{red} 0.77}} &  {\bf {\color{red} 0.72}} &  {\bf {\color{red} 0.67}} &  {\bf {\color{red} 0.63}} &  {\bf {\color{red} 0.59}} \\ 
\hline 
 2053 &  63 &  66 ans 2 mois &  -15.83\% &  2294.77 &  {\bf 38.54} &  2477.35 &  {\bf {\color{red} 0.93}} &  {\bf {\color{red} 0.85}} &  {\bf {\color{red} 0.79}} &  {\bf {\color{red} 0.74}} &  {\bf {\color{red} 0.70}} &  {\bf {\color{red} 0.65}} \\ 
\hline 
 2054 &  64 &  66 ans 3 mois &  -11.25\% &  2525.69 &  {\bf 41.88} &  2509.56 &  {\bf 1.01} &  {\bf {\color{red} 0.93}} &  {\bf {\color{red} 0.87}} &  {\bf {\color{red} 0.82}} &  {\bf {\color{red} 0.77}} &  {\bf {\color{red} 0.72}} \\ 
\hline 
 2055 &  65 &  66 ans 4 mois &  -6.67\% &  2770.02 &  {\bf 45.34} &  2542.18 &  {\bf 1.09} &  {\bf 1.02} &  {\bf {\color{red} 0.96}} &  {\bf {\color{red} 0.90}} &  {\bf {\color{red} 0.84}} &  {\bf {\color{red} 0.79}} \\ 
\hline 
 2056 &  66 &  66 ans 5 mois &  -2.08\% &  3028.18 &  {\bf 48.93} &  2575.23 &  {\bf 1.18} &  {\bf 1.12} &  {\bf 1.05} &  {\bf {\color{red} 0.98}} &  {\bf {\color{red} 0.92}} &  {\bf {\color{red} 0.86}} \\ 
\hline 
 2057 &  67 &  66 ans 6 mois &  2.50\% &  3300.58 &  {\bf 52.65} &  2608.71 &  {\bf 1.27} &  {\bf 1.22} &  {\bf 1.14} &  {\bf 1.07} &  {\bf 1.00} &  {\bf {\color{red} 0.94}} \\ 
\hline 
\hline 
\end{tabular} 
\end{center} } 

 \begin{center}\includegraphics[width=0.9\textwidth]{fig/Infirmier_1990_22_dest_retraite.pdf}\end{center} \label{fig/Infirmier_1990_22_dest_retraite.pdf} 

\newpage 
 
\subsection{Génération 2003 (début en 2025)} 

\paragraph{Retraites possibles et ratios Revenu/SMIC à 70, 75, 80, 85, 90 ans avec le modèle \emph{Gouvernement truqué (âge-pivot bloqué à 65 ans)}}  
 
{ \scriptsize \begin{center} 
\begin{tabular}[htb]{|c|c||c|c||c|c||c||c|c|c|c|c|c|} 
\hline 
 Retraite en &  Âge &  Âge pivot &  Décote/Surcote &  Retraite (\euro{} 2019) &  Tx Rempl(\%) &  SMIC (\euro{} 2019) &  Retraite/SMIC &  Rev70/SMIC &  Rev75/SMIC &  Rev80/SMIC &  Rev85/SMIC &  Rev90/SMIC \\ 
\hline \hline 
 2065 &  62 &  65 ans 0 mois &  -15.00\% &  1896.25 &  {\bf 48.82} &  3076.71 &  {\bf {\color{red} 0.62}} &  {\bf {\color{red} 0.56}} &  {\bf {\color{red} 0.52}} &  {\bf {\color{red} 0.49}} &  {\bf {\color{red} 0.46}} &  {\bf {\color{red} 0.43}} \\ 
\hline 
 2066 &  63 &  65 ans 0 mois &  -10.00\% &  2082.64 &  {\bf 53.52} &  3116.71 &  {\bf {\color{red} 0.67}} &  {\bf {\color{red} 0.61}} &  {\bf {\color{red} 0.57}} &  {\bf {\color{red} 0.54}} &  {\bf {\color{red} 0.50}} &  {\bf {\color{red} 0.47}} \\ 
\hline 
 2067 &  64 &  65 ans 0 mois &  -5.00\% &  2278.46 &  {\bf 58.45} &  3157.23 &  {\bf {\color{red} 0.72}} &  {\bf {\color{red} 0.67}} &  {\bf {\color{red} 0.63}} &  {\bf {\color{red} 0.59}} &  {\bf {\color{red} 0.55}} &  {\bf {\color{red} 0.52}} \\ 
\hline 
 2068 &  65 &  65 ans 0 mois &  0.00\% &  2718.53 &  {\bf 69.62} &  3198.27 &  {\bf {\color{red} 0.85}} &  {\bf {\color{red} 0.80}} &  {\bf {\color{red} 0.75}} &  {\bf {\color{red} 0.70}} &  {\bf {\color{red} 0.66}} &  {\bf {\color{red} 0.62}} \\ 
\hline 
 2069 &  66 &  65 ans 0 mois &  5.00\% &  2753.87 &  {\bf 70.40} &  3239.85 &  {\bf {\color{red} 0.85}} &  {\bf {\color{red} 0.81}} &  {\bf {\color{red} 0.76}} &  {\bf {\color{red} 0.71}} &  {\bf {\color{red} 0.67}} &  {\bf {\color{red} 0.62}} \\ 
\hline 
 2070 &  67 &  65 ans 0 mois &  10.00\% &  2924.48 &  {\bf 74.63} &  3281.97 &  {\bf {\color{red} 0.89}} &  {\bf {\color{red} 0.86}} &  {\bf {\color{red} 0.80}} &  {\bf {\color{red} 0.75}} &  {\bf {\color{red} 0.71}} &  {\bf {\color{red} 0.66}} \\ 
\hline 
\hline 
\end{tabular} 
\end{center} } 
\paragraph{Retraites possibles et ratios Revenu/SMIC à 70, 75, 80, 85, 90 ans avec le modèle \emph{Gouvernement corrigé (âge-pivot glissant)}}  
 
{ \scriptsize \begin{center} 
\begin{tabular}[htb]{|c|c||c|c||c|c||c||c|c|c|c|c|c|} 
\hline 
 Retraite en &  Âge &  Âge pivot &  Décote/Surcote &  Retraite (\euro{} 2019) &  Tx Rempl(\%) &  SMIC (\euro{} 2019) &  Retraite/SMIC &  Rev70/SMIC &  Rev75/SMIC &  Rev80/SMIC &  Rev85/SMIC &  Rev90/SMIC \\ 
\hline \hline 
 2065 &  62 &  67 ans 2 mois &  -25.83\% &  1654.57 &  {\bf 42.60} &  3076.71 &  {\bf {\color{red} 0.54}} &  {\bf {\color{red} 0.48}} &  {\bf {\color{red} 0.45}} &  {\bf {\color{red} 0.43}} &  {\bf {\color{red} 0.40}} &  {\bf {\color{red} 0.37}} \\ 
\hline 
 2066 &  63 &  67 ans 3 mois &  -21.25\% &  1822.31 &  {\bf 46.83} &  3116.71 &  {\bf {\color{red} 0.58}} &  {\bf {\color{red} 0.53}} &  {\bf {\color{red} 0.50}} &  {\bf {\color{red} 0.47}} &  {\bf {\color{red} 0.44}} &  {\bf {\color{red} 0.41}} \\ 
\hline 
 2067 &  64 &  67 ans 4 mois &  -16.67\% &  1998.65 &  {\bf 51.27} &  3157.23 &  {\bf {\color{red} 0.63}} &  {\bf {\color{red} 0.59}} &  {\bf {\color{red} 0.55}} &  {\bf {\color{red} 0.51}} &  {\bf {\color{red} 0.48}} &  {\bf {\color{red} 0.45}} \\ 
\hline 
 2068 &  65 &  67 ans 5 mois &  -12.08\% &  2718.53 &  {\bf 69.62} &  3198.27 &  {\bf {\color{red} 0.85}} &  {\bf {\color{red} 0.80}} &  {\bf {\color{red} 0.75}} &  {\bf {\color{red} 0.70}} &  {\bf {\color{red} 0.66}} &  {\bf {\color{red} 0.62}} \\ 
\hline 
 2069 &  66 &  67 ans 6 mois &  -7.50\% &  2753.87 &  {\bf 70.40} &  3239.85 &  {\bf {\color{red} 0.85}} &  {\bf {\color{red} 0.81}} &  {\bf {\color{red} 0.76}} &  {\bf {\color{red} 0.71}} &  {\bf {\color{red} 0.67}} &  {\bf {\color{red} 0.62}} \\ 
\hline 
 2070 &  67 &  67 ans 7 mois &  -2.92\% &  2789.67 &  {\bf 71.19} &  3281.97 &  {\bf {\color{red} 0.85}} &  {\bf {\color{red} 0.82}} &  {\bf {\color{red} 0.77}} &  {\bf {\color{red} 0.72}} &  {\bf {\color{red} 0.67}} &  {\bf {\color{red} 0.63}} \\ 
\hline 
\hline 
\end{tabular} 
\end{center} } 
\paragraph{Retraites possibles et ratios Revenu/SMIC à 70, 75, 80, 85, 90 ans avec le modèle \emph{Destinie2 (revalorisation de la fonction publique)}}  
 
{ \scriptsize \begin{center} 
\begin{tabular}[htb]{|c|c||c|c||c|c||c||c|c|c|c|c|c|} 
\hline 
 Retraite en &  Âge &  Âge pivot &  Décote/Surcote &  Retraite (\euro{} 2019) &  Tx Rempl(\%) &  SMIC (\euro{} 2019) &  Retraite/SMIC &  Rev70/SMIC &  Rev75/SMIC &  Rev80/SMIC &  Rev85/SMIC &  Rev90/SMIC \\ 
\hline \hline 
 2065 &  62 &  67 ans 2 mois &  -25.83\% &  2408.18 &  {\bf 34.64} &  2892.68 &  {\bf {\color{red} 0.83}} &  {\bf {\color{red} 0.75}} &  {\bf {\color{red} 0.70}} &  {\bf {\color{red} 0.66}} &  {\bf {\color{red} 0.62}} &  {\bf {\color{red} 0.58}} \\ 
\hline 
 2066 &  63 &  67 ans 3 mois &  -21.25\% &  2667.43 &  {\bf 37.88} &  2930.29 &  {\bf {\color{red} 0.91}} &  {\bf {\color{red} 0.83}} &  {\bf {\color{red} 0.78}} &  {\bf {\color{red} 0.73}} &  {\bf {\color{red} 0.69}} &  {\bf {\color{red} 0.64}} \\ 
\hline 
 2067 &  64 &  67 ans 4 mois &  -16.67\% &  2942.12 &  {\bf 41.24} &  2968.38 &  {\bf {\color{red} 0.99}} &  {\bf {\color{red} 0.92}} &  {\bf {\color{red} 0.86}} &  {\bf {\color{red} 0.81}} &  {\bf {\color{red} 0.76}} &  {\bf {\color{red} 0.71}} \\ 
\hline 
 2068 &  65 &  67 ans 5 mois &  -12.08\% &  3232.72 &  {\bf 44.74} &  3006.97 &  {\bf 1.08} &  {\bf 1.01} &  {\bf {\color{red} 0.94}} &  {\bf {\color{red} 0.89}} &  {\bf {\color{red} 0.83}} &  {\bf {\color{red} 0.78}} \\ 
\hline 
 2069 &  66 &  67 ans 6 mois &  -7.50\% &  3539.72 &  {\bf 48.35} &  3046.06 &  {\bf 1.16} &  {\bf 1.10} &  {\bf 1.03} &  {\bf {\color{red} 0.97}} &  {\bf {\color{red} 0.91}} &  {\bf {\color{red} 0.85}} \\ 
\hline 
 2070 &  67 &  67 ans 7 mois &  -2.92\% &  3863.61 &  {\bf 52.10} &  3085.66 &  {\bf 1.25} &  {\bf 1.20} &  {\bf 1.13} &  {\bf 1.06} &  {\bf {\color{red} 0.99}} &  {\bf {\color{red} 0.93}} \\ 
\hline 
\hline 
\end{tabular} 
\end{center} } 

 \begin{center}\includegraphics[width=0.9\textwidth]{fig/Infirmier_2003_22_dest_retraite.pdf}\end{center} \label{fig/Infirmier_2003_22_dest_retraite.pdf} 

\newpage 
 
\chapter{Aide-soignante (CN puis HC)} 

\begin{minipage}{0.55\linewidth}\includegraphics[width=0.7\textwidth]{fig/grille_AideSoignant.pdf}\end{minipage} 
\begin{minipage}{0.3\linewidth} 
 \begin{center} 

\begin{tabular}[htb]{|c|c|} 
\hline 
 Indice majoré &  Durée (années) \\ 
\hline \hline 
 327 &  1.00 \\ 
\hline 
 328 &  2.00 \\ 
\hline 
 329 &  2.00 \\ 
\hline 
 330 &  2.00 \\ 
\hline 
 332 &  2.00 \\ 
\hline 
 334 &  2.00 \\ 
\hline 
 338 &  2.00 \\ 
\hline 
 342 &  2.00 \\ 
\hline 
 346 &  3.00 \\ 
\hline 
 356 &  3.00 \\ 
\hline 
 368 &  3.33 \\ 
\hline 
 380 &  2.00 \\ 
\hline 
 390 &  3.00 \\ 
\hline 
 402 &  3.00 \\ 
\hline 
 411 &  4.00 \\ 
\hline 
 418 &   \\ 
\hline 
\hline 
\end{tabular} 
\end{center} 
 \end{minipage} 


 \addto{\captionsenglish}{ \renewcommand{\mtctitle}{}} \setcounter{minitocdepth}{2} 
 \minitoc \newpage 

\section{Début de carrière à 22 ans} 

\subsection{Génération 1975 (début en 1997)} 

\paragraph{Retraites possibles et ratios Revenu/SMIC à 70, 75, 80, 85, 90 ans avec le modèle \emph{Gouvernement truqué (âge-pivot bloqué à 65 ans)}}  
 
{ \scriptsize \begin{center} 
\begin{tabular}[htb]{|c|c||c|c||c|c||c||c|c|c|c|c|c|} 
\hline 
 Retraite en &  Âge &  Âge pivot &  Décote/Surcote &  Retraite (\euro{} 2019) &  Tx Rempl(\%) &  SMIC (\euro{} 2019) &  Retraite/SMIC &  Rev70/SMIC &  Rev75/SMIC &  Rev80/SMIC &  Rev85/SMIC &  Rev90/SMIC \\ 
\hline \hline 
 2037 &  62 &  64 ans 10 mois &  -14.17\% &  1111.54 &  {\bf 44.91} &  2143.00 &  {\bf {\color{red} 0.52}} &  {\bf {\color{red} 0.47}} &  {\bf {\color{red} 0.44}} &  {\bf {\color{red} 0.41}} &  {\bf {\color{red} 0.39}} &  {\bf {\color{red} 0.36}} \\ 
\hline 
 2038 &  63 &  64 ans 11 mois &  -9.58\% &  1211.04 &  {\bf 48.84} &  2170.86 &  {\bf {\color{red} 0.56}} &  {\bf {\color{red} 0.51}} &  {\bf {\color{red} 0.48}} &  {\bf {\color{red} 0.45}} &  {\bf {\color{red} 0.42}} &  {\bf {\color{red} 0.39}} \\ 
\hline 
 2039 &  64 &  65 ans 0 mois &  -5.00\% &  1315.96 &  {\bf 52.97} &  2199.08 &  {\bf {\color{red} 0.60}} &  {\bf {\color{red} 0.55}} &  {\bf {\color{red} 0.52}} &  {\bf {\color{red} 0.49}} &  {\bf {\color{red} 0.46}} &  {\bf {\color{red} 0.43}} \\ 
\hline 
 2040 &  65 &  65 ans 0 mois &  0.00\% &  1893.52 &  {\bf 76.07} &  2227.67 &  {\bf {\color{red} 0.85}} &  {\bf {\color{red} 0.80}} &  {\bf {\color{red} 0.75}} &  {\bf {\color{red} 0.70}} &  {\bf {\color{red} 0.66}} &  {\bf {\color{red} 0.62}} \\ 
\hline 
 2041 &  66 &  65 ans 0 mois &  5.00\% &  1918.14 &  {\bf 76.92} &  2256.63 &  {\bf {\color{red} 0.85}} &  {\bf {\color{red} 0.81}} &  {\bf {\color{red} 0.76}} &  {\bf {\color{red} 0.71}} &  {\bf {\color{red} 0.67}} &  {\bf {\color{red} 0.62}} \\ 
\hline 
 2042 &  67 &  65 ans 0 mois &  10.00\% &  1943.07 &  {\bf 77.78} &  2285.97 &  {\bf {\color{red} 0.85}} &  {\bf {\color{red} 0.82}} &  {\bf {\color{red} 0.77}} &  {\bf {\color{red} 0.72}} &  {\bf {\color{red} 0.67}} &  {\bf {\color{red} 0.63}} \\ 
\hline 
\hline 
\end{tabular} 
\end{center} } 
\paragraph{Retraites possibles et ratios Revenu/SMIC à 70, 75, 80, 85, 90 ans avec le modèle \emph{Gouvernement corrigé (âge-pivot glissant)}}  
 
{ \scriptsize \begin{center} 
\begin{tabular}[htb]{|c|c||c|c||c|c||c||c|c|c|c|c|c|} 
\hline 
 Retraite en &  Âge &  Âge pivot &  Décote/Surcote &  Retraite (\euro{} 2019) &  Tx Rempl(\%) &  SMIC (\euro{} 2019) &  Retraite/SMIC &  Rev70/SMIC &  Rev75/SMIC &  Rev80/SMIC &  Rev85/SMIC &  Rev90/SMIC \\ 
\hline \hline 
 2037 &  62 &  64 ans 10 mois &  -14.17\% &  1111.54 &  {\bf 44.91} &  2143.00 &  {\bf {\color{red} 0.52}} &  {\bf {\color{red} 0.47}} &  {\bf {\color{red} 0.44}} &  {\bf {\color{red} 0.41}} &  {\bf {\color{red} 0.39}} &  {\bf {\color{red} 0.36}} \\ 
\hline 
 2038 &  63 &  64 ans 11 mois &  -9.58\% &  1211.04 &  {\bf 48.84} &  2170.86 &  {\bf {\color{red} 0.56}} &  {\bf {\color{red} 0.51}} &  {\bf {\color{red} 0.48}} &  {\bf {\color{red} 0.45}} &  {\bf {\color{red} 0.42}} &  {\bf {\color{red} 0.39}} \\ 
\hline 
 2039 &  64 &  65 ans 0 mois &  -5.00\% &  1315.96 &  {\bf 52.97} &  2199.08 &  {\bf {\color{red} 0.60}} &  {\bf {\color{red} 0.55}} &  {\bf {\color{red} 0.52}} &  {\bf {\color{red} 0.49}} &  {\bf {\color{red} 0.46}} &  {\bf {\color{red} 0.43}} \\ 
\hline 
 2040 &  65 &  65 ans 1 mois &  -0.42\% &  1893.52 &  {\bf 76.07} &  2227.67 &  {\bf {\color{red} 0.85}} &  {\bf {\color{red} 0.80}} &  {\bf {\color{red} 0.75}} &  {\bf {\color{red} 0.70}} &  {\bf {\color{red} 0.66}} &  {\bf {\color{red} 0.62}} \\ 
\hline 
 2041 &  66 &  65 ans 2 mois &  4.17\% &  1918.14 &  {\bf 76.92} &  2256.63 &  {\bf {\color{red} 0.85}} &  {\bf {\color{red} 0.81}} &  {\bf {\color{red} 0.76}} &  {\bf {\color{red} 0.71}} &  {\bf {\color{red} 0.67}} &  {\bf {\color{red} 0.62}} \\ 
\hline 
 2042 &  67 &  65 ans 3 mois &  8.75\% &  1943.07 &  {\bf 77.78} &  2285.97 &  {\bf {\color{red} 0.85}} &  {\bf {\color{red} 0.82}} &  {\bf {\color{red} 0.77}} &  {\bf {\color{red} 0.72}} &  {\bf {\color{red} 0.67}} &  {\bf {\color{red} 0.63}} \\ 
\hline 
\hline 
\end{tabular} 
\end{center} } 
\paragraph{Retraites possibles et ratios Revenu/SMIC à 70, 75, 80, 85, 90 ans avec le modèle \emph{Destinie2 (revalorisation de la fonction publique)}}  
 
{ \scriptsize \begin{center} 
\begin{tabular}[htb]{|c|c||c|c||c|c||c||c|c|c|c|c|c|} 
\hline 
 Retraite en &  Âge &  Âge pivot &  Décote/Surcote &  Retraite (\euro{} 2019) &  Tx Rempl(\%) &  SMIC (\euro{} 2019) &  Retraite/SMIC &  Rev70/SMIC &  Rev75/SMIC &  Rev80/SMIC &  Rev85/SMIC &  Rev90/SMIC \\ 
\hline \hline 
 2037 &  62 &  64 ans 10 mois &  -14.17\% &  1232.66 &  {\bf 39.94} &  2014.82 &  {\bf {\color{red} 0.61}} &  {\bf {\color{red} 0.55}} &  {\bf {\color{red} 0.52}} &  {\bf {\color{red} 0.48}} &  {\bf {\color{red} 0.45}} &  {\bf {\color{red} 0.43}} \\ 
\hline 
 2038 &  63 &  64 ans 11 mois &  -9.58\% &  1347.73 &  {\bf 43.11} &  2041.01 &  {\bf {\color{red} 0.66}} &  {\bf {\color{red} 0.60}} &  {\bf {\color{red} 0.57}} &  {\bf {\color{red} 0.53}} &  {\bf {\color{red} 0.50}} &  {\bf {\color{red} 0.47}} \\ 
\hline 
 2039 &  64 &  65 ans 0 mois &  -5.00\% &  1469.81 &  {\bf 46.41} &  2067.55 &  {\bf {\color{red} 0.71}} &  {\bf {\color{red} 0.66}} &  {\bf {\color{red} 0.62}} &  {\bf {\color{red} 0.58}} &  {\bf {\color{red} 0.54}} &  {\bf {\color{red} 0.51}} \\ 
\hline 
 2040 &  65 &  65 ans 1 mois &  -0.42\% &  1780.26 &  {\bf 55.49} &  2094.43 &  {\bf {\color{red} 0.85}} &  {\bf {\color{red} 0.80}} &  {\bf {\color{red} 0.75}} &  {\bf {\color{red} 0.70}} &  {\bf {\color{red} 0.66}} &  {\bf {\color{red} 0.62}} \\ 
\hline 
 2041 &  66 &  65 ans 2 mois &  4.17\% &  1803.40 &  {\bf 55.49} &  2121.65 &  {\bf {\color{red} 0.85}} &  {\bf {\color{red} 0.81}} &  {\bf {\color{red} 0.76}} &  {\bf {\color{red} 0.71}} &  {\bf {\color{red} 0.67}} &  {\bf {\color{red} 0.62}} \\ 
\hline 
 2042 &  67 &  65 ans 3 mois &  8.75\% &  1882.28 &  {\bf 57.17} &  2149.23 &  {\bf {\color{red} 0.88}} &  {\bf {\color{red} 0.84}} &  {\bf {\color{red} 0.79}} &  {\bf {\color{red} 0.74}} &  {\bf {\color{red} 0.69}} &  {\bf {\color{red} 0.65}} \\ 
\hline 
\hline 
\end{tabular} 
\end{center} } 

 \begin{center}\includegraphics[width=0.9\textwidth]{fig/AideSoignant_1975_22_dest_retraite.pdf}\end{center} \label{fig/AideSoignant_1975_22_dest_retraite.pdf} 

\newpage 
 
\subsection{Génération 1980 (début en 2002)} 

\paragraph{Retraites possibles et ratios Revenu/SMIC à 70, 75, 80, 85, 90 ans avec le modèle \emph{Gouvernement truqué (âge-pivot bloqué à 65 ans)}}  
 
{ \scriptsize \begin{center} 
\begin{tabular}[htb]{|c|c||c|c||c|c||c||c|c|c|c|c|c|} 
\hline 
 Retraite en &  Âge &  Âge pivot &  Décote/Surcote &  Retraite (\euro{} 2019) &  Tx Rempl(\%) &  SMIC (\euro{} 2019) &  Retraite/SMIC &  Rev70/SMIC &  Rev75/SMIC &  Rev80/SMIC &  Rev85/SMIC &  Rev90/SMIC \\ 
\hline \hline 
 2042 &  62 &  65 ans 0 mois &  -15.00\% &  1124.91 &  {\bf 45.45} &  2285.97 &  {\bf {\color{red} 0.49}} &  {\bf {\color{red} 0.44}} &  {\bf {\color{red} 0.42}} &  {\bf {\color{red} 0.39}} &  {\bf {\color{red} 0.37}} &  {\bf {\color{red} 0.34}} \\ 
\hline 
 2043 &  63 &  65 ans 0 mois &  -10.00\% &  1235.80 &  {\bf 49.83} &  2315.68 &  {\bf {\color{red} 0.53}} &  {\bf {\color{red} 0.49}} &  {\bf {\color{red} 0.46}} &  {\bf {\color{red} 0.43}} &  {\bf {\color{red} 0.40}} &  {\bf {\color{red} 0.38}} \\ 
\hline 
 2044 &  64 &  65 ans 0 mois &  -5.00\% &  1353.27 &  {\bf 54.47} &  2345.79 &  {\bf {\color{red} 0.58}} &  {\bf {\color{red} 0.53}} &  {\bf {\color{red} 0.50}} &  {\bf {\color{red} 0.47}} &  {\bf {\color{red} 0.44}} &  {\bf {\color{red} 0.41}} \\ 
\hline 
 2045 &  65 &  65 ans 0 mois &  0.00\% &  2019.84 &  {\bf 81.15} &  2376.28 &  {\bf {\color{red} 0.85}} &  {\bf {\color{red} 0.80}} &  {\bf {\color{red} 0.75}} &  {\bf {\color{red} 0.70}} &  {\bf {\color{red} 0.66}} &  {\bf {\color{red} 0.62}} \\ 
\hline 
 2046 &  66 &  65 ans 0 mois &  5.00\% &  2046.10 &  {\bf 82.05} &  2407.18 &  {\bf {\color{red} 0.85}} &  {\bf {\color{red} 0.81}} &  {\bf {\color{red} 0.76}} &  {\bf {\color{red} 0.71}} &  {\bf {\color{red} 0.67}} &  {\bf {\color{red} 0.62}} \\ 
\hline 
 2047 &  67 &  65 ans 0 mois &  10.00\% &  2072.70 &  {\bf 82.97} &  2438.47 &  {\bf {\color{red} 0.85}} &  {\bf {\color{red} 0.82}} &  {\bf {\color{red} 0.77}} &  {\bf {\color{red} 0.72}} &  {\bf {\color{red} 0.67}} &  {\bf {\color{red} 0.63}} \\ 
\hline 
\hline 
\end{tabular} 
\end{center} } 
\paragraph{Retraites possibles et ratios Revenu/SMIC à 70, 75, 80, 85, 90 ans avec le modèle \emph{Gouvernement corrigé (âge-pivot glissant)}}  
 
{ \scriptsize \begin{center} 
\begin{tabular}[htb]{|c|c||c|c||c|c||c||c|c|c|c|c|c|} 
\hline 
 Retraite en &  Âge &  Âge pivot &  Décote/Surcote &  Retraite (\euro{} 2019) &  Tx Rempl(\%) &  SMIC (\euro{} 2019) &  Retraite/SMIC &  Rev70/SMIC &  Rev75/SMIC &  Rev80/SMIC &  Rev85/SMIC &  Rev90/SMIC \\ 
\hline \hline 
 2042 &  62 &  65 ans 3 mois &  -16.25\% &  1108.37 &  {\bf 44.78} &  2285.97 &  {\bf {\color{red} 0.48}} &  {\bf {\color{red} 0.44}} &  {\bf {\color{red} 0.41}} &  {\bf {\color{red} 0.38}} &  {\bf {\color{red} 0.36}} &  {\bf {\color{red} 0.34}} \\ 
\hline 
 2043 &  63 &  65 ans 4 mois &  -11.67\% &  1212.92 &  {\bf 48.91} &  2315.68 &  {\bf {\color{red} 0.52}} &  {\bf {\color{red} 0.48}} &  {\bf {\color{red} 0.45}} &  {\bf {\color{red} 0.42}} &  {\bf {\color{red} 0.39}} &  {\bf {\color{red} 0.37}} \\ 
\hline 
 2044 &  64 &  65 ans 5 mois &  -7.08\% &  1323.59 &  {\bf 53.28} &  2345.79 &  {\bf {\color{red} 0.56}} &  {\bf {\color{red} 0.52}} &  {\bf {\color{red} 0.49}} &  {\bf {\color{red} 0.46}} &  {\bf {\color{red} 0.43}} &  {\bf {\color{red} 0.40}} \\ 
\hline 
 2045 &  65 &  65 ans 6 mois &  -2.50\% &  2019.84 &  {\bf 81.15} &  2376.28 &  {\bf {\color{red} 0.85}} &  {\bf {\color{red} 0.80}} &  {\bf {\color{red} 0.75}} &  {\bf {\color{red} 0.70}} &  {\bf {\color{red} 0.66}} &  {\bf {\color{red} 0.62}} \\ 
\hline 
 2046 &  66 &  65 ans 7 mois &  2.08\% &  2046.10 &  {\bf 82.05} &  2407.18 &  {\bf {\color{red} 0.85}} &  {\bf {\color{red} 0.81}} &  {\bf {\color{red} 0.76}} &  {\bf {\color{red} 0.71}} &  {\bf {\color{red} 0.67}} &  {\bf {\color{red} 0.62}} \\ 
\hline 
 2047 &  67 &  65 ans 8 mois &  6.67\% &  2072.70 &  {\bf 82.97} &  2438.47 &  {\bf {\color{red} 0.85}} &  {\bf {\color{red} 0.82}} &  {\bf {\color{red} 0.77}} &  {\bf {\color{red} 0.72}} &  {\bf {\color{red} 0.67}} &  {\bf {\color{red} 0.63}} \\ 
\hline 
\hline 
\end{tabular} 
\end{center} } 
\paragraph{Retraites possibles et ratios Revenu/SMIC à 70, 75, 80, 85, 90 ans avec le modèle \emph{Destinie2 (revalorisation de la fonction publique)}}  
 
{ \scriptsize \begin{center} 
\begin{tabular}[htb]{|c|c||c|c||c|c||c||c|c|c|c|c|c|} 
\hline 
 Retraite en &  Âge &  Âge pivot &  Décote/Surcote &  Retraite (\euro{} 2019) &  Tx Rempl(\%) &  SMIC (\euro{} 2019) &  Retraite/SMIC &  Rev70/SMIC &  Rev75/SMIC &  Rev80/SMIC &  Rev85/SMIC &  Rev90/SMIC \\ 
\hline \hline 
 2042 &  62 &  65 ans 3 mois &  -16.25\% &  1266.83 &  {\bf 38.48} &  2149.23 &  {\bf {\color{red} 0.59}} &  {\bf {\color{red} 0.53}} &  {\bf {\color{red} 0.50}} &  {\bf {\color{red} 0.47}} &  {\bf {\color{red} 0.44}} &  {\bf {\color{red} 0.41}} \\ 
\hline 
 2043 &  63 &  65 ans 4 mois &  -11.67\% &  1392.48 &  {\bf 41.75} &  2177.17 &  {\bf {\color{red} 0.64}} &  {\bf {\color{red} 0.58}} &  {\bf {\color{red} 0.55}} &  {\bf {\color{red} 0.51}} &  {\bf {\color{red} 0.48}} &  {\bf {\color{red} 0.45}} \\ 
\hline 
 2044 &  64 &  65 ans 5 mois &  -7.08\% &  1526.34 &  {\bf 45.18} &  2205.48 &  {\bf {\color{red} 0.69}} &  {\bf {\color{red} 0.64}} &  {\bf {\color{red} 0.60}} &  {\bf {\color{red} 0.56}} &  {\bf {\color{red} 0.53}} &  {\bf {\color{red} 0.49}} \\ 
\hline 
 2045 &  65 &  65 ans 6 mois &  -2.50\% &  1899.03 &  {\bf 55.49} &  2234.15 &  {\bf {\color{red} 0.85}} &  {\bf {\color{red} 0.80}} &  {\bf {\color{red} 0.75}} &  {\bf {\color{red} 0.70}} &  {\bf {\color{red} 0.66}} &  {\bf {\color{red} 0.62}} \\ 
\hline 
 2046 &  66 &  65 ans 7 mois &  2.08\% &  1923.71 &  {\bf 55.49} &  2263.19 &  {\bf {\color{red} 0.85}} &  {\bf {\color{red} 0.81}} &  {\bf {\color{red} 0.76}} &  {\bf {\color{red} 0.71}} &  {\bf {\color{red} 0.67}} &  {\bf {\color{red} 0.62}} \\ 
\hline 
 2047 &  67 &  65 ans 8 mois &  6.67\% &  1977.86 &  {\bf 56.32} &  2292.61 &  {\bf {\color{red} 0.86}} &  {\bf {\color{red} 0.83}} &  {\bf {\color{red} 0.78}} &  {\bf {\color{red} 0.73}} &  {\bf {\color{red} 0.68}} &  {\bf {\color{red} 0.64}} \\ 
\hline 
\hline 
\end{tabular} 
\end{center} } 

 \begin{center}\includegraphics[width=0.9\textwidth]{fig/AideSoignant_1980_22_dest_retraite.pdf}\end{center} \label{fig/AideSoignant_1980_22_dest_retraite.pdf} 

\newpage 
 
\subsection{Génération 1990 (début en 2012)} 

\paragraph{Retraites possibles et ratios Revenu/SMIC à 70, 75, 80, 85, 90 ans avec le modèle \emph{Gouvernement truqué (âge-pivot bloqué à 65 ans)}}  
 
{ \scriptsize \begin{center} 
\begin{tabular}[htb]{|c|c||c|c||c|c||c||c|c|c|c|c|c|} 
\hline 
 Retraite en &  Âge &  Âge pivot &  Décote/Surcote &  Retraite (\euro{} 2019) &  Tx Rempl(\%) &  SMIC (\euro{} 2019) &  Retraite/SMIC &  Rev70/SMIC &  Rev75/SMIC &  Rev80/SMIC &  Rev85/SMIC &  Rev90/SMIC \\ 
\hline \hline 
 2052 &  62 &  65 ans 0 mois &  -15.00\% &  1199.24 &  {\bf 46.10} &  2601.14 &  {\bf {\color{red} 0.46}} &  {\bf {\color{red} 0.42}} &  {\bf {\color{red} 0.39}} &  {\bf {\color{red} 0.37}} &  {\bf {\color{red} 0.34}} &  {\bf {\color{red} 0.32}} \\ 
\hline 
 2053 &  63 &  65 ans 0 mois &  -10.00\% &  1319.30 &  {\bf 50.07} &  2634.96 &  {\bf {\color{red} 0.50}} &  {\bf {\color{red} 0.46}} &  {\bf {\color{red} 0.43}} &  {\bf {\color{red} 0.40}} &  {\bf {\color{red} 0.38}} &  {\bf {\color{red} 0.35}} \\ 
\hline 
 2054 &  64 &  65 ans 0 mois &  -5.00\% &  1445.99 &  {\bf 54.17} &  2669.21 &  {\bf {\color{red} 0.54}} &  {\bf {\color{red} 0.50}} &  {\bf {\color{red} 0.47}} &  {\bf {\color{red} 0.44}} &  {\bf {\color{red} 0.41}} &  {\bf {\color{red} 0.39}} \\ 
\hline 
 2055 &  65 &  65 ans 0 mois &  0.00\% &  2298.33 &  {\bf 85.00} &  2703.91 &  {\bf {\color{red} 0.85}} &  {\bf {\color{red} 0.80}} &  {\bf {\color{red} 0.75}} &  {\bf {\color{red} 0.70}} &  {\bf {\color{red} 0.66}} &  {\bf {\color{red} 0.62}} \\ 
\hline 
 2056 &  66 &  65 ans 0 mois &  5.00\% &  2328.20 &  {\bf 85.00} &  2739.06 &  {\bf {\color{red} 0.85}} &  {\bf {\color{red} 0.81}} &  {\bf {\color{red} 0.76}} &  {\bf {\color{red} 0.71}} &  {\bf {\color{red} 0.67}} &  {\bf {\color{red} 0.62}} \\ 
\hline 
 2057 &  67 &  65 ans 0 mois &  10.00\% &  2358.47 &  {\bf 85.00} &  2774.67 &  {\bf {\color{red} 0.85}} &  {\bf {\color{red} 0.82}} &  {\bf {\color{red} 0.77}} &  {\bf {\color{red} 0.72}} &  {\bf {\color{red} 0.67}} &  {\bf {\color{red} 0.63}} \\ 
\hline 
\hline 
\end{tabular} 
\end{center} } 
\paragraph{Retraites possibles et ratios Revenu/SMIC à 70, 75, 80, 85, 90 ans avec le modèle \emph{Gouvernement corrigé (âge-pivot glissant)}}  
 
{ \scriptsize \begin{center} 
\begin{tabular}[htb]{|c|c||c|c||c|c||c||c|c|c|c|c|c|} 
\hline 
 Retraite en &  Âge &  Âge pivot &  Décote/Surcote &  Retraite (\euro{} 2019) &  Tx Rempl(\%) &  SMIC (\euro{} 2019) &  Retraite/SMIC &  Rev70/SMIC &  Rev75/SMIC &  Rev80/SMIC &  Rev85/SMIC &  Rev90/SMIC \\ 
\hline \hline 
 2052 &  62 &  66 ans 1 mois &  -20.42\% &  1122.82 &  {\bf 43.17} &  2601.14 &  {\bf {\color{red} 0.43}} &  {\bf {\color{red} 0.39}} &  {\bf {\color{red} 0.36}} &  {\bf {\color{red} 0.34}} &  {\bf {\color{red} 0.32}} &  {\bf {\color{red} 0.30}} \\ 
\hline 
 2053 &  63 &  66 ans 2 mois &  -15.83\% &  1233.79 &  {\bf 46.82} &  2634.96 &  {\bf {\color{red} 0.47}} &  {\bf {\color{red} 0.43}} &  {\bf {\color{red} 0.40}} &  {\bf {\color{red} 0.38}} &  {\bf {\color{red} 0.35}} &  {\bf {\color{red} 0.33}} \\ 
\hline 
 2054 &  64 &  66 ans 3 mois &  -11.25\% &  1350.86 &  {\bf 50.61} &  2669.21 &  {\bf {\color{red} 0.51}} &  {\bf {\color{red} 0.47}} &  {\bf {\color{red} 0.44}} &  {\bf {\color{red} 0.41}} &  {\bf {\color{red} 0.39}} &  {\bf {\color{red} 0.36}} \\ 
\hline 
 2055 &  65 &  66 ans 4 mois &  -6.67\% &  2298.33 &  {\bf 85.00} &  2703.91 &  {\bf {\color{red} 0.85}} &  {\bf {\color{red} 0.80}} &  {\bf {\color{red} 0.75}} &  {\bf {\color{red} 0.70}} &  {\bf {\color{red} 0.66}} &  {\bf {\color{red} 0.62}} \\ 
\hline 
 2056 &  66 &  66 ans 5 mois &  -2.08\% &  2328.20 &  {\bf 85.00} &  2739.06 &  {\bf {\color{red} 0.85}} &  {\bf {\color{red} 0.81}} &  {\bf {\color{red} 0.76}} &  {\bf {\color{red} 0.71}} &  {\bf {\color{red} 0.67}} &  {\bf {\color{red} 0.62}} \\ 
\hline 
 2057 &  67 &  66 ans 6 mois &  2.50\% &  2358.47 &  {\bf 85.00} &  2774.67 &  {\bf {\color{red} 0.85}} &  {\bf {\color{red} 0.82}} &  {\bf {\color{red} 0.77}} &  {\bf {\color{red} 0.72}} &  {\bf {\color{red} 0.67}} &  {\bf {\color{red} 0.63}} \\ 
\hline 
\hline 
\end{tabular} 
\end{center} } 
\paragraph{Retraites possibles et ratios Revenu/SMIC à 70, 75, 80, 85, 90 ans avec le modèle \emph{Destinie2 (revalorisation de la fonction publique)}}  
 
{ \scriptsize \begin{center} 
\begin{tabular}[htb]{|c|c||c|c||c|c||c||c|c|c|c|c|c|} 
\hline 
 Retraite en &  Âge &  Âge pivot &  Décote/Surcote &  Retraite (\euro{} 2019) &  Tx Rempl(\%) &  SMIC (\euro{} 2019) &  Retraite/SMIC &  Rev70/SMIC &  Rev75/SMIC &  Rev80/SMIC &  Rev85/SMIC &  Rev90/SMIC \\ 
\hline \hline 
 2052 &  62 &  66 ans 1 mois &  -20.42\% &  1392.73 &  {\bf 37.18} &  2445.56 &  {\bf {\color{red} 0.57}} &  {\bf {\color{red} 0.51}} &  {\bf {\color{red} 0.48}} &  {\bf {\color{red} 0.45}} &  {\bf {\color{red} 0.42}} &  {\bf {\color{red} 0.40}} \\ 
\hline 
 2053 &  63 &  66 ans 2 mois &  -15.83\% &  1536.55 &  {\bf 40.49} &  2477.35 &  {\bf {\color{red} 0.62}} &  {\bf {\color{red} 0.57}} &  {\bf {\color{red} 0.53}} &  {\bf {\color{red} 0.50}} &  {\bf {\color{red} 0.47}} &  {\bf {\color{red} 0.44}} \\ 
\hline 
 2054 &  64 &  66 ans 3 mois &  -11.25\% &  1688.77 &  {\bf 43.93} &  2509.56 &  {\bf {\color{red} 0.67}} &  {\bf {\color{red} 0.62}} &  {\bf {\color{red} 0.58}} &  {\bf {\color{red} 0.55}} &  {\bf {\color{red} 0.51}} &  {\bf {\color{red} 0.48}} \\ 
\hline 
 2055 &  65 &  66 ans 4 mois &  -6.67\% &  2160.85 &  {\bf 55.49} &  2542.18 &  {\bf {\color{red} 0.85}} &  {\bf {\color{red} 0.80}} &  {\bf {\color{red} 0.75}} &  {\bf {\color{red} 0.70}} &  {\bf {\color{red} 0.66}} &  {\bf {\color{red} 0.62}} \\ 
\hline 
 2056 &  66 &  66 ans 5 mois &  -2.08\% &  2188.95 &  {\bf 55.49} &  2575.23 &  {\bf {\color{red} 0.85}} &  {\bf {\color{red} 0.81}} &  {\bf {\color{red} 0.76}} &  {\bf {\color{red} 0.71}} &  {\bf {\color{red} 0.67}} &  {\bf {\color{red} 0.62}} \\ 
\hline 
 2057 &  67 &  66 ans 6 mois &  2.50\% &  2217.40 &  {\bf 55.49} &  2608.71 &  {\bf {\color{red} 0.85}} &  {\bf {\color{red} 0.82}} &  {\bf {\color{red} 0.77}} &  {\bf {\color{red} 0.72}} &  {\bf {\color{red} 0.67}} &  {\bf {\color{red} 0.63}} \\ 
\hline 
\hline 
\end{tabular} 
\end{center} } 

 \begin{center}\includegraphics[width=0.9\textwidth]{fig/AideSoignant_1990_22_dest_retraite.pdf}\end{center} \label{fig/AideSoignant_1990_22_dest_retraite.pdf} 

\newpage 
 
\subsection{Génération 2003 (début en 2025)} 

\paragraph{Retraites possibles et ratios Revenu/SMIC à 70, 75, 80, 85, 90 ans avec le modèle \emph{Gouvernement truqué (âge-pivot bloqué à 65 ans)}}  
 
{ \scriptsize \begin{center} 
\begin{tabular}[htb]{|c|c||c|c||c|c||c||c|c|c|c|c|c|} 
\hline 
 Retraite en &  Âge &  Âge pivot &  Décote/Surcote &  Retraite (\euro{} 2019) &  Tx Rempl(\%) &  SMIC (\euro{} 2019) &  Retraite/SMIC &  Rev70/SMIC &  Rev75/SMIC &  Rev80/SMIC &  Rev85/SMIC &  Rev90/SMIC \\ 
\hline \hline 
 2065 &  62 &  65 ans 0 mois &  -15.00\% &  1457.41 &  {\bf 47.37} &  3076.71 &  {\bf {\color{red} 0.47}} &  {\bf {\color{red} 0.43}} &  {\bf {\color{red} 0.40}} &  {\bf {\color{red} 0.38}} &  {\bf {\color{red} 0.35}} &  {\bf {\color{red} 0.33}} \\ 
\hline 
 2066 &  63 &  65 ans 0 mois &  -10.00\% &  1602.25 &  {\bf 51.41} &  3116.71 &  {\bf {\color{red} 0.51}} &  {\bf {\color{red} 0.47}} &  {\bf {\color{red} 0.44}} &  {\bf {\color{red} 0.41}} &  {\bf {\color{red} 0.39}} &  {\bf {\color{red} 0.36}} \\ 
\hline 
 2067 &  64 &  65 ans 0 mois &  -5.00\% &  1755.00 &  {\bf 55.59} &  3157.23 &  {\bf {\color{red} 0.56}} &  {\bf {\color{red} 0.51}} &  {\bf {\color{red} 0.48}} &  {\bf {\color{red} 0.45}} &  {\bf {\color{red} 0.42}} &  {\bf {\color{red} 0.40}} \\ 
\hline 
 2068 &  65 &  65 ans 0 mois &  0.00\% &  2718.53 &  {\bf 85.00} &  3198.27 &  {\bf {\color{red} 0.85}} &  {\bf {\color{red} 0.80}} &  {\bf {\color{red} 0.75}} &  {\bf {\color{red} 0.70}} &  {\bf {\color{red} 0.66}} &  {\bf {\color{red} 0.62}} \\ 
\hline 
 2069 &  66 &  65 ans 0 mois &  5.00\% &  2753.87 &  {\bf 85.00} &  3239.85 &  {\bf {\color{red} 0.85}} &  {\bf {\color{red} 0.81}} &  {\bf {\color{red} 0.76}} &  {\bf {\color{red} 0.71}} &  {\bf {\color{red} 0.67}} &  {\bf {\color{red} 0.62}} \\ 
\hline 
 2070 &  67 &  65 ans 0 mois &  10.00\% &  2789.67 &  {\bf 85.00} &  3281.97 &  {\bf {\color{red} 0.85}} &  {\bf {\color{red} 0.82}} &  {\bf {\color{red} 0.77}} &  {\bf {\color{red} 0.72}} &  {\bf {\color{red} 0.67}} &  {\bf {\color{red} 0.63}} \\ 
\hline 
\hline 
\end{tabular} 
\end{center} } 
\paragraph{Retraites possibles et ratios Revenu/SMIC à 70, 75, 80, 85, 90 ans avec le modèle \emph{Gouvernement corrigé (âge-pivot glissant)}}  
 
{ \scriptsize \begin{center} 
\begin{tabular}[htb]{|c|c||c|c||c|c||c||c|c|c|c|c|c|} 
\hline 
 Retraite en &  Âge &  Âge pivot &  Décote/Surcote &  Retraite (\euro{} 2019) &  Tx Rempl(\%) &  SMIC (\euro{} 2019) &  Retraite/SMIC &  Rev70/SMIC &  Rev75/SMIC &  Rev80/SMIC &  Rev85/SMIC &  Rev90/SMIC \\ 
\hline \hline 
 2065 &  62 &  67 ans 2 mois &  -25.83\% &  1271.67 &  {\bf 41.33} &  3076.71 &  {\bf {\color{red} 0.41}} &  {\bf {\color{red} 0.37}} &  {\bf {\color{red} 0.35}} &  {\bf {\color{red} 0.33}} &  {\bf {\color{red} 0.31}} &  {\bf {\color{red} 0.29}} \\ 
\hline 
 2066 &  63 &  67 ans 3 mois &  -21.25\% &  1401.97 &  {\bf 44.98} &  3116.71 &  {\bf {\color{red} 0.45}} &  {\bf {\color{red} 0.41}} &  {\bf {\color{red} 0.39}} &  {\bf {\color{red} 0.36}} &  {\bf {\color{red} 0.34}} &  {\bf {\color{red} 0.32}} \\ 
\hline 
 2067 &  64 &  67 ans 4 mois &  -16.67\% &  1539.47 &  {\bf 48.76} &  3157.23 &  {\bf {\color{red} 0.49}} &  {\bf {\color{red} 0.45}} &  {\bf {\color{red} 0.42}} &  {\bf {\color{red} 0.40}} &  {\bf {\color{red} 0.37}} &  {\bf {\color{red} 0.35}} \\ 
\hline 
 2068 &  65 &  67 ans 5 mois &  -12.08\% &  2718.53 &  {\bf 85.00} &  3198.27 &  {\bf {\color{red} 0.85}} &  {\bf {\color{red} 0.80}} &  {\bf {\color{red} 0.75}} &  {\bf {\color{red} 0.70}} &  {\bf {\color{red} 0.66}} &  {\bf {\color{red} 0.62}} \\ 
\hline 
 2069 &  66 &  67 ans 6 mois &  -7.50\% &  2753.87 &  {\bf 85.00} &  3239.85 &  {\bf {\color{red} 0.85}} &  {\bf {\color{red} 0.81}} &  {\bf {\color{red} 0.76}} &  {\bf {\color{red} 0.71}} &  {\bf {\color{red} 0.67}} &  {\bf {\color{red} 0.62}} \\ 
\hline 
 2070 &  67 &  67 ans 7 mois &  -2.92\% &  2789.67 &  {\bf 85.00} &  3281.97 &  {\bf {\color{red} 0.85}} &  {\bf {\color{red} 0.82}} &  {\bf {\color{red} 0.77}} &  {\bf {\color{red} 0.72}} &  {\bf {\color{red} 0.67}} &  {\bf {\color{red} 0.63}} \\ 
\hline 
\hline 
\end{tabular} 
\end{center} } 
\paragraph{Retraites possibles et ratios Revenu/SMIC à 70, 75, 80, 85, 90 ans avec le modèle \emph{Destinie2 (revalorisation de la fonction publique)}}  
 
{ \scriptsize \begin{center} 
\begin{tabular}[htb]{|c|c||c|c||c|c||c||c|c|c|c|c|c|} 
\hline 
 Retraite en &  Âge &  Âge pivot &  Décote/Surcote &  Retraite (\euro{} 2019) &  Tx Rempl(\%) &  SMIC (\euro{} 2019) &  Retraite/SMIC &  Rev70/SMIC &  Rev75/SMIC &  Rev80/SMIC &  Rev85/SMIC &  Rev90/SMIC \\ 
\hline \hline 
 2065 &  62 &  67 ans 2 mois &  -25.83\% &  1613.13 &  {\bf 36.40} &  2892.68 &  {\bf {\color{red} 0.56}} &  {\bf {\color{red} 0.50}} &  {\bf {\color{red} 0.47}} &  {\bf {\color{red} 0.44}} &  {\bf {\color{red} 0.41}} &  {\bf {\color{red} 0.39}} \\ 
\hline 
 2066 &  63 &  67 ans 3 mois &  -21.25\% &  1784.29 &  {\bf 39.75} &  2930.29 &  {\bf {\color{red} 0.61}} &  {\bf {\color{red} 0.56}} &  {\bf {\color{red} 0.52}} &  {\bf {\color{red} 0.49}} &  {\bf {\color{red} 0.46}} &  {\bf {\color{red} 0.43}} \\ 
\hline 
 2067 &  64 &  67 ans 4 mois &  -16.67\% &  1965.43 &  {\bf 43.22} &  2968.38 &  {\bf {\color{red} 0.66}} &  {\bf {\color{red} 0.61}} &  {\bf {\color{red} 0.57}} &  {\bf {\color{red} 0.54}} &  {\bf {\color{red} 0.50}} &  {\bf {\color{red} 0.47}} \\ 
\hline 
 2068 &  65 &  67 ans 5 mois &  -12.08\% &  2555.93 &  {\bf 55.49} &  3006.97 &  {\bf {\color{red} 0.85}} &  {\bf {\color{red} 0.80}} &  {\bf {\color{red} 0.75}} &  {\bf {\color{red} 0.70}} &  {\bf {\color{red} 0.66}} &  {\bf {\color{red} 0.62}} \\ 
\hline 
 2069 &  66 &  67 ans 6 mois &  -7.50\% &  2589.15 &  {\bf 55.49} &  3046.06 &  {\bf {\color{red} 0.85}} &  {\bf {\color{red} 0.81}} &  {\bf {\color{red} 0.76}} &  {\bf {\color{red} 0.71}} &  {\bf {\color{red} 0.67}} &  {\bf {\color{red} 0.62}} \\ 
\hline 
 2070 &  67 &  67 ans 7 mois &  -2.92\% &  2622.81 &  {\bf 55.49} &  3085.66 &  {\bf {\color{red} 0.85}} &  {\bf {\color{red} 0.82}} &  {\bf {\color{red} 0.77}} &  {\bf {\color{red} 0.72}} &  {\bf {\color{red} 0.67}} &  {\bf {\color{red} 0.63}} \\ 
\hline 
\hline 
\end{tabular} 
\end{center} } 

 \begin{center}\includegraphics[width=0.9\textwidth]{fig/AideSoignant_2003_22_dest_retraite.pdf}\end{center} \label{fig/AideSoignant_2003_22_dest_retraite.pdf} 

\newpage 
 
\chapter{Technicien hospitalier} 

\begin{minipage}{0.55\linewidth}\includegraphics[width=0.7\textwidth]{fig/grille_TechHosp.pdf}\end{minipage} 
\begin{minipage}{0.3\linewidth} 
 \begin{center} 

\begin{tabular}[htb]{|c|c|} 
\hline 
 Indice majoré &  Durée (années) \\ 
\hline \hline 
 339 &  2.00 \\ 
\hline 
 344 &  2.00 \\ 
\hline 
 349 &  2.00 \\ 
\hline 
 356 &  2.00 \\ 
\hline 
 366 &  2.00 \\ 
\hline 
 379 &  2.00 \\ 
\hline 
 394 &  2.00 \\ 
\hline 
 413 &  3.00 \\ 
\hline 
 429 &  3.00 \\ 
\hline 
 440 &  3.00 \\ 
\hline 
 453 &  3.00 \\ 
\hline 
 477 &  4.00 \\ 
\hline 
 503 &   \\ 
\hline 
\hline 
\end{tabular} 
\end{center} 
 \end{minipage} 


 \addto{\captionsenglish}{ \renewcommand{\mtctitle}{}} \setcounter{minitocdepth}{2} 
 \minitoc \newpage 

\section{Début de carrière à 22 ans} 

\subsection{Génération 1975 (début en 1997)} 

\paragraph{Retraites possibles et ratios Revenu/SMIC à 70, 75, 80, 85, 90 ans avec le modèle \emph{Gouvernement truqué (âge-pivot bloqué à 65 ans)}}  
 
{ \scriptsize \begin{center} 
\begin{tabular}[htb]{|c|c||c|c||c|c||c||c|c|c|c|c|c|} 
\hline 
 Retraite en &  Âge &  Âge pivot &  Décote/Surcote &  Retraite (\euro{} 2019) &  Tx Rempl(\%) &  SMIC (\euro{} 2019) &  Retraite/SMIC &  Rev70/SMIC &  Rev75/SMIC &  Rev80/SMIC &  Rev85/SMIC &  Rev90/SMIC \\ 
\hline \hline 
 2037 &  62 &  64 ans 10 mois &  -14.17\% &  1321.60 &  {\bf 43.39} &  2143.00 &  {\bf {\color{red} 0.62}} &  {\bf {\color{red} 0.56}} &  {\bf {\color{red} 0.52}} &  {\bf {\color{red} 0.49}} &  {\bf {\color{red} 0.46}} &  {\bf {\color{red} 0.43}} \\ 
\hline 
 2038 &  63 &  64 ans 11 mois &  -9.58\% &  1441.19 &  {\bf 47.23} &  2170.86 &  {\bf {\color{red} 0.66}} &  {\bf {\color{red} 0.61}} &  {\bf {\color{red} 0.57}} &  {\bf {\color{red} 0.53}} &  {\bf {\color{red} 0.50}} &  {\bf {\color{red} 0.47}} \\ 
\hline 
 2039 &  64 &  65 ans 0 mois &  -5.00\% &  1567.38 &  {\bf 51.27} &  2199.08 &  {\bf {\color{red} 0.71}} &  {\bf {\color{red} 0.66}} &  {\bf {\color{red} 0.62}} &  {\bf {\color{red} 0.58}} &  {\bf {\color{red} 0.54}} &  {\bf {\color{red} 0.51}} \\ 
\hline 
 2040 &  65 &  65 ans 0 mois &  0.00\% &  1893.52 &  {\bf 61.82} &  2227.67 &  {\bf {\color{red} 0.85}} &  {\bf {\color{red} 0.80}} &  {\bf {\color{red} 0.75}} &  {\bf {\color{red} 0.70}} &  {\bf {\color{red} 0.66}} &  {\bf {\color{red} 0.62}} \\ 
\hline 
 2041 &  66 &  65 ans 0 mois &  5.00\% &  1918.14 &  {\bf 62.52} &  2256.63 &  {\bf {\color{red} 0.85}} &  {\bf {\color{red} 0.81}} &  {\bf {\color{red} 0.76}} &  {\bf {\color{red} 0.71}} &  {\bf {\color{red} 0.67}} &  {\bf {\color{red} 0.62}} \\ 
\hline 
 2042 &  67 &  65 ans 0 mois &  10.00\% &  2011.85 &  {\bf 65.45} &  2285.97 &  {\bf {\color{red} 0.88}} &  {\bf {\color{red} 0.85}} &  {\bf {\color{red} 0.79}} &  {\bf {\color{red} 0.74}} &  {\bf {\color{red} 0.70}} &  {\bf {\color{red} 0.65}} \\ 
\hline 
\hline 
\end{tabular} 
\end{center} } 
\paragraph{Retraites possibles et ratios Revenu/SMIC à 70, 75, 80, 85, 90 ans avec le modèle \emph{Gouvernement corrigé (âge-pivot glissant)}}  
 
{ \scriptsize \begin{center} 
\begin{tabular}[htb]{|c|c||c|c||c|c||c||c|c|c|c|c|c|} 
\hline 
 Retraite en &  Âge &  Âge pivot &  Décote/Surcote &  Retraite (\euro{} 2019) &  Tx Rempl(\%) &  SMIC (\euro{} 2019) &  Retraite/SMIC &  Rev70/SMIC &  Rev75/SMIC &  Rev80/SMIC &  Rev85/SMIC &  Rev90/SMIC \\ 
\hline \hline 
 2037 &  62 &  64 ans 10 mois &  -14.17\% &  1321.60 &  {\bf 43.39} &  2143.00 &  {\bf {\color{red} 0.62}} &  {\bf {\color{red} 0.56}} &  {\bf {\color{red} 0.52}} &  {\bf {\color{red} 0.49}} &  {\bf {\color{red} 0.46}} &  {\bf {\color{red} 0.43}} \\ 
\hline 
 2038 &  63 &  64 ans 11 mois &  -9.58\% &  1441.19 &  {\bf 47.23} &  2170.86 &  {\bf {\color{red} 0.66}} &  {\bf {\color{red} 0.61}} &  {\bf {\color{red} 0.57}} &  {\bf {\color{red} 0.53}} &  {\bf {\color{red} 0.50}} &  {\bf {\color{red} 0.47}} \\ 
\hline 
 2039 &  64 &  65 ans 0 mois &  -5.00\% &  1567.38 &  {\bf 51.27} &  2199.08 &  {\bf {\color{red} 0.71}} &  {\bf {\color{red} 0.66}} &  {\bf {\color{red} 0.62}} &  {\bf {\color{red} 0.58}} &  {\bf {\color{red} 0.54}} &  {\bf {\color{red} 0.51}} \\ 
\hline 
 2040 &  65 &  65 ans 1 mois &  -0.42\% &  1893.52 &  {\bf 61.82} &  2227.67 &  {\bf {\color{red} 0.85}} &  {\bf {\color{red} 0.80}} &  {\bf {\color{red} 0.75}} &  {\bf {\color{red} 0.70}} &  {\bf {\color{red} 0.66}} &  {\bf {\color{red} 0.62}} \\ 
\hline 
 2041 &  66 &  65 ans 2 mois &  4.17\% &  1918.14 &  {\bf 62.52} &  2256.63 &  {\bf {\color{red} 0.85}} &  {\bf {\color{red} 0.81}} &  {\bf {\color{red} 0.76}} &  {\bf {\color{red} 0.71}} &  {\bf {\color{red} 0.67}} &  {\bf {\color{red} 0.62}} \\ 
\hline 
 2042 &  67 &  65 ans 3 mois &  8.75\% &  1988.99 &  {\bf 64.71} &  2285.97 &  {\bf {\color{red} 0.87}} &  {\bf {\color{red} 0.84}} &  {\bf {\color{red} 0.78}} &  {\bf {\color{red} 0.74}} &  {\bf {\color{red} 0.69}} &  {\bf {\color{red} 0.65}} \\ 
\hline 
\hline 
\end{tabular} 
\end{center} } 
\paragraph{Retraites possibles et ratios Revenu/SMIC à 70, 75, 80, 85, 90 ans avec le modèle \emph{Destinie2 (revalorisation de la fonction publique)}}  
 
{ \scriptsize \begin{center} 
\begin{tabular}[htb]{|c|c||c|c||c|c||c||c|c|c|c|c|c|} 
\hline 
 Retraite en &  Âge &  Âge pivot &  Décote/Surcote &  Retraite (\euro{} 2019) &  Tx Rempl(\%) &  SMIC (\euro{} 2019) &  Retraite/SMIC &  Rev70/SMIC &  Rev75/SMIC &  Rev80/SMIC &  Rev85/SMIC &  Rev90/SMIC \\ 
\hline \hline 
 2037 &  62 &  64 ans 10 mois &  -14.17\% &  1455.57 &  {\bf 38.33} &  2014.82 &  {\bf {\color{red} 0.72}} &  {\bf {\color{red} 0.65}} &  {\bf {\color{red} 0.61}} &  {\bf {\color{red} 0.57}} &  {\bf {\color{red} 0.54}} &  {\bf {\color{red} 0.50}} \\ 
\hline 
 2038 &  63 &  64 ans 11 mois &  -9.58\% &  1593.40 &  {\bf 41.42} &  2041.01 &  {\bf {\color{red} 0.78}} &  {\bf {\color{red} 0.71}} &  {\bf {\color{red} 0.67}} &  {\bf {\color{red} 0.63}} &  {\bf {\color{red} 0.59}} &  {\bf {\color{red} 0.55}} \\ 
\hline 
 2039 &  64 &  65 ans 0 mois &  -5.00\% &  1739.76 &  {\bf 44.64} &  2067.55 &  {\bf {\color{red} 0.84}} &  {\bf {\color{red} 0.78}} &  {\bf {\color{red} 0.73}} &  {\bf {\color{red} 0.68}} &  {\bf {\color{red} 0.64}} &  {\bf {\color{red} 0.60}} \\ 
\hline 
 2040 &  65 &  65 ans 1 mois &  -0.42\% &  1895.12 &  {\bf 48.01} &  2094.43 &  {\bf {\color{red} 0.90}} &  {\bf {\color{red} 0.85}} &  {\bf {\color{red} 0.80}} &  {\bf {\color{red} 0.75}} &  {\bf {\color{red} 0.70}} &  {\bf {\color{red} 0.66}} \\ 
\hline 
 2041 &  66 &  65 ans 2 mois &  4.17\% &  2060.00 &  {\bf 51.51} &  2121.65 &  {\bf {\color{red} 0.97}} &  {\bf {\color{red} 0.92}} &  {\bf {\color{red} 0.86}} &  {\bf {\color{red} 0.81}} &  {\bf {\color{red} 0.76}} &  {\bf {\color{red} 0.71}} \\ 
\hline 
 2042 &  67 &  65 ans 3 mois &  8.75\% &  2234.96 &  {\bf 55.17} &  2149.23 &  {\bf 1.04} &  {\bf 1.00} &  {\bf {\color{red} 0.94}} &  {\bf {\color{red} 0.88}} &  {\bf {\color{red} 0.82}} &  {\bf {\color{red} 0.77}} \\ 
\hline 
\hline 
\end{tabular} 
\end{center} } 

 \begin{center}\includegraphics[width=0.9\textwidth]{fig/TechHosp_1975_22_dest_retraite.pdf}\end{center} \label{fig/TechHosp_1975_22_dest_retraite.pdf} 

\newpage 
 
\subsection{Génération 1980 (début en 2002)} 

\paragraph{Retraites possibles et ratios Revenu/SMIC à 70, 75, 80, 85, 90 ans avec le modèle \emph{Gouvernement truqué (âge-pivot bloqué à 65 ans)}}  
 
{ \scriptsize \begin{center} 
\begin{tabular}[htb]{|c|c||c|c||c|c||c||c|c|c|c|c|c|} 
\hline 
 Retraite en &  Âge &  Âge pivot &  Décote/Surcote &  Retraite (\euro{} 2019) &  Tx Rempl(\%) &  SMIC (\euro{} 2019) &  Retraite/SMIC &  Rev70/SMIC &  Rev75/SMIC &  Rev80/SMIC &  Rev85/SMIC &  Rev90/SMIC \\ 
\hline \hline 
 2042 &  62 &  65 ans 0 mois &  -15.00\% &  1334.51 &  {\bf 43.81} &  2285.97 &  {\bf {\color{red} 0.58}} &  {\bf {\color{red} 0.53}} &  {\bf {\color{red} 0.49}} &  {\bf {\color{red} 0.46}} &  {\bf {\color{red} 0.43}} &  {\bf {\color{red} 0.41}} \\ 
\hline 
 2043 &  63 &  65 ans 0 mois &  -10.00\% &  1467.43 &  {\bf 48.09} &  2315.68 &  {\bf {\color{red} 0.63}} &  {\bf {\color{red} 0.58}} &  {\bf {\color{red} 0.54}} &  {\bf {\color{red} 0.51}} &  {\bf {\color{red} 0.48}} &  {\bf {\color{red} 0.45}} \\ 
\hline 
 2044 &  64 &  65 ans 0 mois &  -5.00\% &  1608.33 &  {\bf 52.61} &  2345.79 &  {\bf {\color{red} 0.69}} &  {\bf {\color{red} 0.63}} &  {\bf {\color{red} 0.59}} &  {\bf {\color{red} 0.56}} &  {\bf {\color{red} 0.52}} &  {\bf {\color{red} 0.49}} \\ 
\hline 
 2045 &  65 &  65 ans 0 mois &  0.00\% &  2019.84 &  {\bf 65.95} &  2376.28 &  {\bf {\color{red} 0.85}} &  {\bf {\color{red} 0.80}} &  {\bf {\color{red} 0.75}} &  {\bf {\color{red} 0.70}} &  {\bf {\color{red} 0.66}} &  {\bf {\color{red} 0.62}} \\ 
\hline 
 2046 &  66 &  65 ans 0 mois &  5.00\% &  2046.10 &  {\bf 66.69} &  2407.18 &  {\bf {\color{red} 0.85}} &  {\bf {\color{red} 0.81}} &  {\bf {\color{red} 0.76}} &  {\bf {\color{red} 0.71}} &  {\bf {\color{red} 0.67}} &  {\bf {\color{red} 0.62}} \\ 
\hline 
 2047 &  67 &  65 ans 0 mois &  10.00\% &  2078.62 &  {\bf 67.62} &  2438.47 &  {\bf {\color{red} 0.85}} &  {\bf {\color{red} 0.82}} &  {\bf {\color{red} 0.77}} &  {\bf {\color{red} 0.72}} &  {\bf {\color{red} 0.68}} &  {\bf {\color{red} 0.63}} \\ 
\hline 
\hline 
\end{tabular} 
\end{center} } 
\paragraph{Retraites possibles et ratios Revenu/SMIC à 70, 75, 80, 85, 90 ans avec le modèle \emph{Gouvernement corrigé (âge-pivot glissant)}}  
 
{ \scriptsize \begin{center} 
\begin{tabular}[htb]{|c|c||c|c||c|c||c||c|c|c|c|c|c|} 
\hline 
 Retraite en &  Âge &  Âge pivot &  Décote/Surcote &  Retraite (\euro{} 2019) &  Tx Rempl(\%) &  SMIC (\euro{} 2019) &  Retraite/SMIC &  Rev70/SMIC &  Rev75/SMIC &  Rev80/SMIC &  Rev85/SMIC &  Rev90/SMIC \\ 
\hline \hline 
 2042 &  62 &  65 ans 3 mois &  -16.25\% &  1314.88 &  {\bf 43.17} &  2285.97 &  {\bf {\color{red} 0.58}} &  {\bf {\color{red} 0.52}} &  {\bf {\color{red} 0.49}} &  {\bf {\color{red} 0.46}} &  {\bf {\color{red} 0.43}} &  {\bf {\color{red} 0.40}} \\ 
\hline 
 2043 &  63 &  65 ans 4 mois &  -11.67\% &  1440.26 &  {\bf 47.20} &  2315.68 &  {\bf {\color{red} 0.62}} &  {\bf {\color{red} 0.57}} &  {\bf {\color{red} 0.53}} &  {\bf {\color{red} 0.50}} &  {\bf {\color{red} 0.47}} &  {\bf {\color{red} 0.44}} \\ 
\hline 
 2044 &  64 &  65 ans 5 mois &  -7.08\% &  1573.06 &  {\bf 51.45} &  2345.79 &  {\bf {\color{red} 0.67}} &  {\bf {\color{red} 0.62}} &  {\bf {\color{red} 0.58}} &  {\bf {\color{red} 0.55}} &  {\bf {\color{red} 0.51}} &  {\bf {\color{red} 0.48}} \\ 
\hline 
 2045 &  65 &  65 ans 6 mois &  -2.50\% &  2019.84 &  {\bf 65.95} &  2376.28 &  {\bf {\color{red} 0.85}} &  {\bf {\color{red} 0.80}} &  {\bf {\color{red} 0.75}} &  {\bf {\color{red} 0.70}} &  {\bf {\color{red} 0.66}} &  {\bf {\color{red} 0.62}} \\ 
\hline 
 2046 &  66 &  65 ans 7 mois &  2.08\% &  2046.10 &  {\bf 66.69} &  2407.18 &  {\bf {\color{red} 0.85}} &  {\bf {\color{red} 0.81}} &  {\bf {\color{red} 0.76}} &  {\bf {\color{red} 0.71}} &  {\bf {\color{red} 0.67}} &  {\bf {\color{red} 0.62}} \\ 
\hline 
 2047 &  67 &  65 ans 8 mois &  6.67\% &  2072.70 &  {\bf 67.43} &  2438.47 &  {\bf {\color{red} 0.85}} &  {\bf {\color{red} 0.82}} &  {\bf {\color{red} 0.77}} &  {\bf {\color{red} 0.72}} &  {\bf {\color{red} 0.67}} &  {\bf {\color{red} 0.63}} \\ 
\hline 
\hline 
\end{tabular} 
\end{center} } 
\paragraph{Retraites possibles et ratios Revenu/SMIC à 70, 75, 80, 85, 90 ans avec le modèle \emph{Destinie2 (revalorisation de la fonction publique)}}  
 
{ \scriptsize \begin{center} 
\begin{tabular}[htb]{|c|c||c|c||c|c||c||c|c|c|c|c|c|} 
\hline 
 Retraite en &  Âge &  Âge pivot &  Décote/Surcote &  Retraite (\euro{} 2019) &  Tx Rempl(\%) &  SMIC (\euro{} 2019) &  Retraite/SMIC &  Rev70/SMIC &  Rev75/SMIC &  Rev80/SMIC &  Rev85/SMIC &  Rev90/SMIC \\ 
\hline \hline 
 2042 &  62 &  65 ans 3 mois &  -16.25\% &  1499.43 &  {\bf 37.01} &  2149.23 &  {\bf {\color{red} 0.70}} &  {\bf {\color{red} 0.63}} &  {\bf {\color{red} 0.59}} &  {\bf {\color{red} 0.55}} &  {\bf {\color{red} 0.52}} &  {\bf {\color{red} 0.49}} \\ 
\hline 
 2043 &  63 &  65 ans 4 mois &  -11.67\% &  1650.07 &  {\bf 40.21} &  2177.17 &  {\bf {\color{red} 0.76}} &  {\bf {\color{red} 0.69}} &  {\bf {\color{red} 0.65}} &  {\bf {\color{red} 0.61}} &  {\bf {\color{red} 0.57}} &  {\bf {\color{red} 0.53}} \\ 
\hline 
 2044 &  64 &  65 ans 5 mois &  -7.08\% &  1810.68 &  {\bf 43.56} &  2205.48 &  {\bf {\color{red} 0.82}} &  {\bf {\color{red} 0.76}} &  {\bf {\color{red} 0.71}} &  {\bf {\color{red} 0.67}} &  {\bf {\color{red} 0.63}} &  {\bf {\color{red} 0.59}} \\ 
\hline 
 2045 &  65 &  65 ans 6 mois &  -2.50\% &  1981.85 &  {\bf 47.06} &  2234.15 &  {\bf {\color{red} 0.89}} &  {\bf {\color{red} 0.83}} &  {\bf {\color{red} 0.78}} &  {\bf {\color{red} 0.73}} &  {\bf {\color{red} 0.69}} &  {\bf {\color{red} 0.64}} \\ 
\hline 
 2046 &  66 &  65 ans 7 mois &  2.08\% &  2162.60 &  {\bf 50.70} &  2263.19 &  {\bf {\color{red} 0.96}} &  {\bf {\color{red} 0.91}} &  {\bf {\color{red} 0.85}} &  {\bf {\color{red} 0.80}} &  {\bf {\color{red} 0.75}} &  {\bf {\color{red} 0.70}} \\ 
\hline 
 2047 &  67 &  65 ans 8 mois &  6.67\% &  2353.23 &  {\bf 54.46} &  2292.61 &  {\bf 1.03} &  {\bf {\color{red} 0.99}} &  {\bf {\color{red} 0.93}} &  {\bf {\color{red} 0.87}} &  {\bf {\color{red} 0.81}} &  {\bf {\color{red} 0.76}} \\ 
\hline 
\hline 
\end{tabular} 
\end{center} } 

 \begin{center}\includegraphics[width=0.9\textwidth]{fig/TechHosp_1980_22_dest_retraite.pdf}\end{center} \label{fig/TechHosp_1980_22_dest_retraite.pdf} 

\newpage 
 
\subsection{Génération 1990 (début en 2012)} 

\paragraph{Retraites possibles et ratios Revenu/SMIC à 70, 75, 80, 85, 90 ans avec le modèle \emph{Gouvernement truqué (âge-pivot bloqué à 65 ans)}}  
 
{ \scriptsize \begin{center} 
\begin{tabular}[htb]{|c|c||c|c||c|c||c||c|c|c|c|c|c|} 
\hline 
 Retraite en &  Âge &  Âge pivot &  Décote/Surcote &  Retraite (\euro{} 2019) &  Tx Rempl(\%) &  SMIC (\euro{} 2019) &  Retraite/SMIC &  Rev70/SMIC &  Rev75/SMIC &  Rev80/SMIC &  Rev85/SMIC &  Rev90/SMIC \\ 
\hline \hline 
 2052 &  62 &  65 ans 0 mois &  -15.00\% &  1431.21 &  {\bf 46.99} &  2601.14 &  {\bf {\color{red} 0.55}} &  {\bf {\color{red} 0.50}} &  {\bf {\color{red} 0.47}} &  {\bf {\color{red} 0.44}} &  {\bf {\color{red} 0.41}} &  {\bf {\color{red} 0.38}} \\ 
\hline 
 2053 &  63 &  65 ans 0 mois &  -10.00\% &  1573.33 &  {\bf 51.56} &  2634.96 &  {\bf {\color{red} 0.60}} &  {\bf {\color{red} 0.55}} &  {\bf {\color{red} 0.51}} &  {\bf {\color{red} 0.48}} &  {\bf {\color{red} 0.45}} &  {\bf {\color{red} 0.42}} \\ 
\hline 
 2054 &  64 &  65 ans 0 mois &  -5.00\% &  1722.75 &  {\bf 56.35} &  2669.21 &  {\bf {\color{red} 0.65}} &  {\bf {\color{red} 0.60}} &  {\bf {\color{red} 0.56}} &  {\bf {\color{red} 0.52}} &  {\bf {\color{red} 0.49}} &  {\bf {\color{red} 0.46}} \\ 
\hline 
 2055 &  65 &  65 ans 0 mois &  0.00\% &  2298.33 &  {\bf 75.04} &  2703.91 &  {\bf {\color{red} 0.85}} &  {\bf {\color{red} 0.80}} &  {\bf {\color{red} 0.75}} &  {\bf {\color{red} 0.70}} &  {\bf {\color{red} 0.66}} &  {\bf {\color{red} 0.62}} \\ 
\hline 
 2056 &  66 &  65 ans 0 mois &  5.00\% &  2328.20 &  {\bf 75.88} &  2739.06 &  {\bf {\color{red} 0.85}} &  {\bf {\color{red} 0.81}} &  {\bf {\color{red} 0.76}} &  {\bf {\color{red} 0.71}} &  {\bf {\color{red} 0.67}} &  {\bf {\color{red} 0.62}} \\ 
\hline 
 2057 &  67 &  65 ans 0 mois &  10.00\% &  2358.47 &  {\bf 76.73} &  2774.67 &  {\bf {\color{red} 0.85}} &  {\bf {\color{red} 0.82}} &  {\bf {\color{red} 0.77}} &  {\bf {\color{red} 0.72}} &  {\bf {\color{red} 0.67}} &  {\bf {\color{red} 0.63}} \\ 
\hline 
\hline 
\end{tabular} 
\end{center} } 
\paragraph{Retraites possibles et ratios Revenu/SMIC à 70, 75, 80, 85, 90 ans avec le modèle \emph{Gouvernement corrigé (âge-pivot glissant)}}  
 
{ \scriptsize \begin{center} 
\begin{tabular}[htb]{|c|c||c|c||c|c||c||c|c|c|c|c|c|} 
\hline 
 Retraite en &  Âge &  Âge pivot &  Décote/Surcote &  Retraite (\euro{} 2019) &  Tx Rempl(\%) &  SMIC (\euro{} 2019) &  Retraite/SMIC &  Rev70/SMIC &  Rev75/SMIC &  Rev80/SMIC &  Rev85/SMIC &  Rev90/SMIC \\ 
\hline \hline 
 2052 &  62 &  66 ans 1 mois &  -20.42\% &  1340.01 &  {\bf 43.99} &  2601.14 &  {\bf {\color{red} 0.52}} &  {\bf {\color{red} 0.46}} &  {\bf {\color{red} 0.44}} &  {\bf {\color{red} 0.41}} &  {\bf {\color{red} 0.38}} &  {\bf {\color{red} 0.36}} \\ 
\hline 
 2053 &  63 &  66 ans 2 mois &  -15.83\% &  1471.36 &  {\bf 48.22} &  2634.96 &  {\bf {\color{red} 0.56}} &  {\bf {\color{red} 0.51}} &  {\bf {\color{red} 0.48}} &  {\bf {\color{red} 0.45}} &  {\bf {\color{red} 0.42}} &  {\bf {\color{red} 0.39}} \\ 
\hline 
 2054 &  64 &  66 ans 3 mois &  -11.25\% &  1609.41 &  {\bf 52.64} &  2669.21 &  {\bf {\color{red} 0.60}} &  {\bf {\color{red} 0.56}} &  {\bf {\color{red} 0.52}} &  {\bf {\color{red} 0.49}} &  {\bf {\color{red} 0.46}} &  {\bf {\color{red} 0.43}} \\ 
\hline 
 2055 &  65 &  66 ans 4 mois &  -6.67\% &  2298.33 &  {\bf 75.04} &  2703.91 &  {\bf {\color{red} 0.85}} &  {\bf {\color{red} 0.80}} &  {\bf {\color{red} 0.75}} &  {\bf {\color{red} 0.70}} &  {\bf {\color{red} 0.66}} &  {\bf {\color{red} 0.62}} \\ 
\hline 
 2056 &  66 &  66 ans 5 mois &  -2.08\% &  2328.20 &  {\bf 75.88} &  2739.06 &  {\bf {\color{red} 0.85}} &  {\bf {\color{red} 0.81}} &  {\bf {\color{red} 0.76}} &  {\bf {\color{red} 0.71}} &  {\bf {\color{red} 0.67}} &  {\bf {\color{red} 0.62}} \\ 
\hline 
 2057 &  67 &  66 ans 6 mois &  2.50\% &  2358.47 &  {\bf 76.73} &  2774.67 &  {\bf {\color{red} 0.85}} &  {\bf {\color{red} 0.82}} &  {\bf {\color{red} 0.77}} &  {\bf {\color{red} 0.72}} &  {\bf {\color{red} 0.67}} &  {\bf {\color{red} 0.63}} \\ 
\hline 
\hline 
\end{tabular} 
\end{center} } 
\paragraph{Retraites possibles et ratios Revenu/SMIC à 70, 75, 80, 85, 90 ans avec le modèle \emph{Destinie2 (revalorisation de la fonction publique)}}  
 
{ \scriptsize \begin{center} 
\begin{tabular}[htb]{|c|c||c|c||c|c||c||c|c|c|c|c|c|} 
\hline 
 Retraite en &  Âge &  Âge pivot &  Décote/Surcote &  Retraite (\euro{} 2019) &  Tx Rempl(\%) &  SMIC (\euro{} 2019) &  Retraite/SMIC &  Rev70/SMIC &  Rev75/SMIC &  Rev80/SMIC &  Rev85/SMIC &  Rev90/SMIC \\ 
\hline \hline 
 2052 &  62 &  66 ans 1 mois &  -20.42\% &  1675.64 &  {\bf 36.35} &  2445.56 &  {\bf {\color{red} 0.69}} &  {\bf {\color{red} 0.62}} &  {\bf {\color{red} 0.58}} &  {\bf {\color{red} 0.54}} &  {\bf {\color{red} 0.51}} &  {\bf {\color{red} 0.48}} \\ 
\hline 
 2053 &  63 &  66 ans 2 mois &  -15.83\% &  1849.89 &  {\bf 39.62} &  2477.35 &  {\bf {\color{red} 0.75}} &  {\bf {\color{red} 0.68}} &  {\bf {\color{red} 0.64}} &  {\bf {\color{red} 0.60}} &  {\bf {\color{red} 0.56}} &  {\bf {\color{red} 0.53}} \\ 
\hline 
 2054 &  64 &  66 ans 3 mois &  -11.25\% &  2034.42 &  {\bf 43.01} &  2509.56 &  {\bf {\color{red} 0.81}} &  {\bf {\color{red} 0.75}} &  {\bf {\color{red} 0.70}} &  {\bf {\color{red} 0.66}} &  {\bf {\color{red} 0.62}} &  {\bf {\color{red} 0.58}} \\ 
\hline 
 2055 &  65 &  66 ans 4 mois &  -6.67\% &  2229.55 &  {\bf 46.53} &  2542.18 &  {\bf {\color{red} 0.88}} &  {\bf {\color{red} 0.82}} &  {\bf {\color{red} 0.77}} &  {\bf {\color{red} 0.72}} &  {\bf {\color{red} 0.68}} &  {\bf {\color{red} 0.64}} \\ 
\hline 
 2056 &  66 &  66 ans 5 mois &  -2.08\% &  2435.60 &  {\bf 50.18} &  2575.23 &  {\bf {\color{red} 0.95}} &  {\bf {\color{red} 0.90}} &  {\bf {\color{red} 0.84}} &  {\bf {\color{red} 0.79}} &  {\bf {\color{red} 0.74}} &  {\bf {\color{red} 0.69}} \\ 
\hline 
 2057 &  67 &  66 ans 6 mois &  2.50\% &  2652.91 &  {\bf 53.95} &  2608.71 &  {\bf 1.02} &  {\bf {\color{red} 0.98}} &  {\bf {\color{red} 0.92}} &  {\bf {\color{red} 0.86}} &  {\bf {\color{red} 0.81}} &  {\bf {\color{red} 0.76}} \\ 
\hline 
\hline 
\end{tabular} 
\end{center} } 

 \begin{center}\includegraphics[width=0.9\textwidth]{fig/TechHosp_1990_22_dest_retraite.pdf}\end{center} \label{fig/TechHosp_1990_22_dest_retraite.pdf} 

\newpage 
 
\subsection{Génération 2003 (début en 2025)} 

\paragraph{Retraites possibles et ratios Revenu/SMIC à 70, 75, 80, 85, 90 ans avec le modèle \emph{Gouvernement truqué (âge-pivot bloqué à 65 ans)}}  
 
{ \scriptsize \begin{center} 
\begin{tabular}[htb]{|c|c||c|c||c|c||c||c|c|c|c|c|c|} 
\hline 
 Retraite en &  Âge &  Âge pivot &  Décote/Surcote &  Retraite (\euro{} 2019) &  Tx Rempl(\%) &  SMIC (\euro{} 2019) &  Retraite/SMIC &  Rev70/SMIC &  Rev75/SMIC &  Rev80/SMIC &  Rev85/SMIC &  Rev90/SMIC \\ 
\hline \hline 
 2065 &  62 &  65 ans 0 mois &  -15.00\% &  1531.73 &  {\bf 49.78} &  3076.71 &  {\bf {\color{red} 0.50}} &  {\bf {\color{red} 0.45}} &  {\bf {\color{red} 0.42}} &  {\bf {\color{red} 0.39}} &  {\bf {\color{red} 0.37}} &  {\bf {\color{red} 0.35}} \\ 
\hline 
 2066 &  63 &  65 ans 0 mois &  -10.00\% &  1681.96 &  {\bf 53.97} &  3116.71 &  {\bf {\color{red} 0.54}} &  {\bf {\color{red} 0.49}} &  {\bf {\color{red} 0.46}} &  {\bf {\color{red} 0.43}} &  {\bf {\color{red} 0.41}} &  {\bf {\color{red} 0.38}} \\ 
\hline 
 2067 &  64 &  65 ans 0 mois &  -5.00\% &  1840.23 &  {\bf 58.29} &  3157.23 &  {\bf {\color{red} 0.58}} &  {\bf {\color{red} 0.54}} &  {\bf {\color{red} 0.51}} &  {\bf {\color{red} 0.47}} &  {\bf {\color{red} 0.44}} &  {\bf {\color{red} 0.42}} \\ 
\hline 
 2068 &  65 &  65 ans 0 mois &  0.00\% &  2718.53 &  {\bf 85.00} &  3198.27 &  {\bf {\color{red} 0.85}} &  {\bf {\color{red} 0.80}} &  {\bf {\color{red} 0.75}} &  {\bf {\color{red} 0.70}} &  {\bf {\color{red} 0.66}} &  {\bf {\color{red} 0.62}} \\ 
\hline 
 2069 &  66 &  65 ans 0 mois &  5.00\% &  2753.87 &  {\bf 85.00} &  3239.85 &  {\bf {\color{red} 0.85}} &  {\bf {\color{red} 0.81}} &  {\bf {\color{red} 0.76}} &  {\bf {\color{red} 0.71}} &  {\bf {\color{red} 0.67}} &  {\bf {\color{red} 0.62}} \\ 
\hline 
 2070 &  67 &  65 ans 0 mois &  10.00\% &  2789.67 &  {\bf 85.00} &  3281.97 &  {\bf {\color{red} 0.85}} &  {\bf {\color{red} 0.82}} &  {\bf {\color{red} 0.77}} &  {\bf {\color{red} 0.72}} &  {\bf {\color{red} 0.67}} &  {\bf {\color{red} 0.63}} \\ 
\hline 
\hline 
\end{tabular} 
\end{center} } 
\paragraph{Retraites possibles et ratios Revenu/SMIC à 70, 75, 80, 85, 90 ans avec le modèle \emph{Gouvernement corrigé (âge-pivot glissant)}}  
 
{ \scriptsize \begin{center} 
\begin{tabular}[htb]{|c|c||c|c||c|c||c||c|c|c|c|c|c|} 
\hline 
 Retraite en &  Âge &  Âge pivot &  Décote/Surcote &  Retraite (\euro{} 2019) &  Tx Rempl(\%) &  SMIC (\euro{} 2019) &  Retraite/SMIC &  Rev70/SMIC &  Rev75/SMIC &  Rev80/SMIC &  Rev85/SMIC &  Rev90/SMIC \\ 
\hline \hline 
 2065 &  62 &  67 ans 2 mois &  -25.83\% &  1336.51 &  {\bf 43.44} &  3076.71 &  {\bf {\color{red} 0.43}} &  {\bf {\color{red} 0.39}} &  {\bf {\color{red} 0.37}} &  {\bf {\color{red} 0.34}} &  {\bf {\color{red} 0.32}} &  {\bf {\color{red} 0.30}} \\ 
\hline 
 2066 &  63 &  67 ans 3 mois &  -21.25\% &  1471.71 &  {\bf 47.22} &  3116.71 &  {\bf {\color{red} 0.47}} &  {\bf {\color{red} 0.43}} &  {\bf {\color{red} 0.40}} &  {\bf {\color{red} 0.38}} &  {\bf {\color{red} 0.36}} &  {\bf {\color{red} 0.33}} \\ 
\hline 
 2067 &  64 &  67 ans 4 mois &  -16.67\% &  1614.24 &  {\bf 51.13} &  3157.23 &  {\bf {\color{red} 0.51}} &  {\bf {\color{red} 0.47}} &  {\bf {\color{red} 0.44}} &  {\bf {\color{red} 0.42}} &  {\bf {\color{red} 0.39}} &  {\bf {\color{red} 0.37}} \\ 
\hline 
 2068 &  65 &  67 ans 5 mois &  -12.08\% &  2718.53 &  {\bf 85.00} &  3198.27 &  {\bf {\color{red} 0.85}} &  {\bf {\color{red} 0.80}} &  {\bf {\color{red} 0.75}} &  {\bf {\color{red} 0.70}} &  {\bf {\color{red} 0.66}} &  {\bf {\color{red} 0.62}} \\ 
\hline 
 2069 &  66 &  67 ans 6 mois &  -7.50\% &  2753.87 &  {\bf 85.00} &  3239.85 &  {\bf {\color{red} 0.85}} &  {\bf {\color{red} 0.81}} &  {\bf {\color{red} 0.76}} &  {\bf {\color{red} 0.71}} &  {\bf {\color{red} 0.67}} &  {\bf {\color{red} 0.62}} \\ 
\hline 
 2070 &  67 &  67 ans 7 mois &  -2.92\% &  2789.67 &  {\bf 85.00} &  3281.97 &  {\bf {\color{red} 0.85}} &  {\bf {\color{red} 0.82}} &  {\bf {\color{red} 0.77}} &  {\bf {\color{red} 0.72}} &  {\bf {\color{red} 0.67}} &  {\bf {\color{red} 0.63}} \\ 
\hline 
\hline 
\end{tabular} 
\end{center} } 
\paragraph{Retraites possibles et ratios Revenu/SMIC à 70, 75, 80, 85, 90 ans avec le modèle \emph{Destinie2 (revalorisation de la fonction publique)}}  
 
{ \scriptsize \begin{center} 
\begin{tabular}[htb]{|c|c||c|c||c|c||c||c|c|c|c|c|c|} 
\hline 
 Retraite en &  Âge &  Âge pivot &  Décote/Surcote &  Retraite (\euro{} 2019) &  Tx Rempl(\%) &  SMIC (\euro{} 2019) &  Retraite/SMIC &  Rev70/SMIC &  Rev75/SMIC &  Rev80/SMIC &  Rev85/SMIC &  Rev90/SMIC \\ 
\hline \hline 
 2065 &  62 &  67 ans 2 mois &  -25.83\% &  1942.90 &  {\bf 35.63} &  2892.68 &  {\bf {\color{red} 0.67}} &  {\bf {\color{red} 0.61}} &  {\bf {\color{red} 0.57}} &  {\bf {\color{red} 0.53}} &  {\bf {\color{red} 0.50}} &  {\bf {\color{red} 0.47}} \\ 
\hline 
 2066 &  63 &  67 ans 3 mois &  -21.25\% &  2150.33 &  {\bf 38.93} &  2930.29 &  {\bf {\color{red} 0.73}} &  {\bf {\color{red} 0.67}} &  {\bf {\color{red} 0.63}} &  {\bf {\color{red} 0.59}} &  {\bf {\color{red} 0.55}} &  {\bf {\color{red} 0.52}} \\ 
\hline 
 2067 &  64 &  67 ans 4 mois &  -16.67\% &  2369.96 &  {\bf 42.36} &  2968.38 &  {\bf {\color{red} 0.80}} &  {\bf {\color{red} 0.74}} &  {\bf {\color{red} 0.69}} &  {\bf {\color{red} 0.65}} &  {\bf {\color{red} 0.61}} &  {\bf {\color{red} 0.57}} \\ 
\hline 
 2068 &  65 &  67 ans 5 mois &  -12.08\% &  2602.17 &  {\bf 45.91} &  3006.97 &  {\bf {\color{red} 0.87}} &  {\bf {\color{red} 0.81}} &  {\bf {\color{red} 0.76}} &  {\bf {\color{red} 0.71}} &  {\bf {\color{red} 0.67}} &  {\bf {\color{red} 0.63}} \\ 
\hline 
 2069 &  66 &  67 ans 6 mois &  -7.50\% &  2847.34 &  {\bf 49.59} &  3046.06 &  {\bf {\color{red} 0.93}} &  {\bf {\color{red} 0.89}} &  {\bf {\color{red} 0.83}} &  {\bf {\color{red} 0.78}} &  {\bf {\color{red} 0.73}} &  {\bf {\color{red} 0.69}} \\ 
\hline 
 2070 &  67 &  67 ans 7 mois &  -2.92\% &  3105.87 &  {\bf 53.40} &  3085.66 &  {\bf 1.01} &  {\bf {\color{red} 0.97}} &  {\bf {\color{red} 0.91}} &  {\bf {\color{red} 0.85}} &  {\bf {\color{red} 0.80}} &  {\bf {\color{red} 0.75}} \\ 
\hline 
\hline 
\end{tabular} 
\end{center} } 

 \begin{center}\includegraphics[width=0.9\textwidth]{fig/TechHosp_2003_22_dest_retraite.pdf}\end{center} \label{fig/TechHosp_2003_22_dest_retraite.pdf} 

\newpage 
 
\chapter{Adjoint Technique (devenant principal C2 puis C1)} 

\begin{minipage}{0.55\linewidth}\includegraphics[width=0.7\textwidth]{fig/grille_AdjTech.pdf}\end{minipage} 
\begin{minipage}{0.3\linewidth} 
 \begin{center} 

\begin{tabular}[htb]{|c|c|} 
\hline 
 Indice majoré &  Durée (années) \\ 
\hline \hline 
 326 &  1.00 \\ 
\hline 
 327 &  2.00 \\ 
\hline 
 328 &  2.00 \\ 
\hline 
 329 &  2.00 \\ 
\hline 
 330 &  2.00 \\ 
\hline 
 332 &  2.00 \\ 
\hline 
 335 &  2.00 \\ 
\hline 
 339 &  2.00 \\ 
\hline 
 343 &  3.00 \\ 
\hline 
 354 &  3.00 \\ 
\hline 
 367 &  3.33 \\ 
\hline 
 380 &  2.00 \\ 
\hline 
 390 &  3.00 \\ 
\hline 
 402 &  3.00 \\ 
\hline 
 411 &  4.00 \\ 
\hline 
 415 &  3.00 \\ 
\hline 
 430 &  3.00 \\ 
\hline 
 450 &  3.00 \\ 
\hline 
 466 &   \\ 
\hline 
\hline 
\end{tabular} 
\end{center} 
 \end{minipage} 


 \addto{\captionsenglish}{ \renewcommand{\mtctitle}{}} \setcounter{minitocdepth}{2} 
 \minitoc \newpage 

\section{Début de carrière à 22 ans} 

\subsection{Génération 1975 (début en 1997)} 

\paragraph{Retraites possibles et ratios Revenu/SMIC à 70, 75, 80, 85, 90 ans avec le modèle \emph{Gouvernement truqué (âge-pivot bloqué à 65 ans)}}  
 
{ \scriptsize \begin{center} 
\begin{tabular}[htb]{|c|c||c|c||c|c||c||c|c|c|c|c|c|} 
\hline 
 Retraite en &  Âge &  Âge pivot &  Décote/Surcote &  Retraite (\euro{} 2019) &  Tx Rempl(\%) &  SMIC (\euro{} 2019) &  Retraite/SMIC &  Rev70/SMIC &  Rev75/SMIC &  Rev80/SMIC &  Rev85/SMIC &  Rev90/SMIC \\ 
\hline \hline 
 2037 &  62 &  64 ans 10 mois &  -14.17\% &  1091.14 &  {\bf 43.52} &  2143.00 &  {\bf {\color{red} 0.51}} &  {\bf {\color{red} 0.46}} &  {\bf {\color{red} 0.43}} &  {\bf {\color{red} 0.40}} &  {\bf {\color{red} 0.38}} &  {\bf {\color{red} 0.35}} \\ 
\hline 
 2038 &  63 &  64 ans 11 mois &  -9.58\% &  1189.79 &  {\bf 47.37} &  2170.86 &  {\bf {\color{red} 0.55}} &  {\bf {\color{red} 0.50}} &  {\bf {\color{red} 0.47}} &  {\bf {\color{red} 0.44}} &  {\bf {\color{red} 0.41}} &  {\bf {\color{red} 0.39}} \\ 
\hline 
 2039 &  64 &  65 ans 0 mois &  -5.00\% &  1294.91 &  {\bf 49.90} &  2199.08 &  {\bf {\color{red} 0.59}} &  {\bf {\color{red} 0.54}} &  {\bf {\color{red} 0.51}} &  {\bf {\color{red} 0.48}} &  {\bf {\color{red} 0.45}} &  {\bf {\color{red} 0.42}} \\ 
\hline 
 2040 &  65 &  65 ans 0 mois &  0.00\% &  1893.52 &  {\bf 71.76} &  2227.67 &  {\bf {\color{red} 0.85}} &  {\bf {\color{red} 0.80}} &  {\bf {\color{red} 0.75}} &  {\bf {\color{red} 0.70}} &  {\bf {\color{red} 0.66}} &  {\bf {\color{red} 0.62}} \\ 
\hline 
 2041 &  66 &  65 ans 0 mois &  5.00\% &  1918.14 &  {\bf 72.56} &  2256.63 &  {\bf {\color{red} 0.85}} &  {\bf {\color{red} 0.81}} &  {\bf {\color{red} 0.76}} &  {\bf {\color{red} 0.71}} &  {\bf {\color{red} 0.67}} &  {\bf {\color{red} 0.62}} \\ 
\hline 
 2042 &  67 &  65 ans 0 mois &  10.00\% &  1943.07 &  {\bf 71.66} &  2285.97 &  {\bf {\color{red} 0.85}} &  {\bf {\color{red} 0.82}} &  {\bf {\color{red} 0.77}} &  {\bf {\color{red} 0.72}} &  {\bf {\color{red} 0.67}} &  {\bf {\color{red} 0.63}} \\ 
\hline 
\hline 
\end{tabular} 
\end{center} } 
\paragraph{Retraites possibles et ratios Revenu/SMIC à 70, 75, 80, 85, 90 ans avec le modèle \emph{Gouvernement corrigé (âge-pivot glissant)}}  
 
{ \scriptsize \begin{center} 
\begin{tabular}[htb]{|c|c||c|c||c|c||c||c|c|c|c|c|c|} 
\hline 
 Retraite en &  Âge &  Âge pivot &  Décote/Surcote &  Retraite (\euro{} 2019) &  Tx Rempl(\%) &  SMIC (\euro{} 2019) &  Retraite/SMIC &  Rev70/SMIC &  Rev75/SMIC &  Rev80/SMIC &  Rev85/SMIC &  Rev90/SMIC \\ 
\hline \hline 
 2037 &  62 &  64 ans 10 mois &  -14.17\% &  1091.14 &  {\bf 43.52} &  2143.00 &  {\bf {\color{red} 0.51}} &  {\bf {\color{red} 0.46}} &  {\bf {\color{red} 0.43}} &  {\bf {\color{red} 0.40}} &  {\bf {\color{red} 0.38}} &  {\bf {\color{red} 0.35}} \\ 
\hline 
 2038 &  63 &  64 ans 11 mois &  -9.58\% &  1189.79 &  {\bf 47.37} &  2170.86 &  {\bf {\color{red} 0.55}} &  {\bf {\color{red} 0.50}} &  {\bf {\color{red} 0.47}} &  {\bf {\color{red} 0.44}} &  {\bf {\color{red} 0.41}} &  {\bf {\color{red} 0.39}} \\ 
\hline 
 2039 &  64 &  65 ans 0 mois &  -5.00\% &  1294.91 &  {\bf 49.90} &  2199.08 &  {\bf {\color{red} 0.59}} &  {\bf {\color{red} 0.54}} &  {\bf {\color{red} 0.51}} &  {\bf {\color{red} 0.48}} &  {\bf {\color{red} 0.45}} &  {\bf {\color{red} 0.42}} \\ 
\hline 
 2040 &  65 &  65 ans 1 mois &  -0.42\% &  1893.52 &  {\bf 71.76} &  2227.67 &  {\bf {\color{red} 0.85}} &  {\bf {\color{red} 0.80}} &  {\bf {\color{red} 0.75}} &  {\bf {\color{red} 0.70}} &  {\bf {\color{red} 0.66}} &  {\bf {\color{red} 0.62}} \\ 
\hline 
 2041 &  66 &  65 ans 2 mois &  4.17\% &  1918.14 &  {\bf 72.56} &  2256.63 &  {\bf {\color{red} 0.85}} &  {\bf {\color{red} 0.81}} &  {\bf {\color{red} 0.76}} &  {\bf {\color{red} 0.71}} &  {\bf {\color{red} 0.67}} &  {\bf {\color{red} 0.62}} \\ 
\hline 
 2042 &  67 &  65 ans 3 mois &  8.75\% &  1943.07 &  {\bf 71.66} &  2285.97 &  {\bf {\color{red} 0.85}} &  {\bf {\color{red} 0.82}} &  {\bf {\color{red} 0.77}} &  {\bf {\color{red} 0.72}} &  {\bf {\color{red} 0.67}} &  {\bf {\color{red} 0.63}} \\ 
\hline 
\hline 
\end{tabular} 
\end{center} } 
\paragraph{Retraites possibles et ratios Revenu/SMIC à 70, 75, 80, 85, 90 ans avec le modèle \emph{Destinie2 (revalorisation de la fonction publique)}}  
 
{ \scriptsize \begin{center} 
\begin{tabular}[htb]{|c|c||c|c||c|c||c||c|c|c|c|c|c|} 
\hline 
 Retraite en &  Âge &  Âge pivot &  Décote/Surcote &  Retraite (\euro{} 2019) &  Tx Rempl(\%) &  SMIC (\euro{} 2019) &  Retraite/SMIC &  Rev70/SMIC &  Rev75/SMIC &  Rev80/SMIC &  Rev85/SMIC &  Rev90/SMIC \\ 
\hline \hline 
 2037 &  62 &  64 ans 10 mois &  -14.17\% &  1212.60 &  {\bf 38.79} &  2014.82 &  {\bf {\color{red} 0.60}} &  {\bf {\color{red} 0.54}} &  {\bf {\color{red} 0.51}} &  {\bf {\color{red} 0.48}} &  {\bf {\color{red} 0.45}} &  {\bf {\color{red} 0.42}} \\ 
\hline 
 2038 &  63 &  64 ans 11 mois &  -9.58\% &  1326.95 &  {\bf 41.90} &  2041.01 &  {\bf {\color{red} 0.65}} &  {\bf {\color{red} 0.59}} &  {\bf {\color{red} 0.56}} &  {\bf {\color{red} 0.52}} &  {\bf {\color{red} 0.49}} &  {\bf {\color{red} 0.46}} \\ 
\hline 
 2039 &  64 &  65 ans 0 mois &  -5.00\% &  1449.66 &  {\bf 43.82} &  2067.55 &  {\bf {\color{red} 0.70}} &  {\bf {\color{red} 0.65}} &  {\bf {\color{red} 0.61}} &  {\bf {\color{red} 0.57}} &  {\bf {\color{red} 0.53}} &  {\bf {\color{red} 0.50}} \\ 
\hline 
 2040 &  65 &  65 ans 1 mois &  -0.42\% &  1780.26 &  {\bf 52.34} &  2094.43 &  {\bf {\color{red} 0.85}} &  {\bf {\color{red} 0.80}} &  {\bf {\color{red} 0.75}} &  {\bf {\color{red} 0.70}} &  {\bf {\color{red} 0.66}} &  {\bf {\color{red} 0.62}} \\ 
\hline 
 2041 &  66 &  65 ans 2 mois &  4.17\% &  1803.40 &  {\bf 52.34} &  2121.65 &  {\bf {\color{red} 0.85}} &  {\bf {\color{red} 0.81}} &  {\bf {\color{red} 0.76}} &  {\bf {\color{red} 0.71}} &  {\bf {\color{red} 0.67}} &  {\bf {\color{red} 0.62}} \\ 
\hline 
 2042 &  67 &  65 ans 3 mois &  8.75\% &  1868.74 &  {\bf 52.30} &  2149.23 &  {\bf {\color{red} 0.87}} &  {\bf {\color{red} 0.84}} &  {\bf {\color{red} 0.78}} &  {\bf {\color{red} 0.74}} &  {\bf {\color{red} 0.69}} &  {\bf {\color{red} 0.65}} \\ 
\hline 
\hline 
\end{tabular} 
\end{center} } 

 \begin{center}\includegraphics[width=0.9\textwidth]{fig/AdjTech_1975_22_dest_retraite.pdf}\end{center} \label{fig/AdjTech_1975_22_dest_retraite.pdf} 

\newpage 
 
\subsection{Génération 1980 (début en 2002)} 

\paragraph{Retraites possibles et ratios Revenu/SMIC à 70, 75, 80, 85, 90 ans avec le modèle \emph{Gouvernement truqué (âge-pivot bloqué à 65 ans)}}  
 
{ \scriptsize \begin{center} 
\begin{tabular}[htb]{|c|c||c|c||c|c||c||c|c|c|c|c|c|} 
\hline 
 Retraite en &  Âge &  Âge pivot &  Décote/Surcote &  Retraite (\euro{} 2019) &  Tx Rempl(\%) &  SMIC (\euro{} 2019) &  Retraite/SMIC &  Rev70/SMIC &  Rev75/SMIC &  Rev80/SMIC &  Rev85/SMIC &  Rev90/SMIC \\ 
\hline \hline 
 2042 &  62 &  65 ans 0 mois &  -15.00\% &  1105.27 &  {\bf 44.09} &  2285.97 &  {\bf {\color{red} 0.48}} &  {\bf {\color{red} 0.44}} &  {\bf {\color{red} 0.41}} &  {\bf {\color{red} 0.38}} &  {\bf {\color{red} 0.36}} &  {\bf {\color{red} 0.34}} \\ 
\hline 
 2043 &  63 &  65 ans 0 mois &  -10.00\% &  1215.16 &  {\bf 48.38} &  2315.68 &  {\bf {\color{red} 0.52}} &  {\bf {\color{red} 0.48}} &  {\bf {\color{red} 0.45}} &  {\bf {\color{red} 0.42}} &  {\bf {\color{red} 0.39}} &  {\bf {\color{red} 0.37}} \\ 
\hline 
 2044 &  64 &  65 ans 0 mois &  -5.00\% &  1332.67 &  {\bf 51.36} &  2345.79 &  {\bf {\color{red} 0.57}} &  {\bf {\color{red} 0.53}} &  {\bf {\color{red} 0.49}} &  {\bf {\color{red} 0.46}} &  {\bf {\color{red} 0.43}} &  {\bf {\color{red} 0.41}} \\ 
\hline 
 2045 &  65 &  65 ans 0 mois &  0.00\% &  2019.84 &  {\bf 76.55} &  2376.28 &  {\bf {\color{red} 0.85}} &  {\bf {\color{red} 0.80}} &  {\bf {\color{red} 0.75}} &  {\bf {\color{red} 0.70}} &  {\bf {\color{red} 0.66}} &  {\bf {\color{red} 0.62}} \\ 
\hline 
 2046 &  66 &  65 ans 0 mois &  5.00\% &  2046.10 &  {\bf 77.40} &  2407.18 &  {\bf {\color{red} 0.85}} &  {\bf {\color{red} 0.81}} &  {\bf {\color{red} 0.76}} &  {\bf {\color{red} 0.71}} &  {\bf {\color{red} 0.67}} &  {\bf {\color{red} 0.62}} \\ 
\hline 
 2047 &  67 &  65 ans 0 mois &  10.00\% &  2072.70 &  {\bf 76.44} &  2438.47 &  {\bf {\color{red} 0.85}} &  {\bf {\color{red} 0.82}} &  {\bf {\color{red} 0.77}} &  {\bf {\color{red} 0.72}} &  {\bf {\color{red} 0.67}} &  {\bf {\color{red} 0.63}} \\ 
\hline 
\hline 
\end{tabular} 
\end{center} } 
\paragraph{Retraites possibles et ratios Revenu/SMIC à 70, 75, 80, 85, 90 ans avec le modèle \emph{Gouvernement corrigé (âge-pivot glissant)}}  
 
{ \scriptsize \begin{center} 
\begin{tabular}[htb]{|c|c||c|c||c|c||c||c|c|c|c|c|c|} 
\hline 
 Retraite en &  Âge &  Âge pivot &  Décote/Surcote &  Retraite (\euro{} 2019) &  Tx Rempl(\%) &  SMIC (\euro{} 2019) &  Retraite/SMIC &  Rev70/SMIC &  Rev75/SMIC &  Rev80/SMIC &  Rev85/SMIC &  Rev90/SMIC \\ 
\hline \hline 
 2042 &  62 &  65 ans 3 mois &  -16.25\% &  1089.01 &  {\bf 43.44} &  2285.97 &  {\bf {\color{red} 0.48}} &  {\bf {\color{red} 0.43}} &  {\bf {\color{red} 0.40}} &  {\bf {\color{red} 0.38}} &  {\bf {\color{red} 0.35}} &  {\bf {\color{red} 0.33}} \\ 
\hline 
 2043 &  63 &  65 ans 4 mois &  -11.67\% &  1192.66 &  {\bf 47.48} &  2315.68 &  {\bf {\color{red} 0.52}} &  {\bf {\color{red} 0.47}} &  {\bf {\color{red} 0.44}} &  {\bf {\color{red} 0.41}} &  {\bf {\color{red} 0.39}} &  {\bf {\color{red} 0.36}} \\ 
\hline 
 2044 &  64 &  65 ans 5 mois &  -7.08\% &  1303.45 &  {\bf 50.23} &  2345.79 &  {\bf {\color{red} 0.56}} &  {\bf {\color{red} 0.51}} &  {\bf {\color{red} 0.48}} &  {\bf {\color{red} 0.45}} &  {\bf {\color{red} 0.42}} &  {\bf {\color{red} 0.40}} \\ 
\hline 
 2045 &  65 &  65 ans 6 mois &  -2.50\% &  2019.84 &  {\bf 76.55} &  2376.28 &  {\bf {\color{red} 0.85}} &  {\bf {\color{red} 0.80}} &  {\bf {\color{red} 0.75}} &  {\bf {\color{red} 0.70}} &  {\bf {\color{red} 0.66}} &  {\bf {\color{red} 0.62}} \\ 
\hline 
 2046 &  66 &  65 ans 7 mois &  2.08\% &  2046.10 &  {\bf 77.40} &  2407.18 &  {\bf {\color{red} 0.85}} &  {\bf {\color{red} 0.81}} &  {\bf {\color{red} 0.76}} &  {\bf {\color{red} 0.71}} &  {\bf {\color{red} 0.67}} &  {\bf {\color{red} 0.62}} \\ 
\hline 
 2047 &  67 &  65 ans 8 mois &  6.67\% &  2072.70 &  {\bf 76.44} &  2438.47 &  {\bf {\color{red} 0.85}} &  {\bf {\color{red} 0.82}} &  {\bf {\color{red} 0.77}} &  {\bf {\color{red} 0.72}} &  {\bf {\color{red} 0.67}} &  {\bf {\color{red} 0.63}} \\ 
\hline 
\hline 
\end{tabular} 
\end{center} } 
\paragraph{Retraites possibles et ratios Revenu/SMIC à 70, 75, 80, 85, 90 ans avec le modèle \emph{Destinie2 (revalorisation de la fonction publique)}}  
 
{ \scriptsize \begin{center} 
\begin{tabular}[htb]{|c|c||c|c||c|c||c||c|c|c|c|c|c|} 
\hline 
 Retraite en &  Âge &  Âge pivot &  Décote/Surcote &  Retraite (\euro{} 2019) &  Tx Rempl(\%) &  SMIC (\euro{} 2019) &  Retraite/SMIC &  Rev70/SMIC &  Rev75/SMIC &  Rev80/SMIC &  Rev85/SMIC &  Rev90/SMIC \\ 
\hline \hline 
 2042 &  62 &  65 ans 3 mois &  -16.25\% &  1245.98 &  {\bf 37.36} &  2149.23 &  {\bf {\color{red} 0.58}} &  {\bf {\color{red} 0.52}} &  {\bf {\color{red} 0.49}} &  {\bf {\color{red} 0.46}} &  {\bf {\color{red} 0.43}} &  {\bf {\color{red} 0.40}} \\ 
\hline 
 2043 &  63 &  65 ans 4 mois &  -11.67\% &  1370.77 &  {\bf 40.58} &  2177.17 &  {\bf {\color{red} 0.63}} &  {\bf {\color{red} 0.58}} &  {\bf {\color{red} 0.54}} &  {\bf {\color{red} 0.51}} &  {\bf {\color{red} 0.47}} &  {\bf {\color{red} 0.44}} \\ 
\hline 
 2044 &  64 &  65 ans 5 mois &  -7.08\% &  1505.17 &  {\bf 42.65} &  2205.48 &  {\bf {\color{red} 0.68}} &  {\bf {\color{red} 0.63}} &  {\bf {\color{red} 0.59}} &  {\bf {\color{red} 0.56}} &  {\bf {\color{red} 0.52}} &  {\bf {\color{red} 0.49}} \\ 
\hline 
 2045 &  65 &  65 ans 6 mois &  -2.50\% &  1899.03 &  {\bf 52.34} &  2234.15 &  {\bf {\color{red} 0.85}} &  {\bf {\color{red} 0.80}} &  {\bf {\color{red} 0.75}} &  {\bf {\color{red} 0.70}} &  {\bf {\color{red} 0.66}} &  {\bf {\color{red} 0.62}} \\ 
\hline 
 2046 &  66 &  65 ans 7 mois &  2.08\% &  1923.71 &  {\bf 52.34} &  2263.19 &  {\bf {\color{red} 0.85}} &  {\bf {\color{red} 0.81}} &  {\bf {\color{red} 0.76}} &  {\bf {\color{red} 0.71}} &  {\bf {\color{red} 0.67}} &  {\bf {\color{red} 0.62}} \\ 
\hline 
 2047 &  67 &  65 ans 8 mois &  6.67\% &  1963.32 &  {\bf 51.51} &  2292.61 &  {\bf {\color{red} 0.86}} &  {\bf {\color{red} 0.82}} &  {\bf {\color{red} 0.77}} &  {\bf {\color{red} 0.72}} &  {\bf {\color{red} 0.68}} &  {\bf {\color{red} 0.64}} \\ 
\hline 
\hline 
\end{tabular} 
\end{center} } 

 \begin{center}\includegraphics[width=0.9\textwidth]{fig/AdjTech_1980_22_dest_retraite.pdf}\end{center} \label{fig/AdjTech_1980_22_dest_retraite.pdf} 

\newpage 
 
\subsection{Génération 1990 (début en 2012)} 

\paragraph{Retraites possibles et ratios Revenu/SMIC à 70, 75, 80, 85, 90 ans avec le modèle \emph{Gouvernement truqué (âge-pivot bloqué à 65 ans)}}  
 
{ \scriptsize \begin{center} 
\begin{tabular}[htb]{|c|c||c|c||c|c||c||c|c|c|c|c|c|} 
\hline 
 Retraite en &  Âge &  Âge pivot &  Décote/Surcote &  Retraite (\euro{} 2019) &  Tx Rempl(\%) &  SMIC (\euro{} 2019) &  Retraite/SMIC &  Rev70/SMIC &  Rev75/SMIC &  Rev80/SMIC &  Rev85/SMIC &  Rev90/SMIC \\ 
\hline \hline 
 2052 &  62 &  65 ans 0 mois &  -15.00\% &  1181.65 &  {\bf 45.43} &  2601.14 &  {\bf {\color{red} 0.45}} &  {\bf {\color{red} 0.41}} &  {\bf {\color{red} 0.38}} &  {\bf {\color{red} 0.36}} &  {\bf {\color{red} 0.34}} &  {\bf {\color{red} 0.32}} \\ 
\hline 
 2053 &  63 &  65 ans 0 mois &  -10.00\% &  1300.43 &  {\bf 49.35} &  2634.96 &  {\bf {\color{red} 0.49}} &  {\bf {\color{red} 0.45}} &  {\bf {\color{red} 0.42}} &  {\bf {\color{red} 0.40}} &  {\bf {\color{red} 0.37}} &  {\bf {\color{red} 0.35}} \\ 
\hline 
 2054 &  64 &  65 ans 0 mois &  -5.00\% &  1425.82 &  {\bf 53.42} &  2669.21 &  {\bf {\color{red} 0.53}} &  {\bf {\color{red} 0.49}} &  {\bf {\color{red} 0.46}} &  {\bf {\color{red} 0.43}} &  {\bf {\color{red} 0.41}} &  {\bf {\color{red} 0.38}} \\ 
\hline 
 2055 &  65 &  65 ans 0 mois &  0.00\% &  2298.33 &  {\bf 85.00} &  2703.91 &  {\bf {\color{red} 0.85}} &  {\bf {\color{red} 0.80}} &  {\bf {\color{red} 0.75}} &  {\bf {\color{red} 0.70}} &  {\bf {\color{red} 0.66}} &  {\bf {\color{red} 0.62}} \\ 
\hline 
 2056 &  66 &  65 ans 0 mois &  5.00\% &  2328.20 &  {\bf 85.00} &  2739.06 &  {\bf {\color{red} 0.85}} &  {\bf {\color{red} 0.81}} &  {\bf {\color{red} 0.76}} &  {\bf {\color{red} 0.71}} &  {\bf {\color{red} 0.67}} &  {\bf {\color{red} 0.62}} \\ 
\hline 
 2057 &  67 &  65 ans 0 mois &  10.00\% &  2358.47 &  {\bf 85.00} &  2774.67 &  {\bf {\color{red} 0.85}} &  {\bf {\color{red} 0.82}} &  {\bf {\color{red} 0.77}} &  {\bf {\color{red} 0.72}} &  {\bf {\color{red} 0.67}} &  {\bf {\color{red} 0.63}} \\ 
\hline 
\hline 
\end{tabular} 
\end{center} } 
\paragraph{Retraites possibles et ratios Revenu/SMIC à 70, 75, 80, 85, 90 ans avec le modèle \emph{Gouvernement corrigé (âge-pivot glissant)}}  
 
{ \scriptsize \begin{center} 
\begin{tabular}[htb]{|c|c||c|c||c|c||c||c|c|c|c|c|c|} 
\hline 
 Retraite en &  Âge &  Âge pivot &  Décote/Surcote &  Retraite (\euro{} 2019) &  Tx Rempl(\%) &  SMIC (\euro{} 2019) &  Retraite/SMIC &  Rev70/SMIC &  Rev75/SMIC &  Rev80/SMIC &  Rev85/SMIC &  Rev90/SMIC \\ 
\hline \hline 
 2052 &  62 &  66 ans 1 mois &  -20.42\% &  1106.35 &  {\bf 42.53} &  2601.14 &  {\bf {\color{red} 0.43}} &  {\bf {\color{red} 0.38}} &  {\bf {\color{red} 0.36}} &  {\bf {\color{red} 0.34}} &  {\bf {\color{red} 0.32}} &  {\bf {\color{red} 0.30}} \\ 
\hline 
 2053 &  63 &  66 ans 2 mois &  -15.83\% &  1216.15 &  {\bf 46.15} &  2634.96 &  {\bf {\color{red} 0.46}} &  {\bf {\color{red} 0.42}} &  {\bf {\color{red} 0.40}} &  {\bf {\color{red} 0.37}} &  {\bf {\color{red} 0.35}} &  {\bf {\color{red} 0.33}} \\ 
\hline 
 2054 &  64 &  66 ans 3 mois &  -11.25\% &  1332.02 &  {\bf 49.90} &  2669.21 &  {\bf {\color{red} 0.50}} &  {\bf {\color{red} 0.46}} &  {\bf {\color{red} 0.43}} &  {\bf {\color{red} 0.41}} &  {\bf {\color{red} 0.38}} &  {\bf {\color{red} 0.36}} \\ 
\hline 
 2055 &  65 &  66 ans 4 mois &  -6.67\% &  2298.33 &  {\bf 85.00} &  2703.91 &  {\bf {\color{red} 0.85}} &  {\bf {\color{red} 0.80}} &  {\bf {\color{red} 0.75}} &  {\bf {\color{red} 0.70}} &  {\bf {\color{red} 0.66}} &  {\bf {\color{red} 0.62}} \\ 
\hline 
 2056 &  66 &  66 ans 5 mois &  -2.08\% &  2328.20 &  {\bf 85.00} &  2739.06 &  {\bf {\color{red} 0.85}} &  {\bf {\color{red} 0.81}} &  {\bf {\color{red} 0.76}} &  {\bf {\color{red} 0.71}} &  {\bf {\color{red} 0.67}} &  {\bf {\color{red} 0.62}} \\ 
\hline 
 2057 &  67 &  66 ans 6 mois &  2.50\% &  2358.47 &  {\bf 85.00} &  2774.67 &  {\bf {\color{red} 0.85}} &  {\bf {\color{red} 0.82}} &  {\bf {\color{red} 0.77}} &  {\bf {\color{red} 0.72}} &  {\bf {\color{red} 0.67}} &  {\bf {\color{red} 0.63}} \\ 
\hline 
\hline 
\end{tabular} 
\end{center} } 
\paragraph{Retraites possibles et ratios Revenu/SMIC à 70, 75, 80, 85, 90 ans avec le modèle \emph{Destinie2 (revalorisation de la fonction publique)}}  
 
{ \scriptsize \begin{center} 
\begin{tabular}[htb]{|c|c||c|c||c|c||c||c|c|c|c|c|c|} 
\hline 
 Retraite en &  Âge &  Âge pivot &  Décote/Surcote &  Retraite (\euro{} 2019) &  Tx Rempl(\%) &  SMIC (\euro{} 2019) &  Retraite/SMIC &  Rev70/SMIC &  Rev75/SMIC &  Rev80/SMIC &  Rev85/SMIC &  Rev90/SMIC \\ 
\hline \hline 
 2052 &  62 &  66 ans 1 mois &  -20.42\% &  1368.95 &  {\bf 36.07} &  2445.56 &  {\bf {\color{red} 0.56}} &  {\bf {\color{red} 0.50}} &  {\bf {\color{red} 0.47}} &  {\bf {\color{red} 0.44}} &  {\bf {\color{red} 0.42}} &  {\bf {\color{red} 0.39}} \\ 
\hline 
 2053 &  63 &  66 ans 2 mois &  -15.83\% &  1511.64 &  {\bf 39.32} &  2477.35 &  {\bf {\color{red} 0.61}} &  {\bf {\color{red} 0.56}} &  {\bf {\color{red} 0.52}} &  {\bf {\color{red} 0.49}} &  {\bf {\color{red} 0.46}} &  {\bf {\color{red} 0.43}} \\ 
\hline 
 2054 &  64 &  66 ans 3 mois &  -11.25\% &  1664.28 &  {\bf 41.45} &  2509.56 &  {\bf {\color{red} 0.66}} &  {\bf {\color{red} 0.61}} &  {\bf {\color{red} 0.58}} &  {\bf {\color{red} 0.54}} &  {\bf {\color{red} 0.51}} &  {\bf {\color{red} 0.47}} \\ 
\hline 
 2055 &  65 &  66 ans 4 mois &  -6.67\% &  2160.85 &  {\bf 52.34} &  2542.18 &  {\bf {\color{red} 0.85}} &  {\bf {\color{red} 0.80}} &  {\bf {\color{red} 0.75}} &  {\bf {\color{red} 0.70}} &  {\bf {\color{red} 0.66}} &  {\bf {\color{red} 0.62}} \\ 
\hline 
 2056 &  66 &  66 ans 5 mois &  -2.08\% &  2188.95 &  {\bf 52.34} &  2575.23 &  {\bf {\color{red} 0.85}} &  {\bf {\color{red} 0.81}} &  {\bf {\color{red} 0.76}} &  {\bf {\color{red} 0.71}} &  {\bf {\color{red} 0.67}} &  {\bf {\color{red} 0.62}} \\ 
\hline 
 2057 &  67 &  66 ans 6 mois &  2.50\% &  2217.40 &  {\bf 51.13} &  2608.71 &  {\bf {\color{red} 0.85}} &  {\bf {\color{red} 0.82}} &  {\bf {\color{red} 0.77}} &  {\bf {\color{red} 0.72}} &  {\bf {\color{red} 0.67}} &  {\bf {\color{red} 0.63}} \\ 
\hline 
\hline 
\end{tabular} 
\end{center} } 

 \begin{center}\includegraphics[width=0.9\textwidth]{fig/AdjTech_1990_22_dest_retraite.pdf}\end{center} \label{fig/AdjTech_1990_22_dest_retraite.pdf} 

\newpage 
 
\subsection{Génération 2003 (début en 2025)} 

\paragraph{Retraites possibles et ratios Revenu/SMIC à 70, 75, 80, 85, 90 ans avec le modèle \emph{Gouvernement truqué (âge-pivot bloqué à 65 ans)}}  
 
{ \scriptsize \begin{center} 
\begin{tabular}[htb]{|c|c||c|c||c|c||c||c|c|c|c|c|c|} 
\hline 
 Retraite en &  Âge &  Âge pivot &  Décote/Surcote &  Retraite (\euro{} 2019) &  Tx Rempl(\%) &  SMIC (\euro{} 2019) &  Retraite/SMIC &  Rev70/SMIC &  Rev75/SMIC &  Rev80/SMIC &  Rev85/SMIC &  Rev90/SMIC \\ 
\hline \hline 
 2065 &  62 &  65 ans 0 mois &  -15.00\% &  1457.41 &  {\bf 47.37} &  3076.71 &  {\bf {\color{red} 0.47}} &  {\bf {\color{red} 0.43}} &  {\bf {\color{red} 0.40}} &  {\bf {\color{red} 0.38}} &  {\bf {\color{red} 0.35}} &  {\bf {\color{red} 0.33}} \\ 
\hline 
 2066 &  63 &  65 ans 0 mois &  -10.00\% &  1602.25 &  {\bf 51.41} &  3116.71 &  {\bf {\color{red} 0.51}} &  {\bf {\color{red} 0.47}} &  {\bf {\color{red} 0.44}} &  {\bf {\color{red} 0.41}} &  {\bf {\color{red} 0.39}} &  {\bf {\color{red} 0.36}} \\ 
\hline 
 2067 &  64 &  65 ans 0 mois &  -5.00\% &  1755.00 &  {\bf 55.59} &  3157.23 &  {\bf {\color{red} 0.56}} &  {\bf {\color{red} 0.51}} &  {\bf {\color{red} 0.48}} &  {\bf {\color{red} 0.45}} &  {\bf {\color{red} 0.42}} &  {\bf {\color{red} 0.40}} \\ 
\hline 
 2068 &  65 &  65 ans 0 mois &  0.00\% &  2718.53 &  {\bf 85.00} &  3198.27 &  {\bf {\color{red} 0.85}} &  {\bf {\color{red} 0.80}} &  {\bf {\color{red} 0.75}} &  {\bf {\color{red} 0.70}} &  {\bf {\color{red} 0.66}} &  {\bf {\color{red} 0.62}} \\ 
\hline 
 2069 &  66 &  65 ans 0 mois &  5.00\% &  2753.87 &  {\bf 85.00} &  3239.85 &  {\bf {\color{red} 0.85}} &  {\bf {\color{red} 0.81}} &  {\bf {\color{red} 0.76}} &  {\bf {\color{red} 0.71}} &  {\bf {\color{red} 0.67}} &  {\bf {\color{red} 0.62}} \\ 
\hline 
 2070 &  67 &  65 ans 0 mois &  10.00\% &  2789.67 &  {\bf 85.00} &  3281.97 &  {\bf {\color{red} 0.85}} &  {\bf {\color{red} 0.82}} &  {\bf {\color{red} 0.77}} &  {\bf {\color{red} 0.72}} &  {\bf {\color{red} 0.67}} &  {\bf {\color{red} 0.63}} \\ 
\hline 
\hline 
\end{tabular} 
\end{center} } 
\paragraph{Retraites possibles et ratios Revenu/SMIC à 70, 75, 80, 85, 90 ans avec le modèle \emph{Gouvernement corrigé (âge-pivot glissant)}}  
 
{ \scriptsize \begin{center} 
\begin{tabular}[htb]{|c|c||c|c||c|c||c||c|c|c|c|c|c|} 
\hline 
 Retraite en &  Âge &  Âge pivot &  Décote/Surcote &  Retraite (\euro{} 2019) &  Tx Rempl(\%) &  SMIC (\euro{} 2019) &  Retraite/SMIC &  Rev70/SMIC &  Rev75/SMIC &  Rev80/SMIC &  Rev85/SMIC &  Rev90/SMIC \\ 
\hline \hline 
 2065 &  62 &  67 ans 2 mois &  -25.83\% &  1271.67 &  {\bf 41.33} &  3076.71 &  {\bf {\color{red} 0.41}} &  {\bf {\color{red} 0.37}} &  {\bf {\color{red} 0.35}} &  {\bf {\color{red} 0.33}} &  {\bf {\color{red} 0.31}} &  {\bf {\color{red} 0.29}} \\ 
\hline 
 2066 &  63 &  67 ans 3 mois &  -21.25\% &  1401.97 &  {\bf 44.98} &  3116.71 &  {\bf {\color{red} 0.45}} &  {\bf {\color{red} 0.41}} &  {\bf {\color{red} 0.39}} &  {\bf {\color{red} 0.36}} &  {\bf {\color{red} 0.34}} &  {\bf {\color{red} 0.32}} \\ 
\hline 
 2067 &  64 &  67 ans 4 mois &  -16.67\% &  1539.47 &  {\bf 48.76} &  3157.23 &  {\bf {\color{red} 0.49}} &  {\bf {\color{red} 0.45}} &  {\bf {\color{red} 0.42}} &  {\bf {\color{red} 0.40}} &  {\bf {\color{red} 0.37}} &  {\bf {\color{red} 0.35}} \\ 
\hline 
 2068 &  65 &  67 ans 5 mois &  -12.08\% &  2718.53 &  {\bf 85.00} &  3198.27 &  {\bf {\color{red} 0.85}} &  {\bf {\color{red} 0.80}} &  {\bf {\color{red} 0.75}} &  {\bf {\color{red} 0.70}} &  {\bf {\color{red} 0.66}} &  {\bf {\color{red} 0.62}} \\ 
\hline 
 2069 &  66 &  67 ans 6 mois &  -7.50\% &  2753.87 &  {\bf 85.00} &  3239.85 &  {\bf {\color{red} 0.85}} &  {\bf {\color{red} 0.81}} &  {\bf {\color{red} 0.76}} &  {\bf {\color{red} 0.71}} &  {\bf {\color{red} 0.67}} &  {\bf {\color{red} 0.62}} \\ 
\hline 
 2070 &  67 &  67 ans 7 mois &  -2.92\% &  2789.67 &  {\bf 85.00} &  3281.97 &  {\bf {\color{red} 0.85}} &  {\bf {\color{red} 0.82}} &  {\bf {\color{red} 0.77}} &  {\bf {\color{red} 0.72}} &  {\bf {\color{red} 0.67}} &  {\bf {\color{red} 0.63}} \\ 
\hline 
\hline 
\end{tabular} 
\end{center} } 
\paragraph{Retraites possibles et ratios Revenu/SMIC à 70, 75, 80, 85, 90 ans avec le modèle \emph{Destinie2 (revalorisation de la fonction publique)}}  
 
{ \scriptsize \begin{center} 
\begin{tabular}[htb]{|c|c||c|c||c|c||c||c|c|c|c|c|c|} 
\hline 
 Retraite en &  Âge &  Âge pivot &  Décote/Surcote &  Retraite (\euro{} 2019) &  Tx Rempl(\%) &  SMIC (\euro{} 2019) &  Retraite/SMIC &  Rev70/SMIC &  Rev75/SMIC &  Rev80/SMIC &  Rev85/SMIC &  Rev90/SMIC \\ 
\hline \hline 
 2065 &  62 &  67 ans 2 mois &  -25.83\% &  1584.92 &  {\bf 35.31} &  2892.68 &  {\bf {\color{red} 0.55}} &  {\bf {\color{red} 0.49}} &  {\bf {\color{red} 0.46}} &  {\bf {\color{red} 0.43}} &  {\bf {\color{red} 0.41}} &  {\bf {\color{red} 0.38}} \\ 
\hline 
 2066 &  63 &  67 ans 3 mois &  -21.25\% &  1754.58 &  {\bf 38.59} &  2930.29 &  {\bf {\color{red} 0.60}} &  {\bf {\color{red} 0.55}} &  {\bf {\color{red} 0.51}} &  {\bf {\color{red} 0.48}} &  {\bf {\color{red} 0.45}} &  {\bf {\color{red} 0.42}} \\ 
\hline 
 2067 &  64 &  67 ans 4 mois &  -16.67\% &  1935.93 &  {\bf 40.76} &  2968.38 &  {\bf {\color{red} 0.65}} &  {\bf {\color{red} 0.60}} &  {\bf {\color{red} 0.57}} &  {\bf {\color{red} 0.53}} &  {\bf {\color{red} 0.50}} &  {\bf {\color{red} 0.47}} \\ 
\hline 
 2068 &  65 &  67 ans 5 mois &  -12.08\% &  2555.93 &  {\bf 52.34} &  3006.97 &  {\bf {\color{red} 0.85}} &  {\bf {\color{red} 0.80}} &  {\bf {\color{red} 0.75}} &  {\bf {\color{red} 0.70}} &  {\bf {\color{red} 0.66}} &  {\bf {\color{red} 0.62}} \\ 
\hline 
 2069 &  66 &  67 ans 6 mois &  -7.50\% &  2589.15 &  {\bf 52.34} &  3046.06 &  {\bf {\color{red} 0.85}} &  {\bf {\color{red} 0.81}} &  {\bf {\color{red} 0.76}} &  {\bf {\color{red} 0.71}} &  {\bf {\color{red} 0.67}} &  {\bf {\color{red} 0.62}} \\ 
\hline 
 2070 &  67 &  67 ans 7 mois &  -2.92\% &  2622.81 &  {\bf 51.13} &  3085.66 &  {\bf {\color{red} 0.85}} &  {\bf {\color{red} 0.82}} &  {\bf {\color{red} 0.77}} &  {\bf {\color{red} 0.72}} &  {\bf {\color{red} 0.67}} &  {\bf {\color{red} 0.63}} \\ 
\hline 
\hline 
\end{tabular} 
\end{center} } 

 \begin{center}\includegraphics[width=0.9\textwidth]{fig/AdjTech_2003_22_dest_retraite.pdf}\end{center} \label{fig/AdjTech_2003_22_dest_retraite.pdf} 

\newpage 
 
\chapter{Rédacteur territorial (C2 puis C1)} 

\begin{minipage}{0.55\linewidth}\includegraphics[width=0.7\textwidth]{fig/grille_Redacteur.pdf}\end{minipage} 
\begin{minipage}{0.3\linewidth} 
 \begin{center} 

\begin{tabular}[htb]{|c|c|} 
\hline 
 Indice majoré &  Durée (années) \\ 
\hline \hline 
 356 &  2.00 \\ 
\hline 
 362 &  2.00 \\ 
\hline 
 369 &  2.00 \\ 
\hline 
 379 &  2.00 \\ 
\hline 
 390 &  2.00 \\ 
\hline 
 401 &  2.00 \\ 
\hline 
 416 &  2.00 \\ 
\hline 
 436 &  3.00 \\ 
\hline 
 452 &  3.00 \\ 
\hline 
 461 &  3.00 \\ 
\hline 
 480 &  3.00 \\ 
\hline 
 504 &  4.00 \\ 
\hline 
 534 &  3.00 \\ 
\hline 
 551 &  3.00 \\ 
\hline 
 569 &  3.00 \\ 
\hline 
 587 &   \\ 
\hline 
\hline 
\end{tabular} 
\end{center} 
 \end{minipage} 


 \addto{\captionsenglish}{ \renewcommand{\mtctitle}{}} \setcounter{minitocdepth}{2} 
 \minitoc \newpage 

\section{Début de carrière à 22 ans} 

\subsection{Génération 1975 (début en 1997)} 

\paragraph{Retraites possibles et ratios Revenu/SMIC à 70, 75, 80, 85, 90 ans avec le modèle \emph{Gouvernement truqué (âge-pivot bloqué à 65 ans)}}  
 
{ \scriptsize \begin{center} 
\begin{tabular}[htb]{|c|c||c|c||c|c||c||c|c|c|c|c|c|} 
\hline 
 Retraite en &  Âge &  Âge pivot &  Décote/Surcote &  Retraite (\euro{} 2019) &  Tx Rempl(\%) &  SMIC (\euro{} 2019) &  Retraite/SMIC &  Rev70/SMIC &  Rev75/SMIC &  Rev80/SMIC &  Rev85/SMIC &  Rev90/SMIC \\ 
\hline \hline 
 2037 &  62 &  64 ans 10 mois &  -14.17\% &  1430.63 &  {\bf 39.90} &  2143.00 &  {\bf {\color{red} 0.67}} &  {\bf {\color{red} 0.60}} &  {\bf {\color{red} 0.56}} &  {\bf {\color{red} 0.53}} &  {\bf {\color{red} 0.50}} &  {\bf {\color{red} 0.46}} \\ 
\hline 
 2038 &  63 &  64 ans 11 mois &  -9.58\% &  1563.72 &  {\bf 43.53} &  2170.86 &  {\bf {\color{red} 0.72}} &  {\bf {\color{red} 0.66}} &  {\bf {\color{red} 0.62}} &  {\bf {\color{red} 0.58}} &  {\bf {\color{red} 0.54}} &  {\bf {\color{red} 0.51}} \\ 
\hline 
 2039 &  64 &  65 ans 0 mois &  -5.00\% &  1704.37 &  {\bf 47.36} &  2199.08 &  {\bf {\color{red} 0.78}} &  {\bf {\color{red} 0.72}} &  {\bf {\color{red} 0.67}} &  {\bf {\color{red} 0.63}} &  {\bf {\color{red} 0.59}} &  {\bf {\color{red} 0.55}} \\ 
\hline 
 2040 &  65 &  65 ans 0 mois &  0.00\% &  1893.52 &  {\bf 52.52} &  2227.67 &  {\bf {\color{red} 0.85}} &  {\bf {\color{red} 0.80}} &  {\bf {\color{red} 0.75}} &  {\bf {\color{red} 0.70}} &  {\bf {\color{red} 0.66}} &  {\bf {\color{red} 0.62}} \\ 
\hline 
 2041 &  66 &  65 ans 0 mois &  5.00\% &  2025.90 &  {\bf 56.09} &  2256.63 &  {\bf {\color{red} 0.90}} &  {\bf {\color{red} 0.85}} &  {\bf {\color{red} 0.80}} &  {\bf {\color{red} 0.75}} &  {\bf {\color{red} 0.70}} &  {\bf {\color{red} 0.66}} \\ 
\hline 
 2042 &  67 &  65 ans 0 mois &  10.00\% &  2200.46 &  {\bf 60.82} &  2285.97 &  {\bf {\color{red} 0.96}} &  {\bf {\color{red} 0.93}} &  {\bf {\color{red} 0.87}} &  {\bf {\color{red} 0.81}} &  {\bf {\color{red} 0.76}} &  {\bf {\color{red} 0.72}} \\ 
\hline 
\hline 
\end{tabular} 
\end{center} } 
\paragraph{Retraites possibles et ratios Revenu/SMIC à 70, 75, 80, 85, 90 ans avec le modèle \emph{Gouvernement corrigé (âge-pivot glissant)}}  
 
{ \scriptsize \begin{center} 
\begin{tabular}[htb]{|c|c||c|c||c|c||c||c|c|c|c|c|c|} 
\hline 
 Retraite en &  Âge &  Âge pivot &  Décote/Surcote &  Retraite (\euro{} 2019) &  Tx Rempl(\%) &  SMIC (\euro{} 2019) &  Retraite/SMIC &  Rev70/SMIC &  Rev75/SMIC &  Rev80/SMIC &  Rev85/SMIC &  Rev90/SMIC \\ 
\hline \hline 
 2037 &  62 &  64 ans 10 mois &  -14.17\% &  1430.63 &  {\bf 39.90} &  2143.00 &  {\bf {\color{red} 0.67}} &  {\bf {\color{red} 0.60}} &  {\bf {\color{red} 0.56}} &  {\bf {\color{red} 0.53}} &  {\bf {\color{red} 0.50}} &  {\bf {\color{red} 0.46}} \\ 
\hline 
 2038 &  63 &  64 ans 11 mois &  -9.58\% &  1563.72 &  {\bf 43.53} &  2170.86 &  {\bf {\color{red} 0.72}} &  {\bf {\color{red} 0.66}} &  {\bf {\color{red} 0.62}} &  {\bf {\color{red} 0.58}} &  {\bf {\color{red} 0.54}} &  {\bf {\color{red} 0.51}} \\ 
\hline 
 2039 &  64 &  65 ans 0 mois &  -5.00\% &  1704.37 &  {\bf 47.36} &  2199.08 &  {\bf {\color{red} 0.78}} &  {\bf {\color{red} 0.72}} &  {\bf {\color{red} 0.67}} &  {\bf {\color{red} 0.63}} &  {\bf {\color{red} 0.59}} &  {\bf {\color{red} 0.55}} \\ 
\hline 
 2040 &  65 &  65 ans 1 mois &  -0.42\% &  1893.52 &  {\bf 52.52} &  2227.67 &  {\bf {\color{red} 0.85}} &  {\bf {\color{red} 0.80}} &  {\bf {\color{red} 0.75}} &  {\bf {\color{red} 0.70}} &  {\bf {\color{red} 0.66}} &  {\bf {\color{red} 0.62}} \\ 
\hline 
 2041 &  66 &  65 ans 2 mois &  4.17\% &  2009.82 &  {\bf 55.65} &  2256.63 &  {\bf {\color{red} 0.89}} &  {\bf {\color{red} 0.85}} &  {\bf {\color{red} 0.79}} &  {\bf {\color{red} 0.74}} &  {\bf {\color{red} 0.70}} &  {\bf {\color{red} 0.65}} \\ 
\hline 
 2042 &  67 &  65 ans 3 mois &  8.75\% &  2175.46 &  {\bf 60.13} &  2285.97 &  {\bf {\color{red} 0.95}} &  {\bf {\color{red} 0.92}} &  {\bf {\color{red} 0.86}} &  {\bf {\color{red} 0.80}} &  {\bf {\color{red} 0.75}} &  {\bf {\color{red} 0.71}} \\ 
\hline 
\hline 
\end{tabular} 
\end{center} } 
\paragraph{Retraites possibles et ratios Revenu/SMIC à 70, 75, 80, 85, 90 ans avec le modèle \emph{Destinie2 (revalorisation de la fonction publique)}}  
 
{ \scriptsize \begin{center} 
\begin{tabular}[htb]{|c|c||c|c||c|c||c||c|c|c|c|c|c|} 
\hline 
 Retraite en &  Âge &  Âge pivot &  Décote/Surcote &  Retraite (\euro{} 2019) &  Tx Rempl(\%) &  SMIC (\euro{} 2019) &  Retraite/SMIC &  Rev70/SMIC &  Rev75/SMIC &  Rev80/SMIC &  Rev85/SMIC &  Rev90/SMIC \\ 
\hline \hline 
 2037 &  62 &  64 ans 10 mois &  -14.17\% &  1577.19 &  {\bf 35.28} &  2014.82 &  {\bf {\color{red} 0.78}} &  {\bf {\color{red} 0.71}} &  {\bf {\color{red} 0.66}} &  {\bf {\color{red} 0.62}} &  {\bf {\color{red} 0.58}} &  {\bf {\color{red} 0.55}} \\ 
\hline 
 2038 &  63 &  64 ans 11 mois &  -9.58\% &  1731.07 &  {\bf 38.23} &  2041.01 &  {\bf {\color{red} 0.85}} &  {\bf {\color{red} 0.77}} &  {\bf {\color{red} 0.73}} &  {\bf {\color{red} 0.68}} &  {\bf {\color{red} 0.64}} &  {\bf {\color{red} 0.60}} \\ 
\hline 
 2039 &  64 &  65 ans 0 mois &  -5.00\% &  1894.75 &  {\bf 41.30} &  2067.55 &  {\bf {\color{red} 0.92}} &  {\bf {\color{red} 0.85}} &  {\bf {\color{red} 0.80}} &  {\bf {\color{red} 0.75}} &  {\bf {\color{red} 0.70}} &  {\bf {\color{red} 0.66}} \\ 
\hline 
 2040 &  65 &  65 ans 1 mois &  -0.42\% &  2068.76 &  {\bf 44.52} &  2094.43 &  {\bf {\color{red} 0.99}} &  {\bf {\color{red} 0.93}} &  {\bf {\color{red} 0.87}} &  {\bf {\color{red} 0.81}} &  {\bf {\color{red} 0.76}} &  {\bf {\color{red} 0.72}} \\ 
\hline 
 2041 &  66 &  65 ans 2 mois &  4.17\% &  2253.71 &  {\bf 47.88} &  2121.65 &  {\bf 1.06} &  {\bf 1.01} &  {\bf {\color{red} 0.95}} &  {\bf {\color{red} 0.89}} &  {\bf {\color{red} 0.83}} &  {\bf {\color{red} 0.78}} \\ 
\hline 
 2042 &  67 &  65 ans 3 mois &  8.75\% &  2450.22 &  {\bf 51.38} &  2149.23 &  {\bf 1.14} &  {\bf 1.10} &  {\bf 1.03} &  {\bf {\color{red} 0.96}} &  {\bf {\color{red} 0.90}} &  {\bf {\color{red} 0.85}} \\ 
\hline 
\hline 
\end{tabular} 
\end{center} } 

 \begin{center}\includegraphics[width=0.9\textwidth]{fig/Redacteur_1975_22_dest_retraite.pdf}\end{center} \label{fig/Redacteur_1975_22_dest_retraite.pdf} 

\newpage 
 
\subsection{Génération 1980 (début en 2002)} 

\paragraph{Retraites possibles et ratios Revenu/SMIC à 70, 75, 80, 85, 90 ans avec le modèle \emph{Gouvernement truqué (âge-pivot bloqué à 65 ans)}}  
 
{ \scriptsize \begin{center} 
\begin{tabular}[htb]{|c|c||c|c||c|c||c||c|c|c|c|c|c|} 
\hline 
 Retraite en &  Âge &  Âge pivot &  Décote/Surcote &  Retraite (\euro{} 2019) &  Tx Rempl(\%) &  SMIC (\euro{} 2019) &  Retraite/SMIC &  Rev70/SMIC &  Rev75/SMIC &  Rev80/SMIC &  Rev85/SMIC &  Rev90/SMIC \\ 
\hline \hline 
 2042 &  62 &  65 ans 0 mois &  -15.00\% &  1443.92 &  {\bf 40.27} &  2285.97 &  {\bf {\color{red} 0.63}} &  {\bf {\color{red} 0.57}} &  {\bf {\color{red} 0.53}} &  {\bf {\color{red} 0.50}} &  {\bf {\color{red} 0.47}} &  {\bf {\color{red} 0.44}} \\ 
\hline 
 2043 &  63 &  65 ans 0 mois &  -10.00\% &  1591.38 &  {\bf 44.30} &  2315.68 &  {\bf {\color{red} 0.69}} &  {\bf {\color{red} 0.63}} &  {\bf {\color{red} 0.59}} &  {\bf {\color{red} 0.55}} &  {\bf {\color{red} 0.52}} &  {\bf {\color{red} 0.48}} \\ 
\hline 
 2044 &  64 &  65 ans 0 mois &  -5.00\% &  1747.92 &  {\bf 48.57} &  2345.79 &  {\bf {\color{red} 0.75}} &  {\bf {\color{red} 0.69}} &  {\bf {\color{red} 0.65}} &  {\bf {\color{red} 0.61}} &  {\bf {\color{red} 0.57}} &  {\bf {\color{red} 0.53}} \\ 
\hline 
 2045 &  65 &  65 ans 0 mois &  0.00\% &  2019.84 &  {\bf 56.03} &  2376.28 &  {\bf {\color{red} 0.85}} &  {\bf {\color{red} 0.80}} &  {\bf {\color{red} 0.75}} &  {\bf {\color{red} 0.70}} &  {\bf {\color{red} 0.66}} &  {\bf {\color{red} 0.62}} \\ 
\hline 
 2046 &  66 &  65 ans 0 mois &  5.00\% &  2088.63 &  {\bf 57.83} &  2407.18 &  {\bf {\color{red} 0.87}} &  {\bf {\color{red} 0.82}} &  {\bf {\color{red} 0.77}} &  {\bf {\color{red} 0.72}} &  {\bf {\color{red} 0.68}} &  {\bf {\color{red} 0.64}} \\ 
\hline 
 2047 &  67 &  65 ans 0 mois &  10.00\% &  2271.93 &  {\bf 62.79} &  2438.47 &  {\bf {\color{red} 0.93}} &  {\bf {\color{red} 0.90}} &  {\bf {\color{red} 0.84}} &  {\bf {\color{red} 0.79}} &  {\bf {\color{red} 0.74}} &  {\bf {\color{red} 0.69}} \\ 
\hline 
\hline 
\end{tabular} 
\end{center} } 
\paragraph{Retraites possibles et ratios Revenu/SMIC à 70, 75, 80, 85, 90 ans avec le modèle \emph{Gouvernement corrigé (âge-pivot glissant)}}  
 
{ \scriptsize \begin{center} 
\begin{tabular}[htb]{|c|c||c|c||c|c||c||c|c|c|c|c|c|} 
\hline 
 Retraite en &  Âge &  Âge pivot &  Décote/Surcote &  Retraite (\euro{} 2019) &  Tx Rempl(\%) &  SMIC (\euro{} 2019) &  Retraite/SMIC &  Rev70/SMIC &  Rev75/SMIC &  Rev80/SMIC &  Rev85/SMIC &  Rev90/SMIC \\ 
\hline \hline 
 2042 &  62 &  65 ans 3 mois &  -16.25\% &  1422.69 &  {\bf 39.68} &  2285.97 &  {\bf {\color{red} 0.62}} &  {\bf {\color{red} 0.56}} &  {\bf {\color{red} 0.53}} &  {\bf {\color{red} 0.49}} &  {\bf {\color{red} 0.46}} &  {\bf {\color{red} 0.43}} \\ 
\hline 
 2043 &  63 &  65 ans 4 mois &  -11.67\% &  1561.91 &  {\bf 43.48} &  2315.68 &  {\bf {\color{red} 0.67}} &  {\bf {\color{red} 0.62}} &  {\bf {\color{red} 0.58}} &  {\bf {\color{red} 0.54}} &  {\bf {\color{red} 0.51}} &  {\bf {\color{red} 0.48}} \\ 
\hline 
 2044 &  64 &  65 ans 5 mois &  -7.08\% &  1709.59 &  {\bf 47.51} &  2345.79 &  {\bf {\color{red} 0.73}} &  {\bf {\color{red} 0.67}} &  {\bf {\color{red} 0.63}} &  {\bf {\color{red} 0.59}} &  {\bf {\color{red} 0.56}} &  {\bf {\color{red} 0.52}} \\ 
\hline 
 2045 &  65 &  65 ans 6 mois &  -2.50\% &  2019.84 &  {\bf 56.03} &  2376.28 &  {\bf {\color{red} 0.85}} &  {\bf {\color{red} 0.80}} &  {\bf {\color{red} 0.75}} &  {\bf {\color{red} 0.70}} &  {\bf {\color{red} 0.66}} &  {\bf {\color{red} 0.62}} \\ 
\hline 
 2046 &  66 &  65 ans 7 mois &  2.08\% &  2046.10 &  {\bf 56.65} &  2407.18 &  {\bf {\color{red} 0.85}} &  {\bf {\color{red} 0.81}} &  {\bf {\color{red} 0.76}} &  {\bf {\color{red} 0.71}} &  {\bf {\color{red} 0.67}} &  {\bf {\color{red} 0.62}} \\ 
\hline 
 2047 &  67 &  65 ans 8 mois &  6.67\% &  2203.09 &  {\bf 60.89} &  2438.47 &  {\bf {\color{red} 0.90}} &  {\bf {\color{red} 0.87}} &  {\bf {\color{red} 0.81}} &  {\bf {\color{red} 0.76}} &  {\bf {\color{red} 0.72}} &  {\bf {\color{red} 0.67}} \\ 
\hline 
\hline 
\end{tabular} 
\end{center} } 
\paragraph{Retraites possibles et ratios Revenu/SMIC à 70, 75, 80, 85, 90 ans avec le modèle \emph{Destinie2 (revalorisation de la fonction publique)}}  
 
{ \scriptsize \begin{center} 
\begin{tabular}[htb]{|c|c||c|c||c|c||c||c|c|c|c|c|c|} 
\hline 
 Retraite en &  Âge &  Âge pivot &  Décote/Surcote &  Retraite (\euro{} 2019) &  Tx Rempl(\%) &  SMIC (\euro{} 2019) &  Retraite/SMIC &  Rev70/SMIC &  Rev75/SMIC &  Rev80/SMIC &  Rev85/SMIC &  Rev90/SMIC \\ 
\hline \hline 
 2042 &  62 &  65 ans 3 mois &  -16.25\% &  1624.87 &  {\bf 34.07} &  2149.23 &  {\bf {\color{red} 0.76}} &  {\bf {\color{red} 0.68}} &  {\bf {\color{red} 0.64}} &  {\bf {\color{red} 0.60}} &  {\bf {\color{red} 0.56}} &  {\bf {\color{red} 0.53}} \\ 
\hline 
 2043 &  63 &  65 ans 4 mois &  -11.67\% &  1792.83 &  {\bf 37.11} &  2177.17 &  {\bf {\color{red} 0.82}} &  {\bf {\color{red} 0.75}} &  {\bf {\color{red} 0.71}} &  {\bf {\color{red} 0.66}} &  {\bf {\color{red} 0.62}} &  {\bf {\color{red} 0.58}} \\ 
\hline 
 2044 &  64 &  65 ans 5 mois &  -7.08\% &  1972.20 &  {\bf 40.30} &  2205.48 &  {\bf {\color{red} 0.89}} &  {\bf {\color{red} 0.83}} &  {\bf {\color{red} 0.78}} &  {\bf {\color{red} 0.73}} &  {\bf {\color{red} 0.68}} &  {\bf {\color{red} 0.64}} \\ 
\hline 
 2045 &  65 &  65 ans 6 mois &  -2.50\% &  2163.66 &  {\bf 43.65} &  2234.15 &  {\bf {\color{red} 0.97}} &  {\bf {\color{red} 0.91}} &  {\bf {\color{red} 0.85}} &  {\bf {\color{red} 0.80}} &  {\bf {\color{red} 0.75}} &  {\bf {\color{red} 0.70}} \\ 
\hline 
 2046 &  66 &  65 ans 7 mois &  2.08\% &  2366.17 &  {\bf 47.12} &  2263.19 &  {\bf 1.05} &  {\bf {\color{red} 0.99}} &  {\bf {\color{red} 0.93}} &  {\bf {\color{red} 0.87}} &  {\bf {\color{red} 0.82}} &  {\bf {\color{red} 0.77}} \\ 
\hline 
 2047 &  67 &  65 ans 8 mois &  6.67\% &  2580.07 &  {\bf 50.72} &  2292.61 &  {\bf 1.13} &  {\bf 1.08} &  {\bf 1.01} &  {\bf {\color{red} 0.95}} &  {\bf {\color{red} 0.89}} &  {\bf {\color{red} 0.84}} \\ 
\hline 
\hline 
\end{tabular} 
\end{center} } 

 \begin{center}\includegraphics[width=0.9\textwidth]{fig/Redacteur_1980_22_dest_retraite.pdf}\end{center} \label{fig/Redacteur_1980_22_dest_retraite.pdf} 

\newpage 
 
\subsection{Génération 1990 (début en 2012)} 

\paragraph{Retraites possibles et ratios Revenu/SMIC à 70, 75, 80, 85, 90 ans avec le modèle \emph{Gouvernement truqué (âge-pivot bloqué à 65 ans)}}  
 
{ \scriptsize \begin{center} 
\begin{tabular}[htb]{|c|c||c|c||c|c||c||c|c|c|c|c|c|} 
\hline 
 Retraite en &  Âge &  Âge pivot &  Décote/Surcote &  Retraite (\euro{} 2019) &  Tx Rempl(\%) &  SMIC (\euro{} 2019) &  Retraite/SMIC &  Rev70/SMIC &  Rev75/SMIC &  Rev80/SMIC &  Rev85/SMIC &  Rev90/SMIC \\ 
\hline \hline 
 2052 &  62 &  65 ans 0 mois &  -15.00\% &  1546.31 &  {\bf 43.12} &  2601.14 &  {\bf {\color{red} 0.59}} &  {\bf {\color{red} 0.54}} &  {\bf {\color{red} 0.50}} &  {\bf {\color{red} 0.47}} &  {\bf {\color{red} 0.44}} &  {\bf {\color{red} 0.41}} \\ 
\hline 
 2053 &  63 &  65 ans 0 mois &  -10.00\% &  1703.55 &  {\bf 47.42} &  2634.96 &  {\bf {\color{red} 0.65}} &  {\bf {\color{red} 0.59}} &  {\bf {\color{red} 0.55}} &  {\bf {\color{red} 0.52}} &  {\bf {\color{red} 0.49}} &  {\bf {\color{red} 0.46}} \\ 
\hline 
 2054 &  64 &  65 ans 0 mois &  -5.00\% &  1869.16 &  {\bf 51.94} &  2669.21 &  {\bf {\color{red} 0.70}} &  {\bf {\color{red} 0.65}} &  {\bf {\color{red} 0.61}} &  {\bf {\color{red} 0.57}} &  {\bf {\color{red} 0.53}} &  {\bf {\color{red} 0.50}} \\ 
\hline 
 2055 &  65 &  65 ans 0 mois &  0.00\% &  2298.33 &  {\bf 63.75} &  2703.91 &  {\bf {\color{red} 0.85}} &  {\bf {\color{red} 0.80}} &  {\bf {\color{red} 0.75}} &  {\bf {\color{red} 0.70}} &  {\bf {\color{red} 0.66}} &  {\bf {\color{red} 0.62}} \\ 
\hline 
 2056 &  66 &  65 ans 0 mois &  5.00\% &  2328.20 &  {\bf 64.46} &  2739.06 &  {\bf {\color{red} 0.85}} &  {\bf {\color{red} 0.81}} &  {\bf {\color{red} 0.76}} &  {\bf {\color{red} 0.71}} &  {\bf {\color{red} 0.67}} &  {\bf {\color{red} 0.62}} \\ 
\hline 
 2057 &  67 &  65 ans 0 mois &  10.00\% &  2417.86 &  {\bf 66.83} &  2774.67 &  {\bf {\color{red} 0.87}} &  {\bf {\color{red} 0.84}} &  {\bf {\color{red} 0.79}} &  {\bf {\color{red} 0.74}} &  {\bf {\color{red} 0.69}} &  {\bf {\color{red} 0.65}} \\ 
\hline 
\hline 
\end{tabular} 
\end{center} } 
\paragraph{Retraites possibles et ratios Revenu/SMIC à 70, 75, 80, 85, 90 ans avec le modèle \emph{Gouvernement corrigé (âge-pivot glissant)}}  
 
{ \scriptsize \begin{center} 
\begin{tabular}[htb]{|c|c||c|c||c|c||c||c|c|c|c|c|c|} 
\hline 
 Retraite en &  Âge &  Âge pivot &  Décote/Surcote &  Retraite (\euro{} 2019) &  Tx Rempl(\%) &  SMIC (\euro{} 2019) &  Retraite/SMIC &  Rev70/SMIC &  Rev75/SMIC &  Rev80/SMIC &  Rev85/SMIC &  Rev90/SMIC \\ 
\hline \hline 
 2052 &  62 &  66 ans 1 mois &  -20.42\% &  1447.77 &  {\bf 40.38} &  2601.14 &  {\bf {\color{red} 0.56}} &  {\bf {\color{red} 0.50}} &  {\bf {\color{red} 0.47}} &  {\bf {\color{red} 0.44}} &  {\bf {\color{red} 0.41}} &  {\bf {\color{red} 0.39}} \\ 
\hline 
 2053 &  63 &  66 ans 2 mois &  -15.83\% &  1593.14 &  {\bf 44.35} &  2634.96 &  {\bf {\color{red} 0.60}} &  {\bf {\color{red} 0.55}} &  {\bf {\color{red} 0.52}} &  {\bf {\color{red} 0.49}} &  {\bf {\color{red} 0.46}} &  {\bf {\color{red} 0.43}} \\ 
\hline 
 2054 &  64 &  66 ans 3 mois &  -11.25\% &  1746.19 &  {\bf 48.52} &  2669.21 &  {\bf {\color{red} 0.65}} &  {\bf {\color{red} 0.61}} &  {\bf {\color{red} 0.57}} &  {\bf {\color{red} 0.53}} &  {\bf {\color{red} 0.50}} &  {\bf {\color{red} 0.47}} \\ 
\hline 
 2055 &  65 &  66 ans 4 mois &  -6.67\% &  2298.33 &  {\bf 63.75} &  2703.91 &  {\bf {\color{red} 0.85}} &  {\bf {\color{red} 0.80}} &  {\bf {\color{red} 0.75}} &  {\bf {\color{red} 0.70}} &  {\bf {\color{red} 0.66}} &  {\bf {\color{red} 0.62}} \\ 
\hline 
 2056 &  66 &  66 ans 5 mois &  -2.08\% &  2328.20 &  {\bf 64.46} &  2739.06 &  {\bf {\color{red} 0.85}} &  {\bf {\color{red} 0.81}} &  {\bf {\color{red} 0.76}} &  {\bf {\color{red} 0.71}} &  {\bf {\color{red} 0.67}} &  {\bf {\color{red} 0.62}} \\ 
\hline 
 2057 &  67 &  66 ans 6 mois &  2.50\% &  2358.47 &  {\bf 65.19} &  2774.67 &  {\bf {\color{red} 0.85}} &  {\bf {\color{red} 0.82}} &  {\bf {\color{red} 0.77}} &  {\bf {\color{red} 0.72}} &  {\bf {\color{red} 0.67}} &  {\bf {\color{red} 0.63}} \\ 
\hline 
\hline 
\end{tabular} 
\end{center} } 
\paragraph{Retraites possibles et ratios Revenu/SMIC à 70, 75, 80, 85, 90 ans avec le modèle \emph{Destinie2 (revalorisation de la fonction publique)}}  
 
{ \scriptsize \begin{center} 
\begin{tabular}[htb]{|c|c||c|c||c|c||c||c|c|c|c|c|c|} 
\hline 
 Retraite en &  Âge &  Âge pivot &  Décote/Surcote &  Retraite (\euro{} 2019) &  Tx Rempl(\%) &  SMIC (\euro{} 2019) &  Retraite/SMIC &  Rev70/SMIC &  Rev75/SMIC &  Rev80/SMIC &  Rev85/SMIC &  Rev90/SMIC \\ 
\hline \hline 
 2052 &  62 &  66 ans 1 mois &  -20.42\% &  1813.85 &  {\bf 33.43} &  2445.56 &  {\bf {\color{red} 0.74}} &  {\bf {\color{red} 0.67}} &  {\bf {\color{red} 0.63}} &  {\bf {\color{red} 0.59}} &  {\bf {\color{red} 0.55}} &  {\bf {\color{red} 0.52}} \\ 
\hline 
 2053 &  63 &  66 ans 2 mois &  -15.83\% &  2007.64 &  {\bf 36.53} &  2477.35 &  {\bf {\color{red} 0.81}} &  {\bf {\color{red} 0.74}} &  {\bf {\color{red} 0.69}} &  {\bf {\color{red} 0.65}} &  {\bf {\color{red} 0.61}} &  {\bf {\color{red} 0.57}} \\ 
\hline 
 2054 &  64 &  66 ans 3 mois &  -11.25\% &  2213.27 &  {\bf 39.75} &  2509.56 &  {\bf {\color{red} 0.88}} &  {\bf {\color{red} 0.82}} &  {\bf {\color{red} 0.77}} &  {\bf {\color{red} 0.72}} &  {\bf {\color{red} 0.67}} &  {\bf {\color{red} 0.63}} \\ 
\hline 
 2055 &  65 &  66 ans 4 mois &  -6.67\% &  2431.11 &  {\bf 43.10} &  2542.18 &  {\bf {\color{red} 0.96}} &  {\bf {\color{red} 0.90}} &  {\bf {\color{red} 0.84}} &  {\bf {\color{red} 0.79}} &  {\bf {\color{red} 0.74}} &  {\bf {\color{red} 0.69}} \\ 
\hline 
 2056 &  66 &  66 ans 5 mois &  -2.08\% &  2661.53 &  {\bf 46.58} &  2575.23 &  {\bf 1.03} &  {\bf {\color{red} 0.98}} &  {\bf {\color{red} 0.92}} &  {\bf {\color{red} 0.86}} &  {\bf {\color{red} 0.81}} &  {\bf {\color{red} 0.76}} \\ 
\hline 
 2057 &  67 &  66 ans 6 mois &  2.50\% &  2904.91 &  {\bf 50.19} &  2608.71 &  {\bf 1.11} &  {\bf 1.07} &  {\bf 1.00} &  {\bf {\color{red} 0.94}} &  {\bf {\color{red} 0.88}} &  {\bf {\color{red} 0.83}} \\ 
\hline 
\hline 
\end{tabular} 
\end{center} } 

 \begin{center}\includegraphics[width=0.9\textwidth]{fig/Redacteur_1990_22_dest_retraite.pdf}\end{center} \label{fig/Redacteur_1990_22_dest_retraite.pdf} 

\newpage 
 
\subsection{Génération 2003 (début en 2025)} 

\paragraph{Retraites possibles et ratios Revenu/SMIC à 70, 75, 80, 85, 90 ans avec le modèle \emph{Gouvernement truqué (âge-pivot bloqué à 65 ans)}}  
 
{ \scriptsize \begin{center} 
\begin{tabular}[htb]{|c|c||c|c||c|c||c||c|c|c|c|c|c|} 
\hline 
 Retraite en &  Âge &  Âge pivot &  Décote/Surcote &  Retraite (\euro{} 2019) &  Tx Rempl(\%) &  SMIC (\euro{} 2019) &  Retraite/SMIC &  Rev70/SMIC &  Rev75/SMIC &  Rev80/SMIC &  Rev85/SMIC &  Rev90/SMIC \\ 
\hline \hline 
 2065 &  62 &  65 ans 0 mois &  -15.00\% &  1653.71 &  {\bf 46.12} &  3076.71 &  {\bf {\color{red} 0.54}} &  {\bf {\color{red} 0.48}} &  {\bf {\color{red} 0.45}} &  {\bf {\color{red} 0.43}} &  {\bf {\color{red} 0.40}} &  {\bf {\color{red} 0.37}} \\ 
\hline 
 2066 &  63 &  65 ans 0 mois &  -10.00\% &  1818.75 &  {\bf 50.63} &  3116.71 &  {\bf {\color{red} 0.58}} &  {\bf {\color{red} 0.53}} &  {\bf {\color{red} 0.50}} &  {\bf {\color{red} 0.47}} &  {\bf {\color{red} 0.44}} &  {\bf {\color{red} 0.41}} \\ 
\hline 
 2067 &  64 &  65 ans 0 mois &  -5.00\% &  1992.34 &  {\bf 55.36} &  3157.23 &  {\bf {\color{red} 0.63}} &  {\bf {\color{red} 0.58}} &  {\bf {\color{red} 0.55}} &  {\bf {\color{red} 0.51}} &  {\bf {\color{red} 0.48}} &  {\bf {\color{red} 0.45}} \\ 
\hline 
 2068 &  65 &  65 ans 0 mois &  0.00\% &  2718.53 &  {\bf 75.41} &  3198.27 &  {\bf {\color{red} 0.85}} &  {\bf {\color{red} 0.80}} &  {\bf {\color{red} 0.75}} &  {\bf {\color{red} 0.70}} &  {\bf {\color{red} 0.66}} &  {\bf {\color{red} 0.62}} \\ 
\hline 
 2069 &  66 &  65 ans 0 mois &  5.00\% &  2753.87 &  {\bf 76.25} &  3239.85 &  {\bf {\color{red} 0.85}} &  {\bf {\color{red} 0.81}} &  {\bf {\color{red} 0.76}} &  {\bf {\color{red} 0.71}} &  {\bf {\color{red} 0.67}} &  {\bf {\color{red} 0.62}} \\ 
\hline 
 2070 &  67 &  65 ans 0 mois &  10.00\% &  2789.67 &  {\bf 77.10} &  3281.97 &  {\bf {\color{red} 0.85}} &  {\bf {\color{red} 0.82}} &  {\bf {\color{red} 0.77}} &  {\bf {\color{red} 0.72}} &  {\bf {\color{red} 0.67}} &  {\bf {\color{red} 0.63}} \\ 
\hline 
\hline 
\end{tabular} 
\end{center} } 
\paragraph{Retraites possibles et ratios Revenu/SMIC à 70, 75, 80, 85, 90 ans avec le modèle \emph{Gouvernement corrigé (âge-pivot glissant)}}  
 
{ \scriptsize \begin{center} 
\begin{tabular}[htb]{|c|c||c|c||c|c||c||c|c|c|c|c|c|} 
\hline 
 Retraite en &  Âge &  Âge pivot &  Décote/Surcote &  Retraite (\euro{} 2019) &  Tx Rempl(\%) &  SMIC (\euro{} 2019) &  Retraite/SMIC &  Rev70/SMIC &  Rev75/SMIC &  Rev80/SMIC &  Rev85/SMIC &  Rev90/SMIC \\ 
\hline \hline 
 2065 &  62 &  67 ans 2 mois &  -25.83\% &  1442.94 &  {\bf 40.24} &  3076.71 &  {\bf {\color{red} 0.47}} &  {\bf {\color{red} 0.42}} &  {\bf {\color{red} 0.40}} &  {\bf {\color{red} 0.37}} &  {\bf {\color{red} 0.35}} &  {\bf {\color{red} 0.33}} \\ 
\hline 
 2066 &  63 &  67 ans 3 mois &  -21.25\% &  1591.41 &  {\bf 44.30} &  3116.71 &  {\bf {\color{red} 0.51}} &  {\bf {\color{red} 0.47}} &  {\bf {\color{red} 0.44}} &  {\bf {\color{red} 0.41}} &  {\bf {\color{red} 0.38}} &  {\bf {\color{red} 0.36}} \\ 
\hline 
 2067 &  64 &  67 ans 4 mois &  -16.67\% &  1747.66 &  {\bf 48.56} &  3157.23 &  {\bf {\color{red} 0.55}} &  {\bf {\color{red} 0.51}} &  {\bf {\color{red} 0.48}} &  {\bf {\color{red} 0.45}} &  {\bf {\color{red} 0.42}} &  {\bf {\color{red} 0.40}} \\ 
\hline 
 2068 &  65 &  67 ans 5 mois &  -12.08\% &  2718.53 &  {\bf 75.41} &  3198.27 &  {\bf {\color{red} 0.85}} &  {\bf {\color{red} 0.80}} &  {\bf {\color{red} 0.75}} &  {\bf {\color{red} 0.70}} &  {\bf {\color{red} 0.66}} &  {\bf {\color{red} 0.62}} \\ 
\hline 
 2069 &  66 &  67 ans 6 mois &  -7.50\% &  2753.87 &  {\bf 76.25} &  3239.85 &  {\bf {\color{red} 0.85}} &  {\bf {\color{red} 0.81}} &  {\bf {\color{red} 0.76}} &  {\bf {\color{red} 0.71}} &  {\bf {\color{red} 0.67}} &  {\bf {\color{red} 0.62}} \\ 
\hline 
 2070 &  67 &  67 ans 7 mois &  -2.92\% &  2789.67 &  {\bf 77.10} &  3281.97 &  {\bf {\color{red} 0.85}} &  {\bf {\color{red} 0.82}} &  {\bf {\color{red} 0.77}} &  {\bf {\color{red} 0.72}} &  {\bf {\color{red} 0.67}} &  {\bf {\color{red} 0.63}} \\ 
\hline 
\hline 
\end{tabular} 
\end{center} } 
\paragraph{Retraites possibles et ratios Revenu/SMIC à 70, 75, 80, 85, 90 ans avec le modèle \emph{Destinie2 (revalorisation de la fonction publique)}}  
 
{ \scriptsize \begin{center} 
\begin{tabular}[htb]{|c|c||c|c||c|c||c||c|c|c|c|c|c|} 
\hline 
 Retraite en &  Âge &  Âge pivot &  Décote/Surcote &  Retraite (\euro{} 2019) &  Tx Rempl(\%) &  SMIC (\euro{} 2019) &  Retraite/SMIC &  Rev70/SMIC &  Rev75/SMIC &  Rev80/SMIC &  Rev85/SMIC &  Rev90/SMIC \\ 
\hline \hline 
 2065 &  62 &  67 ans 2 mois &  -25.83\% &  2102.63 &  {\bf 32.76} &  2892.68 &  {\bf {\color{red} 0.73}} &  {\bf {\color{red} 0.66}} &  {\bf {\color{red} 0.61}} &  {\bf {\color{red} 0.58}} &  {\bf {\color{red} 0.54}} &  {\bf {\color{red} 0.51}} \\ 
\hline 
 2066 &  63 &  67 ans 3 mois &  -21.25\% &  2332.85 &  {\bf 35.88} &  2930.29 &  {\bf {\color{red} 0.80}} &  {\bf {\color{red} 0.73}} &  {\bf {\color{red} 0.68}} &  {\bf {\color{red} 0.64}} &  {\bf {\color{red} 0.60}} &  {\bf {\color{red} 0.56}} \\ 
\hline 
 2067 &  64 &  67 ans 4 mois &  -16.67\% &  2577.11 &  {\bf 39.13} &  2968.38 &  {\bf {\color{red} 0.87}} &  {\bf {\color{red} 0.80}} &  {\bf {\color{red} 0.75}} &  {\bf {\color{red} 0.71}} &  {\bf {\color{red} 0.66}} &  {\bf {\color{red} 0.62}} \\ 
\hline 
 2068 &  65 &  67 ans 5 mois &  -12.08\% &  2835.84 &  {\bf 42.51} &  3006.97 &  {\bf {\color{red} 0.94}} &  {\bf {\color{red} 0.88}} &  {\bf {\color{red} 0.83}} &  {\bf {\color{red} 0.78}} &  {\bf {\color{red} 0.73}} &  {\bf {\color{red} 0.68}} \\ 
\hline 
 2069 &  66 &  67 ans 6 mois &  -7.50\% &  3109.49 &  {\bf 46.01} &  3046.06 &  {\bf 1.02} &  {\bf {\color{red} 0.97}} &  {\bf {\color{red} 0.91}} &  {\bf {\color{red} 0.85}} &  {\bf {\color{red} 0.80}} &  {\bf {\color{red} 0.75}} \\ 
\hline 
 2070 &  67 &  67 ans 7 mois &  -2.92\% &  3398.50 &  {\bf 49.64} &  3085.66 &  {\bf 1.10} &  {\bf 1.06} &  {\bf {\color{red} 0.99}} &  {\bf {\color{red} 0.93}} &  {\bf {\color{red} 0.87}} &  {\bf {\color{red} 0.82}} \\ 
\hline 
\hline 
\end{tabular} 
\end{center} } 

 \begin{center}\includegraphics[width=0.9\textwidth]{fig/Redacteur_2003_22_dest_retraite.pdf}\end{center} \label{fig/Redacteur_2003_22_dest_retraite.pdf} 

\newpage 
 
\chapter{Secrétaire administratif} 

\begin{minipage}{0.55\linewidth}\includegraphics[width=0.7\textwidth]{fig/grille_SecretaireAdmin.pdf}\end{minipage} 
\begin{minipage}{0.3\linewidth} 
 \begin{center} 

\begin{tabular}[htb]{|c|c|} 
\hline 
 Indice majoré &  Durée (années) \\ 
\hline \hline 
 339 &  2.00 \\ 
\hline 
 344 &  2.00 \\ 
\hline 
 349 &  2.00 \\ 
\hline 
 356 &  2.00 \\ 
\hline 
 366 &  2.00 \\ 
\hline 
 379 &  2.00 \\ 
\hline 
 394 &  2.00 \\ 
\hline 
 413 &  3.00 \\ 
\hline 
 429 &  3.00 \\ 
\hline 
 440 &  3.00 \\ 
\hline 
 453 &  3.00 \\ 
\hline 
 477 &  4.00 \\ 
\hline 
 503 &   \\ 
\hline 
\hline 
\end{tabular} 
\end{center} 
 \end{minipage} 


 \addto{\captionsenglish}{ \renewcommand{\mtctitle}{}} \setcounter{minitocdepth}{2} 
 \minitoc \newpage 

\section{Début de carrière à 22 ans} 

\subsection{Génération 1975 (début en 1997)} 

\paragraph{Retraites possibles et ratios Revenu/SMIC à 70, 75, 80, 85, 90 ans avec le modèle \emph{Gouvernement truqué (âge-pivot bloqué à 65 ans)}}  
 
{ \scriptsize \begin{center} 
\begin{tabular}[htb]{|c|c||c|c||c|c||c||c|c|c|c|c|c|} 
\hline 
 Retraite en &  Âge &  Âge pivot &  Décote/Surcote &  Retraite (\euro{} 2019) &  Tx Rempl(\%) &  SMIC (\euro{} 2019) &  Retraite/SMIC &  Rev70/SMIC &  Rev75/SMIC &  Rev80/SMIC &  Rev85/SMIC &  Rev90/SMIC \\ 
\hline \hline 
 2037 &  62 &  64 ans 10 mois &  -14.17\% &  1465.93 &  {\bf 43.54} &  2143.00 &  {\bf {\color{red} 0.68}} &  {\bf {\color{red} 0.62}} &  {\bf {\color{red} 0.58}} &  {\bf {\color{red} 0.54}} &  {\bf {\color{red} 0.51}} &  {\bf {\color{red} 0.48}} \\ 
\hline 
 2038 &  63 &  64 ans 11 mois &  -9.58\% &  1598.43 &  {\bf 47.40} &  2170.86 &  {\bf {\color{red} 0.74}} &  {\bf {\color{red} 0.67}} &  {\bf {\color{red} 0.63}} &  {\bf {\color{red} 0.59}} &  {\bf {\color{red} 0.55}} &  {\bf {\color{red} 0.52}} \\ 
\hline 
 2039 &  64 &  65 ans 0 mois &  -5.00\% &  1738.22 &  {\bf 51.46} &  2199.08 &  {\bf {\color{red} 0.79}} &  {\bf {\color{red} 0.73}} &  {\bf {\color{red} 0.69}} &  {\bf {\color{red} 0.64}} &  {\bf {\color{red} 0.60}} &  {\bf {\color{red} 0.56}} \\ 
\hline 
 2040 &  65 &  65 ans 0 mois &  0.00\% &  1893.56 &  {\bf 55.97} &  2227.67 &  {\bf {\color{red} 0.85}} &  {\bf {\color{red} 0.80}} &  {\bf {\color{red} 0.75}} &  {\bf {\color{red} 0.70}} &  {\bf {\color{red} 0.66}} &  {\bf {\color{red} 0.62}} \\ 
\hline 
 2041 &  66 &  65 ans 0 mois &  5.00\% &  2057.51 &  {\bf 60.71} &  2256.63 &  {\bf {\color{red} 0.91}} &  {\bf {\color{red} 0.87}} &  {\bf {\color{red} 0.81}} &  {\bf {\color{red} 0.76}} &  {\bf {\color{red} 0.71}} &  {\bf {\color{red} 0.67}} \\ 
\hline 
 2042 &  67 &  65 ans 0 mois &  10.00\% &  2230.51 &  {\bf 65.71} &  2285.97 &  {\bf {\color{red} 0.98}} &  {\bf {\color{red} 0.94}} &  {\bf {\color{red} 0.88}} &  {\bf {\color{red} 0.82}} &  {\bf {\color{red} 0.77}} &  {\bf {\color{red} 0.72}} \\ 
\hline 
\hline 
\end{tabular} 
\end{center} } 
\paragraph{Retraites possibles et ratios Revenu/SMIC à 70, 75, 80, 85, 90 ans avec le modèle \emph{Gouvernement corrigé (âge-pivot glissant)}}  
 
{ \scriptsize \begin{center} 
\begin{tabular}[htb]{|c|c||c|c||c|c||c||c|c|c|c|c|c|} 
\hline 
 Retraite en &  Âge &  Âge pivot &  Décote/Surcote &  Retraite (\euro{} 2019) &  Tx Rempl(\%) &  SMIC (\euro{} 2019) &  Retraite/SMIC &  Rev70/SMIC &  Rev75/SMIC &  Rev80/SMIC &  Rev85/SMIC &  Rev90/SMIC \\ 
\hline \hline 
 2037 &  62 &  64 ans 10 mois &  -14.17\% &  1465.93 &  {\bf 43.54} &  2143.00 &  {\bf {\color{red} 0.68}} &  {\bf {\color{red} 0.62}} &  {\bf {\color{red} 0.58}} &  {\bf {\color{red} 0.54}} &  {\bf {\color{red} 0.51}} &  {\bf {\color{red} 0.48}} \\ 
\hline 
 2038 &  63 &  64 ans 11 mois &  -9.58\% &  1598.43 &  {\bf 47.40} &  2170.86 &  {\bf {\color{red} 0.74}} &  {\bf {\color{red} 0.67}} &  {\bf {\color{red} 0.63}} &  {\bf {\color{red} 0.59}} &  {\bf {\color{red} 0.55}} &  {\bf {\color{red} 0.52}} \\ 
\hline 
 2039 &  64 &  65 ans 0 mois &  -5.00\% &  1738.22 &  {\bf 51.46} &  2199.08 &  {\bf {\color{red} 0.79}} &  {\bf {\color{red} 0.73}} &  {\bf {\color{red} 0.69}} &  {\bf {\color{red} 0.64}} &  {\bf {\color{red} 0.60}} &  {\bf {\color{red} 0.56}} \\ 
\hline 
 2040 &  65 &  65 ans 1 mois &  -0.42\% &  1893.52 &  {\bf 55.96} &  2227.67 &  {\bf {\color{red} 0.85}} &  {\bf {\color{red} 0.80}} &  {\bf {\color{red} 0.75}} &  {\bf {\color{red} 0.70}} &  {\bf {\color{red} 0.66}} &  {\bf {\color{red} 0.62}} \\ 
\hline 
 2041 &  66 &  65 ans 2 mois &  4.17\% &  2041.18 &  {\bf 60.23} &  2256.63 &  {\bf {\color{red} 0.90}} &  {\bf {\color{red} 0.86}} &  {\bf {\color{red} 0.81}} &  {\bf {\color{red} 0.75}} &  {\bf {\color{red} 0.71}} &  {\bf {\color{red} 0.66}} \\ 
\hline 
 2042 &  67 &  65 ans 3 mois &  8.75\% &  2205.16 &  {\bf 64.96} &  2285.97 &  {\bf {\color{red} 0.96}} &  {\bf {\color{red} 0.93}} &  {\bf {\color{red} 0.87}} &  {\bf {\color{red} 0.82}} &  {\bf {\color{red} 0.76}} &  {\bf {\color{red} 0.72}} \\ 
\hline 
\hline 
\end{tabular} 
\end{center} } 
\paragraph{Retraites possibles et ratios Revenu/SMIC à 70, 75, 80, 85, 90 ans avec le modèle \emph{Destinie2 (revalorisation de la fonction publique)}}  
 
{ \scriptsize \begin{center} 
\begin{tabular}[htb]{|c|c||c|c||c|c||c||c|c|c|c|c|c|} 
\hline 
 Retraite en &  Âge &  Âge pivot &  Décote/Surcote &  Retraite (\euro{} 2019) &  Tx Rempl(\%) &  SMIC (\euro{} 2019) &  Retraite/SMIC &  Rev70/SMIC &  Rev75/SMIC &  Rev80/SMIC &  Rev85/SMIC &  Rev90/SMIC \\ 
\hline \hline 
 2037 &  62 &  64 ans 10 mois &  -14.17\% &  1608.00 &  {\bf 38.33} &  2014.82 &  {\bf {\color{red} 0.80}} &  {\bf {\color{red} 0.72}} &  {\bf {\color{red} 0.67}} &  {\bf {\color{red} 0.63}} &  {\bf {\color{red} 0.59}} &  {\bf {\color{red} 0.56}} \\ 
\hline 
 2038 &  63 &  64 ans 11 mois &  -9.58\% &  1760.27 &  {\bf 41.42} &  2041.01 &  {\bf {\color{red} 0.86}} &  {\bf {\color{red} 0.79}} &  {\bf {\color{red} 0.74}} &  {\bf {\color{red} 0.69}} &  {\bf {\color{red} 0.65}} &  {\bf {\color{red} 0.61}} \\ 
\hline 
 2039 &  64 &  65 ans 0 mois &  -5.00\% &  1921.95 &  {\bf 44.64} &  2067.55 &  {\bf {\color{red} 0.93}} &  {\bf {\color{red} 0.86}} &  {\bf {\color{red} 0.81}} &  {\bf {\color{red} 0.76}} &  {\bf {\color{red} 0.71}} &  {\bf {\color{red} 0.66}} \\ 
\hline 
 2040 &  65 &  65 ans 1 mois &  -0.42\% &  2093.58 &  {\bf 48.01} &  2094.43 &  {\bf {\color{red} 1.00}} &  {\bf {\color{red} 0.94}} &  {\bf {\color{red} 0.88}} &  {\bf {\color{red} 0.82}} &  {\bf {\color{red} 0.77}} &  {\bf {\color{red} 0.72}} \\ 
\hline 
 2041 &  66 &  65 ans 2 mois &  4.17\% &  2275.73 &  {\bf 51.51} &  2121.65 &  {\bf 1.07} &  {\bf 1.02} &  {\bf {\color{red} 0.95}} &  {\bf {\color{red} 0.90}} &  {\bf {\color{red} 0.84}} &  {\bf {\color{red} 0.79}} \\ 
\hline 
 2042 &  67 &  65 ans 3 mois &  8.75\% &  2469.02 &  {\bf 55.17} &  2149.23 &  {\bf 1.15} &  {\bf 1.11} &  {\bf 1.04} &  {\bf {\color{red} 0.97}} &  {\bf {\color{red} 0.91}} &  {\bf {\color{red} 0.85}} \\ 
\hline 
\hline 
\end{tabular} 
\end{center} } 

 \begin{center}\includegraphics[width=0.9\textwidth]{fig/SecretaireAdmin_1975_22_dest_retraite.pdf}\end{center} \label{fig/SecretaireAdmin_1975_22_dest_retraite.pdf} 

\newpage 
 
\subsection{Génération 1980 (début en 2002)} 

\paragraph{Retraites possibles et ratios Revenu/SMIC à 70, 75, 80, 85, 90 ans avec le modèle \emph{Gouvernement truqué (âge-pivot bloqué à 65 ans)}}  
 
{ \scriptsize \begin{center} 
\begin{tabular}[htb]{|c|c||c|c||c|c||c||c|c|c|c|c|c|} 
\hline 
 Retraite en &  Âge &  Âge pivot &  Décote/Surcote &  Retraite (\euro{} 2019) &  Tx Rempl(\%) &  SMIC (\euro{} 2019) &  Retraite/SMIC &  Rev70/SMIC &  Rev75/SMIC &  Rev80/SMIC &  Rev85/SMIC &  Rev90/SMIC \\ 
\hline \hline 
 2042 &  62 &  65 ans 0 mois &  -15.00\% &  1480.11 &  {\bf 43.96} &  2285.97 &  {\bf {\color{red} 0.65}} &  {\bf {\color{red} 0.58}} &  {\bf {\color{red} 0.55}} &  {\bf {\color{red} 0.51}} &  {\bf {\color{red} 0.48}} &  {\bf {\color{red} 0.45}} \\ 
\hline 
 2043 &  63 &  65 ans 0 mois &  -10.00\% &  1627.38 &  {\bf 48.26} &  2315.68 &  {\bf {\color{red} 0.70}} &  {\bf {\color{red} 0.64}} &  {\bf {\color{red} 0.60}} &  {\bf {\color{red} 0.56}} &  {\bf {\color{red} 0.53}} &  {\bf {\color{red} 0.50}} \\ 
\hline 
 2044 &  64 &  65 ans 0 mois &  -5.00\% &  1783.47 &  {\bf 52.80} &  2345.79 &  {\bf {\color{red} 0.76}} &  {\bf {\color{red} 0.70}} &  {\bf {\color{red} 0.66}} &  {\bf {\color{red} 0.62}} &  {\bf {\color{red} 0.58}} &  {\bf {\color{red} 0.54}} \\ 
\hline 
 2045 &  65 &  65 ans 0 mois &  0.00\% &  2019.84 &  {\bf 59.70} &  2376.28 &  {\bf {\color{red} 0.85}} &  {\bf {\color{red} 0.80}} &  {\bf {\color{red} 0.75}} &  {\bf {\color{red} 0.70}} &  {\bf {\color{red} 0.66}} &  {\bf {\color{red} 0.62}} \\ 
\hline 
 2046 &  66 &  65 ans 0 mois &  5.00\% &  2122.41 &  {\bf 62.63} &  2407.18 &  {\bf {\color{red} 0.88}} &  {\bf {\color{red} 0.84}} &  {\bf {\color{red} 0.78}} &  {\bf {\color{red} 0.74}} &  {\bf {\color{red} 0.69}} &  {\bf {\color{red} 0.65}} \\ 
\hline 
 2047 &  67 &  65 ans 0 mois &  10.00\% &  2304.36 &  {\bf 67.88} &  2438.47 &  {\bf {\color{red} 0.95}} &  {\bf {\color{red} 0.91}} &  {\bf {\color{red} 0.85}} &  {\bf {\color{red} 0.80}} &  {\bf {\color{red} 0.75}} &  {\bf {\color{red} 0.70}} \\ 
\hline 
\hline 
\end{tabular} 
\end{center} } 
\paragraph{Retraites possibles et ratios Revenu/SMIC à 70, 75, 80, 85, 90 ans avec le modèle \emph{Gouvernement corrigé (âge-pivot glissant)}}  
 
{ \scriptsize \begin{center} 
\begin{tabular}[htb]{|c|c||c|c||c|c||c||c|c|c|c|c|c|} 
\hline 
 Retraite en &  Âge &  Âge pivot &  Décote/Surcote &  Retraite (\euro{} 2019) &  Tx Rempl(\%) &  SMIC (\euro{} 2019) &  Retraite/SMIC &  Rev70/SMIC &  Rev75/SMIC &  Rev80/SMIC &  Rev85/SMIC &  Rev90/SMIC \\ 
\hline \hline 
 2042 &  62 &  65 ans 3 mois &  -16.25\% &  1458.34 &  {\bf 43.31} &  2285.97 &  {\bf {\color{red} 0.64}} &  {\bf {\color{red} 0.58}} &  {\bf {\color{red} 0.54}} &  {\bf {\color{red} 0.51}} &  {\bf {\color{red} 0.47}} &  {\bf {\color{red} 0.44}} \\ 
\hline 
 2043 &  63 &  65 ans 4 mois &  -11.67\% &  1597.24 &  {\bf 47.36} &  2315.68 &  {\bf {\color{red} 0.69}} &  {\bf {\color{red} 0.63}} &  {\bf {\color{red} 0.59}} &  {\bf {\color{red} 0.55}} &  {\bf {\color{red} 0.52}} &  {\bf {\color{red} 0.49}} \\ 
\hline 
 2044 &  64 &  65 ans 5 mois &  -7.08\% &  1744.36 &  {\bf 51.64} &  2345.79 &  {\bf {\color{red} 0.74}} &  {\bf {\color{red} 0.69}} &  {\bf {\color{red} 0.65}} &  {\bf {\color{red} 0.60}} &  {\bf {\color{red} 0.57}} &  {\bf {\color{red} 0.53}} \\ 
\hline 
 2045 &  65 &  65 ans 6 mois &  -2.50\% &  2019.84 &  {\bf 59.70} &  2376.28 &  {\bf {\color{red} 0.85}} &  {\bf {\color{red} 0.80}} &  {\bf {\color{red} 0.75}} &  {\bf {\color{red} 0.70}} &  {\bf {\color{red} 0.66}} &  {\bf {\color{red} 0.62}} \\ 
\hline 
 2046 &  66 &  65 ans 7 mois &  2.08\% &  2063.45 &  {\bf 60.89} &  2407.18 &  {\bf {\color{red} 0.86}} &  {\bf {\color{red} 0.81}} &  {\bf {\color{red} 0.76}} &  {\bf {\color{red} 0.72}} &  {\bf {\color{red} 0.67}} &  {\bf {\color{red} 0.63}} \\ 
\hline 
 2047 &  67 &  65 ans 8 mois &  6.67\% &  2234.53 &  {\bf 65.83} &  2438.47 &  {\bf {\color{red} 0.92}} &  {\bf {\color{red} 0.88}} &  {\bf {\color{red} 0.83}} &  {\bf {\color{red} 0.77}} &  {\bf {\color{red} 0.73}} &  {\bf {\color{red} 0.68}} \\ 
\hline 
\hline 
\end{tabular} 
\end{center} } 
\paragraph{Retraites possibles et ratios Revenu/SMIC à 70, 75, 80, 85, 90 ans avec le modèle \emph{Destinie2 (revalorisation de la fonction publique)}}  
 
{ \scriptsize \begin{center} 
\begin{tabular}[htb]{|c|c||c|c||c|c||c||c|c|c|c|c|c|} 
\hline 
 Retraite en &  Âge &  Âge pivot &  Décote/Surcote &  Retraite (\euro{} 2019) &  Tx Rempl(\%) &  SMIC (\euro{} 2019) &  Retraite/SMIC &  Rev70/SMIC &  Rev75/SMIC &  Rev80/SMIC &  Rev85/SMIC &  Rev90/SMIC \\ 
\hline \hline 
 2042 &  62 &  65 ans 3 mois &  -16.25\% &  1656.45 &  {\bf 37.01} &  2149.23 &  {\bf {\color{red} 0.77}} &  {\bf {\color{red} 0.70}} &  {\bf {\color{red} 0.65}} &  {\bf {\color{red} 0.61}} &  {\bf {\color{red} 0.57}} &  {\bf {\color{red} 0.54}} \\ 
\hline 
 2043 &  63 &  65 ans 4 mois &  -11.67\% &  1822.87 &  {\bf 40.21} &  2177.17 &  {\bf {\color{red} 0.84}} &  {\bf {\color{red} 0.76}} &  {\bf {\color{red} 0.72}} &  {\bf {\color{red} 0.67}} &  {\bf {\color{red} 0.63}} &  {\bf {\color{red} 0.59}} \\ 
\hline 
 2044 &  64 &  65 ans 5 mois &  -7.08\% &  2000.30 &  {\bf 43.56} &  2205.48 &  {\bf {\color{red} 0.91}} &  {\bf {\color{red} 0.84}} &  {\bf {\color{red} 0.79}} &  {\bf {\color{red} 0.74}} &  {\bf {\color{red} 0.69}} &  {\bf {\color{red} 0.65}} \\ 
\hline 
 2045 &  65 &  65 ans 6 mois &  -2.50\% &  2189.39 &  {\bf 47.06} &  2234.15 &  {\bf {\color{red} 0.98}} &  {\bf {\color{red} 0.92}} &  {\bf {\color{red} 0.86}} &  {\bf {\color{red} 0.81}} &  {\bf {\color{red} 0.76}} &  {\bf {\color{red} 0.71}} \\ 
\hline 
 2046 &  66 &  65 ans 7 mois &  2.08\% &  2389.08 &  {\bf 50.70} &  2263.19 &  {\bf 1.06} &  {\bf 1.00} &  {\bf {\color{red} 0.94}} &  {\bf {\color{red} 0.88}} &  {\bf {\color{red} 0.83}} &  {\bf {\color{red} 0.77}} \\ 
\hline 
 2047 &  67 &  65 ans 8 mois &  6.67\% &  2599.67 &  {\bf 54.46} &  2292.61 &  {\bf 1.13} &  {\bf 1.09} &  {\bf 1.02} &  {\bf {\color{red} 0.96}} &  {\bf {\color{red} 0.90}} &  {\bf {\color{red} 0.84}} \\ 
\hline 
\hline 
\end{tabular} 
\end{center} } 

 \begin{center}\includegraphics[width=0.9\textwidth]{fig/SecretaireAdmin_1980_22_dest_retraite.pdf}\end{center} \label{fig/SecretaireAdmin_1980_22_dest_retraite.pdf} 

\newpage 
 
\subsection{Génération 1990 (début en 2012)} 

\paragraph{Retraites possibles et ratios Revenu/SMIC à 70, 75, 80, 85, 90 ans avec le modèle \emph{Gouvernement truqué (âge-pivot bloqué à 65 ans)}}  
 
{ \scriptsize \begin{center} 
\begin{tabular}[htb]{|c|c||c|c||c|c||c||c|c|c|c|c|c|} 
\hline 
 Retraite en &  Âge &  Âge pivot &  Décote/Surcote &  Retraite (\euro{} 2019) &  Tx Rempl(\%) &  SMIC (\euro{} 2019) &  Retraite/SMIC &  Rev70/SMIC &  Rev75/SMIC &  Rev80/SMIC &  Rev85/SMIC &  Rev90/SMIC \\ 
\hline \hline 
 2052 &  62 &  65 ans 0 mois &  -15.00\% &  1587.40 &  {\bf 47.15} &  2601.14 &  {\bf {\color{red} 0.61}} &  {\bf {\color{red} 0.55}} &  {\bf {\color{red} 0.52}} &  {\bf {\color{red} 0.48}} &  {\bf {\color{red} 0.45}} &  {\bf {\color{red} 0.43}} \\ 
\hline 
 2053 &  63 &  65 ans 0 mois &  -10.00\% &  1744.88 &  {\bf 51.74} &  2634.96 &  {\bf {\color{red} 0.66}} &  {\bf {\color{red} 0.60}} &  {\bf {\color{red} 0.57}} &  {\bf {\color{red} 0.53}} &  {\bf {\color{red} 0.50}} &  {\bf {\color{red} 0.47}} \\ 
\hline 
 2054 &  64 &  65 ans 0 mois &  -5.00\% &  1910.43 &  {\bf 56.56} &  2669.21 &  {\bf {\color{red} 0.72}} &  {\bf {\color{red} 0.66}} &  {\bf {\color{red} 0.62}} &  {\bf {\color{red} 0.58}} &  {\bf {\color{red} 0.55}} &  {\bf {\color{red} 0.51}} \\ 
\hline 
 2055 &  65 &  65 ans 0 mois &  0.00\% &  2298.33 &  {\bf 67.93} &  2703.91 &  {\bf {\color{red} 0.85}} &  {\bf {\color{red} 0.80}} &  {\bf {\color{red} 0.75}} &  {\bf {\color{red} 0.70}} &  {\bf {\color{red} 0.66}} &  {\bf {\color{red} 0.62}} \\ 
\hline 
 2056 &  66 &  65 ans 0 mois &  5.00\% &  2328.20 &  {\bf 68.70} &  2739.06 &  {\bf {\color{red} 0.85}} &  {\bf {\color{red} 0.81}} &  {\bf {\color{red} 0.76}} &  {\bf {\color{red} 0.71}} &  {\bf {\color{red} 0.67}} &  {\bf {\color{red} 0.62}} \\ 
\hline 
 2057 &  67 &  65 ans 0 mois &  10.00\% &  2457.17 &  {\bf 72.39} &  2774.67 &  {\bf {\color{red} 0.89}} &  {\bf {\color{red} 0.85}} &  {\bf {\color{red} 0.80}} &  {\bf {\color{red} 0.75}} &  {\bf {\color{red} 0.70}} &  {\bf {\color{red} 0.66}} \\ 
\hline 
\hline 
\end{tabular} 
\end{center} } 
\paragraph{Retraites possibles et ratios Revenu/SMIC à 70, 75, 80, 85, 90 ans avec le modèle \emph{Gouvernement corrigé (âge-pivot glissant)}}  
 
{ \scriptsize \begin{center} 
\begin{tabular}[htb]{|c|c||c|c||c|c||c||c|c|c|c|c|c|} 
\hline 
 Retraite en &  Âge &  Âge pivot &  Décote/Surcote &  Retraite (\euro{} 2019) &  Tx Rempl(\%) &  SMIC (\euro{} 2019) &  Retraite/SMIC &  Rev70/SMIC &  Rev75/SMIC &  Rev80/SMIC &  Rev85/SMIC &  Rev90/SMIC \\ 
\hline \hline 
 2052 &  62 &  66 ans 1 mois &  -20.42\% &  1486.25 &  {\bf 44.14} &  2601.14 &  {\bf {\color{red} 0.57}} &  {\bf {\color{red} 0.52}} &  {\bf {\color{red} 0.48}} &  {\bf {\color{red} 0.45}} &  {\bf {\color{red} 0.42}} &  {\bf {\color{red} 0.40}} \\ 
\hline 
 2053 &  63 &  66 ans 2 mois &  -15.83\% &  1631.78 &  {\bf 48.39} &  2634.96 &  {\bf {\color{red} 0.62}} &  {\bf {\color{red} 0.57}} &  {\bf {\color{red} 0.53}} &  {\bf {\color{red} 0.50}} &  {\bf {\color{red} 0.47}} &  {\bf {\color{red} 0.44}} \\ 
\hline 
 2054 &  64 &  66 ans 3 mois &  -11.25\% &  1784.74 &  {\bf 52.84} &  2669.21 &  {\bf {\color{red} 0.67}} &  {\bf {\color{red} 0.62}} &  {\bf {\color{red} 0.58}} &  {\bf {\color{red} 0.54}} &  {\bf {\color{red} 0.51}} &  {\bf {\color{red} 0.48}} \\ 
\hline 
 2055 &  65 &  66 ans 4 mois &  -6.67\% &  2298.33 &  {\bf 67.93} &  2703.91 &  {\bf {\color{red} 0.85}} &  {\bf {\color{red} 0.80}} &  {\bf {\color{red} 0.75}} &  {\bf {\color{red} 0.70}} &  {\bf {\color{red} 0.66}} &  {\bf {\color{red} 0.62}} \\ 
\hline 
 2056 &  66 &  66 ans 5 mois &  -2.08\% &  2328.20 &  {\bf 68.70} &  2739.06 &  {\bf {\color{red} 0.85}} &  {\bf {\color{red} 0.81}} &  {\bf {\color{red} 0.76}} &  {\bf {\color{red} 0.71}} &  {\bf {\color{red} 0.67}} &  {\bf {\color{red} 0.62}} \\ 
\hline 
 2057 &  67 &  66 ans 6 mois &  2.50\% &  2358.47 &  {\bf 69.48} &  2774.67 &  {\bf {\color{red} 0.85}} &  {\bf {\color{red} 0.82}} &  {\bf {\color{red} 0.77}} &  {\bf {\color{red} 0.72}} &  {\bf {\color{red} 0.67}} &  {\bf {\color{red} 0.63}} \\ 
\hline 
\hline 
\end{tabular} 
\end{center} } 
\paragraph{Retraites possibles et ratios Revenu/SMIC à 70, 75, 80, 85, 90 ans avec le modèle \emph{Destinie2 (revalorisation de la fonction publique)}}  
 
{ \scriptsize \begin{center} 
\begin{tabular}[htb]{|c|c||c|c||c|c||c||c|c|c|c|c|c|} 
\hline 
 Retraite en &  Âge &  Âge pivot &  Décote/Surcote &  Retraite (\euro{} 2019) &  Tx Rempl(\%) &  SMIC (\euro{} 2019) &  Retraite/SMIC &  Rev70/SMIC &  Rev75/SMIC &  Rev80/SMIC &  Rev85/SMIC &  Rev90/SMIC \\ 
\hline \hline 
 2052 &  62 &  66 ans 1 mois &  -20.42\% &  1851.13 &  {\bf 36.35} &  2445.56 &  {\bf {\color{red} 0.76}} &  {\bf {\color{red} 0.68}} &  {\bf {\color{red} 0.64}} &  {\bf {\color{red} 0.60}} &  {\bf {\color{red} 0.56}} &  {\bf {\color{red} 0.53}} \\ 
\hline 
 2053 &  63 &  66 ans 2 mois &  -15.83\% &  2043.62 &  {\bf 39.62} &  2477.35 &  {\bf {\color{red} 0.82}} &  {\bf {\color{red} 0.75}} &  {\bf {\color{red} 0.71}} &  {\bf {\color{red} 0.66}} &  {\bf {\color{red} 0.62}} &  {\bf {\color{red} 0.58}} \\ 
\hline 
 2054 &  64 &  66 ans 3 mois &  -11.25\% &  2247.47 &  {\bf 43.01} &  2509.56 &  {\bf {\color{red} 0.90}} &  {\bf {\color{red} 0.83}} &  {\bf {\color{red} 0.78}} &  {\bf {\color{red} 0.73}} &  {\bf {\color{red} 0.68}} &  {\bf {\color{red} 0.64}} \\ 
\hline 
 2055 &  65 &  66 ans 4 mois &  -6.67\% &  2463.04 &  {\bf 46.53} &  2542.18 &  {\bf {\color{red} 0.97}} &  {\bf {\color{red} 0.91}} &  {\bf {\color{red} 0.85}} &  {\bf {\color{red} 0.80}} &  {\bf {\color{red} 0.75}} &  {\bf {\color{red} 0.70}} \\ 
\hline 
 2056 &  66 &  66 ans 5 mois &  -2.08\% &  2690.67 &  {\bf 50.18} &  2575.23 &  {\bf 1.04} &  {\bf {\color{red} 0.99}} &  {\bf {\color{red} 0.93}} &  {\bf {\color{red} 0.87}} &  {\bf {\color{red} 0.82}} &  {\bf {\color{red} 0.77}} \\ 
\hline 
 2057 &  67 &  66 ans 6 mois &  2.50\% &  2930.73 &  {\bf 53.95} &  2608.71 &  {\bf 1.12} &  {\bf 1.08} &  {\bf 1.01} &  {\bf {\color{red} 0.95}} &  {\bf {\color{red} 0.89}} &  {\bf {\color{red} 0.83}} \\ 
\hline 
\hline 
\end{tabular} 
\end{center} } 

 \begin{center}\includegraphics[width=0.9\textwidth]{fig/SecretaireAdmin_1990_22_dest_retraite.pdf}\end{center} \label{fig/SecretaireAdmin_1990_22_dest_retraite.pdf} 

\newpage 
 
\subsection{Génération 2003 (début en 2025)} 

\paragraph{Retraites possibles et ratios Revenu/SMIC à 70, 75, 80, 85, 90 ans avec le modèle \emph{Gouvernement truqué (âge-pivot bloqué à 65 ans)}}  
 
{ \scriptsize \begin{center} 
\begin{tabular}[htb]{|c|c||c|c||c|c||c||c|c|c|c|c|c|} 
\hline 
 Retraite en &  Âge &  Âge pivot &  Décote/Surcote &  Retraite (\euro{} 2019) &  Tx Rempl(\%) &  SMIC (\euro{} 2019) &  Retraite/SMIC &  Rev70/SMIC &  Rev75/SMIC &  Rev80/SMIC &  Rev85/SMIC &  Rev90/SMIC \\ 
\hline \hline 
 2065 &  62 &  65 ans 0 mois &  -15.00\% &  1698.65 &  {\bf 50.45} &  3076.71 &  {\bf {\color{red} 0.55}} &  {\bf {\color{red} 0.50}} &  {\bf {\color{red} 0.47}} &  {\bf {\color{red} 0.44}} &  {\bf {\color{red} 0.41}} &  {\bf {\color{red} 0.38}} \\ 
\hline 
 2066 &  63 &  65 ans 0 mois &  -10.00\% &  1864.20 &  {\bf 55.28} &  3116.71 &  {\bf {\color{red} 0.60}} &  {\bf {\color{red} 0.55}} &  {\bf {\color{red} 0.51}} &  {\bf {\color{red} 0.48}} &  {\bf {\color{red} 0.45}} &  {\bf {\color{red} 0.42}} \\ 
\hline 
 2067 &  64 &  65 ans 0 mois &  -5.00\% &  2038.01 &  {\bf 60.33} &  3157.23 &  {\bf {\color{red} 0.65}} &  {\bf {\color{red} 0.60}} &  {\bf {\color{red} 0.56}} &  {\bf {\color{red} 0.52}} &  {\bf {\color{red} 0.49}} &  {\bf {\color{red} 0.46}} \\ 
\hline 
 2068 &  65 &  65 ans 0 mois &  0.00\% &  2718.53 &  {\bf 80.35} &  3198.27 &  {\bf {\color{red} 0.85}} &  {\bf {\color{red} 0.80}} &  {\bf {\color{red} 0.75}} &  {\bf {\color{red} 0.70}} &  {\bf {\color{red} 0.66}} &  {\bf {\color{red} 0.62}} \\ 
\hline 
 2069 &  66 &  65 ans 0 mois &  5.00\% &  2753.87 &  {\bf 81.26} &  3239.85 &  {\bf {\color{red} 0.85}} &  {\bf {\color{red} 0.81}} &  {\bf {\color{red} 0.76}} &  {\bf {\color{red} 0.71}} &  {\bf {\color{red} 0.67}} &  {\bf {\color{red} 0.62}} \\ 
\hline 
 2070 &  67 &  65 ans 0 mois &  10.00\% &  2789.67 &  {\bf 82.18} &  3281.97 &  {\bf {\color{red} 0.85}} &  {\bf {\color{red} 0.82}} &  {\bf {\color{red} 0.77}} &  {\bf {\color{red} 0.72}} &  {\bf {\color{red} 0.67}} &  {\bf {\color{red} 0.63}} \\ 
\hline 
\hline 
\end{tabular} 
\end{center} } 
\paragraph{Retraites possibles et ratios Revenu/SMIC à 70, 75, 80, 85, 90 ans avec le modèle \emph{Gouvernement corrigé (âge-pivot glissant)}}  
 
{ \scriptsize \begin{center} 
\begin{tabular}[htb]{|c|c||c|c||c|c||c||c|c|c|c|c|c|} 
\hline 
 Retraite en &  Âge &  Âge pivot &  Décote/Surcote &  Retraite (\euro{} 2019) &  Tx Rempl(\%) &  SMIC (\euro{} 2019) &  Retraite/SMIC &  Rev70/SMIC &  Rev75/SMIC &  Rev80/SMIC &  Rev85/SMIC &  Rev90/SMIC \\ 
\hline \hline 
 2065 &  62 &  67 ans 2 mois &  -25.83\% &  1482.15 &  {\bf 44.02} &  3076.71 &  {\bf {\color{red} 0.48}} &  {\bf {\color{red} 0.43}} &  {\bf {\color{red} 0.41}} &  {\bf {\color{red} 0.38}} &  {\bf {\color{red} 0.36}} &  {\bf {\color{red} 0.34}} \\ 
\hline 
 2066 &  63 &  67 ans 3 mois &  -21.25\% &  1631.17 &  {\bf 48.37} &  3116.71 &  {\bf {\color{red} 0.52}} &  {\bf {\color{red} 0.48}} &  {\bf {\color{red} 0.45}} &  {\bf {\color{red} 0.42}} &  {\bf {\color{red} 0.39}} &  {\bf {\color{red} 0.37}} \\ 
\hline 
 2067 &  64 &  67 ans 4 mois &  -16.67\% &  1787.73 &  {\bf 52.92} &  3157.23 &  {\bf {\color{red} 0.57}} &  {\bf {\color{red} 0.52}} &  {\bf {\color{red} 0.49}} &  {\bf {\color{red} 0.46}} &  {\bf {\color{red} 0.43}} &  {\bf {\color{red} 0.40}} \\ 
\hline 
 2068 &  65 &  67 ans 5 mois &  -12.08\% &  2718.53 &  {\bf 80.35} &  3198.27 &  {\bf {\color{red} 0.85}} &  {\bf {\color{red} 0.80}} &  {\bf {\color{red} 0.75}} &  {\bf {\color{red} 0.70}} &  {\bf {\color{red} 0.66}} &  {\bf {\color{red} 0.62}} \\ 
\hline 
 2069 &  66 &  67 ans 6 mois &  -7.50\% &  2753.87 &  {\bf 81.26} &  3239.85 &  {\bf {\color{red} 0.85}} &  {\bf {\color{red} 0.81}} &  {\bf {\color{red} 0.76}} &  {\bf {\color{red} 0.71}} &  {\bf {\color{red} 0.67}} &  {\bf {\color{red} 0.62}} \\ 
\hline 
 2070 &  67 &  67 ans 7 mois &  -2.92\% &  2789.67 &  {\bf 82.18} &  3281.97 &  {\bf {\color{red} 0.85}} &  {\bf {\color{red} 0.82}} &  {\bf {\color{red} 0.77}} &  {\bf {\color{red} 0.72}} &  {\bf {\color{red} 0.67}} &  {\bf {\color{red} 0.63}} \\ 
\hline 
\hline 
\end{tabular} 
\end{center} } 
\paragraph{Retraites possibles et ratios Revenu/SMIC à 70, 75, 80, 85, 90 ans avec le modèle \emph{Destinie2 (revalorisation de la fonction publique)}}  
 
{ \scriptsize \begin{center} 
\begin{tabular}[htb]{|c|c||c|c||c|c||c||c|c|c|c|c|c|} 
\hline 
 Retraite en &  Âge &  Âge pivot &  Décote/Surcote &  Retraite (\euro{} 2019) &  Tx Rempl(\%) &  SMIC (\euro{} 2019) &  Retraite/SMIC &  Rev70/SMIC &  Rev75/SMIC &  Rev80/SMIC &  Rev85/SMIC &  Rev90/SMIC \\ 
\hline \hline 
 2065 &  62 &  67 ans 2 mois &  -25.83\% &  2146.37 &  {\bf 35.63} &  2892.68 &  {\bf {\color{red} 0.74}} &  {\bf {\color{red} 0.67}} &  {\bf {\color{red} 0.63}} &  {\bf {\color{red} 0.59}} &  {\bf {\color{red} 0.55}} &  {\bf {\color{red} 0.52}} \\ 
\hline 
 2066 &  63 &  67 ans 3 mois &  -21.25\% &  2375.52 &  {\bf 38.93} &  2930.29 &  {\bf {\color{red} 0.81}} &  {\bf {\color{red} 0.74}} &  {\bf {\color{red} 0.69}} &  {\bf {\color{red} 0.65}} &  {\bf {\color{red} 0.61}} &  {\bf {\color{red} 0.57}} \\ 
\hline 
 2067 &  64 &  67 ans 4 mois &  -16.67\% &  2618.15 &  {\bf 42.36} &  2968.38 &  {\bf {\color{red} 0.88}} &  {\bf {\color{red} 0.82}} &  {\bf {\color{red} 0.77}} &  {\bf {\color{red} 0.72}} &  {\bf {\color{red} 0.67}} &  {\bf {\color{red} 0.63}} \\ 
\hline 
 2068 &  65 &  67 ans 5 mois &  -12.08\% &  2874.68 &  {\bf 45.91} &  3006.97 &  {\bf {\color{red} 0.96}} &  {\bf {\color{red} 0.90}} &  {\bf {\color{red} 0.84}} &  {\bf {\color{red} 0.79}} &  {\bf {\color{red} 0.74}} &  {\bf {\color{red} 0.69}} \\ 
\hline 
 2069 &  66 &  67 ans 6 mois &  -7.50\% &  3145.53 &  {\bf 49.59} &  3046.06 &  {\bf 1.03} &  {\bf {\color{red} 0.98}} &  {\bf {\color{red} 0.92}} &  {\bf {\color{red} 0.86}} &  {\bf {\color{red} 0.81}} &  {\bf {\color{red} 0.76}} \\ 
\hline 
 2070 &  67 &  67 ans 7 mois &  -2.92\% &  3431.13 &  {\bf 53.40} &  3085.66 &  {\bf 1.11} &  {\bf 1.07} &  {\bf 1.00} &  {\bf {\color{red} 0.94}} &  {\bf {\color{red} 0.88}} &  {\bf {\color{red} 0.83}} \\ 
\hline 
\hline 
\end{tabular} 
\end{center} } 

 \begin{center}\includegraphics[width=0.9\textwidth]{fig/SecretaireAdmin_2003_22_dest_retraite.pdf}\end{center} \label{fig/SecretaireAdmin_2003_22_dest_retraite.pdf} 

\newpage 
 
\chapter{ATSEM (C2 puis C1)} 

\begin{minipage}{0.55\linewidth}\includegraphics[width=0.7\textwidth]{fig/grille_ATSEM.pdf}\end{minipage} 
\begin{minipage}{0.3\linewidth} 
 \begin{center} 

\begin{tabular}[htb]{|c|c|} 
\hline 
 Indice majoré &  Durée (années) \\ 
\hline \hline 
 329 &  1.00 \\ 
\hline 
 330 &  2.00 \\ 
\hline 
 333 &  2.00 \\ 
\hline 
 336 &  2.00 \\ 
\hline 
 345 &  2.00 \\ 
\hline 
 351 &  2.00 \\ 
\hline 
 364 &  2.00 \\ 
\hline 
 380 &  2.00 \\ 
\hline 
 390 &  3.00 \\ 
\hline 
 402 &  3.00 \\ 
\hline 
 411 &  4.00 \\ 
\hline 
 415 &  3.00 \\ 
\hline 
 430 &  3.00 \\ 
\hline 
 450 &  3.00 \\ 
\hline 
 466 &   \\ 
\hline 
\hline 
\end{tabular} 
\end{center} 
 \end{minipage} 


 \addto{\captionsenglish}{ \renewcommand{\mtctitle}{}} \setcounter{minitocdepth}{2} 
 \minitoc \newpage 

\section{Début de carrière à 22 ans} 

\subsection{Génération 1975 (début en 1997)} 

\paragraph{Retraites possibles et ratios Revenu/SMIC à 70, 75, 80, 85, 90 ans avec le modèle \emph{Gouvernement truqué (âge-pivot bloqué à 65 ans)}}  
 
{ \scriptsize \begin{center} 
\begin{tabular}[htb]{|c|c||c|c||c|c||c||c|c|c|c|c|c|} 
\hline 
 Retraite en &  Âge &  Âge pivot &  Décote/Surcote &  Retraite (\euro{} 2019) &  Tx Rempl(\%) &  SMIC (\euro{} 2019) &  Retraite/SMIC &  Rev70/SMIC &  Rev75/SMIC &  Rev80/SMIC &  Rev85/SMIC &  Rev90/SMIC \\ 
\hline \hline 
 2037 &  62 &  64 ans 10 mois &  -14.17\% &  1115.00 &  {\bf 43.28} &  2143.00 &  {\bf {\color{red} 0.52}} &  {\bf {\color{red} 0.47}} &  {\bf {\color{red} 0.44}} &  {\bf {\color{red} 0.41}} &  {\bf {\color{red} 0.39}} &  {\bf {\color{red} 0.36}} \\ 
\hline 
 2038 &  63 &  64 ans 11 mois &  -9.58\% &  1215.99 &  {\bf 47.11} &  2170.86 &  {\bf {\color{red} 0.56}} &  {\bf {\color{red} 0.51}} &  {\bf {\color{red} 0.48}} &  {\bf {\color{red} 0.45}} &  {\bf {\color{red} 0.42}} &  {\bf {\color{red} 0.40}} \\ 
\hline 
 2039 &  64 &  65 ans 0 mois &  -5.00\% &  1322.55 &  {\bf 51.13} &  2199.08 &  {\bf {\color{red} 0.60}} &  {\bf {\color{red} 0.56}} &  {\bf {\color{red} 0.52}} &  {\bf {\color{red} 0.49}} &  {\bf {\color{red} 0.46}} &  {\bf {\color{red} 0.43}} \\ 
\hline 
 2040 &  65 &  65 ans 0 mois &  0.00\% &  1893.52 &  {\bf 73.06} &  2227.67 &  {\bf {\color{red} 0.85}} &  {\bf {\color{red} 0.80}} &  {\bf {\color{red} 0.75}} &  {\bf {\color{red} 0.70}} &  {\bf {\color{red} 0.66}} &  {\bf {\color{red} 0.62}} \\ 
\hline 
 2041 &  66 &  65 ans 0 mois &  5.00\% &  1918.14 &  {\bf 73.87} &  2256.63 &  {\bf {\color{red} 0.85}} &  {\bf {\color{red} 0.81}} &  {\bf {\color{red} 0.76}} &  {\bf {\color{red} 0.71}} &  {\bf {\color{red} 0.67}} &  {\bf {\color{red} 0.62}} \\ 
\hline 
 2042 &  67 &  65 ans 0 mois &  10.00\% &  1943.07 &  {\bf 74.68} &  2285.97 &  {\bf {\color{red} 0.85}} &  {\bf {\color{red} 0.82}} &  {\bf {\color{red} 0.77}} &  {\bf {\color{red} 0.72}} &  {\bf {\color{red} 0.67}} &  {\bf {\color{red} 0.63}} \\ 
\hline 
\hline 
\end{tabular} 
\end{center} } 
\paragraph{Retraites possibles et ratios Revenu/SMIC à 70, 75, 80, 85, 90 ans avec le modèle \emph{Gouvernement corrigé (âge-pivot glissant)}}  
 
{ \scriptsize \begin{center} 
\begin{tabular}[htb]{|c|c||c|c||c|c||c||c|c|c|c|c|c|} 
\hline 
 Retraite en &  Âge &  Âge pivot &  Décote/Surcote &  Retraite (\euro{} 2019) &  Tx Rempl(\%) &  SMIC (\euro{} 2019) &  Retraite/SMIC &  Rev70/SMIC &  Rev75/SMIC &  Rev80/SMIC &  Rev85/SMIC &  Rev90/SMIC \\ 
\hline \hline 
 2037 &  62 &  64 ans 10 mois &  -14.17\% &  1115.00 &  {\bf 43.28} &  2143.00 &  {\bf {\color{red} 0.52}} &  {\bf {\color{red} 0.47}} &  {\bf {\color{red} 0.44}} &  {\bf {\color{red} 0.41}} &  {\bf {\color{red} 0.39}} &  {\bf {\color{red} 0.36}} \\ 
\hline 
 2038 &  63 &  64 ans 11 mois &  -9.58\% &  1215.99 &  {\bf 47.11} &  2170.86 &  {\bf {\color{red} 0.56}} &  {\bf {\color{red} 0.51}} &  {\bf {\color{red} 0.48}} &  {\bf {\color{red} 0.45}} &  {\bf {\color{red} 0.42}} &  {\bf {\color{red} 0.40}} \\ 
\hline 
 2039 &  64 &  65 ans 0 mois &  -5.00\% &  1322.55 &  {\bf 51.13} &  2199.08 &  {\bf {\color{red} 0.60}} &  {\bf {\color{red} 0.56}} &  {\bf {\color{red} 0.52}} &  {\bf {\color{red} 0.49}} &  {\bf {\color{red} 0.46}} &  {\bf {\color{red} 0.43}} \\ 
\hline 
 2040 &  65 &  65 ans 1 mois &  -0.42\% &  1893.52 &  {\bf 73.06} &  2227.67 &  {\bf {\color{red} 0.85}} &  {\bf {\color{red} 0.80}} &  {\bf {\color{red} 0.75}} &  {\bf {\color{red} 0.70}} &  {\bf {\color{red} 0.66}} &  {\bf {\color{red} 0.62}} \\ 
\hline 
 2041 &  66 &  65 ans 2 mois &  4.17\% &  1918.14 &  {\bf 73.87} &  2256.63 &  {\bf {\color{red} 0.85}} &  {\bf {\color{red} 0.81}} &  {\bf {\color{red} 0.76}} &  {\bf {\color{red} 0.71}} &  {\bf {\color{red} 0.67}} &  {\bf {\color{red} 0.62}} \\ 
\hline 
 2042 &  67 &  65 ans 3 mois &  8.75\% &  1943.07 &  {\bf 74.68} &  2285.97 &  {\bf {\color{red} 0.85}} &  {\bf {\color{red} 0.82}} &  {\bf {\color{red} 0.77}} &  {\bf {\color{red} 0.72}} &  {\bf {\color{red} 0.67}} &  {\bf {\color{red} 0.63}} \\ 
\hline 
\hline 
\end{tabular} 
\end{center} } 
\paragraph{Retraites possibles et ratios Revenu/SMIC à 70, 75, 80, 85, 90 ans avec le modèle \emph{Destinie2 (revalorisation de la fonction publique)}}  
 
{ \scriptsize \begin{center} 
\begin{tabular}[htb]{|c|c||c|c||c|c||c||c|c|c|c|c|c|} 
\hline 
 Retraite en &  Âge &  Âge pivot &  Décote/Surcote &  Retraite (\euro{} 2019) &  Tx Rempl(\%) &  SMIC (\euro{} 2019) &  Retraite/SMIC &  Rev70/SMIC &  Rev75/SMIC &  Rev80/SMIC &  Rev85/SMIC &  Rev90/SMIC \\ 
\hline \hline 
 2037 &  62 &  64 ans 10 mois &  -14.17\% &  1234.07 &  {\bf 38.40} &  2014.82 &  {\bf {\color{red} 0.61}} &  {\bf {\color{red} 0.55}} &  {\bf {\color{red} 0.52}} &  {\bf {\color{red} 0.49}} &  {\bf {\color{red} 0.46}} &  {\bf {\color{red} 0.43}} \\ 
\hline 
 2038 &  63 &  64 ans 11 mois &  -9.58\% &  1350.85 &  {\bf 41.50} &  2041.01 &  {\bf {\color{red} 0.66}} &  {\bf {\color{red} 0.60}} &  {\bf {\color{red} 0.57}} &  {\bf {\color{red} 0.53}} &  {\bf {\color{red} 0.50}} &  {\bf {\color{red} 0.47}} \\ 
\hline 
 2039 &  64 &  65 ans 0 mois &  -5.00\% &  1474.85 &  {\bf 44.72} &  2067.55 &  {\bf {\color{red} 0.71}} &  {\bf {\color{red} 0.66}} &  {\bf {\color{red} 0.62}} &  {\bf {\color{red} 0.58}} &  {\bf {\color{red} 0.54}} &  {\bf {\color{red} 0.51}} \\ 
\hline 
 2040 &  65 &  65 ans 1 mois &  -0.42\% &  1780.26 &  {\bf 53.29} &  2094.43 &  {\bf {\color{red} 0.85}} &  {\bf {\color{red} 0.80}} &  {\bf {\color{red} 0.75}} &  {\bf {\color{red} 0.70}} &  {\bf {\color{red} 0.66}} &  {\bf {\color{red} 0.62}} \\ 
\hline 
 2041 &  66 &  65 ans 2 mois &  4.17\% &  1803.40 &  {\bf 53.29} &  2121.65 &  {\bf {\color{red} 0.85}} &  {\bf {\color{red} 0.81}} &  {\bf {\color{red} 0.76}} &  {\bf {\color{red} 0.71}} &  {\bf {\color{red} 0.67}} &  {\bf {\color{red} 0.62}} \\ 
\hline 
 2042 &  67 &  65 ans 3 mois &  8.75\% &  1894.37 &  {\bf 55.26} &  2149.23 &  {\bf {\color{red} 0.88}} &  {\bf {\color{red} 0.85}} &  {\bf {\color{red} 0.79}} &  {\bf {\color{red} 0.75}} &  {\bf {\color{red} 0.70}} &  {\bf {\color{red} 0.65}} \\ 
\hline 
\hline 
\end{tabular} 
\end{center} } 

 \begin{center}\includegraphics[width=0.9\textwidth]{fig/ATSEM_1975_22_dest_retraite.pdf}\end{center} \label{fig/ATSEM_1975_22_dest_retraite.pdf} 

\newpage 
 
\subsection{Génération 1980 (début en 2002)} 

\paragraph{Retraites possibles et ratios Revenu/SMIC à 70, 75, 80, 85, 90 ans avec le modèle \emph{Gouvernement truqué (âge-pivot bloqué à 65 ans)}}  
 
{ \scriptsize \begin{center} 
\begin{tabular}[htb]{|c|c||c|c||c|c||c||c|c|c|c|c|c|} 
\hline 
 Retraite en &  Âge &  Âge pivot &  Décote/Surcote &  Retraite (\euro{} 2019) &  Tx Rempl(\%) &  SMIC (\euro{} 2019) &  Retraite/SMIC &  Rev70/SMIC &  Rev75/SMIC &  Rev80/SMIC &  Rev85/SMIC &  Rev90/SMIC \\ 
\hline \hline 
 2042 &  62 &  65 ans 0 mois &  -15.00\% &  1123.83 &  {\bf 43.62} &  2285.97 &  {\bf {\color{red} 0.49}} &  {\bf {\color{red} 0.44}} &  {\bf {\color{red} 0.42}} &  {\bf {\color{red} 0.39}} &  {\bf {\color{red} 0.37}} &  {\bf {\color{red} 0.34}} \\ 
\hline 
 2043 &  63 &  65 ans 0 mois &  -10.00\% &  1235.91 &  {\bf 47.88} &  2315.68 &  {\bf {\color{red} 0.53}} &  {\bf {\color{red} 0.49}} &  {\bf {\color{red} 0.46}} &  {\bf {\color{red} 0.43}} &  {\bf {\color{red} 0.40}} &  {\bf {\color{red} 0.38}} \\ 
\hline 
 2044 &  64 &  65 ans 0 mois &  -5.00\% &  1354.73 &  {\bf 52.38} &  2345.79 &  {\bf {\color{red} 0.58}} &  {\bf {\color{red} 0.53}} &  {\bf {\color{red} 0.50}} &  {\bf {\color{red} 0.47}} &  {\bf {\color{red} 0.44}} &  {\bf {\color{red} 0.41}} \\ 
\hline 
 2045 &  65 &  65 ans 0 mois &  0.00\% &  2019.84 &  {\bf 77.94} &  2376.28 &  {\bf {\color{red} 0.85}} &  {\bf {\color{red} 0.80}} &  {\bf {\color{red} 0.75}} &  {\bf {\color{red} 0.70}} &  {\bf {\color{red} 0.66}} &  {\bf {\color{red} 0.62}} \\ 
\hline 
 2046 &  66 &  65 ans 0 mois &  5.00\% &  2046.10 &  {\bf 78.79} &  2407.18 &  {\bf {\color{red} 0.85}} &  {\bf {\color{red} 0.81}} &  {\bf {\color{red} 0.76}} &  {\bf {\color{red} 0.71}} &  {\bf {\color{red} 0.67}} &  {\bf {\color{red} 0.62}} \\ 
\hline 
 2047 &  67 &  65 ans 0 mois &  10.00\% &  2072.70 &  {\bf 79.66} &  2438.47 &  {\bf {\color{red} 0.85}} &  {\bf {\color{red} 0.82}} &  {\bf {\color{red} 0.77}} &  {\bf {\color{red} 0.72}} &  {\bf {\color{red} 0.67}} &  {\bf {\color{red} 0.63}} \\ 
\hline 
\hline 
\end{tabular} 
\end{center} } 
\paragraph{Retraites possibles et ratios Revenu/SMIC à 70, 75, 80, 85, 90 ans avec le modèle \emph{Gouvernement corrigé (âge-pivot glissant)}}  
 
{ \scriptsize \begin{center} 
\begin{tabular}[htb]{|c|c||c|c||c|c||c||c|c|c|c|c|c|} 
\hline 
 Retraite en &  Âge &  Âge pivot &  Décote/Surcote &  Retraite (\euro{} 2019) &  Tx Rempl(\%) &  SMIC (\euro{} 2019) &  Retraite/SMIC &  Rev70/SMIC &  Rev75/SMIC &  Rev80/SMIC &  Rev85/SMIC &  Rev90/SMIC \\ 
\hline \hline 
 2042 &  62 &  65 ans 3 mois &  -16.25\% &  1107.30 &  {\bf 42.98} &  2285.97 &  {\bf {\color{red} 0.48}} &  {\bf {\color{red} 0.44}} &  {\bf {\color{red} 0.41}} &  {\bf {\color{red} 0.38}} &  {\bf {\color{red} 0.36}} &  {\bf {\color{red} 0.34}} \\ 
\hline 
 2043 &  63 &  65 ans 4 mois &  -11.67\% &  1213.03 &  {\bf 46.99} &  2315.68 &  {\bf {\color{red} 0.52}} &  {\bf {\color{red} 0.48}} &  {\bf {\color{red} 0.45}} &  {\bf {\color{red} 0.42}} &  {\bf {\color{red} 0.39}} &  {\bf {\color{red} 0.37}} \\ 
\hline 
 2044 &  64 &  65 ans 5 mois &  -7.08\% &  1325.02 &  {\bf 51.23} &  2345.79 &  {\bf {\color{red} 0.56}} &  {\bf {\color{red} 0.52}} &  {\bf {\color{red} 0.49}} &  {\bf {\color{red} 0.46}} &  {\bf {\color{red} 0.43}} &  {\bf {\color{red} 0.40}} \\ 
\hline 
 2045 &  65 &  65 ans 6 mois &  -2.50\% &  2019.84 &  {\bf 77.94} &  2376.28 &  {\bf {\color{red} 0.85}} &  {\bf {\color{red} 0.80}} &  {\bf {\color{red} 0.75}} &  {\bf {\color{red} 0.70}} &  {\bf {\color{red} 0.66}} &  {\bf {\color{red} 0.62}} \\ 
\hline 
 2046 &  66 &  65 ans 7 mois &  2.08\% &  2046.10 &  {\bf 78.79} &  2407.18 &  {\bf {\color{red} 0.85}} &  {\bf {\color{red} 0.81}} &  {\bf {\color{red} 0.76}} &  {\bf {\color{red} 0.71}} &  {\bf {\color{red} 0.67}} &  {\bf {\color{red} 0.62}} \\ 
\hline 
 2047 &  67 &  65 ans 8 mois &  6.67\% &  2072.70 &  {\bf 79.66} &  2438.47 &  {\bf {\color{red} 0.85}} &  {\bf {\color{red} 0.82}} &  {\bf {\color{red} 0.77}} &  {\bf {\color{red} 0.72}} &  {\bf {\color{red} 0.67}} &  {\bf {\color{red} 0.63}} \\ 
\hline 
\hline 
\end{tabular} 
\end{center} } 
\paragraph{Retraites possibles et ratios Revenu/SMIC à 70, 75, 80, 85, 90 ans avec le modèle \emph{Destinie2 (revalorisation de la fonction publique)}}  
 
{ \scriptsize \begin{center} 
\begin{tabular}[htb]{|c|c||c|c||c|c||c||c|c|c|c|c|c|} 
\hline 
 Retraite en &  Âge &  Âge pivot &  Décote/Surcote &  Retraite (\euro{} 2019) &  Tx Rempl(\%) &  SMIC (\euro{} 2019) &  Retraite/SMIC &  Rev70/SMIC &  Rev75/SMIC &  Rev80/SMIC &  Rev85/SMIC &  Rev90/SMIC \\ 
\hline \hline 
 2042 &  62 &  65 ans 3 mois &  -16.25\% &  1267.25 &  {\bf 36.97} &  2149.23 &  {\bf {\color{red} 0.59}} &  {\bf {\color{red} 0.53}} &  {\bf {\color{red} 0.50}} &  {\bf {\color{red} 0.47}} &  {\bf {\color{red} 0.44}} &  {\bf {\color{red} 0.41}} \\ 
\hline 
 2043 &  63 &  65 ans 4 mois &  -11.67\% &  1394.61 &  {\bf 40.16} &  2177.17 &  {\bf {\color{red} 0.64}} &  {\bf {\color{red} 0.59}} &  {\bf {\color{red} 0.55}} &  {\bf {\color{red} 0.51}} &  {\bf {\color{red} 0.48}} &  {\bf {\color{red} 0.45}} \\ 
\hline 
 2044 &  64 &  65 ans 5 mois &  -7.08\% &  1530.41 &  {\bf 43.51} &  2205.48 &  {\bf {\color{red} 0.69}} &  {\bf {\color{red} 0.64}} &  {\bf {\color{red} 0.60}} &  {\bf {\color{red} 0.56}} &  {\bf {\color{red} 0.53}} &  {\bf {\color{red} 0.50}} \\ 
\hline 
 2045 &  65 &  65 ans 6 mois &  -2.50\% &  1899.03 &  {\bf 53.29} &  2234.15 &  {\bf {\color{red} 0.85}} &  {\bf {\color{red} 0.80}} &  {\bf {\color{red} 0.75}} &  {\bf {\color{red} 0.70}} &  {\bf {\color{red} 0.66}} &  {\bf {\color{red} 0.62}} \\ 
\hline 
 2046 &  66 &  65 ans 7 mois &  2.08\% &  1923.71 &  {\bf 53.29} &  2263.19 &  {\bf {\color{red} 0.85}} &  {\bf {\color{red} 0.81}} &  {\bf {\color{red} 0.76}} &  {\bf {\color{red} 0.71}} &  {\bf {\color{red} 0.67}} &  {\bf {\color{red} 0.62}} \\ 
\hline 
 2047 &  67 &  65 ans 8 mois &  6.67\% &  1989.18 &  {\bf 54.40} &  2292.61 &  {\bf {\color{red} 0.87}} &  {\bf {\color{red} 0.83}} &  {\bf {\color{red} 0.78}} &  {\bf {\color{red} 0.73}} &  {\bf {\color{red} 0.69}} &  {\bf {\color{red} 0.64}} \\ 
\hline 
\hline 
\end{tabular} 
\end{center} } 

 \begin{center}\includegraphics[width=0.9\textwidth]{fig/ATSEM_1980_22_dest_retraite.pdf}\end{center} \label{fig/ATSEM_1980_22_dest_retraite.pdf} 

\newpage 
 
\subsection{Génération 1990 (début en 2012)} 

\paragraph{Retraites possibles et ratios Revenu/SMIC à 70, 75, 80, 85, 90 ans avec le modèle \emph{Gouvernement truqué (âge-pivot bloqué à 65 ans)}}  
 
{ \scriptsize \begin{center} 
\begin{tabular}[htb]{|c|c||c|c||c|c||c||c|c|c|c|c|c|} 
\hline 
 Retraite en &  Âge &  Âge pivot &  Décote/Surcote &  Retraite (\euro{} 2019) &  Tx Rempl(\%) &  SMIC (\euro{} 2019) &  Retraite/SMIC &  Rev70/SMIC &  Rev75/SMIC &  Rev80/SMIC &  Rev85/SMIC &  Rev90/SMIC \\ 
\hline \hline 
 2052 &  62 &  65 ans 0 mois &  -15.00\% &  1206.15 &  {\bf 46.37} &  2601.14 &  {\bf {\color{red} 0.46}} &  {\bf {\color{red} 0.42}} &  {\bf {\color{red} 0.39}} &  {\bf {\color{red} 0.37}} &  {\bf {\color{red} 0.34}} &  {\bf {\color{red} 0.32}} \\ 
\hline 
 2053 &  63 &  65 ans 0 mois &  -10.00\% &  1326.71 &  {\bf 50.35} &  2634.96 &  {\bf {\color{red} 0.50}} &  {\bf {\color{red} 0.46}} &  {\bf {\color{red} 0.43}} &  {\bf {\color{red} 0.40}} &  {\bf {\color{red} 0.38}} &  {\bf {\color{red} 0.36}} \\ 
\hline 
 2054 &  64 &  65 ans 0 mois &  -5.00\% &  1453.92 &  {\bf 54.47} &  2669.21 &  {\bf {\color{red} 0.54}} &  {\bf {\color{red} 0.50}} &  {\bf {\color{red} 0.47}} &  {\bf {\color{red} 0.44}} &  {\bf {\color{red} 0.42}} &  {\bf {\color{red} 0.39}} \\ 
\hline 
 2055 &  65 &  65 ans 0 mois &  0.00\% &  2298.33 &  {\bf 85.00} &  2703.91 &  {\bf {\color{red} 0.85}} &  {\bf {\color{red} 0.80}} &  {\bf {\color{red} 0.75}} &  {\bf {\color{red} 0.70}} &  {\bf {\color{red} 0.66}} &  {\bf {\color{red} 0.62}} \\ 
\hline 
 2056 &  66 &  65 ans 0 mois &  5.00\% &  2328.20 &  {\bf 85.00} &  2739.06 &  {\bf {\color{red} 0.85}} &  {\bf {\color{red} 0.81}} &  {\bf {\color{red} 0.76}} &  {\bf {\color{red} 0.71}} &  {\bf {\color{red} 0.67}} &  {\bf {\color{red} 0.62}} \\ 
\hline 
 2057 &  67 &  65 ans 0 mois &  10.00\% &  2358.47 &  {\bf 85.00} &  2774.67 &  {\bf {\color{red} 0.85}} &  {\bf {\color{red} 0.82}} &  {\bf {\color{red} 0.77}} &  {\bf {\color{red} 0.72}} &  {\bf {\color{red} 0.67}} &  {\bf {\color{red} 0.63}} \\ 
\hline 
\hline 
\end{tabular} 
\end{center} } 
\paragraph{Retraites possibles et ratios Revenu/SMIC à 70, 75, 80, 85, 90 ans avec le modèle \emph{Gouvernement corrigé (âge-pivot glissant)}}  
 
{ \scriptsize \begin{center} 
\begin{tabular}[htb]{|c|c||c|c||c|c||c||c|c|c|c|c|c|} 
\hline 
 Retraite en &  Âge &  Âge pivot &  Décote/Surcote &  Retraite (\euro{} 2019) &  Tx Rempl(\%) &  SMIC (\euro{} 2019) &  Retraite/SMIC &  Rev70/SMIC &  Rev75/SMIC &  Rev80/SMIC &  Rev85/SMIC &  Rev90/SMIC \\ 
\hline \hline 
 2052 &  62 &  66 ans 1 mois &  -20.42\% &  1129.28 &  {\bf 43.41} &  2601.14 &  {\bf {\color{red} 0.43}} &  {\bf {\color{red} 0.39}} &  {\bf {\color{red} 0.37}} &  {\bf {\color{red} 0.34}} &  {\bf {\color{red} 0.32}} &  {\bf {\color{red} 0.30}} \\ 
\hline 
 2053 &  63 &  66 ans 2 mois &  -15.83\% &  1240.72 &  {\bf 47.09} &  2634.96 &  {\bf {\color{red} 0.47}} &  {\bf {\color{red} 0.43}} &  {\bf {\color{red} 0.40}} &  {\bf {\color{red} 0.38}} &  {\bf {\color{red} 0.35}} &  {\bf {\color{red} 0.33}} \\ 
\hline 
 2054 &  64 &  66 ans 3 mois &  -11.25\% &  1358.26 &  {\bf 50.89} &  2669.21 &  {\bf {\color{red} 0.51}} &  {\bf {\color{red} 0.47}} &  {\bf {\color{red} 0.44}} &  {\bf {\color{red} 0.41}} &  {\bf {\color{red} 0.39}} &  {\bf {\color{red} 0.36}} \\ 
\hline 
 2055 &  65 &  66 ans 4 mois &  -6.67\% &  2298.33 &  {\bf 85.00} &  2703.91 &  {\bf {\color{red} 0.85}} &  {\bf {\color{red} 0.80}} &  {\bf {\color{red} 0.75}} &  {\bf {\color{red} 0.70}} &  {\bf {\color{red} 0.66}} &  {\bf {\color{red} 0.62}} \\ 
\hline 
 2056 &  66 &  66 ans 5 mois &  -2.08\% &  2328.20 &  {\bf 85.00} &  2739.06 &  {\bf {\color{red} 0.85}} &  {\bf {\color{red} 0.81}} &  {\bf {\color{red} 0.76}} &  {\bf {\color{red} 0.71}} &  {\bf {\color{red} 0.67}} &  {\bf {\color{red} 0.62}} \\ 
\hline 
 2057 &  67 &  66 ans 6 mois &  2.50\% &  2358.47 &  {\bf 85.00} &  2774.67 &  {\bf {\color{red} 0.85}} &  {\bf {\color{red} 0.82}} &  {\bf {\color{red} 0.77}} &  {\bf {\color{red} 0.72}} &  {\bf {\color{red} 0.67}} &  {\bf {\color{red} 0.63}} \\ 
\hline 
\hline 
\end{tabular} 
\end{center} } 
\paragraph{Retraites possibles et ratios Revenu/SMIC à 70, 75, 80, 85, 90 ans avec le modèle \emph{Destinie2 (revalorisation de la fonction publique)}}  
 
{ \scriptsize \begin{center} 
\begin{tabular}[htb]{|c|c||c|c||c|c||c||c|c|c|c|c|c|} 
\hline 
 Retraite en &  Âge &  Âge pivot &  Décote/Surcote &  Retraite (\euro{} 2019) &  Tx Rempl(\%) &  SMIC (\euro{} 2019) &  Retraite/SMIC &  Rev70/SMIC &  Rev75/SMIC &  Rev80/SMIC &  Rev85/SMIC &  Rev90/SMIC \\ 
\hline \hline 
 2052 &  62 &  66 ans 1 mois &  -20.42\% &  1414.52 &  {\bf 36.26} &  2445.56 &  {\bf {\color{red} 0.58}} &  {\bf {\color{red} 0.52}} &  {\bf {\color{red} 0.49}} &  {\bf {\color{red} 0.46}} &  {\bf {\color{red} 0.43}} &  {\bf {\color{red} 0.40}} \\ 
\hline 
 2053 &  63 &  66 ans 2 mois &  -15.83\% &  1561.72 &  {\bf 39.52} &  2477.35 &  {\bf {\color{red} 0.63}} &  {\bf {\color{red} 0.58}} &  {\bf {\color{red} 0.54}} &  {\bf {\color{red} 0.51}} &  {\bf {\color{red} 0.47}} &  {\bf {\color{red} 0.44}} \\ 
\hline 
 2054 &  64 &  66 ans 3 mois &  -11.25\% &  1717.62 &  {\bf 42.91} &  2509.56 &  {\bf {\color{red} 0.68}} &  {\bf {\color{red} 0.63}} &  {\bf {\color{red} 0.59}} &  {\bf {\color{red} 0.56}} &  {\bf {\color{red} 0.52}} &  {\bf {\color{red} 0.49}} \\ 
\hline 
 2055 &  65 &  66 ans 4 mois &  -6.67\% &  2160.85 &  {\bf 53.29} &  2542.18 &  {\bf {\color{red} 0.85}} &  {\bf {\color{red} 0.80}} &  {\bf {\color{red} 0.75}} &  {\bf {\color{red} 0.70}} &  {\bf {\color{red} 0.66}} &  {\bf {\color{red} 0.62}} \\ 
\hline 
 2056 &  66 &  66 ans 5 mois &  -2.08\% &  2188.95 &  {\bf 53.29} &  2575.23 &  {\bf {\color{red} 0.85}} &  {\bf {\color{red} 0.81}} &  {\bf {\color{red} 0.76}} &  {\bf {\color{red} 0.71}} &  {\bf {\color{red} 0.67}} &  {\bf {\color{red} 0.62}} \\ 
\hline 
 2057 &  67 &  66 ans 6 mois &  2.50\% &  2240.20 &  {\bf 53.84} &  2608.71 &  {\bf {\color{red} 0.86}} &  {\bf {\color{red} 0.83}} &  {\bf {\color{red} 0.77}} &  {\bf {\color{red} 0.73}} &  {\bf {\color{red} 0.68}} &  {\bf {\color{red} 0.64}} \\ 
\hline 
\hline 
\end{tabular} 
\end{center} } 

 \begin{center}\includegraphics[width=0.9\textwidth]{fig/ATSEM_1990_22_dest_retraite.pdf}\end{center} \label{fig/ATSEM_1990_22_dest_retraite.pdf} 

\newpage 
 
\subsection{Génération 2003 (début en 2025)} 

\paragraph{Retraites possibles et ratios Revenu/SMIC à 70, 75, 80, 85, 90 ans avec le modèle \emph{Gouvernement truqué (âge-pivot bloqué à 65 ans)}}  
 
{ \scriptsize \begin{center} 
\begin{tabular}[htb]{|c|c||c|c||c|c||c||c|c|c|c|c|c|} 
\hline 
 Retraite en &  Âge &  Âge pivot &  Décote/Surcote &  Retraite (\euro{} 2019) &  Tx Rempl(\%) &  SMIC (\euro{} 2019) &  Retraite/SMIC &  Rev70/SMIC &  Rev75/SMIC &  Rev80/SMIC &  Rev85/SMIC &  Rev90/SMIC \\ 
\hline \hline 
 2065 &  62 &  65 ans 0 mois &  -15.00\% &  1457.41 &  {\bf 47.37} &  3076.71 &  {\bf {\color{red} 0.47}} &  {\bf {\color{red} 0.43}} &  {\bf {\color{red} 0.40}} &  {\bf {\color{red} 0.38}} &  {\bf {\color{red} 0.35}} &  {\bf {\color{red} 0.33}} \\ 
\hline 
 2066 &  63 &  65 ans 0 mois &  -10.00\% &  1602.25 &  {\bf 51.41} &  3116.71 &  {\bf {\color{red} 0.51}} &  {\bf {\color{red} 0.47}} &  {\bf {\color{red} 0.44}} &  {\bf {\color{red} 0.41}} &  {\bf {\color{red} 0.39}} &  {\bf {\color{red} 0.36}} \\ 
\hline 
 2067 &  64 &  65 ans 0 mois &  -5.00\% &  1755.00 &  {\bf 55.59} &  3157.23 &  {\bf {\color{red} 0.56}} &  {\bf {\color{red} 0.51}} &  {\bf {\color{red} 0.48}} &  {\bf {\color{red} 0.45}} &  {\bf {\color{red} 0.42}} &  {\bf {\color{red} 0.40}} \\ 
\hline 
 2068 &  65 &  65 ans 0 mois &  0.00\% &  2718.53 &  {\bf 85.00} &  3198.27 &  {\bf {\color{red} 0.85}} &  {\bf {\color{red} 0.80}} &  {\bf {\color{red} 0.75}} &  {\bf {\color{red} 0.70}} &  {\bf {\color{red} 0.66}} &  {\bf {\color{red} 0.62}} \\ 
\hline 
 2069 &  66 &  65 ans 0 mois &  5.00\% &  2753.87 &  {\bf 85.00} &  3239.85 &  {\bf {\color{red} 0.85}} &  {\bf {\color{red} 0.81}} &  {\bf {\color{red} 0.76}} &  {\bf {\color{red} 0.71}} &  {\bf {\color{red} 0.67}} &  {\bf {\color{red} 0.62}} \\ 
\hline 
 2070 &  67 &  65 ans 0 mois &  10.00\% &  2789.67 &  {\bf 85.00} &  3281.97 &  {\bf {\color{red} 0.85}} &  {\bf {\color{red} 0.82}} &  {\bf {\color{red} 0.77}} &  {\bf {\color{red} 0.72}} &  {\bf {\color{red} 0.67}} &  {\bf {\color{red} 0.63}} \\ 
\hline 
\hline 
\end{tabular} 
\end{center} } 
\paragraph{Retraites possibles et ratios Revenu/SMIC à 70, 75, 80, 85, 90 ans avec le modèle \emph{Gouvernement corrigé (âge-pivot glissant)}}  
 
{ \scriptsize \begin{center} 
\begin{tabular}[htb]{|c|c||c|c||c|c||c||c|c|c|c|c|c|} 
\hline 
 Retraite en &  Âge &  Âge pivot &  Décote/Surcote &  Retraite (\euro{} 2019) &  Tx Rempl(\%) &  SMIC (\euro{} 2019) &  Retraite/SMIC &  Rev70/SMIC &  Rev75/SMIC &  Rev80/SMIC &  Rev85/SMIC &  Rev90/SMIC \\ 
\hline \hline 
 2065 &  62 &  67 ans 2 mois &  -25.83\% &  1271.67 &  {\bf 41.33} &  3076.71 &  {\bf {\color{red} 0.41}} &  {\bf {\color{red} 0.37}} &  {\bf {\color{red} 0.35}} &  {\bf {\color{red} 0.33}} &  {\bf {\color{red} 0.31}} &  {\bf {\color{red} 0.29}} \\ 
\hline 
 2066 &  63 &  67 ans 3 mois &  -21.25\% &  1401.97 &  {\bf 44.98} &  3116.71 &  {\bf {\color{red} 0.45}} &  {\bf {\color{red} 0.41}} &  {\bf {\color{red} 0.39}} &  {\bf {\color{red} 0.36}} &  {\bf {\color{red} 0.34}} &  {\bf {\color{red} 0.32}} \\ 
\hline 
 2067 &  64 &  67 ans 4 mois &  -16.67\% &  1539.47 &  {\bf 48.76} &  3157.23 &  {\bf {\color{red} 0.49}} &  {\bf {\color{red} 0.45}} &  {\bf {\color{red} 0.42}} &  {\bf {\color{red} 0.40}} &  {\bf {\color{red} 0.37}} &  {\bf {\color{red} 0.35}} \\ 
\hline 
 2068 &  65 &  67 ans 5 mois &  -12.08\% &  2718.53 &  {\bf 85.00} &  3198.27 &  {\bf {\color{red} 0.85}} &  {\bf {\color{red} 0.80}} &  {\bf {\color{red} 0.75}} &  {\bf {\color{red} 0.70}} &  {\bf {\color{red} 0.66}} &  {\bf {\color{red} 0.62}} \\ 
\hline 
 2069 &  66 &  67 ans 6 mois &  -7.50\% &  2753.87 &  {\bf 85.00} &  3239.85 &  {\bf {\color{red} 0.85}} &  {\bf {\color{red} 0.81}} &  {\bf {\color{red} 0.76}} &  {\bf {\color{red} 0.71}} &  {\bf {\color{red} 0.67}} &  {\bf {\color{red} 0.62}} \\ 
\hline 
 2070 &  67 &  67 ans 7 mois &  -2.92\% &  2789.67 &  {\bf 85.00} &  3281.97 &  {\bf {\color{red} 0.85}} &  {\bf {\color{red} 0.82}} &  {\bf {\color{red} 0.77}} &  {\bf {\color{red} 0.72}} &  {\bf {\color{red} 0.67}} &  {\bf {\color{red} 0.63}} \\ 
\hline 
\hline 
\end{tabular} 
\end{center} } 
\paragraph{Retraites possibles et ratios Revenu/SMIC à 70, 75, 80, 85, 90 ans avec le modèle \emph{Destinie2 (revalorisation de la fonction publique)}}  
 
{ \scriptsize \begin{center} 
\begin{tabular}[htb]{|c|c||c|c||c|c||c||c|c|c|c|c|c|} 
\hline 
 Retraite en &  Âge &  Âge pivot &  Décote/Surcote &  Retraite (\euro{} 2019) &  Tx Rempl(\%) &  SMIC (\euro{} 2019) &  Retraite/SMIC &  Rev70/SMIC &  Rev75/SMIC &  Rev80/SMIC &  Rev85/SMIC &  Rev90/SMIC \\ 
\hline \hline 
 2065 &  62 &  67 ans 2 mois &  -25.83\% &  1640.50 &  {\bf 35.56} &  2892.68 &  {\bf {\color{red} 0.57}} &  {\bf {\color{red} 0.51}} &  {\bf {\color{red} 0.48}} &  {\bf {\color{red} 0.45}} &  {\bf {\color{red} 0.42}} &  {\bf {\color{red} 0.40}} \\ 
\hline 
 2066 &  63 &  67 ans 3 mois &  -21.25\% &  1815.76 &  {\bf 38.85} &  2930.29 &  {\bf {\color{red} 0.62}} &  {\bf {\color{red} 0.57}} &  {\bf {\color{red} 0.53}} &  {\bf {\color{red} 0.50}} &  {\bf {\color{red} 0.47}} &  {\bf {\color{red} 0.44}} \\ 
\hline 
 2067 &  64 &  67 ans 4 mois &  -16.67\% &  2001.34 &  {\bf 42.27} &  2968.38 &  {\bf {\color{red} 0.67}} &  {\bf {\color{red} 0.62}} &  {\bf {\color{red} 0.58}} &  {\bf {\color{red} 0.55}} &  {\bf {\color{red} 0.51}} &  {\bf {\color{red} 0.48}} \\ 
\hline 
 2068 &  65 &  67 ans 5 mois &  -12.08\% &  2555.93 &  {\bf 53.29} &  3006.97 &  {\bf {\color{red} 0.85}} &  {\bf {\color{red} 0.80}} &  {\bf {\color{red} 0.75}} &  {\bf {\color{red} 0.70}} &  {\bf {\color{red} 0.66}} &  {\bf {\color{red} 0.62}} \\ 
\hline 
 2069 &  66 &  67 ans 6 mois &  -7.50\% &  2589.15 &  {\bf 53.29} &  3046.06 &  {\bf {\color{red} 0.85}} &  {\bf {\color{red} 0.81}} &  {\bf {\color{red} 0.76}} &  {\bf {\color{red} 0.71}} &  {\bf {\color{red} 0.67}} &  {\bf {\color{red} 0.62}} \\ 
\hline 
 2070 &  67 &  67 ans 7 mois &  -2.92\% &  2623.20 &  {\bf 53.30} &  3085.66 &  {\bf {\color{red} 0.85}} &  {\bf {\color{red} 0.82}} &  {\bf {\color{red} 0.77}} &  {\bf {\color{red} 0.72}} &  {\bf {\color{red} 0.67}} &  {\bf {\color{red} 0.63}} \\ 
\hline 
\hline 
\end{tabular} 
\end{center} } 

 \begin{center}\includegraphics[width=0.9\textwidth]{fig/ATSEM_2003_22_dest_retraite.pdf}\end{center} \label{fig/ATSEM_2003_22_dest_retraite.pdf} 

\newpage 
 
\chapter{Professeur des écoles} 

\begin{minipage}{0.55\linewidth}\includegraphics[width=0.7\textwidth]{fig/grille_ProfEcoles.pdf}\end{minipage} 
\begin{minipage}{0.3\linewidth} 
 \begin{center} 

\begin{tabular}[htb]{|c|c|} 
\hline 
 Indice majoré &  Durée (années) \\ 
\hline \hline 
 450 &  1.00 \\ 
\hline 
 498 &  1.00 \\ 
\hline 
 513 &  2.00 \\ 
\hline 
 542 &  2.00 \\ 
\hline 
 579 &  2.50 \\ 
\hline 
 618 &  3.00 \\ 
\hline 
 710 &  3.50 \\ 
\hline 
 757 &  2.00 \\ 
\hline 
 800 &  2.00 \\ 
\hline 
 830 &   \\ 
\hline 
\hline 
\end{tabular} 
\end{center} 
 \end{minipage} 


 \addto{\captionsenglish}{ \renewcommand{\mtctitle}{}} \setcounter{minitocdepth}{2} 
 \minitoc \newpage 

\section{Début de carrière à 22 ans} 

\subsection{Génération 1975 (début en 1997)} 

\paragraph{Retraites possibles et ratios Revenu/SMIC à 70, 75, 80, 85, 90 ans avec le modèle \emph{Gouvernement truqué (âge-pivot bloqué à 65 ans)}}  
 
{ \scriptsize \begin{center} 
\begin{tabular}[htb]{|c|c||c|c||c|c||c||c|c|c|c|c|c|} 
\hline 
 Retraite en &  Âge &  Âge pivot &  Décote/Surcote &  Retraite (\euro{} 2019) &  Tx Rempl(\%) &  SMIC (\euro{} 2019) &  Retraite/SMIC &  Rev70/SMIC &  Rev75/SMIC &  Rev80/SMIC &  Rev85/SMIC &  Rev90/SMIC \\ 
\hline \hline 
 2037 &  62 &  64 ans 10 mois &  -14.17\% &  1902.09 &  {\bf 44.50} &  2143.00 &  {\bf {\color{red} 0.89}} &  {\bf {\color{red} 0.80}} &  {\bf {\color{red} 0.75}} &  {\bf {\color{red} 0.70}} &  {\bf {\color{red} 0.66}} &  {\bf {\color{red} 0.62}} \\ 
\hline 
 2038 &  63 &  64 ans 11 mois &  -9.58\% &  2072.84 &  {\bf 48.39} &  2170.86 &  {\bf {\color{red} 0.95}} &  {\bf {\color{red} 0.87}} &  {\bf {\color{red} 0.82}} &  {\bf {\color{red} 0.77}} &  {\bf {\color{red} 0.72}} &  {\bf {\color{red} 0.67}} \\ 
\hline 
 2039 &  64 &  65 ans 0 mois &  -5.00\% &  2252.96 &  {\bf 52.49} &  2199.08 &  {\bf 1.02} &  {\bf {\color{red} 0.95}} &  {\bf {\color{red} 0.89}} &  {\bf {\color{red} 0.83}} &  {\bf {\color{red} 0.78}} &  {\bf {\color{red} 0.73}} \\ 
\hline 
 2040 &  65 &  65 ans 0 mois &  0.00\% &  2453.13 &  {\bf 57.03} &  2227.67 &  {\bf 1.10} &  {\bf 1.03} &  {\bf {\color{red} 0.97}} &  {\bf {\color{red} 0.91}} &  {\bf {\color{red} 0.85}} &  {\bf {\color{red} 0.80}} \\ 
\hline 
 2041 &  66 &  65 ans 0 mois &  5.00\% &  2664.36 &  {\bf 61.81} &  2256.63 &  {\bf 1.18} &  {\bf 1.12} &  {\bf 1.05} &  {\bf {\color{red} 0.99}} &  {\bf {\color{red} 0.92}} &  {\bf {\color{red} 0.87}} \\ 
\hline 
 2042 &  67 &  65 ans 0 mois &  10.00\% &  2887.23 &  {\bf 66.83} &  2285.97 &  {\bf 1.26} &  {\bf 1.22} &  {\bf 1.14} &  {\bf 1.07} &  {\bf 1.00} &  {\bf {\color{red} 0.94}} \\ 
\hline 
\hline 
\end{tabular} 
\end{center} } 
\paragraph{Retraites possibles et ratios Revenu/SMIC à 70, 75, 80, 85, 90 ans avec le modèle \emph{Gouvernement corrigé (âge-pivot glissant)}}  
 
{ \scriptsize \begin{center} 
\begin{tabular}[htb]{|c|c||c|c||c|c||c||c|c|c|c|c|c|} 
\hline 
 Retraite en &  Âge &  Âge pivot &  Décote/Surcote &  Retraite (\euro{} 2019) &  Tx Rempl(\%) &  SMIC (\euro{} 2019) &  Retraite/SMIC &  Rev70/SMIC &  Rev75/SMIC &  Rev80/SMIC &  Rev85/SMIC &  Rev90/SMIC \\ 
\hline \hline 
 2037 &  62 &  64 ans 10 mois &  -14.17\% &  1902.09 &  {\bf 44.50} &  2143.00 &  {\bf {\color{red} 0.89}} &  {\bf {\color{red} 0.80}} &  {\bf {\color{red} 0.75}} &  {\bf {\color{red} 0.70}} &  {\bf {\color{red} 0.66}} &  {\bf {\color{red} 0.62}} \\ 
\hline 
 2038 &  63 &  64 ans 11 mois &  -9.58\% &  2072.84 &  {\bf 48.39} &  2170.86 &  {\bf {\color{red} 0.95}} &  {\bf {\color{red} 0.87}} &  {\bf {\color{red} 0.82}} &  {\bf {\color{red} 0.77}} &  {\bf {\color{red} 0.72}} &  {\bf {\color{red} 0.67}} \\ 
\hline 
 2039 &  64 &  65 ans 0 mois &  -5.00\% &  2252.96 &  {\bf 52.49} &  2199.08 &  {\bf 1.02} &  {\bf {\color{red} 0.95}} &  {\bf {\color{red} 0.89}} &  {\bf {\color{red} 0.83}} &  {\bf {\color{red} 0.78}} &  {\bf {\color{red} 0.73}} \\ 
\hline 
 2040 &  65 &  65 ans 1 mois &  -0.42\% &  2442.91 &  {\bf 56.79} &  2227.67 &  {\bf 1.10} &  {\bf 1.03} &  {\bf {\color{red} 0.96}} &  {\bf {\color{red} 0.90}} &  {\bf {\color{red} 0.85}} &  {\bf {\color{red} 0.79}} \\ 
\hline 
 2041 &  66 &  65 ans 2 mois &  4.17\% &  2643.22 &  {\bf 61.32} &  2256.63 &  {\bf 1.17} &  {\bf 1.11} &  {\bf 1.04} &  {\bf {\color{red} 0.98}} &  {\bf {\color{red} 0.92}} &  {\bf {\color{red} 0.86}} \\ 
\hline 
 2042 &  67 &  65 ans 3 mois &  8.75\% &  2854.42 &  {\bf 66.07} &  2285.97 &  {\bf 1.25} &  {\bf 1.20} &  {\bf 1.13} &  {\bf 1.06} &  {\bf {\color{red} 0.99}} &  {\bf {\color{red} 0.93}} \\ 
\hline 
\hline 
\end{tabular} 
\end{center} } 
\paragraph{Retraites possibles et ratios Revenu/SMIC à 70, 75, 80, 85, 90 ans avec le modèle \emph{Destinie2 (revalorisation de la fonction publique)}}  
 
{ \scriptsize \begin{center} 
\begin{tabular}[htb]{|c|c||c|c||c|c||c||c|c|c|c|c|c|} 
\hline 
 Retraite en &  Âge &  Âge pivot &  Décote/Surcote &  Retraite (\euro{} 2019) &  Tx Rempl(\%) &  SMIC (\euro{} 2019) &  Retraite/SMIC &  Rev70/SMIC &  Rev75/SMIC &  Rev80/SMIC &  Rev85/SMIC &  Rev90/SMIC \\ 
\hline \hline 
 2037 &  62 &  64 ans 10 mois &  -14.17\% &  2101.92 &  {\bf 39.41} &  2014.82 &  {\bf 1.04} &  {\bf {\color{red} 0.94}} &  {\bf {\color{red} 0.88}} &  {\bf {\color{red} 0.83}} &  {\bf {\color{red} 0.78}} &  {\bf {\color{red} 0.73}} \\ 
\hline 
 2038 &  63 &  64 ans 11 mois &  -9.58\% &  2299.05 &  {\bf 42.55} &  2041.01 &  {\bf 1.13} &  {\bf 1.03} &  {\bf {\color{red} 0.96}} &  {\bf {\color{red} 0.90}} &  {\bf {\color{red} 0.85}} &  {\bf {\color{red} 0.79}} \\ 
\hline 
 2039 &  64 &  65 ans 0 mois &  -5.00\% &  2508.24 &  {\bf 45.82} &  2067.55 &  {\bf 1.21} &  {\bf 1.12} &  {\bf 1.05} &  {\bf {\color{red} 0.99}} &  {\bf {\color{red} 0.92}} &  {\bf {\color{red} 0.87}} \\ 
\hline 
 2040 &  65 &  65 ans 1 mois &  -0.42\% &  2730.19 &  {\bf 49.24} &  2094.43 &  {\bf 1.30} &  {\bf 1.22} &  {\bf 1.15} &  {\bf 1.07} &  {\bf 1.01} &  {\bf {\color{red} 0.94}} \\ 
\hline 
 2041 &  66 &  65 ans 2 mois &  4.17\% &  2965.64 &  {\bf 52.80} &  2121.65 &  {\bf 1.40} &  {\bf 1.33} &  {\bf 1.24} &  {\bf 1.17} &  {\bf 1.09} &  {\bf 1.03} \\ 
\hline 
 2042 &  67 &  65 ans 3 mois &  8.75\% &  3215.36 &  {\bf 56.51} &  2149.23 &  {\bf 1.50} &  {\bf 1.44} &  {\bf 1.35} &  {\bf 1.26} &  {\bf 1.19} &  {\bf 1.11} \\ 
\hline 
\hline 
\end{tabular} 
\end{center} } 

 \begin{center}\includegraphics[width=0.9\textwidth]{fig/ProfEcoles_1975_22_dest_retraite.pdf}\end{center} \label{fig/ProfEcoles_1975_22_dest_retraite.pdf} 

\newpage 
 
\subsection{Génération 1980 (début en 2002)} 

\paragraph{Retraites possibles et ratios Revenu/SMIC à 70, 75, 80, 85, 90 ans avec le modèle \emph{Gouvernement truqué (âge-pivot bloqué à 65 ans)}}  
 
{ \scriptsize \begin{center} 
\begin{tabular}[htb]{|c|c||c|c||c|c||c||c|c|c|c|c|c|} 
\hline 
 Retraite en &  Âge &  Âge pivot &  Décote/Surcote &  Retraite (\euro{} 2019) &  Tx Rempl(\%) &  SMIC (\euro{} 2019) &  Retraite/SMIC &  Rev70/SMIC &  Rev75/SMIC &  Rev80/SMIC &  Rev85/SMIC &  Rev90/SMIC \\ 
\hline \hline 
 2042 &  62 &  65 ans 0 mois &  -15.00\% &  1925.24 &  {\bf 45.04} &  2285.97 &  {\bf {\color{red} 0.84}} &  {\bf {\color{red} 0.76}} &  {\bf {\color{red} 0.71}} &  {\bf {\color{red} 0.67}} &  {\bf {\color{red} 0.63}} &  {\bf {\color{red} 0.59}} \\ 
\hline 
 2043 &  63 &  65 ans 0 mois &  -10.00\% &  2115.51 &  {\bf 49.39} &  2315.68 &  {\bf {\color{red} 0.91}} &  {\bf {\color{red} 0.83}} &  {\bf {\color{red} 0.78}} &  {\bf {\color{red} 0.73}} &  {\bf {\color{red} 0.69}} &  {\bf {\color{red} 0.64}} \\ 
\hline 
 2044 &  64 &  65 ans 0 mois &  -5.00\% &  2317.12 &  {\bf 53.98} &  2345.79 &  {\bf {\color{red} 0.99}} &  {\bf {\color{red} 0.91}} &  {\bf {\color{red} 0.86}} &  {\bf {\color{red} 0.80}} &  {\bf {\color{red} 0.75}} &  {\bf {\color{red} 0.71}} \\ 
\hline 
 2045 &  65 &  65 ans 0 mois &  0.00\% &  2530.65 &  {\bf 58.83} &  2376.28 &  {\bf 1.06} &  {\bf {\color{red} 1.00}} &  {\bf {\color{red} 0.94}} &  {\bf {\color{red} 0.88}} &  {\bf {\color{red} 0.82}} &  {\bf {\color{red} 0.77}} \\ 
\hline 
 2046 &  66 &  65 ans 0 mois &  5.00\% &  2754.73 &  {\bf 63.90} &  2407.18 &  {\bf 1.14} &  {\bf 1.09} &  {\bf 1.02} &  {\bf {\color{red} 0.96}} &  {\bf {\color{red} 0.90}} &  {\bf {\color{red} 0.84}} \\ 
\hline 
 2047 &  67 &  65 ans 0 mois &  10.00\% &  2989.57 &  {\bf 69.20} &  2438.47 &  {\bf 1.23} &  {\bf 1.18} &  {\bf 1.11} &  {\bf 1.04} &  {\bf {\color{red} 0.97}} &  {\bf {\color{red} 0.91}} \\ 
\hline 
\hline 
\end{tabular} 
\end{center} } 
\paragraph{Retraites possibles et ratios Revenu/SMIC à 70, 75, 80, 85, 90 ans avec le modèle \emph{Gouvernement corrigé (âge-pivot glissant)}}  
 
{ \scriptsize \begin{center} 
\begin{tabular}[htb]{|c|c||c|c||c|c||c||c|c|c|c|c|c|} 
\hline 
 Retraite en &  Âge &  Âge pivot &  Décote/Surcote &  Retraite (\euro{} 2019) &  Tx Rempl(\%) &  SMIC (\euro{} 2019) &  Retraite/SMIC &  Rev70/SMIC &  Rev75/SMIC &  Rev80/SMIC &  Rev85/SMIC &  Rev90/SMIC \\ 
\hline \hline 
 2042 &  62 &  65 ans 3 mois &  -16.25\% &  1896.93 &  {\bf 44.38} &  2285.97 &  {\bf {\color{red} 0.83}} &  {\bf {\color{red} 0.75}} &  {\bf {\color{red} 0.70}} &  {\bf {\color{red} 0.66}} &  {\bf {\color{red} 0.62}} &  {\bf {\color{red} 0.58}} \\ 
\hline 
 2043 &  63 &  65 ans 4 mois &  -11.67\% &  2076.34 &  {\bf 48.47} &  2315.68 &  {\bf {\color{red} 0.90}} &  {\bf {\color{red} 0.82}} &  {\bf {\color{red} 0.77}} &  {\bf {\color{red} 0.72}} &  {\bf {\color{red} 0.67}} &  {\bf {\color{red} 0.63}} \\ 
\hline 
 2044 &  64 &  65 ans 5 mois &  -7.08\% &  2266.30 &  {\bf 52.80} &  2345.79 &  {\bf {\color{red} 0.97}} &  {\bf {\color{red} 0.89}} &  {\bf {\color{red} 0.84}} &  {\bf {\color{red} 0.79}} &  {\bf {\color{red} 0.74}} &  {\bf {\color{red} 0.69}} \\ 
\hline 
 2045 &  65 &  65 ans 6 mois &  -2.50\% &  2467.39 &  {\bf 57.36} &  2376.28 &  {\bf 1.04} &  {\bf {\color{red} 0.97}} &  {\bf {\color{red} 0.91}} &  {\bf {\color{red} 0.86}} &  {\bf {\color{red} 0.80}} &  {\bf {\color{red} 0.75}} \\ 
\hline 
 2046 &  66 &  65 ans 7 mois &  2.08\% &  2678.21 &  {\bf 62.13} &  2407.18 &  {\bf 1.11} &  {\bf 1.06} &  {\bf {\color{red} 0.99}} &  {\bf {\color{red} 0.93}} &  {\bf {\color{red} 0.87}} &  {\bf {\color{red} 0.82}} \\ 
\hline 
 2047 &  67 &  65 ans 8 mois &  6.67\% &  2898.98 &  {\bf 67.11} &  2438.47 &  {\bf 1.19} &  {\bf 1.14} &  {\bf 1.07} &  {\bf 1.01} &  {\bf {\color{red} 0.94}} &  {\bf {\color{red} 0.88}} \\ 
\hline 
\hline 
\end{tabular} 
\end{center} } 
\paragraph{Retraites possibles et ratios Revenu/SMIC à 70, 75, 80, 85, 90 ans avec le modèle \emph{Destinie2 (revalorisation de la fonction publique)}}  
 
{ \scriptsize \begin{center} 
\begin{tabular}[htb]{|c|c||c|c||c|c||c||c|c|c|c|c|c|} 
\hline 
 Retraite en &  Âge &  Âge pivot &  Décote/Surcote &  Retraite (\euro{} 2019) &  Tx Rempl(\%) &  SMIC (\euro{} 2019) &  Retraite/SMIC &  Rev70/SMIC &  Rev75/SMIC &  Rev80/SMIC &  Rev85/SMIC &  Rev90/SMIC \\ 
\hline \hline 
 2042 &  62 &  65 ans 3 mois &  -16.25\% &  2171.58 &  {\bf 38.17} &  2149.23 &  {\bf 1.01} &  {\bf {\color{red} 0.91}} &  {\bf {\color{red} 0.85}} &  {\bf {\color{red} 0.80}} &  {\bf {\color{red} 0.75}} &  {\bf {\color{red} 0.70}} \\ 
\hline 
 2043 &  63 &  65 ans 4 mois &  -11.67\% &  2387.54 &  {\bf 41.42} &  2177.17 &  {\bf 1.10} &  {\bf 1.00} &  {\bf {\color{red} 0.94}} &  {\bf {\color{red} 0.88}} &  {\bf {\color{red} 0.83}} &  {\bf {\color{red} 0.77}} \\ 
\hline 
 2044 &  64 &  65 ans 5 mois &  -7.08\% &  2617.66 &  {\bf 44.83} &  2205.48 &  {\bf 1.19} &  {\bf 1.10} &  {\bf 1.03} &  {\bf {\color{red} 0.97}} &  {\bf {\color{red} 0.90}} &  {\bf {\color{red} 0.85}} \\ 
\hline 
 2045 &  65 &  65 ans 6 mois &  -2.50\% &  2862.76 &  {\bf 48.40} &  2234.15 &  {\bf 1.28} &  {\bf 1.20} &  {\bf 1.13} &  {\bf 1.06} &  {\bf {\color{red} 0.99}} &  {\bf {\color{red} 0.93}} \\ 
\hline 
 2046 &  66 &  65 ans 7 mois &  2.08\% &  3121.43 &  {\bf 52.10} &  2263.19 &  {\bf 1.38} &  {\bf 1.31} &  {\bf 1.23} &  {\bf 1.15} &  {\bf 1.08} &  {\bf 1.01} \\ 
\hline 
 2047 &  67 &  65 ans 8 mois &  6.67\% &  3394.09 &  {\bf 55.92} &  2292.61 &  {\bf 1.48} &  {\bf 1.42} &  {\bf 1.34} &  {\bf 1.25} &  {\bf 1.17} &  {\bf 1.10} \\ 
\hline 
\hline 
\end{tabular} 
\end{center} } 

 \begin{center}\includegraphics[width=0.9\textwidth]{fig/ProfEcoles_1980_22_dest_retraite.pdf}\end{center} \label{fig/ProfEcoles_1980_22_dest_retraite.pdf} 

\newpage 
 
\subsection{Génération 1990 (début en 2012)} 

\paragraph{Retraites possibles et ratios Revenu/SMIC à 70, 75, 80, 85, 90 ans avec le modèle \emph{Gouvernement truqué (âge-pivot bloqué à 65 ans)}}  
 
{ \scriptsize \begin{center} 
\begin{tabular}[htb]{|c|c||c|c||c|c||c||c|c|c|c|c|c|} 
\hline 
 Retraite en &  Âge &  Âge pivot &  Décote/Surcote &  Retraite (\euro{} 2019) &  Tx Rempl(\%) &  SMIC (\euro{} 2019) &  Retraite/SMIC &  Rev70/SMIC &  Rev75/SMIC &  Rev80/SMIC &  Rev85/SMIC &  Rev90/SMIC \\ 
\hline \hline 
 2052 &  62 &  65 ans 0 mois &  -15.00\% &  2069.04 &  {\bf 48.41} &  2601.14 &  {\bf {\color{red} 0.80}} &  {\bf {\color{red} 0.72}} &  {\bf {\color{red} 0.67}} &  {\bf {\color{red} 0.63}} &  {\bf {\color{red} 0.59}} &  {\bf {\color{red} 0.55}} \\ 
\hline 
 2053 &  63 &  65 ans 0 mois &  -10.00\% &  2272.88 &  {\bf 53.06} &  2634.96 &  {\bf {\color{red} 0.86}} &  {\bf {\color{red} 0.79}} &  {\bf {\color{red} 0.74}} &  {\bf {\color{red} 0.69}} &  {\bf {\color{red} 0.65}} &  {\bf {\color{red} 0.61}} \\ 
\hline 
 2054 &  64 &  65 ans 0 mois &  -5.00\% &  2487.11 &  {\bf 57.94} &  2669.21 &  {\bf {\color{red} 0.93}} &  {\bf {\color{red} 0.86}} &  {\bf {\color{red} 0.81}} &  {\bf {\color{red} 0.76}} &  {\bf {\color{red} 0.71}} &  {\bf {\color{red} 0.67}} \\ 
\hline 
 2055 &  65 &  65 ans 0 mois &  0.00\% &  2711.92 &  {\bf 63.04} &  2703.91 &  {\bf 1.00} &  {\bf {\color{red} 0.94}} &  {\bf {\color{red} 0.88}} &  {\bf {\color{red} 0.83}} &  {\bf {\color{red} 0.77}} &  {\bf {\color{red} 0.73}} \\ 
\hline 
 2056 &  66 &  65 ans 0 mois &  5.00\% &  2947.54 &  {\bf 68.37} &  2739.06 &  {\bf 1.08} &  {\bf 1.02} &  {\bf {\color{red} 0.96}} &  {\bf {\color{red} 0.90}} &  {\bf {\color{red} 0.84}} &  {\bf {\color{red} 0.79}} \\ 
\hline 
 2057 &  67 &  65 ans 0 mois &  10.00\% &  3194.18 &  {\bf 73.94} &  2774.67 &  {\bf 1.15} &  {\bf 1.11} &  {\bf 1.04} &  {\bf {\color{red} 0.97}} &  {\bf {\color{red} 0.91}} &  {\bf {\color{red} 0.86}} \\ 
\hline 
\hline 
\end{tabular} 
\end{center} } 
\paragraph{Retraites possibles et ratios Revenu/SMIC à 70, 75, 80, 85, 90 ans avec le modèle \emph{Gouvernement corrigé (âge-pivot glissant)}}  
 
{ \scriptsize \begin{center} 
\begin{tabular}[htb]{|c|c||c|c||c|c||c||c|c|c|c|c|c|} 
\hline 
 Retraite en &  Âge &  Âge pivot &  Décote/Surcote &  Retraite (\euro{} 2019) &  Tx Rempl(\%) &  SMIC (\euro{} 2019) &  Retraite/SMIC &  Rev70/SMIC &  Rev75/SMIC &  Rev80/SMIC &  Rev85/SMIC &  Rev90/SMIC \\ 
\hline \hline 
 2052 &  62 &  66 ans 1 mois &  -20.42\% &  1937.19 &  {\bf 45.32} &  2601.14 &  {\bf {\color{red} 0.74}} &  {\bf {\color{red} 0.67}} &  {\bf {\color{red} 0.63}} &  {\bf {\color{red} 0.59}} &  {\bf {\color{red} 0.55}} &  {\bf {\color{red} 0.52}} \\ 
\hline 
 2053 &  63 &  66 ans 2 mois &  -15.83\% &  2125.57 &  {\bf 49.62} &  2634.96 &  {\bf {\color{red} 0.81}} &  {\bf {\color{red} 0.74}} &  {\bf {\color{red} 0.69}} &  {\bf {\color{red} 0.65}} &  {\bf {\color{red} 0.61}} &  {\bf {\color{red} 0.57}} \\ 
\hline 
 2054 &  64 &  66 ans 3 mois &  -11.25\% &  2323.48 &  {\bf 54.13} &  2669.21 &  {\bf {\color{red} 0.87}} &  {\bf {\color{red} 0.81}} &  {\bf {\color{red} 0.76}} &  {\bf {\color{red} 0.71}} &  {\bf {\color{red} 0.66}} &  {\bf {\color{red} 0.62}} \\ 
\hline 
 2055 &  65 &  66 ans 4 mois &  -6.67\% &  2531.12 &  {\bf 58.84} &  2703.91 &  {\bf {\color{red} 0.94}} &  {\bf {\color{red} 0.88}} &  {\bf {\color{red} 0.82}} &  {\bf {\color{red} 0.77}} &  {\bf {\color{red} 0.72}} &  {\bf {\color{red} 0.68}} \\ 
\hline 
 2056 &  66 &  66 ans 5 mois &  -2.08\% &  2748.69 &  {\bf 63.76} &  2739.06 &  {\bf 1.00} &  {\bf {\color{red} 0.95}} &  {\bf {\color{red} 0.89}} &  {\bf {\color{red} 0.84}} &  {\bf {\color{red} 0.79}} &  {\bf {\color{red} 0.74}} \\ 
\hline 
 2057 &  67 &  66 ans 6 mois &  2.50\% &  2976.40 &  {\bf 68.90} &  2774.67 &  {\bf 1.07} &  {\bf 1.03} &  {\bf {\color{red} 0.97}} &  {\bf {\color{red} 0.91}} &  {\bf {\color{red} 0.85}} &  {\bf {\color{red} 0.80}} \\ 
\hline 
\hline 
\end{tabular} 
\end{center} } 
\paragraph{Retraites possibles et ratios Revenu/SMIC à 70, 75, 80, 85, 90 ans avec le modèle \emph{Destinie2 (revalorisation de la fonction publique)}}  
 
{ \scriptsize \begin{center} 
\begin{tabular}[htb]{|c|c||c|c||c|c||c||c|c|c|c|c|c|} 
\hline 
 Retraite en &  Âge &  Âge pivot &  Décote/Surcote &  Retraite (\euro{} 2019) &  Tx Rempl(\%) &  SMIC (\euro{} 2019) &  Retraite/SMIC &  Rev70/SMIC &  Rev75/SMIC &  Rev80/SMIC &  Rev85/SMIC &  Rev90/SMIC \\ 
\hline \hline 
 2052 &  62 &  66 ans 1 mois &  -20.42\% &  2436.82 &  {\bf 37.64} &  2445.56 &  {\bf {\color{red} 1.00}} &  {\bf {\color{red} 0.90}} &  {\bf {\color{red} 0.84}} &  {\bf {\color{red} 0.79}} &  {\bf {\color{red} 0.74}} &  {\bf {\color{red} 0.69}} \\ 
\hline 
 2053 &  63 &  66 ans 2 mois &  -15.83\% &  2687.50 &  {\bf 40.98} &  2477.35 &  {\bf 1.08} &  {\bf {\color{red} 0.99}} &  {\bf {\color{red} 0.93}} &  {\bf {\color{red} 0.87}} &  {\bf {\color{red} 0.82}} &  {\bf {\color{red} 0.77}} \\ 
\hline 
 2054 &  64 &  66 ans 3 mois &  -11.25\% &  2952.76 &  {\bf 44.44} &  2509.56 &  {\bf 1.18} &  {\bf 1.09} &  {\bf 1.02} &  {\bf {\color{red} 0.96}} &  {\bf {\color{red} 0.90}} &  {\bf {\color{red} 0.84}} \\ 
\hline 
 2055 &  65 &  66 ans 4 mois &  -6.67\% &  3233.05 &  {\bf 48.04} &  2542.18 &  {\bf 1.27} &  {\bf 1.19} &  {\bf 1.12} &  {\bf 1.05} &  {\bf {\color{red} 0.98}} &  {\bf {\color{red} 0.92}} \\ 
\hline 
 2056 &  66 &  66 ans 5 mois &  -2.08\% &  3528.83 &  {\bf 51.76} &  2575.23 &  {\bf 1.37} &  {\bf 1.30} &  {\bf 1.22} &  {\bf 1.14} &  {\bf 1.07} &  {\bf 1.01} \\ 
\hline 
 2057 &  67 &  66 ans 6 mois &  2.50\% &  3840.56 &  {\bf 55.61} &  2608.71 &  {\bf 1.47} &  {\bf 1.42} &  {\bf 1.33} &  {\bf 1.24} &  {\bf 1.17} &  {\bf 1.09} \\ 
\hline 
\hline 
\end{tabular} 
\end{center} } 

 \begin{center}\includegraphics[width=0.9\textwidth]{fig/ProfEcoles_1990_22_dest_retraite.pdf}\end{center} \label{fig/ProfEcoles_1990_22_dest_retraite.pdf} 

\newpage 
 
\subsection{Génération 2003 (début en 2025)} 

\paragraph{Retraites possibles et ratios Revenu/SMIC à 70, 75, 80, 85, 90 ans avec le modèle \emph{Gouvernement truqué (âge-pivot bloqué à 65 ans)}}  
 
{ \scriptsize \begin{center} 
\begin{tabular}[htb]{|c|c||c|c||c|c||c||c|c|c|c|c|c|} 
\hline 
 Retraite en &  Âge &  Âge pivot &  Décote/Surcote &  Retraite (\euro{} 2019) &  Tx Rempl(\%) &  SMIC (\euro{} 2019) &  Retraite/SMIC &  Rev70/SMIC &  Rev75/SMIC &  Rev80/SMIC &  Rev85/SMIC &  Rev90/SMIC \\ 
\hline \hline 
 2065 &  62 &  65 ans 0 mois &  -15.00\% &  2212.01 &  {\bf 51.75} &  3076.71 &  {\bf {\color{red} 0.72}} &  {\bf {\color{red} 0.65}} &  {\bf {\color{red} 0.61}} &  {\bf {\color{red} 0.57}} &  {\bf {\color{red} 0.53}} &  {\bf {\color{red} 0.50}} \\ 
\hline 
 2066 &  63 &  65 ans 0 mois &  -10.00\% &  2426.24 &  {\bf 56.64} &  3116.71 &  {\bf {\color{red} 0.78}} &  {\bf {\color{red} 0.71}} &  {\bf {\color{red} 0.67}} &  {\bf {\color{red} 0.62}} &  {\bf {\color{red} 0.59}} &  {\bf {\color{red} 0.55}} \\ 
\hline 
 2067 &  64 &  65 ans 0 mois &  -5.00\% &  2651.08 &  {\bf 61.76} &  3157.23 &  {\bf {\color{red} 0.84}} &  {\bf {\color{red} 0.78}} &  {\bf {\color{red} 0.73}} &  {\bf {\color{red} 0.68}} &  {\bf {\color{red} 0.64}} &  {\bf {\color{red} 0.60}} \\ 
\hline 
 2068 &  65 &  65 ans 0 mois &  0.00\% &  2886.77 &  {\bf 67.11} &  3198.27 &  {\bf {\color{red} 0.90}} &  {\bf {\color{red} 0.85}} &  {\bf {\color{red} 0.79}} &  {\bf {\color{red} 0.74}} &  {\bf {\color{red} 0.70}} &  {\bf {\color{red} 0.65}} \\ 
\hline 
 2069 &  66 &  65 ans 0 mois &  5.00\% &  3133.52 &  {\bf 72.69} &  3239.85 &  {\bf {\color{red} 0.97}} &  {\bf {\color{red} 0.92}} &  {\bf {\color{red} 0.86}} &  {\bf {\color{red} 0.81}} &  {\bf {\color{red} 0.76}} &  {\bf {\color{red} 0.71}} \\ 
\hline 
 2070 &  67 &  65 ans 0 mois &  10.00\% &  3391.55 &  {\bf 78.51} &  3281.97 &  {\bf 1.03} &  {\bf {\color{red} 0.99}} &  {\bf {\color{red} 0.93}} &  {\bf {\color{red} 0.87}} &  {\bf {\color{red} 0.82}} &  {\bf {\color{red} 0.77}} \\ 
\hline 
\hline 
\end{tabular} 
\end{center} } 
\paragraph{Retraites possibles et ratios Revenu/SMIC à 70, 75, 80, 85, 90 ans avec le modèle \emph{Gouvernement corrigé (âge-pivot glissant)}}  
 
{ \scriptsize \begin{center} 
\begin{tabular}[htb]{|c|c||c|c||c|c||c||c|c|c|c|c|c|} 
\hline 
 Retraite en &  Âge &  Âge pivot &  Décote/Surcote &  Retraite (\euro{} 2019) &  Tx Rempl(\%) &  SMIC (\euro{} 2019) &  Retraite/SMIC &  Rev70/SMIC &  Rev75/SMIC &  Rev80/SMIC &  Rev85/SMIC &  Rev90/SMIC \\ 
\hline \hline 
 2065 &  62 &  67 ans 2 mois &  -25.83\% &  1930.09 &  {\bf 45.16} &  3076.71 &  {\bf {\color{red} 0.63}} &  {\bf {\color{red} 0.57}} &  {\bf {\color{red} 0.53}} &  {\bf {\color{red} 0.50}} &  {\bf {\color{red} 0.47}} &  {\bf {\color{red} 0.44}} \\ 
\hline 
 2066 &  63 &  67 ans 3 mois &  -21.25\% &  2122.96 &  {\bf 49.56} &  3116.71 &  {\bf {\color{red} 0.68}} &  {\bf {\color{red} 0.62}} &  {\bf {\color{red} 0.58}} &  {\bf {\color{red} 0.55}} &  {\bf {\color{red} 0.51}} &  {\bf {\color{red} 0.48}} \\ 
\hline 
 2067 &  64 &  67 ans 4 mois &  -16.67\% &  2325.51 &  {\bf 54.18} &  3157.23 &  {\bf {\color{red} 0.74}} &  {\bf {\color{red} 0.68}} &  {\bf {\color{red} 0.64}} &  {\bf {\color{red} 0.60}} &  {\bf {\color{red} 0.56}} &  {\bf {\color{red} 0.53}} \\ 
\hline 
 2068 &  65 &  67 ans 5 mois &  -12.08\% &  2718.53 &  {\bf 63.20} &  3198.27 &  {\bf {\color{red} 0.85}} &  {\bf {\color{red} 0.80}} &  {\bf {\color{red} 0.75}} &  {\bf {\color{red} 0.70}} &  {\bf {\color{red} 0.66}} &  {\bf {\color{red} 0.62}} \\ 
\hline 
 2069 &  66 &  67 ans 6 mois &  -7.50\% &  2760.48 &  {\bf 64.04} &  3239.85 &  {\bf {\color{red} 0.85}} &  {\bf {\color{red} 0.81}} &  {\bf {\color{red} 0.76}} &  {\bf {\color{red} 0.71}} &  {\bf {\color{red} 0.67}} &  {\bf {\color{red} 0.62}} \\ 
\hline 
 2070 &  67 &  67 ans 7 mois &  -2.92\% &  2993.30 &  {\bf 69.29} &  3281.97 &  {\bf {\color{red} 0.91}} &  {\bf {\color{red} 0.88}} &  {\bf {\color{red} 0.82}} &  {\bf {\color{red} 0.77}} &  {\bf {\color{red} 0.72}} &  {\bf {\color{red} 0.68}} \\ 
\hline 
\hline 
\end{tabular} 
\end{center} } 
\paragraph{Retraites possibles et ratios Revenu/SMIC à 70, 75, 80, 85, 90 ans avec le modèle \emph{Destinie2 (revalorisation de la fonction publique)}}  
 
{ \scriptsize \begin{center} 
\begin{tabular}[htb]{|c|c||c|c||c|c||c||c|c|c|c|c|c|} 
\hline 
 Retraite en &  Âge &  Âge pivot &  Décote/Surcote &  Retraite (\euro{} 2019) &  Tx Rempl(\%) &  SMIC (\euro{} 2019) &  Retraite/SMIC &  Rev70/SMIC &  Rev75/SMIC &  Rev80/SMIC &  Rev85/SMIC &  Rev90/SMIC \\ 
\hline \hline 
 2065 &  62 &  67 ans 2 mois &  -25.83\% &  2828.53 &  {\bf 36.94} &  2892.68 &  {\bf {\color{red} 0.98}} &  {\bf {\color{red} 0.88}} &  {\bf {\color{red} 0.83}} &  {\bf {\color{red} 0.77}} &  {\bf {\color{red} 0.73}} &  {\bf {\color{red} 0.68}} \\ 
\hline 
 2066 &  63 &  67 ans 3 mois &  -21.25\% &  3127.40 &  {\bf 40.31} &  2930.29 &  {\bf 1.07} &  {\bf {\color{red} 0.98}} &  {\bf {\color{red} 0.91}} &  {\bf {\color{red} 0.86}} &  {\bf {\color{red} 0.80}} &  {\bf {\color{red} 0.75}} \\ 
\hline 
 2067 &  64 &  67 ans 4 mois &  -16.67\% &  3443.59 &  {\bf 43.82} &  2968.38 &  {\bf 1.16} &  {\bf 1.07} &  {\bf 1.01} &  {\bf {\color{red} 0.94}} &  {\bf {\color{red} 0.88}} &  {\bf {\color{red} 0.83}} \\ 
\hline 
 2068 &  65 &  67 ans 5 mois &  -12.08\% &  3777.64 &  {\bf 47.45} &  3006.97 &  {\bf 1.26} &  {\bf 1.18} &  {\bf 1.10} &  {\bf 1.04} &  {\bf {\color{red} 0.97}} &  {\bf {\color{red} 0.91}} \\ 
\hline 
 2069 &  66 &  67 ans 6 mois &  -7.50\% &  4130.08 &  {\bf 51.22} &  3046.06 &  {\bf 1.36} &  {\bf 1.29} &  {\bf 1.21} &  {\bf 1.13} &  {\bf 1.06} &  {\bf {\color{red} 0.99}} \\ 
\hline 
 2070 &  67 &  67 ans 7 mois &  -2.92\% &  4501.46 &  {\bf 55.10} &  3085.66 &  {\bf 1.46} &  {\bf 1.40} &  {\bf 1.32} &  {\bf 1.23} &  {\bf 1.16} &  {\bf 1.08} \\ 
\hline 
\hline 
\end{tabular} 
\end{center} } 

 \begin{center}\includegraphics[width=0.9\textwidth]{fig/ProfEcoles_2003_22_dest_retraite.pdf}\end{center} \label{fig/ProfEcoles_2003_22_dest_retraite.pdf} 

\newpage 
 
\chapter{Professeur certifié} 

\begin{minipage}{0.55\linewidth}\includegraphics[width=0.7\textwidth]{fig/grille_ProfCertifie.pdf}\end{minipage} 
\begin{minipage}{0.3\linewidth} 
 \begin{center} 

\begin{tabular}[htb]{|c|c|} 
\hline 
 Indice majoré &  Durée (années) \\ 
\hline \hline 
 450 &  1.00 \\ 
\hline 
 498 &  1.00 \\ 
\hline 
 513 &  2.00 \\ 
\hline 
 542 &  2.00 \\ 
\hline 
 579 &  2.50 \\ 
\hline 
 618 &  3.00 \\ 
\hline 
 710 &  3.50 \\ 
\hline 
 757 &  2.00 \\ 
\hline 
 800 &  2.00 \\ 
\hline 
 830 &   \\ 
\hline 
\hline 
\end{tabular} 
\end{center} 
 \end{minipage} 


 \addto{\captionsenglish}{ \renewcommand{\mtctitle}{}} \setcounter{minitocdepth}{2} 
 \minitoc \newpage 

\section{Début de carrière à 22 ans} 

\subsection{Génération 1975 (début en 1997)} 

\paragraph{Retraites possibles et ratios Revenu/SMIC à 70, 75, 80, 85, 90 ans avec le modèle \emph{Gouvernement truqué (âge-pivot bloqué à 65 ans)}}  
 
{ \scriptsize \begin{center} 
\begin{tabular}[htb]{|c|c||c|c||c|c||c||c|c|c|c|c|c|} 
\hline 
 Retraite en &  Âge &  Âge pivot &  Décote/Surcote &  Retraite (\euro{} 2019) &  Tx Rempl(\%) &  SMIC (\euro{} 2019) &  Retraite/SMIC &  Rev70/SMIC &  Rev75/SMIC &  Rev80/SMIC &  Rev85/SMIC &  Rev90/SMIC \\ 
\hline \hline 
 2037 &  62 &  64 ans 10 mois &  -14.17\% &  1927.13 &  {\bf 44.47} &  2143.00 &  {\bf {\color{red} 0.90}} &  {\bf {\color{red} 0.81}} &  {\bf {\color{red} 0.76}} &  {\bf {\color{red} 0.71}} &  {\bf {\color{red} 0.67}} &  {\bf {\color{red} 0.63}} \\ 
\hline 
 2038 &  63 &  64 ans 11 mois &  -9.58\% &  2100.17 &  {\bf 48.36} &  2170.86 &  {\bf {\color{red} 0.97}} &  {\bf {\color{red} 0.88}} &  {\bf {\color{red} 0.83}} &  {\bf {\color{red} 0.78}} &  {\bf {\color{red} 0.73}} &  {\bf {\color{red} 0.68}} \\ 
\hline 
 2039 &  64 &  65 ans 0 mois &  -5.00\% &  2282.71 &  {\bf 52.45} &  2199.08 &  {\bf 1.04} &  {\bf {\color{red} 0.96}} &  {\bf {\color{red} 0.90}} &  {\bf {\color{red} 0.84}} &  {\bf {\color{red} 0.79}} &  {\bf {\color{red} 0.74}} \\ 
\hline 
 2040 &  65 &  65 ans 0 mois &  0.00\% &  2485.56 &  {\bf 56.99} &  2227.67 &  {\bf 1.12} &  {\bf 1.05} &  {\bf {\color{red} 0.98}} &  {\bf {\color{red} 0.92}} &  {\bf {\color{red} 0.86}} &  {\bf {\color{red} 0.81}} \\ 
\hline 
 2041 &  66 &  65 ans 0 mois &  5.00\% &  2699.62 &  {\bf 61.77} &  2256.63 &  {\bf 1.20} &  {\bf 1.14} &  {\bf 1.07} &  {\bf {\color{red} 1.00}} &  {\bf {\color{red} 0.94}} &  {\bf {\color{red} 0.88}} \\ 
\hline 
 2042 &  67 &  65 ans 0 mois &  10.00\% &  2925.48 &  {\bf 66.80} &  2285.97 &  {\bf 1.28} &  {\bf 1.23} &  {\bf 1.15} &  {\bf 1.08} &  {\bf 1.01} &  {\bf {\color{red} 0.95}} \\ 
\hline 
\hline 
\end{tabular} 
\end{center} } 
\paragraph{Retraites possibles et ratios Revenu/SMIC à 70, 75, 80, 85, 90 ans avec le modèle \emph{Gouvernement corrigé (âge-pivot glissant)}}  
 
{ \scriptsize \begin{center} 
\begin{tabular}[htb]{|c|c||c|c||c|c||c||c|c|c|c|c|c|} 
\hline 
 Retraite en &  Âge &  Âge pivot &  Décote/Surcote &  Retraite (\euro{} 2019) &  Tx Rempl(\%) &  SMIC (\euro{} 2019) &  Retraite/SMIC &  Rev70/SMIC &  Rev75/SMIC &  Rev80/SMIC &  Rev85/SMIC &  Rev90/SMIC \\ 
\hline \hline 
 2037 &  62 &  64 ans 10 mois &  -14.17\% &  1927.13 &  {\bf 44.47} &  2143.00 &  {\bf {\color{red} 0.90}} &  {\bf {\color{red} 0.81}} &  {\bf {\color{red} 0.76}} &  {\bf {\color{red} 0.71}} &  {\bf {\color{red} 0.67}} &  {\bf {\color{red} 0.63}} \\ 
\hline 
 2038 &  63 &  64 ans 11 mois &  -9.58\% &  2100.17 &  {\bf 48.36} &  2170.86 &  {\bf {\color{red} 0.97}} &  {\bf {\color{red} 0.88}} &  {\bf {\color{red} 0.83}} &  {\bf {\color{red} 0.78}} &  {\bf {\color{red} 0.73}} &  {\bf {\color{red} 0.68}} \\ 
\hline 
 2039 &  64 &  65 ans 0 mois &  -5.00\% &  2282.71 &  {\bf 52.45} &  2199.08 &  {\bf 1.04} &  {\bf {\color{red} 0.96}} &  {\bf {\color{red} 0.90}} &  {\bf {\color{red} 0.84}} &  {\bf {\color{red} 0.79}} &  {\bf {\color{red} 0.74}} \\ 
\hline 
 2040 &  65 &  65 ans 1 mois &  -0.42\% &  2475.21 &  {\bf 56.75} &  2227.67 &  {\bf 1.11} &  {\bf 1.04} &  {\bf {\color{red} 0.98}} &  {\bf {\color{red} 0.92}} &  {\bf {\color{red} 0.86}} &  {\bf {\color{red} 0.80}} \\ 
\hline 
 2041 &  66 &  65 ans 2 mois &  4.17\% &  2678.20 &  {\bf 61.28} &  2256.63 &  {\bf 1.19} &  {\bf 1.13} &  {\bf 1.06} &  {\bf {\color{red} 0.99}} &  {\bf {\color{red} 0.93}} &  {\bf {\color{red} 0.87}} \\ 
\hline 
 2042 &  67 &  65 ans 3 mois &  8.75\% &  2892.24 &  {\bf 66.04} &  2285.97 &  {\bf 1.27} &  {\bf 1.22} &  {\bf 1.14} &  {\bf 1.07} &  {\bf 1.00} &  {\bf {\color{red} 0.94}} \\ 
\hline 
\hline 
\end{tabular} 
\end{center} } 
\paragraph{Retraites possibles et ratios Revenu/SMIC à 70, 75, 80, 85, 90 ans avec le modèle \emph{Destinie2 (revalorisation de la fonction publique)}}  
 
{ \scriptsize \begin{center} 
\begin{tabular}[htb]{|c|c||c|c||c|c||c||c|c|c|c|c|c|} 
\hline 
 Retraite en &  Âge &  Âge pivot &  Décote/Surcote &  Retraite (\euro{} 2019) &  Tx Rempl(\%) &  SMIC (\euro{} 2019) &  Retraite/SMIC &  Rev70/SMIC &  Rev75/SMIC &  Rev80/SMIC &  Rev85/SMIC &  Rev90/SMIC \\ 
\hline \hline 
 2037 &  62 &  64 ans 10 mois &  -14.17\% &  2131.08 &  {\bf 39.41} &  2014.82 &  {\bf 1.06} &  {\bf {\color{red} 0.95}} &  {\bf {\color{red} 0.89}} &  {\bf {\color{red} 0.84}} &  {\bf {\color{red} 0.79}} &  {\bf {\color{red} 0.74}} \\ 
\hline 
 2038 &  63 &  64 ans 11 mois &  -9.58\% &  2330.95 &  {\bf 42.55} &  2041.01 &  {\bf 1.14} &  {\bf 1.04} &  {\bf {\color{red} 0.98}} &  {\bf {\color{red} 0.92}} &  {\bf {\color{red} 0.86}} &  {\bf {\color{red} 0.81}} \\ 
\hline 
 2039 &  64 &  65 ans 0 mois &  -5.00\% &  2543.05 &  {\bf 45.82} &  2067.55 &  {\bf 1.23} &  {\bf 1.14} &  {\bf 1.07} &  {\bf 1.00} &  {\bf {\color{red} 0.94}} &  {\bf {\color{red} 0.88}} \\ 
\hline 
 2040 &  65 &  65 ans 1 mois &  -0.42\% &  2768.08 &  {\bf 49.24} &  2094.43 &  {\bf 1.32} &  {\bf 1.24} &  {\bf 1.16} &  {\bf 1.09} &  {\bf 1.02} &  {\bf {\color{red} 0.96}} \\ 
\hline 
 2041 &  66 &  65 ans 2 mois &  4.17\% &  3006.79 &  {\bf 52.80} &  2121.65 &  {\bf 1.42} &  {\bf 1.35} &  {\bf 1.26} &  {\bf 1.18} &  {\bf 1.11} &  {\bf 1.04} \\ 
\hline 
 2042 &  67 &  65 ans 3 mois &  8.75\% &  3259.98 &  {\bf 56.51} &  2149.23 &  {\bf 1.52} &  {\bf 1.46} &  {\bf 1.37} &  {\bf 1.28} &  {\bf 1.20} &  {\bf 1.13} \\ 
\hline 
\hline 
\end{tabular} 
\end{center} } 

 \begin{center}\includegraphics[width=0.9\textwidth]{fig/ProfCertifie_1975_22_dest_retraite.pdf}\end{center} \label{fig/ProfCertifie_1975_22_dest_retraite.pdf} 

\newpage 
 
\subsection{Génération 1980 (début en 2002)} 

\paragraph{Retraites possibles et ratios Revenu/SMIC à 70, 75, 80, 85, 90 ans avec le modèle \emph{Gouvernement truqué (âge-pivot bloqué à 65 ans)}}  
 
{ \scriptsize \begin{center} 
\begin{tabular}[htb]{|c|c||c|c||c|c||c||c|c|c|c|c|c|} 
\hline 
 Retraite en &  Âge &  Âge pivot &  Décote/Surcote &  Retraite (\euro{} 2019) &  Tx Rempl(\%) &  SMIC (\euro{} 2019) &  Retraite/SMIC &  Rev70/SMIC &  Rev75/SMIC &  Rev80/SMIC &  Rev85/SMIC &  Rev90/SMIC \\ 
\hline \hline 
 2042 &  62 &  65 ans 0 mois &  -15.00\% &  1950.60 &  {\bf 45.01} &  2285.97 &  {\bf {\color{red} 0.85}} &  {\bf {\color{red} 0.77}} &  {\bf {\color{red} 0.72}} &  {\bf {\color{red} 0.68}} &  {\bf {\color{red} 0.63}} &  {\bf {\color{red} 0.59}} \\ 
\hline 
 2043 &  63 &  65 ans 0 mois &  -10.00\% &  2143.43 &  {\bf 49.35} &  2315.68 &  {\bf {\color{red} 0.93}} &  {\bf {\color{red} 0.85}} &  {\bf {\color{red} 0.79}} &  {\bf {\color{red} 0.74}} &  {\bf {\color{red} 0.70}} &  {\bf {\color{red} 0.65}} \\ 
\hline 
 2044 &  64 &  65 ans 0 mois &  -5.00\% &  2347.73 &  {\bf 53.94} &  2345.79 &  {\bf 1.00} &  {\bf {\color{red} 0.93}} &  {\bf {\color{red} 0.87}} &  {\bf {\color{red} 0.81}} &  {\bf {\color{red} 0.76}} &  {\bf {\color{red} 0.72}} \\ 
\hline 
 2045 &  65 &  65 ans 0 mois &  0.00\% &  2564.13 &  {\bf 58.79} &  2376.28 &  {\bf 1.08} &  {\bf 1.01} &  {\bf {\color{red} 0.95}} &  {\bf {\color{red} 0.89}} &  {\bf {\color{red} 0.83}} &  {\bf {\color{red} 0.78}} \\ 
\hline 
 2046 &  66 &  65 ans 0 mois &  5.00\% &  2791.21 &  {\bf 63.86} &  2407.18 &  {\bf 1.16} &  {\bf 1.10} &  {\bf 1.03} &  {\bf {\color{red} 0.97}} &  {\bf {\color{red} 0.91}} &  {\bf {\color{red} 0.85}} \\ 
\hline 
 2047 &  67 &  65 ans 0 mois &  10.00\% &  3029.20 &  {\bf 69.16} &  2438.47 &  {\bf 1.24} &  {\bf 1.20} &  {\bf 1.12} &  {\bf 1.05} &  {\bf {\color{red} 0.98}} &  {\bf {\color{red} 0.92}} \\ 
\hline 
\hline 
\end{tabular} 
\end{center} } 
\paragraph{Retraites possibles et ratios Revenu/SMIC à 70, 75, 80, 85, 90 ans avec le modèle \emph{Gouvernement corrigé (âge-pivot glissant)}}  
 
{ \scriptsize \begin{center} 
\begin{tabular}[htb]{|c|c||c|c||c|c||c||c|c|c|c|c|c|} 
\hline 
 Retraite en &  Âge &  Âge pivot &  Décote/Surcote &  Retraite (\euro{} 2019) &  Tx Rempl(\%) &  SMIC (\euro{} 2019) &  Retraite/SMIC &  Rev70/SMIC &  Rev75/SMIC &  Rev80/SMIC &  Rev85/SMIC &  Rev90/SMIC \\ 
\hline \hline 
 2042 &  62 &  65 ans 3 mois &  -16.25\% &  1921.92 &  {\bf 44.35} &  2285.97 &  {\bf {\color{red} 0.84}} &  {\bf {\color{red} 0.76}} &  {\bf {\color{red} 0.71}} &  {\bf {\color{red} 0.67}} &  {\bf {\color{red} 0.62}} &  {\bf {\color{red} 0.59}} \\ 
\hline 
 2043 &  63 &  65 ans 4 mois &  -11.67\% &  2103.73 &  {\bf 48.44} &  2315.68 &  {\bf {\color{red} 0.91}} &  {\bf {\color{red} 0.83}} &  {\bf {\color{red} 0.78}} &  {\bf {\color{red} 0.73}} &  {\bf {\color{red} 0.68}} &  {\bf {\color{red} 0.64}} \\ 
\hline 
 2044 &  64 &  65 ans 5 mois &  -7.08\% &  2296.24 &  {\bf 52.76} &  2345.79 &  {\bf {\color{red} 0.98}} &  {\bf {\color{red} 0.91}} &  {\bf {\color{red} 0.85}} &  {\bf {\color{red} 0.80}} &  {\bf {\color{red} 0.75}} &  {\bf {\color{red} 0.70}} \\ 
\hline 
 2045 &  65 &  65 ans 6 mois &  -2.50\% &  2500.02 &  {\bf 57.32} &  2376.28 &  {\bf 1.05} &  {\bf {\color{red} 0.99}} &  {\bf {\color{red} 0.92}} &  {\bf {\color{red} 0.87}} &  {\bf {\color{red} 0.81}} &  {\bf {\color{red} 0.76}} \\ 
\hline 
 2046 &  66 &  65 ans 7 mois &  2.08\% &  2713.68 &  {\bf 62.09} &  2407.18 &  {\bf 1.13} &  {\bf 1.07} &  {\bf 1.00} &  {\bf {\color{red} 0.94}} &  {\bf {\color{red} 0.88}} &  {\bf {\color{red} 0.83}} \\ 
\hline 
 2047 &  67 &  65 ans 8 mois &  6.67\% &  2937.40 &  {\bf 67.07} &  2438.47 &  {\bf 1.20} &  {\bf 1.16} &  {\bf 1.09} &  {\bf 1.02} &  {\bf {\color{red} 0.95}} &  {\bf {\color{red} 0.90}} \\ 
\hline 
\hline 
\end{tabular} 
\end{center} } 
\paragraph{Retraites possibles et ratios Revenu/SMIC à 70, 75, 80, 85, 90 ans avec le modèle \emph{Destinie2 (revalorisation de la fonction publique)}}  
 
{ \scriptsize \begin{center} 
\begin{tabular}[htb]{|c|c||c|c||c|c||c||c|c|c|c|c|c|} 
\hline 
 Retraite en &  Âge &  Âge pivot &  Décote/Surcote &  Retraite (\euro{} 2019) &  Tx Rempl(\%) &  SMIC (\euro{} 2019) &  Retraite/SMIC &  Rev70/SMIC &  Rev75/SMIC &  Rev80/SMIC &  Rev85/SMIC &  Rev90/SMIC \\ 
\hline \hline 
 2042 &  62 &  65 ans 3 mois &  -16.25\% &  2201.71 &  {\bf 38.17} &  2149.23 &  {\bf 1.02} &  {\bf {\color{red} 0.92}} &  {\bf {\color{red} 0.87}} &  {\bf {\color{red} 0.81}} &  {\bf {\color{red} 0.76}} &  {\bf {\color{red} 0.71}} \\ 
\hline 
 2043 &  63 &  65 ans 4 mois &  -11.67\% &  2420.67 &  {\bf 41.42} &  2177.17 &  {\bf 1.11} &  {\bf 1.02} &  {\bf {\color{red} 0.95}} &  {\bf {\color{red} 0.89}} &  {\bf {\color{red} 0.84}} &  {\bf {\color{red} 0.78}} \\ 
\hline 
 2044 &  64 &  65 ans 5 mois &  -7.08\% &  2653.98 &  {\bf 44.83} &  2205.48 &  {\bf 1.20} &  {\bf 1.11} &  {\bf 1.04} &  {\bf {\color{red} 0.98}} &  {\bf {\color{red} 0.92}} &  {\bf {\color{red} 0.86}} \\ 
\hline 
 2045 &  65 &  65 ans 6 mois &  -2.50\% &  2902.48 &  {\bf 48.40} &  2234.15 &  {\bf 1.30} &  {\bf 1.22} &  {\bf 1.14} &  {\bf 1.07} &  {\bf 1.00} &  {\bf {\color{red} 0.94}} \\ 
\hline 
 2046 &  66 &  65 ans 7 mois &  2.08\% &  3164.75 &  {\bf 52.10} &  2263.19 &  {\bf 1.40} &  {\bf 1.33} &  {\bf 1.24} &  {\bf 1.17} &  {\bf 1.09} &  {\bf 1.03} \\ 
\hline 
 2047 &  67 &  65 ans 8 mois &  6.67\% &  3441.19 &  {\bf 55.92} &  2292.61 &  {\bf 1.50} &  {\bf 1.44} &  {\bf 1.35} &  {\bf 1.27} &  {\bf 1.19} &  {\bf 1.12} \\ 
\hline 
\hline 
\end{tabular} 
\end{center} } 

 \begin{center}\includegraphics[width=0.9\textwidth]{fig/ProfCertifie_1980_22_dest_retraite.pdf}\end{center} \label{fig/ProfCertifie_1980_22_dest_retraite.pdf} 

\newpage 
 
\subsection{Génération 1990 (début en 2012)} 

\paragraph{Retraites possibles et ratios Revenu/SMIC à 70, 75, 80, 85, 90 ans avec le modèle \emph{Gouvernement truqué (âge-pivot bloqué à 65 ans)}}  
 
{ \scriptsize \begin{center} 
\begin{tabular}[htb]{|c|c||c|c||c|c||c||c|c|c|c|c|c|} 
\hline 
 Retraite en &  Âge &  Âge pivot &  Décote/Surcote &  Retraite (\euro{} 2019) &  Tx Rempl(\%) &  SMIC (\euro{} 2019) &  Retraite/SMIC &  Rev70/SMIC &  Rev75/SMIC &  Rev80/SMIC &  Rev85/SMIC &  Rev90/SMIC \\ 
\hline \hline 
 2052 &  62 &  65 ans 0 mois &  -15.00\% &  2096.42 &  {\bf 48.37} &  2601.14 &  {\bf {\color{red} 0.81}} &  {\bf {\color{red} 0.73}} &  {\bf {\color{red} 0.68}} &  {\bf {\color{red} 0.64}} &  {\bf {\color{red} 0.60}} &  {\bf {\color{red} 0.56}} \\ 
\hline 
 2053 &  63 &  65 ans 0 mois &  -10.00\% &  2303.01 &  {\bf 53.03} &  2634.96 &  {\bf {\color{red} 0.87}} &  {\bf {\color{red} 0.80}} &  {\bf {\color{red} 0.75}} &  {\bf {\color{red} 0.70}} &  {\bf {\color{red} 0.66}} &  {\bf {\color{red} 0.62}} \\ 
\hline 
 2054 &  64 &  65 ans 0 mois &  -5.00\% &  2520.11 &  {\bf 57.90} &  2669.21 &  {\bf {\color{red} 0.94}} &  {\bf {\color{red} 0.87}} &  {\bf {\color{red} 0.82}} &  {\bf {\color{red} 0.77}} &  {\bf {\color{red} 0.72}} &  {\bf {\color{red} 0.67}} \\ 
\hline 
 2055 &  65 &  65 ans 0 mois &  0.00\% &  2747.94 &  {\bf 63.01} &  2703.91 &  {\bf 1.02} &  {\bf {\color{red} 0.95}} &  {\bf {\color{red} 0.89}} &  {\bf {\color{red} 0.84}} &  {\bf {\color{red} 0.78}} &  {\bf {\color{red} 0.74}} \\ 
\hline 
 2056 &  66 &  65 ans 0 mois &  5.00\% &  2986.72 &  {\bf 68.34} &  2739.06 &  {\bf 1.09} &  {\bf 1.04} &  {\bf {\color{red} 0.97}} &  {\bf {\color{red} 0.91}} &  {\bf {\color{red} 0.85}} &  {\bf {\color{red} 0.80}} \\ 
\hline 
 2057 &  67 &  65 ans 0 mois &  10.00\% &  3236.68 &  {\bf 73.90} &  2774.67 &  {\bf 1.17} &  {\bf 1.12} &  {\bf 1.05} &  {\bf {\color{red} 0.99}} &  {\bf {\color{red} 0.92}} &  {\bf {\color{red} 0.87}} \\ 
\hline 
\hline 
\end{tabular} 
\end{center} } 
\paragraph{Retraites possibles et ratios Revenu/SMIC à 70, 75, 80, 85, 90 ans avec le modèle \emph{Gouvernement corrigé (âge-pivot glissant)}}  
 
{ \scriptsize \begin{center} 
\begin{tabular}[htb]{|c|c||c|c||c|c||c||c|c|c|c|c|c|} 
\hline 
 Retraite en &  Âge &  Âge pivot &  Décote/Surcote &  Retraite (\euro{} 2019) &  Tx Rempl(\%) &  SMIC (\euro{} 2019) &  Retraite/SMIC &  Rev70/SMIC &  Rev75/SMIC &  Rev80/SMIC &  Rev85/SMIC &  Rev90/SMIC \\ 
\hline \hline 
 2052 &  62 &  66 ans 1 mois &  -20.42\% &  1962.83 &  {\bf 45.29} &  2601.14 &  {\bf {\color{red} 0.75}} &  {\bf {\color{red} 0.68}} &  {\bf {\color{red} 0.64}} &  {\bf {\color{red} 0.60}} &  {\bf {\color{red} 0.56}} &  {\bf {\color{red} 0.53}} \\ 
\hline 
 2053 &  63 &  66 ans 2 mois &  -15.83\% &  2153.74 &  {\bf 49.59} &  2634.96 &  {\bf {\color{red} 0.82}} &  {\bf {\color{red} 0.75}} &  {\bf {\color{red} 0.70}} &  {\bf {\color{red} 0.66}} &  {\bf {\color{red} 0.62}} &  {\bf {\color{red} 0.58}} \\ 
\hline 
 2054 &  64 &  66 ans 3 mois &  -11.25\% &  2354.31 &  {\bf 54.09} &  2669.21 &  {\bf {\color{red} 0.88}} &  {\bf {\color{red} 0.82}} &  {\bf {\color{red} 0.77}} &  {\bf {\color{red} 0.72}} &  {\bf {\color{red} 0.67}} &  {\bf {\color{red} 0.63}} \\ 
\hline 
 2055 &  65 &  66 ans 4 mois &  -6.67\% &  2564.74 &  {\bf 58.81} &  2703.91 &  {\bf {\color{red} 0.95}} &  {\bf {\color{red} 0.89}} &  {\bf {\color{red} 0.83}} &  {\bf {\color{red} 0.78}} &  {\bf {\color{red} 0.73}} &  {\bf {\color{red} 0.69}} \\ 
\hline 
 2056 &  66 &  66 ans 5 mois &  -2.08\% &  2785.23 &  {\bf 63.73} &  2739.06 &  {\bf 1.02} &  {\bf {\color{red} 0.97}} &  {\bf {\color{red} 0.91}} &  {\bf {\color{red} 0.85}} &  {\bf {\color{red} 0.80}} &  {\bf {\color{red} 0.75}} \\ 
\hline 
 2057 &  67 &  66 ans 6 mois &  2.50\% &  3016.00 &  {\bf 68.86} &  2774.67 &  {\bf 1.09} &  {\bf 1.05} &  {\bf {\color{red} 0.98}} &  {\bf {\color{red} 0.92}} &  {\bf {\color{red} 0.86}} &  {\bf {\color{red} 0.81}} \\ 
\hline 
\hline 
\end{tabular} 
\end{center} } 
\paragraph{Retraites possibles et ratios Revenu/SMIC à 70, 75, 80, 85, 90 ans avec le modèle \emph{Destinie2 (revalorisation de la fonction publique)}}  
 
{ \scriptsize \begin{center} 
\begin{tabular}[htb]{|c|c||c|c||c|c||c||c|c|c|c|c|c|} 
\hline 
 Retraite en &  Âge &  Âge pivot &  Décote/Surcote &  Retraite (\euro{} 2019) &  Tx Rempl(\%) &  SMIC (\euro{} 2019) &  Retraite/SMIC &  Rev70/SMIC &  Rev75/SMIC &  Rev80/SMIC &  Rev85/SMIC &  Rev90/SMIC \\ 
\hline \hline 
 2052 &  62 &  66 ans 1 mois &  -20.42\% &  2470.63 &  {\bf 37.64} &  2445.56 &  {\bf 1.01} &  {\bf {\color{red} 0.91}} &  {\bf {\color{red} 0.85}} &  {\bf {\color{red} 0.80}} &  {\bf {\color{red} 0.75}} &  {\bf {\color{red} 0.70}} \\ 
\hline 
 2053 &  63 &  66 ans 2 mois &  -15.83\% &  2724.79 &  {\bf 40.98} &  2477.35 &  {\bf 1.10} &  {\bf 1.00} &  {\bf {\color{red} 0.94}} &  {\bf {\color{red} 0.88}} &  {\bf {\color{red} 0.83}} &  {\bf {\color{red} 0.78}} \\ 
\hline 
 2054 &  64 &  66 ans 3 mois &  -11.25\% &  2993.73 &  {\bf 44.44} &  2509.56 &  {\bf 1.19} &  {\bf 1.10} &  {\bf 1.03} &  {\bf {\color{red} 0.97}} &  {\bf {\color{red} 0.91}} &  {\bf {\color{red} 0.85}} \\ 
\hline 
 2055 &  65 &  66 ans 4 mois &  -6.67\% &  3277.91 &  {\bf 48.04} &  2542.18 &  {\bf 1.29} &  {\bf 1.21} &  {\bf 1.13} &  {\bf 1.06} &  {\bf {\color{red} 1.00}} &  {\bf {\color{red} 0.93}} \\ 
\hline 
 2056 &  66 &  66 ans 5 mois &  -2.08\% &  3577.80 &  {\bf 51.76} &  2575.23 &  {\bf 1.39} &  {\bf 1.32} &  {\bf 1.24} &  {\bf 1.16} &  {\bf 1.09} &  {\bf 1.02} \\ 
\hline 
 2057 &  67 &  66 ans 6 mois &  2.50\% &  3893.86 &  {\bf 55.61} &  2608.71 &  {\bf 1.49} &  {\bf 1.44} &  {\bf 1.35} &  {\bf 1.26} &  {\bf 1.18} &  {\bf 1.11} \\ 
\hline 
\hline 
\end{tabular} 
\end{center} } 

 \begin{center}\includegraphics[width=0.9\textwidth]{fig/ProfCertifie_1990_22_dest_retraite.pdf}\end{center} \label{fig/ProfCertifie_1990_22_dest_retraite.pdf} 

\newpage 
 
\subsection{Génération 2003 (début en 2025)} 

\paragraph{Retraites possibles et ratios Revenu/SMIC à 70, 75, 80, 85, 90 ans avec le modèle \emph{Gouvernement truqué (âge-pivot bloqué à 65 ans)}}  
 
{ \scriptsize \begin{center} 
\begin{tabular}[htb]{|c|c||c|c||c|c||c||c|c|c|c|c|c|} 
\hline 
 Retraite en &  Âge &  Âge pivot &  Décote/Surcote &  Retraite (\euro{} 2019) &  Tx Rempl(\%) &  SMIC (\euro{} 2019) &  Retraite/SMIC &  Rev70/SMIC &  Rev75/SMIC &  Rev80/SMIC &  Rev85/SMIC &  Rev90/SMIC \\ 
\hline \hline 
 2065 &  62 &  65 ans 0 mois &  -15.00\% &  2241.05 &  {\bf 51.71} &  3076.71 &  {\bf {\color{red} 0.73}} &  {\bf {\color{red} 0.66}} &  {\bf {\color{red} 0.62}} &  {\bf {\color{red} 0.58}} &  {\bf {\color{red} 0.54}} &  {\bf {\color{red} 0.51}} \\ 
\hline 
 2066 &  63 &  65 ans 0 mois &  -10.00\% &  2458.13 &  {\bf 56.60} &  3116.71 &  {\bf {\color{red} 0.79}} &  {\bf {\color{red} 0.72}} &  {\bf {\color{red} 0.68}} &  {\bf {\color{red} 0.63}} &  {\bf {\color{red} 0.59}} &  {\bf {\color{red} 0.56}} \\ 
\hline 
 2067 &  64 &  65 ans 0 mois &  -5.00\% &  2685.98 &  {\bf 61.71} &  3157.23 &  {\bf {\color{red} 0.85}} &  {\bf {\color{red} 0.79}} &  {\bf {\color{red} 0.74}} &  {\bf {\color{red} 0.69}} &  {\bf {\color{red} 0.65}} &  {\bf {\color{red} 0.61}} \\ 
\hline 
 2068 &  65 &  65 ans 0 mois &  0.00\% &  2924.81 &  {\bf 67.06} &  3198.27 &  {\bf {\color{red} 0.91}} &  {\bf {\color{red} 0.86}} &  {\bf {\color{red} 0.80}} &  {\bf {\color{red} 0.75}} &  {\bf {\color{red} 0.71}} &  {\bf {\color{red} 0.66}} \\ 
\hline 
 2069 &  66 &  65 ans 0 mois &  5.00\% &  3174.85 &  {\bf 72.64} &  3239.85 &  {\bf {\color{red} 0.98}} &  {\bf {\color{red} 0.93}} &  {\bf {\color{red} 0.87}} &  {\bf {\color{red} 0.82}} &  {\bf {\color{red} 0.77}} &  {\bf {\color{red} 0.72}} \\ 
\hline 
 2070 &  67 &  65 ans 0 mois &  10.00\% &  3436.33 &  {\bf 78.46} &  3281.97 &  {\bf 1.05} &  {\bf 1.01} &  {\bf {\color{red} 0.94}} &  {\bf {\color{red} 0.89}} &  {\bf {\color{red} 0.83}} &  {\bf {\color{red} 0.78}} \\ 
\hline 
\hline 
\end{tabular} 
\end{center} } 
\paragraph{Retraites possibles et ratios Revenu/SMIC à 70, 75, 80, 85, 90 ans avec le modèle \emph{Gouvernement corrigé (âge-pivot glissant)}}  
 
{ \scriptsize \begin{center} 
\begin{tabular}[htb]{|c|c||c|c||c|c||c||c|c|c|c|c|c|} 
\hline 
 Retraite en &  Âge &  Âge pivot &  Décote/Surcote &  Retraite (\euro{} 2019) &  Tx Rempl(\%) &  SMIC (\euro{} 2019) &  Retraite/SMIC &  Rev70/SMIC &  Rev75/SMIC &  Rev80/SMIC &  Rev85/SMIC &  Rev90/SMIC \\ 
\hline \hline 
 2065 &  62 &  67 ans 2 mois &  -25.83\% &  1955.43 &  {\bf 45.12} &  3076.71 &  {\bf {\color{red} 0.64}} &  {\bf {\color{red} 0.57}} &  {\bf {\color{red} 0.54}} &  {\bf {\color{red} 0.50}} &  {\bf {\color{red} 0.47}} &  {\bf {\color{red} 0.44}} \\ 
\hline 
 2066 &  63 &  67 ans 3 mois &  -21.25\% &  2150.87 &  {\bf 49.52} &  3116.71 &  {\bf {\color{red} 0.69}} &  {\bf {\color{red} 0.63}} &  {\bf {\color{red} 0.59}} &  {\bf {\color{red} 0.55}} &  {\bf {\color{red} 0.52}} &  {\bf {\color{red} 0.49}} \\ 
\hline 
 2067 &  64 &  67 ans 4 mois &  -16.67\% &  2356.12 &  {\bf 54.14} &  3157.23 &  {\bf {\color{red} 0.75}} &  {\bf {\color{red} 0.69}} &  {\bf {\color{red} 0.65}} &  {\bf {\color{red} 0.61}} &  {\bf {\color{red} 0.57}} &  {\bf {\color{red} 0.53}} \\ 
\hline 
 2068 &  65 &  67 ans 5 mois &  -12.08\% &  2718.53 &  {\bf 62.33} &  3198.27 &  {\bf {\color{red} 0.85}} &  {\bf {\color{red} 0.80}} &  {\bf {\color{red} 0.75}} &  {\bf {\color{red} 0.70}} &  {\bf {\color{red} 0.66}} &  {\bf {\color{red} 0.62}} \\ 
\hline 
 2069 &  66 &  67 ans 6 mois &  -7.50\% &  2796.89 &  {\bf 63.99} &  3239.85 &  {\bf {\color{red} 0.86}} &  {\bf {\color{red} 0.82}} &  {\bf {\color{red} 0.77}} &  {\bf {\color{red} 0.72}} &  {\bf {\color{red} 0.68}} &  {\bf {\color{red} 0.63}} \\ 
\hline 
 2070 &  67 &  67 ans 7 mois &  -2.92\% &  3032.82 &  {\bf 69.25} &  3281.97 &  {\bf {\color{red} 0.92}} &  {\bf {\color{red} 0.89}} &  {\bf {\color{red} 0.83}} &  {\bf {\color{red} 0.78}} &  {\bf {\color{red} 0.73}} &  {\bf {\color{red} 0.69}} \\ 
\hline 
\hline 
\end{tabular} 
\end{center} } 
\paragraph{Retraites possibles et ratios Revenu/SMIC à 70, 75, 80, 85, 90 ans avec le modèle \emph{Destinie2 (revalorisation de la fonction publique)}}  
 
{ \scriptsize \begin{center} 
\begin{tabular}[htb]{|c|c||c|c||c|c||c||c|c|c|c|c|c|} 
\hline 
 Retraite en &  Âge &  Âge pivot &  Décote/Surcote &  Retraite (\euro{} 2019) &  Tx Rempl(\%) &  SMIC (\euro{} 2019) &  Retraite/SMIC &  Rev70/SMIC &  Rev75/SMIC &  Rev80/SMIC &  Rev85/SMIC &  Rev90/SMIC \\ 
\hline \hline 
 2065 &  62 &  67 ans 2 mois &  -25.83\% &  2867.77 &  {\bf 36.94} &  2892.68 &  {\bf {\color{red} 0.99}} &  {\bf {\color{red} 0.89}} &  {\bf {\color{red} 0.84}} &  {\bf {\color{red} 0.79}} &  {\bf {\color{red} 0.74}} &  {\bf {\color{red} 0.69}} \\ 
\hline 
 2066 &  63 &  67 ans 3 mois &  -21.25\% &  3170.80 &  {\bf 40.31} &  2930.29 &  {\bf 1.08} &  {\bf {\color{red} 0.99}} &  {\bf {\color{red} 0.93}} &  {\bf {\color{red} 0.87}} &  {\bf {\color{red} 0.81}} &  {\bf {\color{red} 0.76}} \\ 
\hline 
 2067 &  64 &  67 ans 4 mois &  -16.67\% &  3491.38 &  {\bf 43.82} &  2968.38 &  {\bf 1.18} &  {\bf 1.09} &  {\bf 1.02} &  {\bf {\color{red} 0.96}} &  {\bf {\color{red} 0.90}} &  {\bf {\color{red} 0.84}} \\ 
\hline 
 2068 &  65 &  67 ans 5 mois &  -12.08\% &  3830.06 &  {\bf 47.45} &  3006.97 &  {\bf 1.27} &  {\bf 1.19} &  {\bf 1.12} &  {\bf 1.05} &  {\bf {\color{red} 0.98}} &  {\bf {\color{red} 0.92}} \\ 
\hline 
 2069 &  66 &  67 ans 6 mois &  -7.50\% &  4187.38 &  {\bf 51.22} &  3046.06 &  {\bf 1.37} &  {\bf 1.31} &  {\bf 1.22} &  {\bf 1.15} &  {\bf 1.08} &  {\bf 1.01} \\ 
\hline 
 2070 &  67 &  67 ans 7 mois &  -2.92\% &  4563.92 &  {\bf 55.10} &  3085.66 &  {\bf 1.48} &  {\bf 1.42} &  {\bf 1.33} &  {\bf 1.25} &  {\bf 1.17} &  {\bf 1.10} \\ 
\hline 
\hline 
\end{tabular} 
\end{center} } 

 \begin{center}\includegraphics[width=0.9\textwidth]{fig/ProfCertifie_2003_22_dest_retraite.pdf}\end{center} \label{fig/ProfCertifie_2003_22_dest_retraite.pdf} 

\newpage 
 
\chapter{Professeur agrégé} 

\begin{minipage}{0.55\linewidth}\includegraphics[width=0.7\textwidth]{fig/grille_ProfAgrege.pdf}\end{minipage} 
\begin{minipage}{0.3\linewidth} 
 \begin{center} 

\begin{tabular}[htb]{|c|c|} 
\hline 
 Indice majoré &  Durée (années) \\ 
\hline \hline 
 450 &  1.00 \\ 
\hline 
 498 &  1.00 \\ 
\hline 
 513 &  2.00 \\ 
\hline 
 542 &  2.00 \\ 
\hline 
 579 &  2.50 \\ 
\hline 
 618 &  3.00 \\ 
\hline 
 710 &  3.50 \\ 
\hline 
 757 &  2.00 \\ 
\hline 
 800 &  2.00 \\ 
\hline 
 830 &  3.00 \\ 
\hline 
 890 &  1.00 \\ 
\hline 
 925 &  1.00 \\ 
\hline 
 972 &   \\ 
\hline 
\hline 
\end{tabular} 
\end{center} 
 \end{minipage} 


 \addto{\captionsenglish}{ \renewcommand{\mtctitle}{}} \setcounter{minitocdepth}{2} 
 \minitoc \newpage 

\section{Début de carrière à 22 ans} 

\subsection{Génération 1975 (début en 1997)} 

\paragraph{Retraites possibles et ratios Revenu/SMIC à 70, 75, 80, 85, 90 ans avec le modèle \emph{Gouvernement truqué (âge-pivot bloqué à 65 ans)}}  
 
{ \scriptsize \begin{center} 
\begin{tabular}[htb]{|c|c||c|c||c|c||c||c|c|c|c|c|c|} 
\hline 
 Retraite en &  Âge &  Âge pivot &  Décote/Surcote &  Retraite (\euro{} 2019) &  Tx Rempl(\%) &  SMIC (\euro{} 2019) &  Retraite/SMIC &  Rev70/SMIC &  Rev75/SMIC &  Rev80/SMIC &  Rev85/SMIC &  Rev90/SMIC \\ 
\hline \hline 
 2037 &  62 &  64 ans 10 mois &  -14.17\% &  2088.51 &  {\bf 41.15} &  2143.00 &  {\bf {\color{red} 0.97}} &  {\bf {\color{red} 0.88}} &  {\bf {\color{red} 0.82}} &  {\bf {\color{red} 0.77}} &  {\bf {\color{red} 0.72}} &  {\bf {\color{red} 0.68}} \\ 
\hline 
 2038 &  63 &  64 ans 11 mois &  -9.58\% &  2280.82 &  {\bf 44.84} &  2170.86 &  {\bf 1.05} &  {\bf {\color{red} 0.96}} &  {\bf {\color{red} 0.90}} &  {\bf {\color{red} 0.84}} &  {\bf {\color{red} 0.79}} &  {\bf {\color{red} 0.74}} \\ 
\hline 
 2039 &  64 &  65 ans 0 mois &  -5.00\% &  2483.95 &  {\bf 48.74} &  2199.08 &  {\bf 1.13} &  {\bf 1.05} &  {\bf {\color{red} 0.98}} &  {\bf {\color{red} 0.92}} &  {\bf {\color{red} 0.86}} &  {\bf {\color{red} 0.81}} \\ 
\hline 
 2040 &  65 &  65 ans 0 mois &  0.00\% &  2709.72 &  {\bf 53.05} &  2227.67 &  {\bf 1.22} &  {\bf 1.14} &  {\bf 1.07} &  {\bf 1.00} &  {\bf {\color{red} 0.94}} &  {\bf {\color{red} 0.88}} \\ 
\hline 
 2041 &  66 &  65 ans 0 mois &  5.00\% &  2948.26 &  {\bf 57.60} &  2256.63 &  {\bf 1.31} &  {\bf 1.24} &  {\bf 1.16} &  {\bf 1.09} &  {\bf 1.02} &  {\bf {\color{red} 0.96}} \\ 
\hline 
 2042 &  67 &  65 ans 0 mois &  10.00\% &  3200.21 &  {\bf 62.39} &  2285.97 &  {\bf 1.40} &  {\bf 1.35} &  {\bf 1.26} &  {\bf 1.18} &  {\bf 1.11} &  {\bf 1.04} \\ 
\hline 
\hline 
\end{tabular} 
\end{center} } 
\paragraph{Retraites possibles et ratios Revenu/SMIC à 70, 75, 80, 85, 90 ans avec le modèle \emph{Gouvernement corrigé (âge-pivot glissant)}}  
 
{ \scriptsize \begin{center} 
\begin{tabular}[htb]{|c|c||c|c||c|c||c||c|c|c|c|c|c|} 
\hline 
 Retraite en &  Âge &  Âge pivot &  Décote/Surcote &  Retraite (\euro{} 2019) &  Tx Rempl(\%) &  SMIC (\euro{} 2019) &  Retraite/SMIC &  Rev70/SMIC &  Rev75/SMIC &  Rev80/SMIC &  Rev85/SMIC &  Rev90/SMIC \\ 
\hline \hline 
 2037 &  62 &  64 ans 10 mois &  -14.17\% &  2088.51 &  {\bf 41.15} &  2143.00 &  {\bf {\color{red} 0.97}} &  {\bf {\color{red} 0.88}} &  {\bf {\color{red} 0.82}} &  {\bf {\color{red} 0.77}} &  {\bf {\color{red} 0.72}} &  {\bf {\color{red} 0.68}} \\ 
\hline 
 2038 &  63 &  64 ans 11 mois &  -9.58\% &  2280.82 &  {\bf 44.84} &  2170.86 &  {\bf 1.05} &  {\bf {\color{red} 0.96}} &  {\bf {\color{red} 0.90}} &  {\bf {\color{red} 0.84}} &  {\bf {\color{red} 0.79}} &  {\bf {\color{red} 0.74}} \\ 
\hline 
 2039 &  64 &  65 ans 0 mois &  -5.00\% &  2483.95 &  {\bf 48.74} &  2199.08 &  {\bf 1.13} &  {\bf 1.05} &  {\bf {\color{red} 0.98}} &  {\bf {\color{red} 0.92}} &  {\bf {\color{red} 0.86}} &  {\bf {\color{red} 0.81}} \\ 
\hline 
 2040 &  65 &  65 ans 1 mois &  -0.42\% &  2698.43 &  {\bf 52.83} &  2227.67 &  {\bf 1.21} &  {\bf 1.14} &  {\bf 1.06} &  {\bf {\color{red} 1.00}} &  {\bf {\color{red} 0.94}} &  {\bf {\color{red} 0.88}} \\ 
\hline 
 2041 &  66 &  65 ans 2 mois &  4.17\% &  2924.86 &  {\bf 57.15} &  2256.63 &  {\bf 1.30} &  {\bf 1.23} &  {\bf 1.15} &  {\bf 1.08} &  {\bf 1.01} &  {\bf {\color{red} 0.95}} \\ 
\hline 
 2042 &  67 &  65 ans 3 mois &  8.75\% &  3163.85 &  {\bf 61.69} &  2285.97 &  {\bf 1.38} &  {\bf 1.33} &  {\bf 1.25} &  {\bf 1.17} &  {\bf 1.10} &  {\bf 1.03} \\ 
\hline 
\hline 
\end{tabular} 
\end{center} } 
\paragraph{Retraites possibles et ratios Revenu/SMIC à 70, 75, 80, 85, 90 ans avec le modèle \emph{Destinie2 (revalorisation de la fonction publique)}}  
 
{ \scriptsize \begin{center} 
\begin{tabular}[htb]{|c|c||c|c||c|c||c||c|c|c|c|c|c|} 
\hline 
 Retraite en &  Âge &  Âge pivot &  Décote/Surcote &  Retraite (\euro{} 2019) &  Tx Rempl(\%) &  SMIC (\euro{} 2019) &  Retraite/SMIC &  Rev70/SMIC &  Rev75/SMIC &  Rev80/SMIC &  Rev85/SMIC &  Rev90/SMIC \\ 
\hline \hline 
 2037 &  62 &  64 ans 10 mois &  -14.17\% &  2314.94 &  {\bf 36.55} &  2014.82 &  {\bf 1.15} &  {\bf 1.04} &  {\bf {\color{red} 0.97}} &  {\bf {\color{red} 0.91}} &  {\bf {\color{red} 0.85}} &  {\bf {\color{red} 0.80}} \\ 
\hline 
 2038 &  63 &  64 ans 11 mois &  -9.58\% &  2537.90 &  {\bf 39.56} &  2041.01 &  {\bf 1.24} &  {\bf 1.14} &  {\bf 1.06} &  {\bf {\color{red} 1.00}} &  {\bf {\color{red} 0.94}} &  {\bf {\color{red} 0.88}} \\ 
\hline 
 2039 &  64 &  65 ans 0 mois &  -5.00\% &  2774.86 &  {\bf 42.70} &  2067.55 &  {\bf 1.34} &  {\bf 1.24} &  {\bf 1.16} &  {\bf 1.09} &  {\bf 1.02} &  {\bf {\color{red} 0.96}} \\ 
\hline 
 2040 &  65 &  65 ans 1 mois &  -0.42\% &  3026.63 &  {\bf 45.97} &  2094.43 &  {\bf 1.45} &  {\bf 1.35} &  {\bf 1.27} &  {\bf 1.19} &  {\bf 1.12} &  {\bf 1.05} \\ 
\hline 
 2041 &  66 &  65 ans 2 mois &  4.17\% &  3294.06 &  {\bf 49.39} &  2121.65 &  {\bf 1.55} &  {\bf 1.47} &  {\bf 1.38} &  {\bf 1.30} &  {\bf 1.21} &  {\bf 1.14} \\ 
\hline 
 2042 &  67 &  65 ans 3 mois &  8.75\% &  3578.04 &  {\bf 52.96} &  2149.23 &  {\bf 1.66} &  {\bf 1.60} &  {\bf 1.50} &  {\bf 1.41} &  {\bf 1.32} &  {\bf 1.24} \\ 
\hline 
\hline 
\end{tabular} 
\end{center} } 

 \begin{center}\includegraphics[width=0.9\textwidth]{fig/ProfAgrege_1975_22_dest_retraite.pdf}\end{center} \label{fig/ProfAgrege_1975_22_dest_retraite.pdf} 

\newpage 
 
\subsection{Génération 1980 (début en 2002)} 

\paragraph{Retraites possibles et ratios Revenu/SMIC à 70, 75, 80, 85, 90 ans avec le modèle \emph{Gouvernement truqué (âge-pivot bloqué à 65 ans)}}  
 
{ \scriptsize \begin{center} 
\begin{tabular}[htb]{|c|c||c|c||c|c||c||c|c|c|c|c|c|} 
\hline 
 Retraite en &  Âge &  Âge pivot &  Décote/Surcote &  Retraite (\euro{} 2019) &  Tx Rempl(\%) &  SMIC (\euro{} 2019) &  Retraite/SMIC &  Rev70/SMIC &  Rev75/SMIC &  Rev80/SMIC &  Rev85/SMIC &  Rev90/SMIC \\ 
\hline \hline 
 2042 &  62 &  65 ans 0 mois &  -15.00\% &  2115.47 &  {\bf 41.68} &  2285.97 &  {\bf {\color{red} 0.93}} &  {\bf {\color{red} 0.83}} &  {\bf {\color{red} 0.78}} &  {\bf {\color{red} 0.73}} &  {\bf {\color{red} 0.69}} &  {\bf {\color{red} 0.64}} \\ 
\hline 
 2043 &  63 &  65 ans 0 mois &  -10.00\% &  2329.30 &  {\bf 45.80} &  2315.68 &  {\bf 1.01} &  {\bf {\color{red} 0.92}} &  {\bf {\color{red} 0.86}} &  {\bf {\color{red} 0.81}} &  {\bf {\color{red} 0.76}} &  {\bf {\color{red} 0.71}} \\ 
\hline 
 2044 &  64 &  65 ans 0 mois &  -5.00\% &  2556.18 &  {\bf 50.15} &  2345.79 &  {\bf 1.09} &  {\bf 1.01} &  {\bf {\color{red} 0.95}} &  {\bf {\color{red} 0.89}} &  {\bf {\color{red} 0.83}} &  {\bf {\color{red} 0.78}} \\ 
\hline 
 2045 &  65 &  65 ans 0 mois &  0.00\% &  2796.78 &  {\bf 54.76} &  2376.28 &  {\bf 1.18} &  {\bf 1.10} &  {\bf 1.03} &  {\bf {\color{red} 0.97}} &  {\bf {\color{red} 0.91}} &  {\bf {\color{red} 0.85}} \\ 
\hline 
 2046 &  66 &  65 ans 0 mois &  5.00\% &  3049.60 &  {\bf 59.58} &  2407.18 &  {\bf 1.27} &  {\bf 1.20} &  {\bf 1.13} &  {\bf 1.06} &  {\bf {\color{red} 0.99}} &  {\bf {\color{red} 0.93}} \\ 
\hline 
 2047 &  67 &  65 ans 0 mois &  10.00\% &  3314.89 &  {\bf 64.63} &  2438.47 &  {\bf 1.36} &  {\bf 1.31} &  {\bf 1.23} &  {\bf 1.15} &  {\bf 1.08} &  {\bf 1.01} \\ 
\hline 
\hline 
\end{tabular} 
\end{center} } 
\paragraph{Retraites possibles et ratios Revenu/SMIC à 70, 75, 80, 85, 90 ans avec le modèle \emph{Gouvernement corrigé (âge-pivot glissant)}}  
 
{ \scriptsize \begin{center} 
\begin{tabular}[htb]{|c|c||c|c||c|c||c||c|c|c|c|c|c|} 
\hline 
 Retraite en &  Âge &  Âge pivot &  Décote/Surcote &  Retraite (\euro{} 2019) &  Tx Rempl(\%) &  SMIC (\euro{} 2019) &  Retraite/SMIC &  Rev70/SMIC &  Rev75/SMIC &  Rev80/SMIC &  Rev85/SMIC &  Rev90/SMIC \\ 
\hline \hline 
 2042 &  62 &  65 ans 3 mois &  -16.25\% &  2084.36 &  {\bf 41.07} &  2285.97 &  {\bf {\color{red} 0.91}} &  {\bf {\color{red} 0.82}} &  {\bf {\color{red} 0.77}} &  {\bf {\color{red} 0.72}} &  {\bf {\color{red} 0.68}} &  {\bf {\color{red} 0.64}} \\ 
\hline 
 2043 &  63 &  65 ans 4 mois &  -11.67\% &  2286.17 &  {\bf 44.95} &  2315.68 &  {\bf {\color{red} 0.99}} &  {\bf {\color{red} 0.90}} &  {\bf {\color{red} 0.85}} &  {\bf {\color{red} 0.79}} &  {\bf {\color{red} 0.74}} &  {\bf {\color{red} 0.70}} \\ 
\hline 
 2044 &  64 &  65 ans 5 mois &  -7.08\% &  2500.12 &  {\bf 49.05} &  2345.79 &  {\bf 1.07} &  {\bf {\color{red} 0.99}} &  {\bf {\color{red} 0.92}} &  {\bf {\color{red} 0.87}} &  {\bf {\color{red} 0.81}} &  {\bf {\color{red} 0.76}} \\ 
\hline 
 2045 &  65 &  65 ans 6 mois &  -2.50\% &  2726.86 &  {\bf 53.39} &  2376.28 &  {\bf 1.15} &  {\bf 1.08} &  {\bf 1.01} &  {\bf {\color{red} 0.95}} &  {\bf {\color{red} 0.89}} &  {\bf {\color{red} 0.83}} \\ 
\hline 
 2046 &  66 &  65 ans 7 mois &  2.08\% &  2964.89 &  {\bf 57.93} &  2407.18 &  {\bf 1.23} &  {\bf 1.17} &  {\bf 1.10} &  {\bf 1.03} &  {\bf {\color{red} 0.96}} &  {\bf {\color{red} 0.90}} \\ 
\hline 
 2047 &  67 &  65 ans 8 mois &  6.67\% &  3214.44 &  {\bf 62.67} &  2438.47 &  {\bf 1.32} &  {\bf 1.27} &  {\bf 1.19} &  {\bf 1.11} &  {\bf 1.04} &  {\bf {\color{red} 0.98}} \\ 
\hline 
\hline 
\end{tabular} 
\end{center} } 
\paragraph{Retraites possibles et ratios Revenu/SMIC à 70, 75, 80, 85, 90 ans avec le modèle \emph{Destinie2 (revalorisation de la fonction publique)}}  
 
{ \scriptsize \begin{center} 
\begin{tabular}[htb]{|c|c||c|c||c|c||c||c|c|c|c|c|c|} 
\hline 
 Retraite en &  Âge &  Âge pivot &  Décote/Surcote &  Retraite (\euro{} 2019) &  Tx Rempl(\%) &  SMIC (\euro{} 2019) &  Retraite/SMIC &  Rev70/SMIC &  Rev75/SMIC &  Rev80/SMIC &  Rev85/SMIC &  Rev90/SMIC \\ 
\hline \hline 
 2042 &  62 &  65 ans 3 mois &  -16.25\% &  2399.13 &  {\bf 35.51} &  2149.23 &  {\bf 1.12} &  {\bf 1.01} &  {\bf {\color{red} 0.94}} &  {\bf {\color{red} 0.88}} &  {\bf {\color{red} 0.83}} &  {\bf {\color{red} 0.78}} \\ 
\hline 
 2043 &  63 &  65 ans 4 mois &  -11.67\% &  2643.57 &  {\bf 38.63} &  2177.17 &  {\bf 1.21} &  {\bf 1.11} &  {\bf 1.04} &  {\bf {\color{red} 0.97}} &  {\bf {\color{red} 0.91}} &  {\bf {\color{red} 0.86}} \\ 
\hline 
 2044 &  64 &  65 ans 5 mois &  -7.08\% &  2904.41 &  {\bf 41.90} &  2205.48 &  {\bf 1.32} &  {\bf 1.22} &  {\bf 1.14} &  {\bf 1.07} &  {\bf 1.00} &  {\bf {\color{red} 0.94}} \\ 
\hline 
 2045 &  65 &  65 ans 6 mois &  -2.50\% &  3182.61 &  {\bf 45.32} &  2234.15 &  {\bf 1.42} &  {\bf 1.34} &  {\bf 1.25} &  {\bf 1.17} &  {\bf 1.10} &  {\bf 1.03} \\ 
\hline 
 2046 &  66 &  65 ans 7 mois &  2.08\% &  3476.62 &  {\bf 48.87} &  2263.19 &  {\bf 1.54} &  {\bf 1.46} &  {\bf 1.37} &  {\bf 1.28} &  {\bf 1.20} &  {\bf 1.13} \\ 
\hline 
 2047 &  67 &  65 ans 8 mois &  6.67\% &  3786.93 &  {\bf 52.55} &  2292.61 &  {\bf 1.65} &  {\bf 1.59} &  {\bf 1.49} &  {\bf 1.40} &  {\bf 1.31} &  {\bf 1.23} \\ 
\hline 
\hline 
\end{tabular} 
\end{center} } 

 \begin{center}\includegraphics[width=0.9\textwidth]{fig/ProfAgrege_1980_22_dest_retraite.pdf}\end{center} \label{fig/ProfAgrege_1980_22_dest_retraite.pdf} 

\newpage 
 
\subsection{Génération 1990 (début en 2012)} 

\paragraph{Retraites possibles et ratios Revenu/SMIC à 70, 75, 80, 85, 90 ans avec le modèle \emph{Gouvernement truqué (âge-pivot bloqué à 65 ans)}}  
 
{ \scriptsize \begin{center} 
\begin{tabular}[htb]{|c|c||c|c||c|c||c||c|c|c|c|c|c|} 
\hline 
 Retraite en &  Âge &  Âge pivot &  Décote/Surcote &  Retraite (\euro{} 2019) &  Tx Rempl(\%) &  SMIC (\euro{} 2019) &  Retraite/SMIC &  Rev70/SMIC &  Rev75/SMIC &  Rev80/SMIC &  Rev85/SMIC &  Rev90/SMIC \\ 
\hline \hline 
 2052 &  62 &  65 ans 0 mois &  -15.00\% &  2269.42 &  {\bf 44.71} &  2601.14 &  {\bf {\color{red} 0.87}} &  {\bf {\color{red} 0.79}} &  {\bf {\color{red} 0.74}} &  {\bf {\color{red} 0.69}} &  {\bf {\color{red} 0.65}} &  {\bf {\color{red} 0.61}} \\ 
\hline 
 2053 &  63 &  65 ans 0 mois &  -10.00\% &  2497.87 &  {\bf 49.11} &  2634.96 &  {\bf {\color{red} 0.95}} &  {\bf {\color{red} 0.87}} &  {\bf {\color{red} 0.81}} &  {\bf {\color{red} 0.76}} &  {\bf {\color{red} 0.71}} &  {\bf {\color{red} 0.67}} \\ 
\hline 
 2054 &  64 &  65 ans 0 mois &  -5.00\% &  2738.31 &  {\bf 53.73} &  2669.21 &  {\bf 1.03} &  {\bf {\color{red} 0.95}} &  {\bf {\color{red} 0.89}} &  {\bf {\color{red} 0.83}} &  {\bf {\color{red} 0.78}} &  {\bf {\color{red} 0.73}} \\ 
\hline 
 2055 &  65 &  65 ans 0 mois &  0.00\% &  2991.00 &  {\bf 58.56} &  2703.91 &  {\bf 1.11} &  {\bf 1.04} &  {\bf {\color{red} 0.97}} &  {\bf {\color{red} 0.91}} &  {\bf {\color{red} 0.85}} &  {\bf {\color{red} 0.80}} \\ 
\hline 
 2056 &  66 &  65 ans 0 mois &  5.00\% &  3256.18 &  {\bf 63.62} &  2739.06 &  {\bf 1.19} &  {\bf 1.13} &  {\bf 1.06} &  {\bf {\color{red} 0.99}} &  {\bf {\color{red} 0.93}} &  {\bf {\color{red} 0.87}} \\ 
\hline 
 2057 &  67 &  65 ans 0 mois &  10.00\% &  3534.12 &  {\bf 68.90} &  2774.67 &  {\bf 1.27} &  {\bf 1.23} &  {\bf 1.15} &  {\bf 1.08} &  {\bf 1.01} &  {\bf {\color{red} 0.95}} \\ 
\hline 
\hline 
\end{tabular} 
\end{center} } 
\paragraph{Retraites possibles et ratios Revenu/SMIC à 70, 75, 80, 85, 90 ans avec le modèle \emph{Gouvernement corrigé (âge-pivot glissant)}}  
 
{ \scriptsize \begin{center} 
\begin{tabular}[htb]{|c|c||c|c||c|c||c||c|c|c|c|c|c|} 
\hline 
 Retraite en &  Âge &  Âge pivot &  Décote/Surcote &  Retraite (\euro{} 2019) &  Tx Rempl(\%) &  SMIC (\euro{} 2019) &  Retraite/SMIC &  Rev70/SMIC &  Rev75/SMIC &  Rev80/SMIC &  Rev85/SMIC &  Rev90/SMIC \\ 
\hline \hline 
 2052 &  62 &  66 ans 1 mois &  -20.42\% &  2124.80 &  {\bf 41.86} &  2601.14 &  {\bf {\color{red} 0.82}} &  {\bf {\color{red} 0.74}} &  {\bf {\color{red} 0.69}} &  {\bf {\color{red} 0.65}} &  {\bf {\color{red} 0.61}} &  {\bf {\color{red} 0.57}} \\ 
\hline 
 2053 &  63 &  66 ans 2 mois &  -15.83\% &  2335.97 &  {\bf 45.93} &  2634.96 &  {\bf {\color{red} 0.89}} &  {\bf {\color{red} 0.81}} &  {\bf {\color{red} 0.76}} &  {\bf {\color{red} 0.71}} &  {\bf {\color{red} 0.67}} &  {\bf {\color{red} 0.63}} \\ 
\hline 
 2054 &  64 &  66 ans 3 mois &  -11.25\% &  2558.16 &  {\bf 50.19} &  2669.21 &  {\bf {\color{red} 0.96}} &  {\bf {\color{red} 0.89}} &  {\bf {\color{red} 0.83}} &  {\bf {\color{red} 0.78}} &  {\bf {\color{red} 0.73}} &  {\bf {\color{red} 0.69}} \\ 
\hline 
 2055 &  65 &  66 ans 4 mois &  -6.67\% &  2791.60 &  {\bf 54.66} &  2703.91 &  {\bf 1.03} &  {\bf {\color{red} 0.97}} &  {\bf {\color{red} 0.91}} &  {\bf {\color{red} 0.85}} &  {\bf {\color{red} 0.80}} &  {\bf {\color{red} 0.75}} \\ 
\hline 
 2056 &  66 &  66 ans 5 mois &  -2.08\% &  3036.52 &  {\bf 59.33} &  2739.06 &  {\bf 1.11} &  {\bf 1.05} &  {\bf {\color{red} 0.99}} &  {\bf {\color{red} 0.93}} &  {\bf {\color{red} 0.87}} &  {\bf {\color{red} 0.81}} \\ 
\hline 
 2057 &  67 &  66 ans 6 mois &  2.50\% &  3293.15 &  {\bf 64.21} &  2774.67 &  {\bf 1.19} &  {\bf 1.14} &  {\bf 1.07} &  {\bf 1.00} &  {\bf {\color{red} 0.94}} &  {\bf {\color{red} 0.88}} \\ 
\hline 
\hline 
\end{tabular} 
\end{center} } 
\paragraph{Retraites possibles et ratios Revenu/SMIC à 70, 75, 80, 85, 90 ans avec le modèle \emph{Destinie2 (revalorisation de la fonction publique)}}  
 
{ \scriptsize \begin{center} 
\begin{tabular}[htb]{|c|c||c|c||c|c||c||c|c|c|c|c|c|} 
\hline 
 Retraite en &  Âge &  Âge pivot &  Décote/Surcote &  Retraite (\euro{} 2019) &  Tx Rempl(\%) &  SMIC (\euro{} 2019) &  Retraite/SMIC &  Rev70/SMIC &  Rev75/SMIC &  Rev80/SMIC &  Rev85/SMIC &  Rev90/SMIC \\ 
\hline \hline 
 2052 &  62 &  66 ans 1 mois &  -20.42\% &  2694.08 &  {\bf 35.05} &  2445.56 &  {\bf 1.10} &  {\bf {\color{red} 0.99}} &  {\bf {\color{red} 0.93}} &  {\bf {\color{red} 0.87}} &  {\bf {\color{red} 0.82}} &  {\bf {\color{red} 0.77}} \\ 
\hline 
 2053 &  63 &  66 ans 2 mois &  -15.83\% &  2977.50 &  {\bf 38.24} &  2477.35 &  {\bf 1.20} &  {\bf 1.10} &  {\bf 1.03} &  {\bf {\color{red} 0.96}} &  {\bf {\color{red} 0.90}} &  {\bf {\color{red} 0.85}} \\ 
\hline 
 2054 &  64 &  66 ans 3 mois &  -11.25\% &  3277.91 &  {\bf 41.55} &  2509.56 &  {\bf 1.31} &  {\bf 1.21} &  {\bf 1.13} &  {\bf 1.06} &  {\bf {\color{red} 1.00}} &  {\bf {\color{red} 0.93}} \\ 
\hline 
 2055 &  65 &  66 ans 4 mois &  -6.67\% &  3595.82 &  {\bf 45.00} &  2542.18 &  {\bf 1.41} &  {\bf 1.33} &  {\bf 1.24} &  {\bf 1.17} &  {\bf 1.09} &  {\bf 1.02} \\ 
\hline 
 2056 &  66 &  66 ans 5 mois &  -2.08\% &  3931.76 &  {\bf 48.57} &  2575.23 &  {\bf 1.53} &  {\bf 1.45} &  {\bf 1.36} &  {\bf 1.27} &  {\bf 1.19} &  {\bf 1.12} \\ 
\hline 
 2057 &  67 &  66 ans 6 mois &  2.50\% &  4286.30 &  {\bf 52.27} &  2608.71 &  {\bf 1.64} &  {\bf 1.58} &  {\bf 1.48} &  {\bf 1.39} &  {\bf 1.30} &  {\bf 1.22} \\ 
\hline 
\hline 
\end{tabular} 
\end{center} } 

 \begin{center}\includegraphics[width=0.9\textwidth]{fig/ProfAgrege_1990_22_dest_retraite.pdf}\end{center} \label{fig/ProfAgrege_1990_22_dest_retraite.pdf} 

\newpage 
 
\subsection{Génération 2003 (début en 2025)} 

\paragraph{Retraites possibles et ratios Revenu/SMIC à 70, 75, 80, 85, 90 ans avec le modèle \emph{Gouvernement truqué (âge-pivot bloqué à 65 ans)}}  
 
{ \scriptsize \begin{center} 
\begin{tabular}[htb]{|c|c||c|c||c|c||c||c|c|c|c|c|c|} 
\hline 
 Retraite en &  Âge &  Âge pivot &  Décote/Surcote &  Retraite (\euro{} 2019) &  Tx Rempl(\%) &  SMIC (\euro{} 2019) &  Retraite/SMIC &  Rev70/SMIC &  Rev75/SMIC &  Rev80/SMIC &  Rev85/SMIC &  Rev90/SMIC \\ 
\hline \hline 
 2065 &  62 &  65 ans 0 mois &  -15.00\% &  2415.39 &  {\bf 47.59} &  3076.71 &  {\bf {\color{red} 0.79}} &  {\bf {\color{red} 0.71}} &  {\bf {\color{red} 0.66}} &  {\bf {\color{red} 0.62}} &  {\bf {\color{red} 0.58}} &  {\bf {\color{red} 0.55}} \\ 
\hline 
 2066 &  63 &  65 ans 0 mois &  -10.00\% &  2654.44 &  {\bf 52.19} &  3116.71 &  {\bf {\color{red} 0.85}} &  {\bf {\color{red} 0.78}} &  {\bf {\color{red} 0.73}} &  {\bf {\color{red} 0.68}} &  {\bf {\color{red} 0.64}} &  {\bf {\color{red} 0.60}} \\ 
\hline 
 2067 &  64 &  65 ans 0 mois &  -5.00\% &  2905.73 &  {\bf 57.01} &  3157.23 &  {\bf {\color{red} 0.92}} &  {\bf {\color{red} 0.85}} &  {\bf {\color{red} 0.80}} &  {\bf {\color{red} 0.75}} &  {\bf {\color{red} 0.70}} &  {\bf {\color{red} 0.66}} \\ 
\hline 
 2068 &  65 &  65 ans 0 mois &  0.00\% &  3169.52 &  {\bf 62.06} &  3198.27 &  {\bf {\color{red} 0.99}} &  {\bf {\color{red} 0.93}} &  {\bf {\color{red} 0.87}} &  {\bf {\color{red} 0.82}} &  {\bf {\color{red} 0.77}} &  {\bf {\color{red} 0.72}} \\ 
\hline 
 2069 &  66 &  65 ans 0 mois &  5.00\% &  3446.06 &  {\bf 67.33} &  3239.85 &  {\bf 1.06} &  {\bf 1.01} &  {\bf {\color{red} 0.95}} &  {\bf {\color{red} 0.89}} &  {\bf {\color{red} 0.83}} &  {\bf {\color{red} 0.78}} \\ 
\hline 
 2070 &  67 &  65 ans 0 mois &  10.00\% &  3735.62 &  {\bf 72.83} &  3281.97 &  {\bf 1.14} &  {\bf 1.09} &  {\bf 1.03} &  {\bf {\color{red} 0.96}} &  {\bf {\color{red} 0.90}} &  {\bf {\color{red} 0.85}} \\ 
\hline 
\hline 
\end{tabular} 
\end{center} } 
\paragraph{Retraites possibles et ratios Revenu/SMIC à 70, 75, 80, 85, 90 ans avec le modèle \emph{Gouvernement corrigé (âge-pivot glissant)}}  
 
{ \scriptsize \begin{center} 
\begin{tabular}[htb]{|c|c||c|c||c|c||c||c|c|c|c|c|c|} 
\hline 
 Retraite en &  Âge &  Âge pivot &  Décote/Surcote &  Retraite (\euro{} 2019) &  Tx Rempl(\%) &  SMIC (\euro{} 2019) &  Retraite/SMIC &  Rev70/SMIC &  Rev75/SMIC &  Rev80/SMIC &  Rev85/SMIC &  Rev90/SMIC \\ 
\hline \hline 
 2065 &  62 &  67 ans 2 mois &  -25.83\% &  2107.55 &  {\bf 41.52} &  3076.71 &  {\bf {\color{red} 0.68}} &  {\bf {\color{red} 0.62}} &  {\bf {\color{red} 0.58}} &  {\bf {\color{red} 0.54}} &  {\bf {\color{red} 0.51}} &  {\bf {\color{red} 0.48}} \\ 
\hline 
 2066 &  63 &  67 ans 3 mois &  -21.25\% &  2322.63 &  {\bf 45.67} &  3116.71 &  {\bf {\color{red} 0.75}} &  {\bf {\color{red} 0.68}} &  {\bf {\color{red} 0.64}} &  {\bf {\color{red} 0.60}} &  {\bf {\color{red} 0.56}} &  {\bf {\color{red} 0.53}} \\ 
\hline 
 2067 &  64 &  67 ans 4 mois &  -16.67\% &  2548.88 &  {\bf 50.01} &  3157.23 &  {\bf {\color{red} 0.81}} &  {\bf {\color{red} 0.75}} &  {\bf {\color{red} 0.70}} &  {\bf {\color{red} 0.66}} &  {\bf {\color{red} 0.62}} &  {\bf {\color{red} 0.58}} \\ 
\hline 
 2068 &  65 &  67 ans 5 mois &  -12.08\% &  2786.53 &  {\bf 54.56} &  3198.27 &  {\bf {\color{red} 0.87}} &  {\bf {\color{red} 0.82}} &  {\bf {\color{red} 0.77}} &  {\bf {\color{red} 0.72}} &  {\bf {\color{red} 0.67}} &  {\bf {\color{red} 0.63}} \\ 
\hline 
 2069 &  66 &  67 ans 6 mois &  -7.50\% &  3035.82 &  {\bf 59.31} &  3239.85 &  {\bf {\color{red} 0.94}} &  {\bf {\color{red} 0.89}} &  {\bf {\color{red} 0.83}} &  {\bf {\color{red} 0.78}} &  {\bf {\color{red} 0.73}} &  {\bf {\color{red} 0.69}} \\ 
\hline 
 2070 &  67 &  67 ans 7 mois &  -2.92\% &  3296.97 &  {\bf 64.28} &  3281.97 &  {\bf 1.00} &  {\bf {\color{red} 0.97}} &  {\bf {\color{red} 0.91}} &  {\bf {\color{red} 0.85}} &  {\bf {\color{red} 0.80}} &  {\bf {\color{red} 0.75}} \\ 
\hline 
\hline 
\end{tabular} 
\end{center} } 
\paragraph{Retraites possibles et ratios Revenu/SMIC à 70, 75, 80, 85, 90 ans avec le modèle \emph{Destinie2 (revalorisation de la fonction publique)}}  
 
{ \scriptsize \begin{center} 
\begin{tabular}[htb]{|c|c||c|c||c|c||c||c|c|c|c|c|c|} 
\hline 
 Retraite en &  Âge &  Âge pivot &  Décote/Surcote &  Retraite (\euro{} 2019) &  Tx Rempl(\%) &  SMIC (\euro{} 2019) &  Retraite/SMIC &  Rev70/SMIC &  Rev75/SMIC &  Rev80/SMIC &  Rev85/SMIC &  Rev90/SMIC \\ 
\hline \hline 
 2065 &  62 &  67 ans 2 mois &  -25.83\% &  3115.87 &  {\bf 34.27} &  2892.68 &  {\bf 1.08} &  {\bf {\color{red} 0.97}} &  {\bf {\color{red} 0.91}} &  {\bf {\color{red} 0.85}} &  {\bf {\color{red} 0.80}} &  {\bf {\color{red} 0.75}} \\ 
\hline 
 2066 &  63 &  67 ans 3 mois &  -21.25\% &  3452.40 &  {\bf 37.48} &  2930.29 &  {\bf 1.18} &  {\bf 1.08} &  {\bf 1.01} &  {\bf {\color{red} 0.95}} &  {\bf {\color{red} 0.89}} &  {\bf {\color{red} 0.83}} \\ 
\hline 
 2067 &  64 &  67 ans 4 mois &  -16.67\% &  3809.05 &  {\bf 40.82} &  2968.38 &  {\bf 1.28} &  {\bf 1.19} &  {\bf 1.11} &  {\bf 1.04} &  {\bf {\color{red} 0.98}} &  {\bf {\color{red} 0.92}} \\ 
\hline 
 2068 &  65 &  67 ans 5 mois &  -12.08\% &  4186.46 &  {\bf 44.29} &  3006.97 &  {\bf 1.39} &  {\bf 1.31} &  {\bf 1.22} &  {\bf 1.15} &  {\bf 1.08} &  {\bf 1.01} \\ 
\hline 
 2069 &  66 &  67 ans 6 mois &  -7.50\% &  4585.25 &  {\bf 47.89} &  3046.06 &  {\bf 1.51} &  {\bf 1.43} &  {\bf 1.34} &  {\bf 1.26} &  {\bf 1.18} &  {\bf 1.10} \\ 
\hline 
 2070 &  67 &  67 ans 7 mois &  -2.92\% &  5006.08 &  {\bf 51.61} &  3085.66 &  {\bf 1.62} &  {\bf 1.56} &  {\bf 1.46} &  {\bf 1.37} &  {\bf 1.29} &  {\bf 1.21} \\ 
\hline 
\hline 
\end{tabular} 
\end{center} } 

 \begin{center}\includegraphics[width=0.9\textwidth]{fig/ProfAgrege_2003_22_dest_retraite.pdf}\end{center} \label{fig/ProfAgrege_2003_22_dest_retraite.pdf} 

\newpage 
 
\chapter{BIATSS (CN puis CS)} 

\begin{minipage}{0.55\linewidth}\includegraphics[width=0.7\textwidth]{fig/grille_BIATSS.pdf}\end{minipage} 
\begin{minipage}{0.3\linewidth} 
 \begin{center} 

\begin{tabular}[htb]{|c|c|} 
\hline 
 Indice majoré &  Durée (années) \\ 
\hline \hline 
 343 &  2.00 \\ 
\hline 
 349 &  2.00 \\ 
\hline 
 355 &  2.00 \\ 
\hline 
 361 &  2.00 \\ 
\hline 
 369 &  2.00 \\ 
\hline 
 381 &  2.00 \\ 
\hline 
 396 &  2.00 \\ 
\hline 
 415 &  3.00 \\ 
\hline 
 431 &  3.00 \\ 
\hline 
 441 &  3.00 \\ 
\hline 
 457 &  3.00 \\ 
\hline 
 477 &  4.00 \\ 
\hline 
 503 &  4.00 \\ 
\hline 
 534 &   \\ 
\hline 
\hline 
\end{tabular} 
\end{center} 
 \end{minipage} 


 \addto{\captionsenglish}{ \renewcommand{\mtctitle}{}} \setcounter{minitocdepth}{2} 
 \minitoc \newpage 

\section{Début de carrière à 22 ans} 

\subsection{Génération 1975 (début en 1997)} 

\paragraph{Retraites possibles et ratios Revenu/SMIC à 70, 75, 80, 85, 90 ans avec le modèle \emph{Gouvernement truqué (âge-pivot bloqué à 65 ans)}}  
 
{ \scriptsize \begin{center} 
\begin{tabular}[htb]{|c|c||c|c||c|c||c||c|c|c|c|c|c|} 
\hline 
 Retraite en &  Âge &  Âge pivot &  Décote/Surcote &  Retraite (\euro{} 2019) &  Tx Rempl(\%) &  SMIC (\euro{} 2019) &  Retraite/SMIC &  Rev70/SMIC &  Rev75/SMIC &  Rev80/SMIC &  Rev85/SMIC &  Rev90/SMIC \\ 
\hline \hline 
 2037 &  62 &  64 ans 10 mois &  -14.17\% &  1156.45 &  {\bf 41.32} &  2143.00 &  {\bf {\color{red} 0.54}} &  {\bf {\color{red} 0.49}} &  {\bf {\color{red} 0.46}} &  {\bf {\color{red} 0.43}} &  {\bf {\color{red} 0.40}} &  {\bf {\color{red} 0.38}} \\ 
\hline 
 2038 &  63 &  64 ans 11 mois &  -9.58\% &  1262.78 &  {\bf 45.03} &  2170.86 &  {\bf {\color{red} 0.58}} &  {\bf {\color{red} 0.53}} &  {\bf {\color{red} 0.50}} &  {\bf {\color{red} 0.47}} &  {\bf {\color{red} 0.44}} &  {\bf {\color{red} 0.41}} \\ 
\hline 
 2039 &  64 &  65 ans 0 mois &  -5.00\% &  1375.09 &  {\bf 48.93} &  2199.08 &  {\bf {\color{red} 0.63}} &  {\bf {\color{red} 0.58}} &  {\bf {\color{red} 0.54}} &  {\bf {\color{red} 0.51}} &  {\bf {\color{red} 0.48}} &  {\bf {\color{red} 0.45}} \\ 
\hline 
 2040 &  65 &  65 ans 0 mois &  0.00\% &  1893.52 &  {\bf 67.24} &  2227.67 &  {\bf {\color{red} 0.85}} &  {\bf {\color{red} 0.80}} &  {\bf {\color{red} 0.75}} &  {\bf {\color{red} 0.70}} &  {\bf {\color{red} 0.66}} &  {\bf {\color{red} 0.62}} \\ 
\hline 
 2041 &  66 &  65 ans 0 mois &  5.00\% &  1918.14 &  {\bf 67.97} &  2256.63 &  {\bf {\color{red} 0.85}} &  {\bf {\color{red} 0.81}} &  {\bf {\color{red} 0.76}} &  {\bf {\color{red} 0.71}} &  {\bf {\color{red} 0.67}} &  {\bf {\color{red} 0.62}} \\ 
\hline 
 2042 &  67 &  65 ans 0 mois &  10.00\% &  1943.07 &  {\bf 68.71} &  2285.97 &  {\bf {\color{red} 0.85}} &  {\bf {\color{red} 0.82}} &  {\bf {\color{red} 0.77}} &  {\bf {\color{red} 0.72}} &  {\bf {\color{red} 0.67}} &  {\bf {\color{red} 0.63}} \\ 
\hline 
\hline 
\end{tabular} 
\end{center} } 
\paragraph{Retraites possibles et ratios Revenu/SMIC à 70, 75, 80, 85, 90 ans avec le modèle \emph{Gouvernement corrigé (âge-pivot glissant)}}  
 
{ \scriptsize \begin{center} 
\begin{tabular}[htb]{|c|c||c|c||c|c||c||c|c|c|c|c|c|} 
\hline 
 Retraite en &  Âge &  Âge pivot &  Décote/Surcote &  Retraite (\euro{} 2019) &  Tx Rempl(\%) &  SMIC (\euro{} 2019) &  Retraite/SMIC &  Rev70/SMIC &  Rev75/SMIC &  Rev80/SMIC &  Rev85/SMIC &  Rev90/SMIC \\ 
\hline \hline 
 2037 &  62 &  64 ans 10 mois &  -14.17\% &  1156.45 &  {\bf 41.32} &  2143.00 &  {\bf {\color{red} 0.54}} &  {\bf {\color{red} 0.49}} &  {\bf {\color{red} 0.46}} &  {\bf {\color{red} 0.43}} &  {\bf {\color{red} 0.40}} &  {\bf {\color{red} 0.38}} \\ 
\hline 
 2038 &  63 &  64 ans 11 mois &  -9.58\% &  1262.78 &  {\bf 45.03} &  2170.86 &  {\bf {\color{red} 0.58}} &  {\bf {\color{red} 0.53}} &  {\bf {\color{red} 0.50}} &  {\bf {\color{red} 0.47}} &  {\bf {\color{red} 0.44}} &  {\bf {\color{red} 0.41}} \\ 
\hline 
 2039 &  64 &  65 ans 0 mois &  -5.00\% &  1375.09 &  {\bf 48.93} &  2199.08 &  {\bf {\color{red} 0.63}} &  {\bf {\color{red} 0.58}} &  {\bf {\color{red} 0.54}} &  {\bf {\color{red} 0.51}} &  {\bf {\color{red} 0.48}} &  {\bf {\color{red} 0.45}} \\ 
\hline 
 2040 &  65 &  65 ans 1 mois &  -0.42\% &  1893.52 &  {\bf 67.24} &  2227.67 &  {\bf {\color{red} 0.85}} &  {\bf {\color{red} 0.80}} &  {\bf {\color{red} 0.75}} &  {\bf {\color{red} 0.70}} &  {\bf {\color{red} 0.66}} &  {\bf {\color{red} 0.62}} \\ 
\hline 
 2041 &  66 &  65 ans 2 mois &  4.17\% &  1918.14 &  {\bf 67.97} &  2256.63 &  {\bf {\color{red} 0.85}} &  {\bf {\color{red} 0.81}} &  {\bf {\color{red} 0.76}} &  {\bf {\color{red} 0.71}} &  {\bf {\color{red} 0.67}} &  {\bf {\color{red} 0.62}} \\ 
\hline 
 2042 &  67 &  65 ans 3 mois &  8.75\% &  1943.07 &  {\bf 68.71} &  2285.97 &  {\bf {\color{red} 0.85}} &  {\bf {\color{red} 0.82}} &  {\bf {\color{red} 0.77}} &  {\bf {\color{red} 0.72}} &  {\bf {\color{red} 0.67}} &  {\bf {\color{red} 0.63}} \\ 
\hline 
\hline 
\end{tabular} 
\end{center} } 
\paragraph{Retraites possibles et ratios Revenu/SMIC à 70, 75, 80, 85, 90 ans avec le modèle \emph{Destinie2 (revalorisation de la fonction publique)}}  
 
{ \scriptsize \begin{center} 
\begin{tabular}[htb]{|c|c||c|c||c|c||c||c|c|c|c|c|c|} 
\hline 
 Retraite en &  Âge &  Âge pivot &  Décote/Surcote &  Retraite (\euro{} 2019) &  Tx Rempl(\%) &  SMIC (\euro{} 2019) &  Retraite/SMIC &  Rev70/SMIC &  Rev75/SMIC &  Rev80/SMIC &  Rev85/SMIC &  Rev90/SMIC \\ 
\hline \hline 
 2037 &  62 &  64 ans 10 mois &  -14.17\% &  1283.31 &  {\bf 36.75} &  2014.82 &  {\bf {\color{red} 0.64}} &  {\bf {\color{red} 0.57}} &  {\bf {\color{red} 0.54}} &  {\bf {\color{red} 0.50}} &  {\bf {\color{red} 0.47}} &  {\bf {\color{red} 0.44}} \\ 
\hline 
 2038 &  63 &  64 ans 11 mois &  -9.58\% &  1406.67 &  {\bf 39.76} &  2041.01 &  {\bf {\color{red} 0.69}} &  {\bf {\color{red} 0.63}} &  {\bf {\color{red} 0.59}} &  {\bf {\color{red} 0.55}} &  {\bf {\color{red} 0.52}} &  {\bf {\color{red} 0.49}} \\ 
\hline 
 2039 &  64 &  65 ans 0 mois &  -5.00\% &  1537.76 &  {\bf 42.91} &  2067.55 &  {\bf {\color{red} 0.74}} &  {\bf {\color{red} 0.69}} &  {\bf {\color{red} 0.65}} &  {\bf {\color{red} 0.60}} &  {\bf {\color{red} 0.57}} &  {\bf {\color{red} 0.53}} \\ 
\hline 
 2040 &  65 &  65 ans 1 mois &  -0.42\% &  1780.26 &  {\bf 49.04} &  2094.43 &  {\bf {\color{red} 0.85}} &  {\bf {\color{red} 0.80}} &  {\bf {\color{red} 0.75}} &  {\bf {\color{red} 0.70}} &  {\bf {\color{red} 0.66}} &  {\bf {\color{red} 0.62}} \\ 
\hline 
 2041 &  66 &  65 ans 2 mois &  4.17\% &  1824.95 &  {\bf 49.63} &  2121.65 &  {\bf {\color{red} 0.86}} &  {\bf {\color{red} 0.82}} &  {\bf {\color{red} 0.77}} &  {\bf {\color{red} 0.72}} &  {\bf {\color{red} 0.67}} &  {\bf {\color{red} 0.63}} \\ 
\hline 
 2042 &  67 &  65 ans 3 mois &  8.75\% &  1982.01 &  {\bf 53.21} &  2149.23 &  {\bf {\color{red} 0.92}} &  {\bf {\color{red} 0.89}} &  {\bf {\color{red} 0.83}} &  {\bf {\color{red} 0.78}} &  {\bf {\color{red} 0.73}} &  {\bf {\color{red} 0.69}} \\ 
\hline 
\hline 
\end{tabular} 
\end{center} } 

 \begin{center}\includegraphics[width=0.9\textwidth]{fig/BIATSS_1975_22_dest_retraite.pdf}\end{center} \label{fig/BIATSS_1975_22_dest_retraite.pdf} 

\newpage 
 
\subsection{Génération 1980 (début en 2002)} 

\paragraph{Retraites possibles et ratios Revenu/SMIC à 70, 75, 80, 85, 90 ans avec le modèle \emph{Gouvernement truqué (âge-pivot bloqué à 65 ans)}}  
 
{ \scriptsize \begin{center} 
\begin{tabular}[htb]{|c|c||c|c||c|c||c||c|c|c|c|c|c|} 
\hline 
 Retraite en &  Âge &  Âge pivot &  Décote/Surcote &  Retraite (\euro{} 2019) &  Tx Rempl(\%) &  SMIC (\euro{} 2019) &  Retraite/SMIC &  Rev70/SMIC &  Rev75/SMIC &  Rev80/SMIC &  Rev85/SMIC &  Rev90/SMIC \\ 
\hline \hline 
 2042 &  62 &  65 ans 0 mois &  -15.00\% &  1167.64 &  {\bf 41.72} &  2285.97 &  {\bf {\color{red} 0.51}} &  {\bf {\color{red} 0.46}} &  {\bf {\color{red} 0.43}} &  {\bf {\color{red} 0.40}} &  {\bf {\color{red} 0.38}} &  {\bf {\color{red} 0.36}} \\ 
\hline 
 2043 &  63 &  65 ans 0 mois &  -10.00\% &  1285.62 &  {\bf 45.84} &  2315.68 &  {\bf {\color{red} 0.56}} &  {\bf {\color{red} 0.51}} &  {\bf {\color{red} 0.48}} &  {\bf {\color{red} 0.45}} &  {\bf {\color{red} 0.42}} &  {\bf {\color{red} 0.39}} \\ 
\hline 
 2044 &  64 &  65 ans 0 mois &  -5.00\% &  1410.81 &  {\bf 50.20} &  2345.79 &  {\bf {\color{red} 0.60}} &  {\bf {\color{red} 0.56}} &  {\bf {\color{red} 0.52}} &  {\bf {\color{red} 0.49}} &  {\bf {\color{red} 0.46}} &  {\bf {\color{red} 0.43}} \\ 
\hline 
 2045 &  65 &  65 ans 0 mois &  0.00\% &  2019.84 &  {\bf 71.72} &  2376.28 &  {\bf {\color{red} 0.85}} &  {\bf {\color{red} 0.80}} &  {\bf {\color{red} 0.75}} &  {\bf {\color{red} 0.70}} &  {\bf {\color{red} 0.66}} &  {\bf {\color{red} 0.62}} \\ 
\hline 
 2046 &  66 &  65 ans 0 mois &  5.00\% &  2046.10 &  {\bf 72.50} &  2407.18 &  {\bf {\color{red} 0.85}} &  {\bf {\color{red} 0.81}} &  {\bf {\color{red} 0.76}} &  {\bf {\color{red} 0.71}} &  {\bf {\color{red} 0.67}} &  {\bf {\color{red} 0.62}} \\ 
\hline 
 2047 &  67 &  65 ans 0 mois &  10.00\% &  2072.70 &  {\bf 73.29} &  2438.47 &  {\bf {\color{red} 0.85}} &  {\bf {\color{red} 0.82}} &  {\bf {\color{red} 0.77}} &  {\bf {\color{red} 0.72}} &  {\bf {\color{red} 0.67}} &  {\bf {\color{red} 0.63}} \\ 
\hline 
\hline 
\end{tabular} 
\end{center} } 
\paragraph{Retraites possibles et ratios Revenu/SMIC à 70, 75, 80, 85, 90 ans avec le modèle \emph{Gouvernement corrigé (âge-pivot glissant)}}  
 
{ \scriptsize \begin{center} 
\begin{tabular}[htb]{|c|c||c|c||c|c||c||c|c|c|c|c|c|} 
\hline 
 Retraite en &  Âge &  Âge pivot &  Décote/Surcote &  Retraite (\euro{} 2019) &  Tx Rempl(\%) &  SMIC (\euro{} 2019) &  Retraite/SMIC &  Rev70/SMIC &  Rev75/SMIC &  Rev80/SMIC &  Rev85/SMIC &  Rev90/SMIC \\ 
\hline \hline 
 2042 &  62 &  65 ans 3 mois &  -16.25\% &  1150.47 &  {\bf 41.11} &  2285.97 &  {\bf {\color{red} 0.50}} &  {\bf {\color{red} 0.45}} &  {\bf {\color{red} 0.43}} &  {\bf {\color{red} 0.40}} &  {\bf {\color{red} 0.37}} &  {\bf {\color{red} 0.35}} \\ 
\hline 
 2043 &  63 &  65 ans 4 mois &  -11.67\% &  1261.82 &  {\bf 44.99} &  2315.68 &  {\bf {\color{red} 0.54}} &  {\bf {\color{red} 0.50}} &  {\bf {\color{red} 0.47}} &  {\bf {\color{red} 0.44}} &  {\bf {\color{red} 0.41}} &  {\bf {\color{red} 0.38}} \\ 
\hline 
 2044 &  64 &  65 ans 5 mois &  -7.08\% &  1379.87 &  {\bf 49.10} &  2345.79 &  {\bf {\color{red} 0.59}} &  {\bf {\color{red} 0.54}} &  {\bf {\color{red} 0.51}} &  {\bf {\color{red} 0.48}} &  {\bf {\color{red} 0.45}} &  {\bf {\color{red} 0.42}} \\ 
\hline 
 2045 &  65 &  65 ans 6 mois &  -2.50\% &  2019.84 &  {\bf 71.72} &  2376.28 &  {\bf {\color{red} 0.85}} &  {\bf {\color{red} 0.80}} &  {\bf {\color{red} 0.75}} &  {\bf {\color{red} 0.70}} &  {\bf {\color{red} 0.66}} &  {\bf {\color{red} 0.62}} \\ 
\hline 
 2046 &  66 &  65 ans 7 mois &  2.08\% &  2046.10 &  {\bf 72.50} &  2407.18 &  {\bf {\color{red} 0.85}} &  {\bf {\color{red} 0.81}} &  {\bf {\color{red} 0.76}} &  {\bf {\color{red} 0.71}} &  {\bf {\color{red} 0.67}} &  {\bf {\color{red} 0.62}} \\ 
\hline 
 2047 &  67 &  65 ans 8 mois &  6.67\% &  2072.70 &  {\bf 73.29} &  2438.47 &  {\bf {\color{red} 0.85}} &  {\bf {\color{red} 0.82}} &  {\bf {\color{red} 0.77}} &  {\bf {\color{red} 0.72}} &  {\bf {\color{red} 0.67}} &  {\bf {\color{red} 0.63}} \\ 
\hline 
\hline 
\end{tabular} 
\end{center} } 
\paragraph{Retraites possibles et ratios Revenu/SMIC à 70, 75, 80, 85, 90 ans avec le modèle \emph{Destinie2 (revalorisation de la fonction publique)}}  
 
{ \scriptsize \begin{center} 
\begin{tabular}[htb]{|c|c||c|c||c|c||c||c|c|c|c|c|c|} 
\hline 
 Retraite en &  Âge &  Âge pivot &  Décote/Surcote &  Retraite (\euro{} 2019) &  Tx Rempl(\%) &  SMIC (\euro{} 2019) &  Retraite/SMIC &  Rev70/SMIC &  Rev75/SMIC &  Rev80/SMIC &  Rev85/SMIC &  Rev90/SMIC \\ 
\hline \hline 
 2042 &  62 &  65 ans 3 mois &  -16.25\% &  1322.18 &  {\bf 35.49} &  2149.23 &  {\bf {\color{red} 0.62}} &  {\bf {\color{red} 0.55}} &  {\bf {\color{red} 0.52}} &  {\bf {\color{red} 0.49}} &  {\bf {\color{red} 0.46}} &  {\bf {\color{red} 0.43}} \\ 
\hline 
 2043 &  63 &  65 ans 4 mois &  -11.67\% &  1456.92 &  {\bf 38.61} &  2177.17 &  {\bf {\color{red} 0.67}} &  {\bf {\color{red} 0.61}} &  {\bf {\color{red} 0.57}} &  {\bf {\color{red} 0.54}} &  {\bf {\color{red} 0.50}} &  {\bf {\color{red} 0.47}} \\ 
\hline 
 2044 &  64 &  65 ans 5 mois &  -7.08\% &  1600.69 &  {\bf 41.88} &  2205.48 &  {\bf {\color{red} 0.73}} &  {\bf {\color{red} 0.67}} &  {\bf {\color{red} 0.63}} &  {\bf {\color{red} 0.59}} &  {\bf {\color{red} 0.55}} &  {\bf {\color{red} 0.52}} \\ 
\hline 
 2045 &  65 &  65 ans 6 mois &  -2.50\% &  1899.03 &  {\bf 49.04} &  2234.15 &  {\bf {\color{red} 0.85}} &  {\bf {\color{red} 0.80}} &  {\bf {\color{red} 0.75}} &  {\bf {\color{red} 0.70}} &  {\bf {\color{red} 0.66}} &  {\bf {\color{red} 0.62}} \\ 
\hline 
 2046 &  66 &  65 ans 7 mois &  2.08\% &  1923.71 &  {\bf 49.04} &  2263.19 &  {\bf {\color{red} 0.85}} &  {\bf {\color{red} 0.81}} &  {\bf {\color{red} 0.76}} &  {\bf {\color{red} 0.71}} &  {\bf {\color{red} 0.67}} &  {\bf {\color{red} 0.62}} \\ 
\hline 
 2047 &  67 &  65 ans 8 mois &  6.67\% &  2087.16 &  {\bf 52.53} &  2292.61 &  {\bf {\color{red} 0.91}} &  {\bf {\color{red} 0.88}} &  {\bf {\color{red} 0.82}} &  {\bf {\color{red} 0.77}} &  {\bf {\color{red} 0.72}} &  {\bf {\color{red} 0.68}} \\ 
\hline 
\hline 
\end{tabular} 
\end{center} } 

 \begin{center}\includegraphics[width=0.9\textwidth]{fig/BIATSS_1980_22_dest_retraite.pdf}\end{center} \label{fig/BIATSS_1980_22_dest_retraite.pdf} 

\newpage 
 
\subsection{Génération 1990 (début en 2012)} 

\paragraph{Retraites possibles et ratios Revenu/SMIC à 70, 75, 80, 85, 90 ans avec le modèle \emph{Gouvernement truqué (âge-pivot bloqué à 65 ans)}}  
 
{ \scriptsize \begin{center} 
\begin{tabular}[htb]{|c|c||c|c||c|c||c||c|c|c|c|c|c|} 
\hline 
 Retraite en &  Âge &  Âge pivot &  Décote/Surcote &  Retraite (\euro{} 2019) &  Tx Rempl(\%) &  SMIC (\euro{} 2019) &  Retraite/SMIC &  Rev70/SMIC &  Rev75/SMIC &  Rev80/SMIC &  Rev85/SMIC &  Rev90/SMIC \\ 
\hline \hline 
 2052 &  62 &  65 ans 0 mois &  -15.00\% &  1251.06 &  {\bf 44.70} &  2601.14 &  {\bf {\color{red} 0.48}} &  {\bf {\color{red} 0.43}} &  {\bf {\color{red} 0.41}} &  {\bf {\color{red} 0.38}} &  {\bf {\color{red} 0.36}} &  {\bf {\color{red} 0.34}} \\ 
\hline 
 2053 &  63 &  65 ans 0 mois &  -10.00\% &  1377.00 &  {\bf 49.10} &  2634.96 &  {\bf {\color{red} 0.52}} &  {\bf {\color{red} 0.48}} &  {\bf {\color{red} 0.45}} &  {\bf {\color{red} 0.42}} &  {\bf {\color{red} 0.39}} &  {\bf {\color{red} 0.37}} \\ 
\hline 
 2054 &  64 &  65 ans 0 mois &  -5.00\% &  1509.56 &  {\bf 53.71} &  2669.21 &  {\bf {\color{red} 0.57}} &  {\bf {\color{red} 0.52}} &  {\bf {\color{red} 0.49}} &  {\bf {\color{red} 0.46}} &  {\bf {\color{red} 0.43}} &  {\bf {\color{red} 0.40}} \\ 
\hline 
 2055 &  65 &  65 ans 0 mois &  0.00\% &  2298.33 &  {\bf 81.61} &  2703.91 &  {\bf {\color{red} 0.85}} &  {\bf {\color{red} 0.80}} &  {\bf {\color{red} 0.75}} &  {\bf {\color{red} 0.70}} &  {\bf {\color{red} 0.66}} &  {\bf {\color{red} 0.62}} \\ 
\hline 
 2056 &  66 &  65 ans 0 mois &  5.00\% &  2328.20 &  {\bf 82.50} &  2739.06 &  {\bf {\color{red} 0.85}} &  {\bf {\color{red} 0.81}} &  {\bf {\color{red} 0.76}} &  {\bf {\color{red} 0.71}} &  {\bf {\color{red} 0.67}} &  {\bf {\color{red} 0.62}} \\ 
\hline 
 2057 &  67 &  65 ans 0 mois &  10.00\% &  2358.47 &  {\bf 83.40} &  2774.67 &  {\bf {\color{red} 0.85}} &  {\bf {\color{red} 0.82}} &  {\bf {\color{red} 0.77}} &  {\bf {\color{red} 0.72}} &  {\bf {\color{red} 0.67}} &  {\bf {\color{red} 0.63}} \\ 
\hline 
\hline 
\end{tabular} 
\end{center} } 
\paragraph{Retraites possibles et ratios Revenu/SMIC à 70, 75, 80, 85, 90 ans avec le modèle \emph{Gouvernement corrigé (âge-pivot glissant)}}  
 
{ \scriptsize \begin{center} 
\begin{tabular}[htb]{|c|c||c|c||c|c||c||c|c|c|c|c|c|} 
\hline 
 Retraite en &  Âge &  Âge pivot &  Décote/Surcote &  Retraite (\euro{} 2019) &  Tx Rempl(\%) &  SMIC (\euro{} 2019) &  Retraite/SMIC &  Rev70/SMIC &  Rev75/SMIC &  Rev80/SMIC &  Rev85/SMIC &  Rev90/SMIC \\ 
\hline \hline 
 2052 &  62 &  66 ans 1 mois &  -20.42\% &  1171.33 &  {\bf 41.85} &  2601.14 &  {\bf {\color{red} 0.45}} &  {\bf {\color{red} 0.41}} &  {\bf {\color{red} 0.38}} &  {\bf {\color{red} 0.36}} &  {\bf {\color{red} 0.33}} &  {\bf {\color{red} 0.31}} \\ 
\hline 
 2053 &  63 &  66 ans 2 mois &  -15.83\% &  1287.75 &  {\bf 45.92} &  2634.96 &  {\bf {\color{red} 0.49}} &  {\bf {\color{red} 0.45}} &  {\bf {\color{red} 0.42}} &  {\bf {\color{red} 0.39}} &  {\bf {\color{red} 0.37}} &  {\bf {\color{red} 0.34}} \\ 
\hline 
 2054 &  64 &  66 ans 3 mois &  -11.25\% &  1410.25 &  {\bf 50.18} &  2669.21 &  {\bf {\color{red} 0.53}} &  {\bf {\color{red} 0.49}} &  {\bf {\color{red} 0.46}} &  {\bf {\color{red} 0.43}} &  {\bf {\color{red} 0.40}} &  {\bf {\color{red} 0.38}} \\ 
\hline 
 2055 &  65 &  66 ans 4 mois &  -6.67\% &  2298.33 &  {\bf 81.61} &  2703.91 &  {\bf {\color{red} 0.85}} &  {\bf {\color{red} 0.80}} &  {\bf {\color{red} 0.75}} &  {\bf {\color{red} 0.70}} &  {\bf {\color{red} 0.66}} &  {\bf {\color{red} 0.62}} \\ 
\hline 
 2056 &  66 &  66 ans 5 mois &  -2.08\% &  2328.20 &  {\bf 82.50} &  2739.06 &  {\bf {\color{red} 0.85}} &  {\bf {\color{red} 0.81}} &  {\bf {\color{red} 0.76}} &  {\bf {\color{red} 0.71}} &  {\bf {\color{red} 0.67}} &  {\bf {\color{red} 0.62}} \\ 
\hline 
 2057 &  67 &  66 ans 6 mois &  2.50\% &  2358.47 &  {\bf 83.40} &  2774.67 &  {\bf {\color{red} 0.85}} &  {\bf {\color{red} 0.82}} &  {\bf {\color{red} 0.77}} &  {\bf {\color{red} 0.72}} &  {\bf {\color{red} 0.67}} &  {\bf {\color{red} 0.63}} \\ 
\hline 
\hline 
\end{tabular} 
\end{center} } 
\paragraph{Retraites possibles et ratios Revenu/SMIC à 70, 75, 80, 85, 90 ans avec le modèle \emph{Destinie2 (revalorisation de la fonction publique)}}  
 
{ \scriptsize \begin{center} 
\begin{tabular}[htb]{|c|c||c|c||c|c||c||c|c|c|c|c|c|} 
\hline 
 Retraite en &  Âge &  Âge pivot &  Décote/Surcote &  Retraite (\euro{} 2019) &  Tx Rempl(\%) &  SMIC (\euro{} 2019) &  Retraite/SMIC &  Rev70/SMIC &  Rev75/SMIC &  Rev80/SMIC &  Rev85/SMIC &  Rev90/SMIC \\ 
\hline \hline 
 2052 &  62 &  66 ans 1 mois &  -20.42\% &  1476.39 &  {\bf 34.83} &  2445.56 &  {\bf {\color{red} 0.60}} &  {\bf {\color{red} 0.54}} &  {\bf {\color{red} 0.51}} &  {\bf {\color{red} 0.48}} &  {\bf {\color{red} 0.45}} &  {\bf {\color{red} 0.42}} \\ 
\hline 
 2053 &  63 &  66 ans 2 mois &  -15.83\% &  1632.02 &  {\bf 38.01} &  2477.35 &  {\bf {\color{red} 0.66}} &  {\bf {\color{red} 0.60}} &  {\bf {\color{red} 0.56}} &  {\bf {\color{red} 0.53}} &  {\bf {\color{red} 0.50}} &  {\bf {\color{red} 0.46}} \\ 
\hline 
 2054 &  64 &  66 ans 3 mois &  -11.25\% &  1797.00 &  {\bf 41.31} &  2509.56 &  {\bf {\color{red} 0.72}} &  {\bf {\color{red} 0.66}} &  {\bf {\color{red} 0.62}} &  {\bf {\color{red} 0.58}} &  {\bf {\color{red} 0.55}} &  {\bf {\color{red} 0.51}} \\ 
\hline 
 2055 &  65 &  66 ans 4 mois &  -6.67\% &  2160.85 &  {\bf 49.04} &  2542.18 &  {\bf {\color{red} 0.85}} &  {\bf {\color{red} 0.80}} &  {\bf {\color{red} 0.75}} &  {\bf {\color{red} 0.70}} &  {\bf {\color{red} 0.66}} &  {\bf {\color{red} 0.62}} \\ 
\hline 
 2056 &  66 &  66 ans 5 mois &  -2.08\% &  2188.95 &  {\bf 49.04} &  2575.23 &  {\bf {\color{red} 0.85}} &  {\bf {\color{red} 0.81}} &  {\bf {\color{red} 0.76}} &  {\bf {\color{red} 0.71}} &  {\bf {\color{red} 0.67}} &  {\bf {\color{red} 0.62}} \\ 
\hline 
 2057 &  67 &  66 ans 6 mois &  2.50\% &  2350.93 &  {\bf 52.00} &  2608.71 &  {\bf {\color{red} 0.90}} &  {\bf {\color{red} 0.87}} &  {\bf {\color{red} 0.81}} &  {\bf {\color{red} 0.76}} &  {\bf {\color{red} 0.71}} &  {\bf {\color{red} 0.67}} \\ 
\hline 
\hline 
\end{tabular} 
\end{center} } 

 \begin{center}\includegraphics[width=0.9\textwidth]{fig/BIATSS_1990_22_dest_retraite.pdf}\end{center} \label{fig/BIATSS_1990_22_dest_retraite.pdf} 

\newpage 
 
\subsection{Génération 2003 (début en 2025)} 

\paragraph{Retraites possibles et ratios Revenu/SMIC à 70, 75, 80, 85, 90 ans avec le modèle \emph{Gouvernement truqué (âge-pivot bloqué à 65 ans)}}  
 
{ \scriptsize \begin{center} 
\begin{tabular}[htb]{|c|c||c|c||c|c||c||c|c|c|c|c|c|} 
\hline 
 Retraite en &  Âge &  Âge pivot &  Décote/Surcote &  Retraite (\euro{} 2019) &  Tx Rempl(\%) &  SMIC (\euro{} 2019) &  Retraite/SMIC &  Rev70/SMIC &  Rev75/SMIC &  Rev80/SMIC &  Rev85/SMIC &  Rev90/SMIC \\ 
\hline \hline 
 2065 &  62 &  65 ans 0 mois &  -15.00\% &  1457.41 &  {\bf 47.37} &  3076.71 &  {\bf {\color{red} 0.47}} &  {\bf {\color{red} 0.43}} &  {\bf {\color{red} 0.40}} &  {\bf {\color{red} 0.38}} &  {\bf {\color{red} 0.35}} &  {\bf {\color{red} 0.33}} \\ 
\hline 
 2066 &  63 &  65 ans 0 mois &  -10.00\% &  1602.25 &  {\bf 51.41} &  3116.71 &  {\bf {\color{red} 0.51}} &  {\bf {\color{red} 0.47}} &  {\bf {\color{red} 0.44}} &  {\bf {\color{red} 0.41}} &  {\bf {\color{red} 0.39}} &  {\bf {\color{red} 0.36}} \\ 
\hline 
 2067 &  64 &  65 ans 0 mois &  -5.00\% &  1755.00 &  {\bf 55.59} &  3157.23 &  {\bf {\color{red} 0.56}} &  {\bf {\color{red} 0.51}} &  {\bf {\color{red} 0.48}} &  {\bf {\color{red} 0.45}} &  {\bf {\color{red} 0.42}} &  {\bf {\color{red} 0.40}} \\ 
\hline 
 2068 &  65 &  65 ans 0 mois &  0.00\% &  2718.53 &  {\bf 85.00} &  3198.27 &  {\bf {\color{red} 0.85}} &  {\bf {\color{red} 0.80}} &  {\bf {\color{red} 0.75}} &  {\bf {\color{red} 0.70}} &  {\bf {\color{red} 0.66}} &  {\bf {\color{red} 0.62}} \\ 
\hline 
 2069 &  66 &  65 ans 0 mois &  5.00\% &  2753.87 &  {\bf 85.00} &  3239.85 &  {\bf {\color{red} 0.85}} &  {\bf {\color{red} 0.81}} &  {\bf {\color{red} 0.76}} &  {\bf {\color{red} 0.71}} &  {\bf {\color{red} 0.67}} &  {\bf {\color{red} 0.62}} \\ 
\hline 
 2070 &  67 &  65 ans 0 mois &  10.00\% &  2789.67 &  {\bf 85.00} &  3281.97 &  {\bf {\color{red} 0.85}} &  {\bf {\color{red} 0.82}} &  {\bf {\color{red} 0.77}} &  {\bf {\color{red} 0.72}} &  {\bf {\color{red} 0.67}} &  {\bf {\color{red} 0.63}} \\ 
\hline 
\hline 
\end{tabular} 
\end{center} } 
\paragraph{Retraites possibles et ratios Revenu/SMIC à 70, 75, 80, 85, 90 ans avec le modèle \emph{Gouvernement corrigé (âge-pivot glissant)}}  
 
{ \scriptsize \begin{center} 
\begin{tabular}[htb]{|c|c||c|c||c|c||c||c|c|c|c|c|c|} 
\hline 
 Retraite en &  Âge &  Âge pivot &  Décote/Surcote &  Retraite (\euro{} 2019) &  Tx Rempl(\%) &  SMIC (\euro{} 2019) &  Retraite/SMIC &  Rev70/SMIC &  Rev75/SMIC &  Rev80/SMIC &  Rev85/SMIC &  Rev90/SMIC \\ 
\hline \hline 
 2065 &  62 &  67 ans 2 mois &  -25.83\% &  1271.67 &  {\bf 41.33} &  3076.71 &  {\bf {\color{red} 0.41}} &  {\bf {\color{red} 0.37}} &  {\bf {\color{red} 0.35}} &  {\bf {\color{red} 0.33}} &  {\bf {\color{red} 0.31}} &  {\bf {\color{red} 0.29}} \\ 
\hline 
 2066 &  63 &  67 ans 3 mois &  -21.25\% &  1401.97 &  {\bf 44.98} &  3116.71 &  {\bf {\color{red} 0.45}} &  {\bf {\color{red} 0.41}} &  {\bf {\color{red} 0.39}} &  {\bf {\color{red} 0.36}} &  {\bf {\color{red} 0.34}} &  {\bf {\color{red} 0.32}} \\ 
\hline 
 2067 &  64 &  67 ans 4 mois &  -16.67\% &  1539.47 &  {\bf 48.76} &  3157.23 &  {\bf {\color{red} 0.49}} &  {\bf {\color{red} 0.45}} &  {\bf {\color{red} 0.42}} &  {\bf {\color{red} 0.40}} &  {\bf {\color{red} 0.37}} &  {\bf {\color{red} 0.35}} \\ 
\hline 
 2068 &  65 &  67 ans 5 mois &  -12.08\% &  2718.53 &  {\bf 85.00} &  3198.27 &  {\bf {\color{red} 0.85}} &  {\bf {\color{red} 0.80}} &  {\bf {\color{red} 0.75}} &  {\bf {\color{red} 0.70}} &  {\bf {\color{red} 0.66}} &  {\bf {\color{red} 0.62}} \\ 
\hline 
 2069 &  66 &  67 ans 6 mois &  -7.50\% &  2753.87 &  {\bf 85.00} &  3239.85 &  {\bf {\color{red} 0.85}} &  {\bf {\color{red} 0.81}} &  {\bf {\color{red} 0.76}} &  {\bf {\color{red} 0.71}} &  {\bf {\color{red} 0.67}} &  {\bf {\color{red} 0.62}} \\ 
\hline 
 2070 &  67 &  67 ans 7 mois &  -2.92\% &  2789.67 &  {\bf 85.00} &  3281.97 &  {\bf {\color{red} 0.85}} &  {\bf {\color{red} 0.82}} &  {\bf {\color{red} 0.77}} &  {\bf {\color{red} 0.72}} &  {\bf {\color{red} 0.67}} &  {\bf {\color{red} 0.63}} \\ 
\hline 
\hline 
\end{tabular} 
\end{center} } 
\paragraph{Retraites possibles et ratios Revenu/SMIC à 70, 75, 80, 85, 90 ans avec le modèle \emph{Destinie2 (revalorisation de la fonction publique)}}  
 
{ \scriptsize \begin{center} 
\begin{tabular}[htb]{|c|c||c|c||c|c||c||c|c|c|c|c|c|} 
\hline 
 Retraite en &  Âge &  Âge pivot &  Décote/Surcote &  Retraite (\euro{} 2019) &  Tx Rempl(\%) &  SMIC (\euro{} 2019) &  Retraite/SMIC &  Rev70/SMIC &  Rev75/SMIC &  Rev80/SMIC &  Rev85/SMIC &  Rev90/SMIC \\ 
\hline \hline 
 2065 &  62 &  67 ans 2 mois &  -25.83\% &  1711.31 &  {\bf 34.13} &  2892.68 &  {\bf {\color{red} 0.59}} &  {\bf {\color{red} 0.53}} &  {\bf {\color{red} 0.50}} &  {\bf {\color{red} 0.47}} &  {\bf {\color{red} 0.44}} &  {\bf {\color{red} 0.41}} \\ 
\hline 
 2066 &  63 &  67 ans 3 mois &  -21.25\% &  1896.35 &  {\bf 37.34} &  2930.29 &  {\bf {\color{red} 0.65}} &  {\bf {\color{red} 0.59}} &  {\bf {\color{red} 0.55}} &  {\bf {\color{red} 0.52}} &  {\bf {\color{red} 0.49}} &  {\bf {\color{red} 0.46}} \\ 
\hline 
 2067 &  64 &  67 ans 4 mois &  -16.67\% &  2092.49 &  {\bf 40.67} &  2968.38 &  {\bf {\color{red} 0.70}} &  {\bf {\color{red} 0.65}} &  {\bf {\color{red} 0.61}} &  {\bf {\color{red} 0.57}} &  {\bf {\color{red} 0.54}} &  {\bf {\color{red} 0.50}} \\ 
\hline 
 2068 &  65 &  67 ans 5 mois &  -12.08\% &  2555.93 &  {\bf 49.04} &  3006.97 &  {\bf {\color{red} 0.85}} &  {\bf {\color{red} 0.80}} &  {\bf {\color{red} 0.75}} &  {\bf {\color{red} 0.70}} &  {\bf {\color{red} 0.66}} &  {\bf {\color{red} 0.62}} \\ 
\hline 
 2069 &  66 &  67 ans 6 mois &  -7.50\% &  2589.15 &  {\bf 49.04} &  3046.06 &  {\bf {\color{red} 0.85}} &  {\bf {\color{red} 0.81}} &  {\bf {\color{red} 0.76}} &  {\bf {\color{red} 0.71}} &  {\bf {\color{red} 0.67}} &  {\bf {\color{red} 0.62}} \\ 
\hline 
 2070 &  67 &  67 ans 7 mois &  -2.92\% &  2750.87 &  {\bf 51.44} &  3085.66 &  {\bf {\color{red} 0.89}} &  {\bf {\color{red} 0.86}} &  {\bf {\color{red} 0.80}} &  {\bf {\color{red} 0.75}} &  {\bf {\color{red} 0.71}} &  {\bf {\color{red} 0.66}} \\ 
\hline 
\hline 
\end{tabular} 
\end{center} } 

 \begin{center}\includegraphics[width=0.9\textwidth]{fig/BIATSS_2003_22_dest_retraite.pdf}\end{center} \label{fig/BIATSS_2003_22_dest_retraite.pdf} 

\newpage 
 
\chapter{Maître de Conférences (thèse, CN puis HC)} 

\begin{minipage}{0.55\linewidth}\includegraphics[width=0.7\textwidth]{fig/grille_MCF.pdf}\end{minipage} 
\begin{minipage}{0.3\linewidth} 
 \begin{center} 

\begin{tabular}[htb]{|c|c|} 
\hline 
 Indice majoré &  Durée (années) \\ 
\hline \hline 
 430 &  3.00 \\ 
\hline 
 474 &  1.00 \\ 
\hline 
 510 &  2.00 \\ 
\hline 
 560 &  2.25 \\ 
\hline 
 600 &  2.50 \\ 
\hline 
 643 &  2.50 \\ 
\hline 
 693 &  2.50 \\ 
\hline 
 739 &  3.00 \\ 
\hline 
 769 &  3.00 \\ 
\hline 
 803 &  2.75 \\ 
\hline 
 830 &  5.00 \\ 
\hline 
 890 &  1.00 \\ 
\hline 
 925 &  1.00 \\ 
\hline 
 972 &   \\ 
\hline 
\hline 
\end{tabular} 
\end{center} 
 \end{minipage} 


 \addto{\captionsenglish}{ \renewcommand{\mtctitle}{}} \setcounter{minitocdepth}{2} 
 \minitoc \newpage 

\section{Début de carrière à 25 ans} 

\subsection{Génération 1975 (début en 2000)} 

\paragraph{Retraites possibles et ratios Revenu/SMIC à 70, 75, 80, 85, 90 ans avec le modèle \emph{Gouvernement truqué (âge-pivot bloqué à 65 ans)}}  
 
{ \scriptsize \begin{center} 
\begin{tabular}[htb]{|c|c||c|c||c|c||c||c|c|c|c|c|c|} 
\hline 
 Retraite en &  Âge &  Âge pivot &  Décote/Surcote &  Retraite (\euro{} 2019) &  Tx Rempl(\%) &  SMIC (\euro{} 2019) &  Retraite/SMIC &  Rev70/SMIC &  Rev75/SMIC &  Rev80/SMIC &  Rev85/SMIC &  Rev90/SMIC \\ 
\hline \hline 
 2037 &  62 &  64 ans 10 mois &  -14.17\% &  1688.23 &  {\bf 35.51} &  2143.00 &  {\bf {\color{red} 0.79}} &  {\bf {\color{red} 0.71}} &  {\bf {\color{red} 0.67}} &  {\bf {\color{red} 0.62}} &  {\bf {\color{red} 0.59}} &  {\bf {\color{red} 0.55}} \\ 
\hline 
 2038 &  63 &  64 ans 11 mois &  -9.58\% &  1851.91 &  {\bf 38.86} &  2170.86 &  {\bf {\color{red} 0.85}} &  {\bf {\color{red} 0.78}} &  {\bf {\color{red} 0.73}} &  {\bf {\color{red} 0.68}} &  {\bf {\color{red} 0.64}} &  {\bf {\color{red} 0.60}} \\ 
\hline 
 2039 &  64 &  65 ans 0 mois &  -5.00\% &  2025.27 &  {\bf 42.41} &  2199.08 &  {\bf {\color{red} 0.92}} &  {\bf {\color{red} 0.85}} &  {\bf {\color{red} 0.80}} &  {\bf {\color{red} 0.75}} &  {\bf {\color{red} 0.70}} &  {\bf {\color{red} 0.66}} \\ 
\hline 
 2040 &  65 &  65 ans 0 mois &  0.00\% &  2218.01 &  {\bf 46.34} &  2227.67 &  {\bf {\color{red} 1.00}} &  {\bf {\color{red} 0.93}} &  {\bf {\color{red} 0.88}} &  {\bf {\color{red} 0.82}} &  {\bf {\color{red} 0.77}} &  {\bf {\color{red} 0.72}} \\ 
\hline 
 2041 &  66 &  65 ans 0 mois &  5.00\% &  2422.14 &  {\bf 50.49} &  2256.63 &  {\bf 1.07} &  {\bf 1.02} &  {\bf {\color{red} 0.96}} &  {\bf {\color{red} 0.90}} &  {\bf {\color{red} 0.84}} &  {\bf {\color{red} 0.79}} \\ 
\hline 
 2042 &  67 &  65 ans 0 mois &  10.00\% &  2638.23 &  {\bf 54.87} &  2285.97 &  {\bf 1.15} &  {\bf 1.11} &  {\bf 1.04} &  {\bf {\color{red} 0.98}} &  {\bf {\color{red} 0.91}} &  {\bf {\color{red} 0.86}} \\ 
\hline 
\hline 
\end{tabular} 
\end{center} } 
\paragraph{Retraites possibles et ratios Revenu/SMIC à 70, 75, 80, 85, 90 ans avec le modèle \emph{Gouvernement corrigé (âge-pivot glissant)}}  
 
{ \scriptsize \begin{center} 
\begin{tabular}[htb]{|c|c||c|c||c|c||c||c|c|c|c|c|c|} 
\hline 
 Retraite en &  Âge &  Âge pivot &  Décote/Surcote &  Retraite (\euro{} 2019) &  Tx Rempl(\%) &  SMIC (\euro{} 2019) &  Retraite/SMIC &  Rev70/SMIC &  Rev75/SMIC &  Rev80/SMIC &  Rev85/SMIC &  Rev90/SMIC \\ 
\hline \hline 
 2037 &  62 &  64 ans 10 mois &  -14.17\% &  1688.23 &  {\bf 35.51} &  2143.00 &  {\bf {\color{red} 0.79}} &  {\bf {\color{red} 0.71}} &  {\bf {\color{red} 0.67}} &  {\bf {\color{red} 0.62}} &  {\bf {\color{red} 0.59}} &  {\bf {\color{red} 0.55}} \\ 
\hline 
 2038 &  63 &  64 ans 11 mois &  -9.58\% &  1851.91 &  {\bf 38.86} &  2170.86 &  {\bf {\color{red} 0.85}} &  {\bf {\color{red} 0.78}} &  {\bf {\color{red} 0.73}} &  {\bf {\color{red} 0.68}} &  {\bf {\color{red} 0.64}} &  {\bf {\color{red} 0.60}} \\ 
\hline 
 2039 &  64 &  65 ans 0 mois &  -5.00\% &  2025.27 &  {\bf 42.41} &  2199.08 &  {\bf {\color{red} 0.92}} &  {\bf {\color{red} 0.85}} &  {\bf {\color{red} 0.80}} &  {\bf {\color{red} 0.75}} &  {\bf {\color{red} 0.70}} &  {\bf {\color{red} 0.66}} \\ 
\hline 
 2040 &  65 &  65 ans 1 mois &  -0.42\% &  2208.77 &  {\bf 46.15} &  2227.67 &  {\bf {\color{red} 0.99}} &  {\bf {\color{red} 0.93}} &  {\bf {\color{red} 0.87}} &  {\bf {\color{red} 0.82}} &  {\bf {\color{red} 0.77}} &  {\bf {\color{red} 0.72}} \\ 
\hline 
 2041 &  66 &  65 ans 2 mois &  4.17\% &  2402.92 &  {\bf 50.09} &  2256.63 &  {\bf 1.06} &  {\bf 1.01} &  {\bf {\color{red} 0.95}} &  {\bf {\color{red} 0.89}} &  {\bf {\color{red} 0.83}} &  {\bf {\color{red} 0.78}} \\ 
\hline 
 2042 &  67 &  65 ans 3 mois &  8.75\% &  2608.25 &  {\bf 54.25} &  2285.97 &  {\bf 1.14} &  {\bf 1.10} &  {\bf 1.03} &  {\bf {\color{red} 0.96}} &  {\bf {\color{red} 0.90}} &  {\bf {\color{red} 0.85}} \\ 
\hline 
\hline 
\end{tabular} 
\end{center} } 
\paragraph{Retraites possibles et ratios Revenu/SMIC à 70, 75, 80, 85, 90 ans avec le modèle \emph{Destinie2 (revalorisation de la fonction publique)}}  
 
{ \scriptsize \begin{center} 
\begin{tabular}[htb]{|c|c||c|c||c|c||c||c|c|c|c|c|c|} 
\hline 
 Retraite en &  Âge &  Âge pivot &  Décote/Surcote &  Retraite (\euro{} 2019) &  Tx Rempl(\%) &  SMIC (\euro{} 2019) &  Retraite/SMIC &  Rev70/SMIC &  Rev75/SMIC &  Rev80/SMIC &  Rev85/SMIC &  Rev90/SMIC \\ 
\hline \hline 
 2037 &  62 &  64 ans 10 mois &  -14.17\% &  1856.50 &  {\bf 31.07} &  2014.82 &  {\bf {\color{red} 0.92}} &  {\bf {\color{red} 0.83}} &  {\bf {\color{red} 0.78}} &  {\bf {\color{red} 0.73}} &  {\bf {\color{red} 0.68}} &  {\bf {\color{red} 0.64}} \\ 
\hline 
 2038 &  63 &  64 ans 11 mois &  -9.58\% &  2046.73 &  {\bf 33.82} &  2041.01 &  {\bf 1.00} &  {\bf {\color{red} 0.92}} &  {\bf {\color{red} 0.86}} &  {\bf {\color{red} 0.81}} &  {\bf {\color{red} 0.75}} &  {\bf {\color{red} 0.71}} \\ 
\hline 
 2039 &  64 &  65 ans 0 mois &  -5.00\% &  2249.61 &  {\bf 36.69} &  2067.55 &  {\bf 1.09} &  {\bf 1.01} &  {\bf {\color{red} 0.94}} &  {\bf {\color{red} 0.88}} &  {\bf {\color{red} 0.83}} &  {\bf {\color{red} 0.78}} \\ 
\hline 
 2040 &  65 &  65 ans 1 mois &  -0.42\% &  2465.83 &  {\bf 39.70} &  2094.43 &  {\bf 1.18} &  {\bf 1.10} &  {\bf 1.03} &  {\bf {\color{red} 0.97}} &  {\bf {\color{red} 0.91}} &  {\bf {\color{red} 0.85}} \\ 
\hline 
 2041 &  66 &  65 ans 2 mois &  4.17\% &  2696.15 &  {\bf 42.85} &  2121.65 &  {\bf 1.27} &  {\bf 1.21} &  {\bf 1.13} &  {\bf 1.06} &  {\bf {\color{red} 0.99}} &  {\bf {\color{red} 0.93}} \\ 
\hline 
 2042 &  67 &  65 ans 3 mois &  8.75\% &  2941.36 &  {\bf 46.15} &  2149.23 &  {\bf 1.37} &  {\bf 1.32} &  {\bf 1.23} &  {\bf 1.16} &  {\bf 1.08} &  {\bf 1.02} \\ 
\hline 
\hline 
\end{tabular} 
\end{center} } 

 \begin{center}\includegraphics[width=0.9\textwidth]{fig/MCF_1975_25_dest_retraite.pdf}\end{center} \label{fig/MCF_1975_25_dest_retraite.pdf} 

\newpage 
 
\subsection{Génération 1980 (début en 2005)} 

\paragraph{Retraites possibles et ratios Revenu/SMIC à 70, 75, 80, 85, 90 ans avec le modèle \emph{Gouvernement truqué (âge-pivot bloqué à 65 ans)}}  
 
{ \scriptsize \begin{center} 
\begin{tabular}[htb]{|c|c||c|c||c|c||c||c|c|c|c|c|c|} 
\hline 
 Retraite en &  Âge &  Âge pivot &  Décote/Surcote &  Retraite (\euro{} 2019) &  Tx Rempl(\%) &  SMIC (\euro{} 2019) &  Retraite/SMIC &  Rev70/SMIC &  Rev75/SMIC &  Rev80/SMIC &  Rev85/SMIC &  Rev90/SMIC \\ 
\hline \hline 
 2042 &  62 &  65 ans 0 mois &  -15.00\% &  1710.55 &  {\bf 35.98} &  2285.97 &  {\bf {\color{red} 0.75}} &  {\bf {\color{red} 0.67}} &  {\bf {\color{red} 0.63}} &  {\bf {\color{red} 0.59}} &  {\bf {\color{red} 0.56}} &  {\bf {\color{red} 0.52}} \\ 
\hline 
 2043 &  63 &  65 ans 0 mois &  -10.00\% &  1891.62 &  {\bf 39.70} &  2315.68 &  {\bf {\color{red} 0.82}} &  {\bf {\color{red} 0.75}} &  {\bf {\color{red} 0.70}} &  {\bf {\color{red} 0.66}} &  {\bf {\color{red} 0.61}} &  {\bf {\color{red} 0.58}} \\ 
\hline 
 2044 &  64 &  65 ans 0 mois &  -5.00\% &  2084.28 &  {\bf 43.64} &  2345.79 &  {\bf {\color{red} 0.89}} &  {\bf {\color{red} 0.82}} &  {\bf {\color{red} 0.77}} &  {\bf {\color{red} 0.72}} &  {\bf {\color{red} 0.68}} &  {\bf {\color{red} 0.64}} \\ 
\hline 
 2045 &  65 &  65 ans 0 mois &  0.00\% &  2289.13 &  {\bf 47.82} &  2376.28 &  {\bf {\color{red} 0.96}} &  {\bf {\color{red} 0.90}} &  {\bf {\color{red} 0.85}} &  {\bf {\color{red} 0.79}} &  {\bf {\color{red} 0.74}} &  {\bf {\color{red} 0.70}} \\ 
\hline 
 2046 &  66 &  65 ans 0 mois &  5.00\% &  2504.94 &  {\bf 52.22} &  2407.18 &  {\bf 1.04} &  {\bf {\color{red} 0.99}} &  {\bf {\color{red} 0.93}} &  {\bf {\color{red} 0.87}} &  {\bf {\color{red} 0.81}} &  {\bf {\color{red} 0.76}} \\ 
\hline 
 2047 &  67 &  65 ans 0 mois &  10.00\% &  2731.96 &  {\bf 56.82} &  2438.47 &  {\bf 1.12} &  {\bf 1.08} &  {\bf 1.01} &  {\bf {\color{red} 0.95}} &  {\bf {\color{red} 0.89}} &  {\bf {\color{red} 0.83}} \\ 
\hline 
\hline 
\end{tabular} 
\end{center} } 
\paragraph{Retraites possibles et ratios Revenu/SMIC à 70, 75, 80, 85, 90 ans avec le modèle \emph{Gouvernement corrigé (âge-pivot glissant)}}  
 
{ \scriptsize \begin{center} 
\begin{tabular}[htb]{|c|c||c|c||c|c||c||c|c|c|c|c|c|} 
\hline 
 Retraite en &  Âge &  Âge pivot &  Décote/Surcote &  Retraite (\euro{} 2019) &  Tx Rempl(\%) &  SMIC (\euro{} 2019) &  Retraite/SMIC &  Rev70/SMIC &  Rev75/SMIC &  Rev80/SMIC &  Rev85/SMIC &  Rev90/SMIC \\ 
\hline \hline 
 2042 &  62 &  65 ans 3 mois &  -16.25\% &  1685.39 &  {\bf 35.45} &  2285.97 &  {\bf {\color{red} 0.74}} &  {\bf {\color{red} 0.66}} &  {\bf {\color{red} 0.62}} &  {\bf {\color{red} 0.58}} &  {\bf {\color{red} 0.55}} &  {\bf {\color{red} 0.51}} \\ 
\hline 
 2043 &  63 &  65 ans 4 mois &  -11.67\% &  1856.59 &  {\bf 38.96} &  2315.68 &  {\bf {\color{red} 0.80}} &  {\bf {\color{red} 0.73}} &  {\bf {\color{red} 0.69}} &  {\bf {\color{red} 0.64}} &  {\bf {\color{red} 0.60}} &  {\bf {\color{red} 0.57}} \\ 
\hline 
 2044 &  64 &  65 ans 5 mois &  -7.08\% &  2038.57 &  {\bf 42.69} &  2345.79 &  {\bf {\color{red} 0.87}} &  {\bf {\color{red} 0.80}} &  {\bf {\color{red} 0.75}} &  {\bf {\color{red} 0.71}} &  {\bf {\color{red} 0.66}} &  {\bf {\color{red} 0.62}} \\ 
\hline 
 2045 &  65 &  65 ans 6 mois &  -2.50\% &  2231.90 &  {\bf 46.63} &  2376.28 &  {\bf {\color{red} 0.94}} &  {\bf {\color{red} 0.88}} &  {\bf {\color{red} 0.83}} &  {\bf {\color{red} 0.77}} &  {\bf {\color{red} 0.73}} &  {\bf {\color{red} 0.68}} \\ 
\hline 
 2046 &  66 &  65 ans 7 mois &  2.08\% &  2435.36 &  {\bf 50.77} &  2407.18 &  {\bf 1.01} &  {\bf {\color{red} 0.96}} &  {\bf {\color{red} 0.90}} &  {\bf {\color{red} 0.84}} &  {\bf {\color{red} 0.79}} &  {\bf {\color{red} 0.74}} \\ 
\hline 
 2047 &  67 &  65 ans 8 mois &  6.67\% &  2649.17 &  {\bf 55.10} &  2438.47 &  {\bf 1.09} &  {\bf 1.05} &  {\bf {\color{red} 0.98}} &  {\bf {\color{red} 0.92}} &  {\bf {\color{red} 0.86}} &  {\bf {\color{red} 0.81}} \\ 
\hline 
\hline 
\end{tabular} 
\end{center} } 
\paragraph{Retraites possibles et ratios Revenu/SMIC à 70, 75, 80, 85, 90 ans avec le modèle \emph{Destinie2 (revalorisation de la fonction publique)}}  
 
{ \scriptsize \begin{center} 
\begin{tabular}[htb]{|c|c||c|c||c|c||c||c|c|c|c|c|c|} 
\hline 
 Retraite en &  Âge &  Âge pivot &  Décote/Surcote &  Retraite (\euro{} 2019) &  Tx Rempl(\%) &  SMIC (\euro{} 2019) &  Retraite/SMIC &  Rev70/SMIC &  Rev75/SMIC &  Rev80/SMIC &  Rev85/SMIC &  Rev90/SMIC \\ 
\hline \hline 
 2042 &  62 &  65 ans 3 mois &  -16.25\% &  1930.10 &  {\bf 30.28} &  2149.23 &  {\bf {\color{red} 0.90}} &  {\bf {\color{red} 0.81}} &  {\bf {\color{red} 0.76}} &  {\bf {\color{red} 0.71}} &  {\bf {\color{red} 0.67}} &  {\bf {\color{red} 0.63}} \\ 
\hline 
 2043 &  63 &  65 ans 4 mois &  -11.67\% &  2138.45 &  {\bf 33.12} &  2177.17 &  {\bf {\color{red} 0.98}} &  {\bf {\color{red} 0.90}} &  {\bf {\color{red} 0.84}} &  {\bf {\color{red} 0.79}} &  {\bf {\color{red} 0.74}} &  {\bf {\color{red} 0.69}} \\ 
\hline 
 2044 &  64 &  65 ans 5 mois &  -7.08\% &  2361.51 &  {\bf 36.11} &  2205.48 &  {\bf 1.07} &  {\bf {\color{red} 0.99}} &  {\bf {\color{red} 0.93}} &  {\bf {\color{red} 0.87}} &  {\bf {\color{red} 0.82}} &  {\bf {\color{red} 0.77}} \\ 
\hline 
 2045 &  65 &  65 ans 6 mois &  -2.50\% &  2600.12 &  {\bf 39.24} &  2234.15 &  {\bf 1.16} &  {\bf 1.09} &  {\bf 1.02} &  {\bf {\color{red} 0.96}} &  {\bf {\color{red} 0.90}} &  {\bf {\color{red} 0.84}} \\ 
\hline 
 2046 &  66 &  65 ans 7 mois &  2.08\% &  2853.11 &  {\bf 42.51} &  2263.19 &  {\bf 1.26} &  {\bf 1.20} &  {\bf 1.12} &  {\bf 1.05} &  {\bf {\color{red} 0.99}} &  {\bf {\color{red} 0.92}} \\ 
\hline 
 2047 &  67 &  65 ans 8 mois &  6.67\% &  3120.91 &  {\bf 45.90} &  2292.61 &  {\bf 1.36} &  {\bf 1.31} &  {\bf 1.23} &  {\bf 1.15} &  {\bf 1.08} &  {\bf 1.01} \\ 
\hline 
\hline 
\end{tabular} 
\end{center} } 

 \begin{center}\includegraphics[width=0.9\textwidth]{fig/MCF_1980_25_dest_retraite.pdf}\end{center} \label{fig/MCF_1980_25_dest_retraite.pdf} 

\newpage 
 
\subsection{Génération 1990 (début en 2015)} 

\paragraph{Retraites possibles et ratios Revenu/SMIC à 70, 75, 80, 85, 90 ans avec le modèle \emph{Gouvernement truqué (âge-pivot bloqué à 65 ans)}}  
 
{ \scriptsize \begin{center} 
\begin{tabular}[htb]{|c|c||c|c||c|c||c||c|c|c|c|c|c|} 
\hline 
 Retraite en &  Âge &  Âge pivot &  Décote/Surcote &  Retraite (\euro{} 2019) &  Tx Rempl(\%) &  SMIC (\euro{} 2019) &  Retraite/SMIC &  Rev70/SMIC &  Rev75/SMIC &  Rev80/SMIC &  Rev85/SMIC &  Rev90/SMIC \\ 
\hline \hline 
 2052 &  62 &  65 ans 0 mois &  -15.00\% &  1837.05 &  {\bf 38.64} &  2601.14 &  {\bf {\color{red} 0.71}} &  {\bf {\color{red} 0.64}} &  {\bf {\color{red} 0.60}} &  {\bf {\color{red} 0.56}} &  {\bf {\color{red} 0.52}} &  {\bf {\color{red} 0.49}} \\ 
\hline 
 2053 &  63 &  65 ans 0 mois &  -10.00\% &  2030.09 &  {\bf 42.60} &  2634.96 &  {\bf {\color{red} 0.77}} &  {\bf {\color{red} 0.70}} &  {\bf {\color{red} 0.66}} &  {\bf {\color{red} 0.62}} &  {\bf {\color{red} 0.58}} &  {\bf {\color{red} 0.54}} \\ 
\hline 
 2054 &  64 &  65 ans 0 mois &  -5.00\% &  2233.88 &  {\bf 46.78} &  2669.21 &  {\bf {\color{red} 0.84}} &  {\bf {\color{red} 0.77}} &  {\bf {\color{red} 0.73}} &  {\bf {\color{red} 0.68}} &  {\bf {\color{red} 0.64}} &  {\bf {\color{red} 0.60}} \\ 
\hline 
 2055 &  65 &  65 ans 0 mois &  0.00\% &  2448.65 &  {\bf 51.16} &  2703.91 &  {\bf {\color{red} 0.91}} &  {\bf {\color{red} 0.85}} &  {\bf {\color{red} 0.80}} &  {\bf {\color{red} 0.75}} &  {\bf {\color{red} 0.70}} &  {\bf {\color{red} 0.66}} \\ 
\hline 
 2056 &  66 &  65 ans 0 mois &  5.00\% &  2674.62 &  {\bf 55.75} &  2739.06 &  {\bf {\color{red} 0.98}} &  {\bf {\color{red} 0.93}} &  {\bf {\color{red} 0.87}} &  {\bf {\color{red} 0.81}} &  {\bf {\color{red} 0.76}} &  {\bf {\color{red} 0.72}} \\ 
\hline 
 2057 &  67 &  65 ans 0 mois &  10.00\% &  2912.02 &  {\bf 60.57} &  2774.67 &  {\bf 1.05} &  {\bf 1.01} &  {\bf {\color{red} 0.95}} &  {\bf {\color{red} 0.89}} &  {\bf {\color{red} 0.83}} &  {\bf {\color{red} 0.78}} \\ 
\hline 
\hline 
\end{tabular} 
\end{center} } 
\paragraph{Retraites possibles et ratios Revenu/SMIC à 70, 75, 80, 85, 90 ans avec le modèle \emph{Gouvernement corrigé (âge-pivot glissant)}}  
 
{ \scriptsize \begin{center} 
\begin{tabular}[htb]{|c|c||c|c||c|c||c||c|c|c|c|c|c|} 
\hline 
 Retraite en &  Âge &  Âge pivot &  Décote/Surcote &  Retraite (\euro{} 2019) &  Tx Rempl(\%) &  SMIC (\euro{} 2019) &  Retraite/SMIC &  Rev70/SMIC &  Rev75/SMIC &  Rev80/SMIC &  Rev85/SMIC &  Rev90/SMIC \\ 
\hline \hline 
 2052 &  62 &  66 ans 1 mois &  -20.42\% &  1719.98 &  {\bf 36.18} &  2601.14 &  {\bf {\color{red} 0.66}} &  {\bf {\color{red} 0.60}} &  {\bf {\color{red} 0.56}} &  {\bf {\color{red} 0.52}} &  {\bf {\color{red} 0.49}} &  {\bf {\color{red} 0.46}} \\ 
\hline 
 2053 &  63 &  66 ans 2 mois &  -15.83\% &  1898.51 &  {\bf 39.84} &  2634.96 &  {\bf {\color{red} 0.72}} &  {\bf {\color{red} 0.66}} &  {\bf {\color{red} 0.62}} &  {\bf {\color{red} 0.58}} &  {\bf {\color{red} 0.54}} &  {\bf {\color{red} 0.51}} \\ 
\hline 
 2054 &  64 &  66 ans 3 mois &  -11.25\% &  2086.92 &  {\bf 43.70} &  2669.21 &  {\bf {\color{red} 0.78}} &  {\bf {\color{red} 0.72}} &  {\bf {\color{red} 0.68}} &  {\bf {\color{red} 0.64}} &  {\bf {\color{red} 0.60}} &  {\bf {\color{red} 0.56}} \\ 
\hline 
 2055 &  65 &  66 ans 4 mois &  -6.67\% &  2285.41 &  {\bf 47.75} &  2703.91 &  {\bf {\color{red} 0.85}} &  {\bf {\color{red} 0.79}} &  {\bf {\color{red} 0.74}} &  {\bf {\color{red} 0.70}} &  {\bf {\color{red} 0.65}} &  {\bf {\color{red} 0.61}} \\ 
\hline 
 2056 &  66 &  66 ans 5 mois &  -2.08\% &  2494.19 &  {\bf 51.99} &  2739.06 &  {\bf {\color{red} 0.91}} &  {\bf {\color{red} 0.86}} &  {\bf {\color{red} 0.81}} &  {\bf {\color{red} 0.76}} &  {\bf {\color{red} 0.71}} &  {\bf {\color{red} 0.67}} \\ 
\hline 
 2057 &  67 &  66 ans 6 mois &  2.50\% &  2713.48 &  {\bf 56.44} &  2774.67 &  {\bf {\color{red} 0.98}} &  {\bf {\color{red} 0.94}} &  {\bf {\color{red} 0.88}} &  {\bf {\color{red} 0.83}} &  {\bf {\color{red} 0.78}} &  {\bf {\color{red} 0.73}} \\ 
\hline 
\hline 
\end{tabular} 
\end{center} } 
\paragraph{Retraites possibles et ratios Revenu/SMIC à 70, 75, 80, 85, 90 ans avec le modèle \emph{Destinie2 (revalorisation de la fonction publique)}}  
 
{ \scriptsize \begin{center} 
\begin{tabular}[htb]{|c|c||c|c||c|c||c||c|c|c|c|c|c|} 
\hline 
 Retraite en &  Âge &  Âge pivot &  Décote/Surcote &  Retraite (\euro{} 2019) &  Tx Rempl(\%) &  SMIC (\euro{} 2019) &  Retraite/SMIC &  Rev70/SMIC &  Rev75/SMIC &  Rev80/SMIC &  Rev85/SMIC &  Rev90/SMIC \\ 
\hline \hline 
 2052 &  62 &  66 ans 1 mois &  -20.42\% &  2179.37 &  {\bf 30.05} &  2445.56 &  {\bf {\color{red} 0.89}} &  {\bf {\color{red} 0.80}} &  {\bf {\color{red} 0.75}} &  {\bf {\color{red} 0.71}} &  {\bf {\color{red} 0.66}} &  {\bf {\color{red} 0.62}} \\ 
\hline 
 2053 &  63 &  66 ans 2 mois &  -15.83\% &  2420.92 &  {\bf 32.95} &  2477.35 &  {\bf {\color{red} 0.98}} &  {\bf {\color{red} 0.89}} &  {\bf {\color{red} 0.84}} &  {\bf {\color{red} 0.78}} &  {\bf {\color{red} 0.74}} &  {\bf {\color{red} 0.69}} \\ 
\hline 
 2054 &  64 &  66 ans 3 mois &  -11.25\% &  2677.87 &  {\bf 35.98} &  2509.56 &  {\bf 1.07} &  {\bf {\color{red} 0.99}} &  {\bf {\color{red} 0.93}} &  {\bf {\color{red} 0.87}} &  {\bf {\color{red} 0.81}} &  {\bf {\color{red} 0.76}} \\ 
\hline 
 2055 &  65 &  66 ans 4 mois &  -6.67\% &  2950.72 &  {\bf 39.14} &  2542.18 &  {\bf 1.16} &  {\bf 1.09} &  {\bf 1.02} &  {\bf {\color{red} 0.96}} &  {\bf {\color{red} 0.90}} &  {\bf {\color{red} 0.84}} \\ 
\hline 
 2056 &  66 &  66 ans 5 mois &  -2.08\% &  3239.95 &  {\bf 42.43} &  2575.23 &  {\bf 1.26} &  {\bf 1.19} &  {\bf 1.12} &  {\bf 1.05} &  {\bf {\color{red} 0.98}} &  {\bf {\color{red} 0.92}} \\ 
\hline 
 2057 &  67 &  66 ans 6 mois &  2.50\% &  3546.07 &  {\bf 45.84} &  2608.71 &  {\bf 1.36} &  {\bf 1.31} &  {\bf 1.23} &  {\bf 1.15} &  {\bf 1.08} &  {\bf 1.01} \\ 
\hline 
\hline 
\end{tabular} 
\end{center} } 

 \begin{center}\includegraphics[width=0.9\textwidth]{fig/MCF_1990_25_dest_retraite.pdf}\end{center} \label{fig/MCF_1990_25_dest_retraite.pdf} 

\newpage 
 
\subsection{Génération 2003 (début en 2028)} 

\paragraph{Retraites possibles et ratios Revenu/SMIC à 70, 75, 80, 85, 90 ans avec le modèle \emph{Gouvernement truqué (âge-pivot bloqué à 65 ans)}}  
 
{ \scriptsize \begin{center} 
\begin{tabular}[htb]{|c|c||c|c||c|c||c||c|c|c|c|c|c|} 
\hline 
 Retraite en &  Âge &  Âge pivot &  Décote/Surcote &  Retraite (\euro{} 2019) &  Tx Rempl(\%) &  SMIC (\euro{} 2019) &  Retraite/SMIC &  Rev70/SMIC &  Rev75/SMIC &  Rev80/SMIC &  Rev85/SMIC &  Rev90/SMIC \\ 
\hline \hline 
 2065 &  62 &  65 ans 0 mois &  -15.00\% &  1944.78 &  {\bf 40.91} &  3076.71 &  {\bf {\color{red} 0.63}} &  {\bf {\color{red} 0.57}} &  {\bf {\color{red} 0.53}} &  {\bf {\color{red} 0.50}} &  {\bf {\color{red} 0.47}} &  {\bf {\color{red} 0.44}} \\ 
\hline 
 2066 &  63 &  65 ans 0 mois &  -10.00\% &  2145.65 &  {\bf 45.03} &  3116.71 &  {\bf {\color{red} 0.69}} &  {\bf {\color{red} 0.63}} &  {\bf {\color{red} 0.59}} &  {\bf {\color{red} 0.55}} &  {\bf {\color{red} 0.52}} &  {\bf {\color{red} 0.49}} \\ 
\hline 
 2067 &  64 &  65 ans 0 mois &  -5.00\% &  2357.44 &  {\bf 49.36} &  3157.23 &  {\bf {\color{red} 0.75}} &  {\bf {\color{red} 0.69}} &  {\bf {\color{red} 0.65}} &  {\bf {\color{red} 0.61}} &  {\bf {\color{red} 0.57}} &  {\bf {\color{red} 0.53}} \\ 
\hline 
 2068 &  65 &  65 ans 0 mois &  0.00\% &  2580.41 &  {\bf 53.91} &  3198.27 &  {\bf {\color{red} 0.81}} &  {\bf {\color{red} 0.76}} &  {\bf {\color{red} 0.71}} &  {\bf {\color{red} 0.66}} &  {\bf {\color{red} 0.62}} &  {\bf {\color{red} 0.58}} \\ 
\hline 
 2069 &  66 &  65 ans 0 mois &  5.00\% &  2814.76 &  {\bf 58.68} &  3239.85 &  {\bf {\color{red} 0.87}} &  {\bf {\color{red} 0.83}} &  {\bf {\color{red} 0.77}} &  {\bf {\color{red} 0.73}} &  {\bf {\color{red} 0.68}} &  {\bf {\color{red} 0.64}} \\ 
\hline 
 2070 &  67 &  65 ans 0 mois &  10.00\% &  3060.75 &  {\bf 63.66} &  3281.97 &  {\bf {\color{red} 0.93}} &  {\bf {\color{red} 0.90}} &  {\bf {\color{red} 0.84}} &  {\bf {\color{red} 0.79}} &  {\bf {\color{red} 0.74}} &  {\bf {\color{red} 0.69}} \\ 
\hline 
\hline 
\end{tabular} 
\end{center} } 
\paragraph{Retraites possibles et ratios Revenu/SMIC à 70, 75, 80, 85, 90 ans avec le modèle \emph{Gouvernement corrigé (âge-pivot glissant)}}  
 
{ \scriptsize \begin{center} 
\begin{tabular}[htb]{|c|c||c|c||c|c||c||c|c|c|c|c|c|} 
\hline 
 Retraite en &  Âge &  Âge pivot &  Décote/Surcote &  Retraite (\euro{} 2019) &  Tx Rempl(\%) &  SMIC (\euro{} 2019) &  Retraite/SMIC &  Rev70/SMIC &  Rev75/SMIC &  Rev80/SMIC &  Rev85/SMIC &  Rev90/SMIC \\ 
\hline \hline 
 2065 &  62 &  67 ans 2 mois &  -25.83\% &  1696.92 &  {\bf 35.69} &  3076.71 &  {\bf {\color{red} 0.55}} &  {\bf {\color{red} 0.50}} &  {\bf {\color{red} 0.47}} &  {\bf {\color{red} 0.44}} &  {\bf {\color{red} 0.41}} &  {\bf {\color{red} 0.38}} \\ 
\hline 
 2066 &  63 &  67 ans 3 mois &  -21.25\% &  1877.44 &  {\bf 39.40} &  3116.71 &  {\bf {\color{red} 0.60}} &  {\bf {\color{red} 0.55}} &  {\bf {\color{red} 0.52}} &  {\bf {\color{red} 0.48}} &  {\bf {\color{red} 0.45}} &  {\bf {\color{red} 0.43}} \\ 
\hline 
 2067 &  64 &  67 ans 4 mois &  -16.67\% &  2067.93 &  {\bf 43.30} &  3157.23 &  {\bf {\color{red} 0.65}} &  {\bf {\color{red} 0.61}} &  {\bf {\color{red} 0.57}} &  {\bf {\color{red} 0.53}} &  {\bf {\color{red} 0.50}} &  {\bf {\color{red} 0.47}} \\ 
\hline 
 2068 &  65 &  67 ans 5 mois &  -12.08\% &  2268.61 &  {\bf 47.40} &  3198.27 &  {\bf {\color{red} 0.71}} &  {\bf {\color{red} 0.66}} &  {\bf {\color{red} 0.62}} &  {\bf {\color{red} 0.58}} &  {\bf {\color{red} 0.55}} &  {\bf {\color{red} 0.51}} \\ 
\hline 
 2069 &  66 &  67 ans 6 mois &  -7.50\% &  2479.67 &  {\bf 51.69} &  3239.85 &  {\bf {\color{red} 0.77}} &  {\bf {\color{red} 0.73}} &  {\bf {\color{red} 0.68}} &  {\bf {\color{red} 0.64}} &  {\bf {\color{red} 0.60}} &  {\bf {\color{red} 0.56}} \\ 
\hline 
 2070 &  67 &  67 ans 7 mois &  -2.92\% &  2701.34 &  {\bf 56.19} &  3281.97 &  {\bf {\color{red} 0.82}} &  {\bf {\color{red} 0.79}} &  {\bf {\color{red} 0.74}} &  {\bf {\color{red} 0.70}} &  {\bf {\color{red} 0.65}} &  {\bf {\color{red} 0.61}} \\ 
\hline 
\hline 
\end{tabular} 
\end{center} } 
\paragraph{Retraites possibles et ratios Revenu/SMIC à 70, 75, 80, 85, 90 ans avec le modèle \emph{Destinie2 (revalorisation de la fonction publique)}}  
 
{ \scriptsize \begin{center} 
\begin{tabular}[htb]{|c|c||c|c||c|c||c||c|c|c|c|c|c|} 
\hline 
 Retraite en &  Âge &  Âge pivot &  Décote/Surcote &  Retraite (\euro{} 2019) &  Tx Rempl(\%) &  SMIC (\euro{} 2019) &  Retraite/SMIC &  Rev70/SMIC &  Rev75/SMIC &  Rev80/SMIC &  Rev85/SMIC &  Rev90/SMIC \\ 
\hline \hline 
 2065 &  62 &  67 ans 2 mois &  -25.83\% &  2515.02 &  {\bf 29.32} &  2892.68 &  {\bf {\color{red} 0.87}} &  {\bf {\color{red} 0.78}} &  {\bf {\color{red} 0.74}} &  {\bf {\color{red} 0.69}} &  {\bf {\color{red} 0.65}} &  {\bf {\color{red} 0.61}} \\ 
\hline 
 2066 &  63 &  67 ans 3 mois &  -21.25\% &  2800.41 &  {\bf 32.23} &  2930.29 &  {\bf {\color{red} 0.96}} &  {\bf {\color{red} 0.87}} &  {\bf {\color{red} 0.82}} &  {\bf {\color{red} 0.77}} &  {\bf {\color{red} 0.72}} &  {\bf {\color{red} 0.67}} \\ 
\hline 
 2067 &  64 &  67 ans 4 mois &  -16.67\% &  3104.03 &  {\bf 35.26} &  2968.38 &  {\bf 1.05} &  {\bf {\color{red} 0.97}} &  {\bf {\color{red} 0.91}} &  {\bf {\color{red} 0.85}} &  {\bf {\color{red} 0.80}} &  {\bf {\color{red} 0.75}} \\ 
\hline 
 2068 &  65 &  67 ans 5 mois &  -12.08\% &  3426.44 &  {\bf 38.43} &  3006.97 &  {\bf 1.14} &  {\bf 1.07} &  {\bf 1.00} &  {\bf {\color{red} 0.94}} &  {\bf {\color{red} 0.88}} &  {\bf {\color{red} 0.83}} \\ 
\hline 
 2069 &  66 &  67 ans 6 mois &  -7.50\% &  3768.24 &  {\bf 41.72} &  3046.06 &  {\bf 1.24} &  {\bf 1.17} &  {\bf 1.10} &  {\bf 1.03} &  {\bf {\color{red} 0.97}} &  {\bf {\color{red} 0.91}} \\ 
\hline 
 2070 &  67 &  67 ans 7 mois &  -2.92\% &  4130.03 &  {\bf 45.13} &  3085.66 &  {\bf 1.34} &  {\bf 1.29} &  {\bf 1.21} &  {\bf 1.13} &  {\bf 1.06} &  {\bf {\color{red} 0.99}} \\ 
\hline 
\hline 
\end{tabular} 
\end{center} } 

 \begin{center}\includegraphics[width=0.9\textwidth]{fig/MCF_2003_25_dest_retraite.pdf}\end{center} \label{fig/MCF_2003_25_dest_retraite.pdf} 

\newpage 
 
\chapter{Chargé de Recherche (thèse, CN puis HC)} 

\begin{minipage}{0.55\linewidth}\includegraphics[width=0.7\textwidth]{fig/grille_CR.pdf}\end{minipage} 
\begin{minipage}{0.3\linewidth} 
 \begin{center} 

\begin{tabular}[htb]{|c|c|} 
\hline 
 Indice majoré &  Durée (années) \\ 
\hline \hline 
 430 &  3.00 \\ 
\hline 
 474 &  1.00 \\ 
\hline 
 510 &  2.00 \\ 
\hline 
 560 &  2.25 \\ 
\hline 
 600 &  2.50 \\ 
\hline 
 643 &  2.50 \\ 
\hline 
 693 &  2.50 \\ 
\hline 
 739 &  3.00 \\ 
\hline 
 769 &  3.00 \\ 
\hline 
 803 &  2.75 \\ 
\hline 
 830 &  5.00 \\ 
\hline 
 890 &  1.00 \\ 
\hline 
 925 &  1.00 \\ 
\hline 
 972 &   \\ 
\hline 
\hline 
\end{tabular} 
\end{center} 
 \end{minipage} 


 \addto{\captionsenglish}{ \renewcommand{\mtctitle}{}} \setcounter{minitocdepth}{2} 
 \minitoc \newpage 

\section{Début de carrière à 25 ans} 

\subsection{Génération 1975 (début en 2000)} 

\paragraph{Retraites possibles et ratios Revenu/SMIC à 70, 75, 80, 85, 90 ans avec le modèle \emph{Gouvernement truqué (âge-pivot bloqué à 65 ans)}}  
 
{ \scriptsize \begin{center} 
\begin{tabular}[htb]{|c|c||c|c||c|c||c||c|c|c|c|c|c|} 
\hline 
 Retraite en &  Âge &  Âge pivot &  Décote/Surcote &  Retraite (\euro{} 2019) &  Tx Rempl(\%) &  SMIC (\euro{} 2019) &  Retraite/SMIC &  Rev70/SMIC &  Rev75/SMIC &  Rev80/SMIC &  Rev85/SMIC &  Rev90/SMIC \\ 
\hline \hline 
 2037 &  62 &  64 ans 10 mois &  -14.17\% &  1688.23 &  {\bf 35.51} &  2143.00 &  {\bf {\color{red} 0.79}} &  {\bf {\color{red} 0.71}} &  {\bf {\color{red} 0.67}} &  {\bf {\color{red} 0.62}} &  {\bf {\color{red} 0.59}} &  {\bf {\color{red} 0.55}} \\ 
\hline 
 2038 &  63 &  64 ans 11 mois &  -9.58\% &  1851.91 &  {\bf 38.86} &  2170.86 &  {\bf {\color{red} 0.85}} &  {\bf {\color{red} 0.78}} &  {\bf {\color{red} 0.73}} &  {\bf {\color{red} 0.68}} &  {\bf {\color{red} 0.64}} &  {\bf {\color{red} 0.60}} \\ 
\hline 
 2039 &  64 &  65 ans 0 mois &  -5.00\% &  2025.27 &  {\bf 42.41} &  2199.08 &  {\bf {\color{red} 0.92}} &  {\bf {\color{red} 0.85}} &  {\bf {\color{red} 0.80}} &  {\bf {\color{red} 0.75}} &  {\bf {\color{red} 0.70}} &  {\bf {\color{red} 0.66}} \\ 
\hline 
 2040 &  65 &  65 ans 0 mois &  0.00\% &  2218.01 &  {\bf 46.34} &  2227.67 &  {\bf {\color{red} 1.00}} &  {\bf {\color{red} 0.93}} &  {\bf {\color{red} 0.88}} &  {\bf {\color{red} 0.82}} &  {\bf {\color{red} 0.77}} &  {\bf {\color{red} 0.72}} \\ 
\hline 
 2041 &  66 &  65 ans 0 mois &  5.00\% &  2422.14 &  {\bf 50.49} &  2256.63 &  {\bf 1.07} &  {\bf 1.02} &  {\bf {\color{red} 0.96}} &  {\bf {\color{red} 0.90}} &  {\bf {\color{red} 0.84}} &  {\bf {\color{red} 0.79}} \\ 
\hline 
 2042 &  67 &  65 ans 0 mois &  10.00\% &  2638.23 &  {\bf 54.87} &  2285.97 &  {\bf 1.15} &  {\bf 1.11} &  {\bf 1.04} &  {\bf {\color{red} 0.98}} &  {\bf {\color{red} 0.91}} &  {\bf {\color{red} 0.86}} \\ 
\hline 
\hline 
\end{tabular} 
\end{center} } 
\paragraph{Retraites possibles et ratios Revenu/SMIC à 70, 75, 80, 85, 90 ans avec le modèle \emph{Gouvernement corrigé (âge-pivot glissant)}}  
 
{ \scriptsize \begin{center} 
\begin{tabular}[htb]{|c|c||c|c||c|c||c||c|c|c|c|c|c|} 
\hline 
 Retraite en &  Âge &  Âge pivot &  Décote/Surcote &  Retraite (\euro{} 2019) &  Tx Rempl(\%) &  SMIC (\euro{} 2019) &  Retraite/SMIC &  Rev70/SMIC &  Rev75/SMIC &  Rev80/SMIC &  Rev85/SMIC &  Rev90/SMIC \\ 
\hline \hline 
 2037 &  62 &  64 ans 10 mois &  -14.17\% &  1688.23 &  {\bf 35.51} &  2143.00 &  {\bf {\color{red} 0.79}} &  {\bf {\color{red} 0.71}} &  {\bf {\color{red} 0.67}} &  {\bf {\color{red} 0.62}} &  {\bf {\color{red} 0.59}} &  {\bf {\color{red} 0.55}} \\ 
\hline 
 2038 &  63 &  64 ans 11 mois &  -9.58\% &  1851.91 &  {\bf 38.86} &  2170.86 &  {\bf {\color{red} 0.85}} &  {\bf {\color{red} 0.78}} &  {\bf {\color{red} 0.73}} &  {\bf {\color{red} 0.68}} &  {\bf {\color{red} 0.64}} &  {\bf {\color{red} 0.60}} \\ 
\hline 
 2039 &  64 &  65 ans 0 mois &  -5.00\% &  2025.27 &  {\bf 42.41} &  2199.08 &  {\bf {\color{red} 0.92}} &  {\bf {\color{red} 0.85}} &  {\bf {\color{red} 0.80}} &  {\bf {\color{red} 0.75}} &  {\bf {\color{red} 0.70}} &  {\bf {\color{red} 0.66}} \\ 
\hline 
 2040 &  65 &  65 ans 1 mois &  -0.42\% &  2208.77 &  {\bf 46.15} &  2227.67 &  {\bf {\color{red} 0.99}} &  {\bf {\color{red} 0.93}} &  {\bf {\color{red} 0.87}} &  {\bf {\color{red} 0.82}} &  {\bf {\color{red} 0.77}} &  {\bf {\color{red} 0.72}} \\ 
\hline 
 2041 &  66 &  65 ans 2 mois &  4.17\% &  2402.92 &  {\bf 50.09} &  2256.63 &  {\bf 1.06} &  {\bf 1.01} &  {\bf {\color{red} 0.95}} &  {\bf {\color{red} 0.89}} &  {\bf {\color{red} 0.83}} &  {\bf {\color{red} 0.78}} \\ 
\hline 
 2042 &  67 &  65 ans 3 mois &  8.75\% &  2608.25 &  {\bf 54.25} &  2285.97 &  {\bf 1.14} &  {\bf 1.10} &  {\bf 1.03} &  {\bf {\color{red} 0.96}} &  {\bf {\color{red} 0.90}} &  {\bf {\color{red} 0.85}} \\ 
\hline 
\hline 
\end{tabular} 
\end{center} } 
\paragraph{Retraites possibles et ratios Revenu/SMIC à 70, 75, 80, 85, 90 ans avec le modèle \emph{Destinie2 (revalorisation de la fonction publique)}}  
 
{ \scriptsize \begin{center} 
\begin{tabular}[htb]{|c|c||c|c||c|c||c||c|c|c|c|c|c|} 
\hline 
 Retraite en &  Âge &  Âge pivot &  Décote/Surcote &  Retraite (\euro{} 2019) &  Tx Rempl(\%) &  SMIC (\euro{} 2019) &  Retraite/SMIC &  Rev70/SMIC &  Rev75/SMIC &  Rev80/SMIC &  Rev85/SMIC &  Rev90/SMIC \\ 
\hline \hline 
 2037 &  62 &  64 ans 10 mois &  -14.17\% &  1856.50 &  {\bf 31.07} &  2014.82 &  {\bf {\color{red} 0.92}} &  {\bf {\color{red} 0.83}} &  {\bf {\color{red} 0.78}} &  {\bf {\color{red} 0.73}} &  {\bf {\color{red} 0.68}} &  {\bf {\color{red} 0.64}} \\ 
\hline 
 2038 &  63 &  64 ans 11 mois &  -9.58\% &  2046.73 &  {\bf 33.82} &  2041.01 &  {\bf 1.00} &  {\bf {\color{red} 0.92}} &  {\bf {\color{red} 0.86}} &  {\bf {\color{red} 0.81}} &  {\bf {\color{red} 0.75}} &  {\bf {\color{red} 0.71}} \\ 
\hline 
 2039 &  64 &  65 ans 0 mois &  -5.00\% &  2249.61 &  {\bf 36.69} &  2067.55 &  {\bf 1.09} &  {\bf 1.01} &  {\bf {\color{red} 0.94}} &  {\bf {\color{red} 0.88}} &  {\bf {\color{red} 0.83}} &  {\bf {\color{red} 0.78}} \\ 
\hline 
 2040 &  65 &  65 ans 1 mois &  -0.42\% &  2465.83 &  {\bf 39.70} &  2094.43 &  {\bf 1.18} &  {\bf 1.10} &  {\bf 1.03} &  {\bf {\color{red} 0.97}} &  {\bf {\color{red} 0.91}} &  {\bf {\color{red} 0.85}} \\ 
\hline 
 2041 &  66 &  65 ans 2 mois &  4.17\% &  2696.15 &  {\bf 42.85} &  2121.65 &  {\bf 1.27} &  {\bf 1.21} &  {\bf 1.13} &  {\bf 1.06} &  {\bf {\color{red} 0.99}} &  {\bf {\color{red} 0.93}} \\ 
\hline 
 2042 &  67 &  65 ans 3 mois &  8.75\% &  2941.36 &  {\bf 46.15} &  2149.23 &  {\bf 1.37} &  {\bf 1.32} &  {\bf 1.23} &  {\bf 1.16} &  {\bf 1.08} &  {\bf 1.02} \\ 
\hline 
\hline 
\end{tabular} 
\end{center} } 

 \begin{center}\includegraphics[width=0.9\textwidth]{fig/CR_1975_25_dest_retraite.pdf}\end{center} \label{fig/CR_1975_25_dest_retraite.pdf} 

\newpage 
 
\subsection{Génération 1980 (début en 2005)} 

\paragraph{Retraites possibles et ratios Revenu/SMIC à 70, 75, 80, 85, 90 ans avec le modèle \emph{Gouvernement truqué (âge-pivot bloqué à 65 ans)}}  
 
{ \scriptsize \begin{center} 
\begin{tabular}[htb]{|c|c||c|c||c|c||c||c|c|c|c|c|c|} 
\hline 
 Retraite en &  Âge &  Âge pivot &  Décote/Surcote &  Retraite (\euro{} 2019) &  Tx Rempl(\%) &  SMIC (\euro{} 2019) &  Retraite/SMIC &  Rev70/SMIC &  Rev75/SMIC &  Rev80/SMIC &  Rev85/SMIC &  Rev90/SMIC \\ 
\hline \hline 
 2042 &  62 &  65 ans 0 mois &  -15.00\% &  1710.55 &  {\bf 35.98} &  2285.97 &  {\bf {\color{red} 0.75}} &  {\bf {\color{red} 0.67}} &  {\bf {\color{red} 0.63}} &  {\bf {\color{red} 0.59}} &  {\bf {\color{red} 0.56}} &  {\bf {\color{red} 0.52}} \\ 
\hline 
 2043 &  63 &  65 ans 0 mois &  -10.00\% &  1891.62 &  {\bf 39.70} &  2315.68 &  {\bf {\color{red} 0.82}} &  {\bf {\color{red} 0.75}} &  {\bf {\color{red} 0.70}} &  {\bf {\color{red} 0.66}} &  {\bf {\color{red} 0.61}} &  {\bf {\color{red} 0.58}} \\ 
\hline 
 2044 &  64 &  65 ans 0 mois &  -5.00\% &  2084.28 &  {\bf 43.64} &  2345.79 &  {\bf {\color{red} 0.89}} &  {\bf {\color{red} 0.82}} &  {\bf {\color{red} 0.77}} &  {\bf {\color{red} 0.72}} &  {\bf {\color{red} 0.68}} &  {\bf {\color{red} 0.64}} \\ 
\hline 
 2045 &  65 &  65 ans 0 mois &  0.00\% &  2289.13 &  {\bf 47.82} &  2376.28 &  {\bf {\color{red} 0.96}} &  {\bf {\color{red} 0.90}} &  {\bf {\color{red} 0.85}} &  {\bf {\color{red} 0.79}} &  {\bf {\color{red} 0.74}} &  {\bf {\color{red} 0.70}} \\ 
\hline 
 2046 &  66 &  65 ans 0 mois &  5.00\% &  2504.94 &  {\bf 52.22} &  2407.18 &  {\bf 1.04} &  {\bf {\color{red} 0.99}} &  {\bf {\color{red} 0.93}} &  {\bf {\color{red} 0.87}} &  {\bf {\color{red} 0.81}} &  {\bf {\color{red} 0.76}} \\ 
\hline 
 2047 &  67 &  65 ans 0 mois &  10.00\% &  2731.96 &  {\bf 56.82} &  2438.47 &  {\bf 1.12} &  {\bf 1.08} &  {\bf 1.01} &  {\bf {\color{red} 0.95}} &  {\bf {\color{red} 0.89}} &  {\bf {\color{red} 0.83}} \\ 
\hline 
\hline 
\end{tabular} 
\end{center} } 
\paragraph{Retraites possibles et ratios Revenu/SMIC à 70, 75, 80, 85, 90 ans avec le modèle \emph{Gouvernement corrigé (âge-pivot glissant)}}  
 
{ \scriptsize \begin{center} 
\begin{tabular}[htb]{|c|c||c|c||c|c||c||c|c|c|c|c|c|} 
\hline 
 Retraite en &  Âge &  Âge pivot &  Décote/Surcote &  Retraite (\euro{} 2019) &  Tx Rempl(\%) &  SMIC (\euro{} 2019) &  Retraite/SMIC &  Rev70/SMIC &  Rev75/SMIC &  Rev80/SMIC &  Rev85/SMIC &  Rev90/SMIC \\ 
\hline \hline 
 2042 &  62 &  65 ans 3 mois &  -16.25\% &  1685.39 &  {\bf 35.45} &  2285.97 &  {\bf {\color{red} 0.74}} &  {\bf {\color{red} 0.66}} &  {\bf {\color{red} 0.62}} &  {\bf {\color{red} 0.58}} &  {\bf {\color{red} 0.55}} &  {\bf {\color{red} 0.51}} \\ 
\hline 
 2043 &  63 &  65 ans 4 mois &  -11.67\% &  1856.59 &  {\bf 38.96} &  2315.68 &  {\bf {\color{red} 0.80}} &  {\bf {\color{red} 0.73}} &  {\bf {\color{red} 0.69}} &  {\bf {\color{red} 0.64}} &  {\bf {\color{red} 0.60}} &  {\bf {\color{red} 0.57}} \\ 
\hline 
 2044 &  64 &  65 ans 5 mois &  -7.08\% &  2038.57 &  {\bf 42.69} &  2345.79 &  {\bf {\color{red} 0.87}} &  {\bf {\color{red} 0.80}} &  {\bf {\color{red} 0.75}} &  {\bf {\color{red} 0.71}} &  {\bf {\color{red} 0.66}} &  {\bf {\color{red} 0.62}} \\ 
\hline 
 2045 &  65 &  65 ans 6 mois &  -2.50\% &  2231.90 &  {\bf 46.63} &  2376.28 &  {\bf {\color{red} 0.94}} &  {\bf {\color{red} 0.88}} &  {\bf {\color{red} 0.83}} &  {\bf {\color{red} 0.77}} &  {\bf {\color{red} 0.73}} &  {\bf {\color{red} 0.68}} \\ 
\hline 
 2046 &  66 &  65 ans 7 mois &  2.08\% &  2435.36 &  {\bf 50.77} &  2407.18 &  {\bf 1.01} &  {\bf {\color{red} 0.96}} &  {\bf {\color{red} 0.90}} &  {\bf {\color{red} 0.84}} &  {\bf {\color{red} 0.79}} &  {\bf {\color{red} 0.74}} \\ 
\hline 
 2047 &  67 &  65 ans 8 mois &  6.67\% &  2649.17 &  {\bf 55.10} &  2438.47 &  {\bf 1.09} &  {\bf 1.05} &  {\bf {\color{red} 0.98}} &  {\bf {\color{red} 0.92}} &  {\bf {\color{red} 0.86}} &  {\bf {\color{red} 0.81}} \\ 
\hline 
\hline 
\end{tabular} 
\end{center} } 
\paragraph{Retraites possibles et ratios Revenu/SMIC à 70, 75, 80, 85, 90 ans avec le modèle \emph{Destinie2 (revalorisation de la fonction publique)}}  
 
{ \scriptsize \begin{center} 
\begin{tabular}[htb]{|c|c||c|c||c|c||c||c|c|c|c|c|c|} 
\hline 
 Retraite en &  Âge &  Âge pivot &  Décote/Surcote &  Retraite (\euro{} 2019) &  Tx Rempl(\%) &  SMIC (\euro{} 2019) &  Retraite/SMIC &  Rev70/SMIC &  Rev75/SMIC &  Rev80/SMIC &  Rev85/SMIC &  Rev90/SMIC \\ 
\hline \hline 
 2042 &  62 &  65 ans 3 mois &  -16.25\% &  1930.10 &  {\bf 30.28} &  2149.23 &  {\bf {\color{red} 0.90}} &  {\bf {\color{red} 0.81}} &  {\bf {\color{red} 0.76}} &  {\bf {\color{red} 0.71}} &  {\bf {\color{red} 0.67}} &  {\bf {\color{red} 0.63}} \\ 
\hline 
 2043 &  63 &  65 ans 4 mois &  -11.67\% &  2138.45 &  {\bf 33.12} &  2177.17 &  {\bf {\color{red} 0.98}} &  {\bf {\color{red} 0.90}} &  {\bf {\color{red} 0.84}} &  {\bf {\color{red} 0.79}} &  {\bf {\color{red} 0.74}} &  {\bf {\color{red} 0.69}} \\ 
\hline 
 2044 &  64 &  65 ans 5 mois &  -7.08\% &  2361.51 &  {\bf 36.11} &  2205.48 &  {\bf 1.07} &  {\bf {\color{red} 0.99}} &  {\bf {\color{red} 0.93}} &  {\bf {\color{red} 0.87}} &  {\bf {\color{red} 0.82}} &  {\bf {\color{red} 0.77}} \\ 
\hline 
 2045 &  65 &  65 ans 6 mois &  -2.50\% &  2600.12 &  {\bf 39.24} &  2234.15 &  {\bf 1.16} &  {\bf 1.09} &  {\bf 1.02} &  {\bf {\color{red} 0.96}} &  {\bf {\color{red} 0.90}} &  {\bf {\color{red} 0.84}} \\ 
\hline 
 2046 &  66 &  65 ans 7 mois &  2.08\% &  2853.11 &  {\bf 42.51} &  2263.19 &  {\bf 1.26} &  {\bf 1.20} &  {\bf 1.12} &  {\bf 1.05} &  {\bf {\color{red} 0.99}} &  {\bf {\color{red} 0.92}} \\ 
\hline 
 2047 &  67 &  65 ans 8 mois &  6.67\% &  3120.91 &  {\bf 45.90} &  2292.61 &  {\bf 1.36} &  {\bf 1.31} &  {\bf 1.23} &  {\bf 1.15} &  {\bf 1.08} &  {\bf 1.01} \\ 
\hline 
\hline 
\end{tabular} 
\end{center} } 

 \begin{center}\includegraphics[width=0.9\textwidth]{fig/CR_1980_25_dest_retraite.pdf}\end{center} \label{fig/CR_1980_25_dest_retraite.pdf} 

\newpage 
 
\subsection{Génération 1990 (début en 2015)} 

\paragraph{Retraites possibles et ratios Revenu/SMIC à 70, 75, 80, 85, 90 ans avec le modèle \emph{Gouvernement truqué (âge-pivot bloqué à 65 ans)}}  
 
{ \scriptsize \begin{center} 
\begin{tabular}[htb]{|c|c||c|c||c|c||c||c|c|c|c|c|c|} 
\hline 
 Retraite en &  Âge &  Âge pivot &  Décote/Surcote &  Retraite (\euro{} 2019) &  Tx Rempl(\%) &  SMIC (\euro{} 2019) &  Retraite/SMIC &  Rev70/SMIC &  Rev75/SMIC &  Rev80/SMIC &  Rev85/SMIC &  Rev90/SMIC \\ 
\hline \hline 
 2052 &  62 &  65 ans 0 mois &  -15.00\% &  1837.05 &  {\bf 38.64} &  2601.14 &  {\bf {\color{red} 0.71}} &  {\bf {\color{red} 0.64}} &  {\bf {\color{red} 0.60}} &  {\bf {\color{red} 0.56}} &  {\bf {\color{red} 0.52}} &  {\bf {\color{red} 0.49}} \\ 
\hline 
 2053 &  63 &  65 ans 0 mois &  -10.00\% &  2030.09 &  {\bf 42.60} &  2634.96 &  {\bf {\color{red} 0.77}} &  {\bf {\color{red} 0.70}} &  {\bf {\color{red} 0.66}} &  {\bf {\color{red} 0.62}} &  {\bf {\color{red} 0.58}} &  {\bf {\color{red} 0.54}} \\ 
\hline 
 2054 &  64 &  65 ans 0 mois &  -5.00\% &  2233.88 &  {\bf 46.78} &  2669.21 &  {\bf {\color{red} 0.84}} &  {\bf {\color{red} 0.77}} &  {\bf {\color{red} 0.73}} &  {\bf {\color{red} 0.68}} &  {\bf {\color{red} 0.64}} &  {\bf {\color{red} 0.60}} \\ 
\hline 
 2055 &  65 &  65 ans 0 mois &  0.00\% &  2448.65 &  {\bf 51.16} &  2703.91 &  {\bf {\color{red} 0.91}} &  {\bf {\color{red} 0.85}} &  {\bf {\color{red} 0.80}} &  {\bf {\color{red} 0.75}} &  {\bf {\color{red} 0.70}} &  {\bf {\color{red} 0.66}} \\ 
\hline 
 2056 &  66 &  65 ans 0 mois &  5.00\% &  2674.62 &  {\bf 55.75} &  2739.06 &  {\bf {\color{red} 0.98}} &  {\bf {\color{red} 0.93}} &  {\bf {\color{red} 0.87}} &  {\bf {\color{red} 0.81}} &  {\bf {\color{red} 0.76}} &  {\bf {\color{red} 0.72}} \\ 
\hline 
 2057 &  67 &  65 ans 0 mois &  10.00\% &  2912.02 &  {\bf 60.57} &  2774.67 &  {\bf 1.05} &  {\bf 1.01} &  {\bf {\color{red} 0.95}} &  {\bf {\color{red} 0.89}} &  {\bf {\color{red} 0.83}} &  {\bf {\color{red} 0.78}} \\ 
\hline 
\hline 
\end{tabular} 
\end{center} } 
\paragraph{Retraites possibles et ratios Revenu/SMIC à 70, 75, 80, 85, 90 ans avec le modèle \emph{Gouvernement corrigé (âge-pivot glissant)}}  
 
{ \scriptsize \begin{center} 
\begin{tabular}[htb]{|c|c||c|c||c|c||c||c|c|c|c|c|c|} 
\hline 
 Retraite en &  Âge &  Âge pivot &  Décote/Surcote &  Retraite (\euro{} 2019) &  Tx Rempl(\%) &  SMIC (\euro{} 2019) &  Retraite/SMIC &  Rev70/SMIC &  Rev75/SMIC &  Rev80/SMIC &  Rev85/SMIC &  Rev90/SMIC \\ 
\hline \hline 
 2052 &  62 &  66 ans 1 mois &  -20.42\% &  1719.98 &  {\bf 36.18} &  2601.14 &  {\bf {\color{red} 0.66}} &  {\bf {\color{red} 0.60}} &  {\bf {\color{red} 0.56}} &  {\bf {\color{red} 0.52}} &  {\bf {\color{red} 0.49}} &  {\bf {\color{red} 0.46}} \\ 
\hline 
 2053 &  63 &  66 ans 2 mois &  -15.83\% &  1898.51 &  {\bf 39.84} &  2634.96 &  {\bf {\color{red} 0.72}} &  {\bf {\color{red} 0.66}} &  {\bf {\color{red} 0.62}} &  {\bf {\color{red} 0.58}} &  {\bf {\color{red} 0.54}} &  {\bf {\color{red} 0.51}} \\ 
\hline 
 2054 &  64 &  66 ans 3 mois &  -11.25\% &  2086.92 &  {\bf 43.70} &  2669.21 &  {\bf {\color{red} 0.78}} &  {\bf {\color{red} 0.72}} &  {\bf {\color{red} 0.68}} &  {\bf {\color{red} 0.64}} &  {\bf {\color{red} 0.60}} &  {\bf {\color{red} 0.56}} \\ 
\hline 
 2055 &  65 &  66 ans 4 mois &  -6.67\% &  2285.41 &  {\bf 47.75} &  2703.91 &  {\bf {\color{red} 0.85}} &  {\bf {\color{red} 0.79}} &  {\bf {\color{red} 0.74}} &  {\bf {\color{red} 0.70}} &  {\bf {\color{red} 0.65}} &  {\bf {\color{red} 0.61}} \\ 
\hline 
 2056 &  66 &  66 ans 5 mois &  -2.08\% &  2494.19 &  {\bf 51.99} &  2739.06 &  {\bf {\color{red} 0.91}} &  {\bf {\color{red} 0.86}} &  {\bf {\color{red} 0.81}} &  {\bf {\color{red} 0.76}} &  {\bf {\color{red} 0.71}} &  {\bf {\color{red} 0.67}} \\ 
\hline 
 2057 &  67 &  66 ans 6 mois &  2.50\% &  2713.48 &  {\bf 56.44} &  2774.67 &  {\bf {\color{red} 0.98}} &  {\bf {\color{red} 0.94}} &  {\bf {\color{red} 0.88}} &  {\bf {\color{red} 0.83}} &  {\bf {\color{red} 0.78}} &  {\bf {\color{red} 0.73}} \\ 
\hline 
\hline 
\end{tabular} 
\end{center} } 
\paragraph{Retraites possibles et ratios Revenu/SMIC à 70, 75, 80, 85, 90 ans avec le modèle \emph{Destinie2 (revalorisation de la fonction publique)}}  
 
{ \scriptsize \begin{center} 
\begin{tabular}[htb]{|c|c||c|c||c|c||c||c|c|c|c|c|c|} 
\hline 
 Retraite en &  Âge &  Âge pivot &  Décote/Surcote &  Retraite (\euro{} 2019) &  Tx Rempl(\%) &  SMIC (\euro{} 2019) &  Retraite/SMIC &  Rev70/SMIC &  Rev75/SMIC &  Rev80/SMIC &  Rev85/SMIC &  Rev90/SMIC \\ 
\hline \hline 
 2052 &  62 &  66 ans 1 mois &  -20.42\% &  2179.37 &  {\bf 30.05} &  2445.56 &  {\bf {\color{red} 0.89}} &  {\bf {\color{red} 0.80}} &  {\bf {\color{red} 0.75}} &  {\bf {\color{red} 0.71}} &  {\bf {\color{red} 0.66}} &  {\bf {\color{red} 0.62}} \\ 
\hline 
 2053 &  63 &  66 ans 2 mois &  -15.83\% &  2420.92 &  {\bf 32.95} &  2477.35 &  {\bf {\color{red} 0.98}} &  {\bf {\color{red} 0.89}} &  {\bf {\color{red} 0.84}} &  {\bf {\color{red} 0.78}} &  {\bf {\color{red} 0.74}} &  {\bf {\color{red} 0.69}} \\ 
\hline 
 2054 &  64 &  66 ans 3 mois &  -11.25\% &  2677.87 &  {\bf 35.98} &  2509.56 &  {\bf 1.07} &  {\bf {\color{red} 0.99}} &  {\bf {\color{red} 0.93}} &  {\bf {\color{red} 0.87}} &  {\bf {\color{red} 0.81}} &  {\bf {\color{red} 0.76}} \\ 
\hline 
 2055 &  65 &  66 ans 4 mois &  -6.67\% &  2950.72 &  {\bf 39.14} &  2542.18 &  {\bf 1.16} &  {\bf 1.09} &  {\bf 1.02} &  {\bf {\color{red} 0.96}} &  {\bf {\color{red} 0.90}} &  {\bf {\color{red} 0.84}} \\ 
\hline 
 2056 &  66 &  66 ans 5 mois &  -2.08\% &  3239.95 &  {\bf 42.43} &  2575.23 &  {\bf 1.26} &  {\bf 1.19} &  {\bf 1.12} &  {\bf 1.05} &  {\bf {\color{red} 0.98}} &  {\bf {\color{red} 0.92}} \\ 
\hline 
 2057 &  67 &  66 ans 6 mois &  2.50\% &  3546.07 &  {\bf 45.84} &  2608.71 &  {\bf 1.36} &  {\bf 1.31} &  {\bf 1.23} &  {\bf 1.15} &  {\bf 1.08} &  {\bf 1.01} \\ 
\hline 
\hline 
\end{tabular} 
\end{center} } 

 \begin{center}\includegraphics[width=0.9\textwidth]{fig/CR_1990_25_dest_retraite.pdf}\end{center} \label{fig/CR_1990_25_dest_retraite.pdf} 

\newpage 
 
\subsection{Génération 2003 (début en 2028)} 

\paragraph{Retraites possibles et ratios Revenu/SMIC à 70, 75, 80, 85, 90 ans avec le modèle \emph{Gouvernement truqué (âge-pivot bloqué à 65 ans)}}  
 
{ \scriptsize \begin{center} 
\begin{tabular}[htb]{|c|c||c|c||c|c||c||c|c|c|c|c|c|} 
\hline 
 Retraite en &  Âge &  Âge pivot &  Décote/Surcote &  Retraite (\euro{} 2019) &  Tx Rempl(\%) &  SMIC (\euro{} 2019) &  Retraite/SMIC &  Rev70/SMIC &  Rev75/SMIC &  Rev80/SMIC &  Rev85/SMIC &  Rev90/SMIC \\ 
\hline \hline 
 2065 &  62 &  65 ans 0 mois &  -15.00\% &  1944.78 &  {\bf 40.91} &  3076.71 &  {\bf {\color{red} 0.63}} &  {\bf {\color{red} 0.57}} &  {\bf {\color{red} 0.53}} &  {\bf {\color{red} 0.50}} &  {\bf {\color{red} 0.47}} &  {\bf {\color{red} 0.44}} \\ 
\hline 
 2066 &  63 &  65 ans 0 mois &  -10.00\% &  2145.65 &  {\bf 45.03} &  3116.71 &  {\bf {\color{red} 0.69}} &  {\bf {\color{red} 0.63}} &  {\bf {\color{red} 0.59}} &  {\bf {\color{red} 0.55}} &  {\bf {\color{red} 0.52}} &  {\bf {\color{red} 0.49}} \\ 
\hline 
 2067 &  64 &  65 ans 0 mois &  -5.00\% &  2357.44 &  {\bf 49.36} &  3157.23 &  {\bf {\color{red} 0.75}} &  {\bf {\color{red} 0.69}} &  {\bf {\color{red} 0.65}} &  {\bf {\color{red} 0.61}} &  {\bf {\color{red} 0.57}} &  {\bf {\color{red} 0.53}} \\ 
\hline 
 2068 &  65 &  65 ans 0 mois &  0.00\% &  2580.41 &  {\bf 53.91} &  3198.27 &  {\bf {\color{red} 0.81}} &  {\bf {\color{red} 0.76}} &  {\bf {\color{red} 0.71}} &  {\bf {\color{red} 0.66}} &  {\bf {\color{red} 0.62}} &  {\bf {\color{red} 0.58}} \\ 
\hline 
 2069 &  66 &  65 ans 0 mois &  5.00\% &  2814.76 &  {\bf 58.68} &  3239.85 &  {\bf {\color{red} 0.87}} &  {\bf {\color{red} 0.83}} &  {\bf {\color{red} 0.77}} &  {\bf {\color{red} 0.73}} &  {\bf {\color{red} 0.68}} &  {\bf {\color{red} 0.64}} \\ 
\hline 
 2070 &  67 &  65 ans 0 mois &  10.00\% &  3060.75 &  {\bf 63.66} &  3281.97 &  {\bf {\color{red} 0.93}} &  {\bf {\color{red} 0.90}} &  {\bf {\color{red} 0.84}} &  {\bf {\color{red} 0.79}} &  {\bf {\color{red} 0.74}} &  {\bf {\color{red} 0.69}} \\ 
\hline 
\hline 
\end{tabular} 
\end{center} } 
\paragraph{Retraites possibles et ratios Revenu/SMIC à 70, 75, 80, 85, 90 ans avec le modèle \emph{Gouvernement corrigé (âge-pivot glissant)}}  
 
{ \scriptsize \begin{center} 
\begin{tabular}[htb]{|c|c||c|c||c|c||c||c|c|c|c|c|c|} 
\hline 
 Retraite en &  Âge &  Âge pivot &  Décote/Surcote &  Retraite (\euro{} 2019) &  Tx Rempl(\%) &  SMIC (\euro{} 2019) &  Retraite/SMIC &  Rev70/SMIC &  Rev75/SMIC &  Rev80/SMIC &  Rev85/SMIC &  Rev90/SMIC \\ 
\hline \hline 
 2065 &  62 &  67 ans 2 mois &  -25.83\% &  1696.92 &  {\bf 35.69} &  3076.71 &  {\bf {\color{red} 0.55}} &  {\bf {\color{red} 0.50}} &  {\bf {\color{red} 0.47}} &  {\bf {\color{red} 0.44}} &  {\bf {\color{red} 0.41}} &  {\bf {\color{red} 0.38}} \\ 
\hline 
 2066 &  63 &  67 ans 3 mois &  -21.25\% &  1877.44 &  {\bf 39.40} &  3116.71 &  {\bf {\color{red} 0.60}} &  {\bf {\color{red} 0.55}} &  {\bf {\color{red} 0.52}} &  {\bf {\color{red} 0.48}} &  {\bf {\color{red} 0.45}} &  {\bf {\color{red} 0.43}} \\ 
\hline 
 2067 &  64 &  67 ans 4 mois &  -16.67\% &  2067.93 &  {\bf 43.30} &  3157.23 &  {\bf {\color{red} 0.65}} &  {\bf {\color{red} 0.61}} &  {\bf {\color{red} 0.57}} &  {\bf {\color{red} 0.53}} &  {\bf {\color{red} 0.50}} &  {\bf {\color{red} 0.47}} \\ 
\hline 
 2068 &  65 &  67 ans 5 mois &  -12.08\% &  2268.61 &  {\bf 47.40} &  3198.27 &  {\bf {\color{red} 0.71}} &  {\bf {\color{red} 0.66}} &  {\bf {\color{red} 0.62}} &  {\bf {\color{red} 0.58}} &  {\bf {\color{red} 0.55}} &  {\bf {\color{red} 0.51}} \\ 
\hline 
 2069 &  66 &  67 ans 6 mois &  -7.50\% &  2479.67 &  {\bf 51.69} &  3239.85 &  {\bf {\color{red} 0.77}} &  {\bf {\color{red} 0.73}} &  {\bf {\color{red} 0.68}} &  {\bf {\color{red} 0.64}} &  {\bf {\color{red} 0.60}} &  {\bf {\color{red} 0.56}} \\ 
\hline 
 2070 &  67 &  67 ans 7 mois &  -2.92\% &  2701.34 &  {\bf 56.19} &  3281.97 &  {\bf {\color{red} 0.82}} &  {\bf {\color{red} 0.79}} &  {\bf {\color{red} 0.74}} &  {\bf {\color{red} 0.70}} &  {\bf {\color{red} 0.65}} &  {\bf {\color{red} 0.61}} \\ 
\hline 
\hline 
\end{tabular} 
\end{center} } 
\paragraph{Retraites possibles et ratios Revenu/SMIC à 70, 75, 80, 85, 90 ans avec le modèle \emph{Destinie2 (revalorisation de la fonction publique)}}  
 
{ \scriptsize \begin{center} 
\begin{tabular}[htb]{|c|c||c|c||c|c||c||c|c|c|c|c|c|} 
\hline 
 Retraite en &  Âge &  Âge pivot &  Décote/Surcote &  Retraite (\euro{} 2019) &  Tx Rempl(\%) &  SMIC (\euro{} 2019) &  Retraite/SMIC &  Rev70/SMIC &  Rev75/SMIC &  Rev80/SMIC &  Rev85/SMIC &  Rev90/SMIC \\ 
\hline \hline 
 2065 &  62 &  67 ans 2 mois &  -25.83\% &  2515.02 &  {\bf 29.32} &  2892.68 &  {\bf {\color{red} 0.87}} &  {\bf {\color{red} 0.78}} &  {\bf {\color{red} 0.74}} &  {\bf {\color{red} 0.69}} &  {\bf {\color{red} 0.65}} &  {\bf {\color{red} 0.61}} \\ 
\hline 
 2066 &  63 &  67 ans 3 mois &  -21.25\% &  2800.41 &  {\bf 32.23} &  2930.29 &  {\bf {\color{red} 0.96}} &  {\bf {\color{red} 0.87}} &  {\bf {\color{red} 0.82}} &  {\bf {\color{red} 0.77}} &  {\bf {\color{red} 0.72}} &  {\bf {\color{red} 0.67}} \\ 
\hline 
 2067 &  64 &  67 ans 4 mois &  -16.67\% &  3104.03 &  {\bf 35.26} &  2968.38 &  {\bf 1.05} &  {\bf {\color{red} 0.97}} &  {\bf {\color{red} 0.91}} &  {\bf {\color{red} 0.85}} &  {\bf {\color{red} 0.80}} &  {\bf {\color{red} 0.75}} \\ 
\hline 
 2068 &  65 &  67 ans 5 mois &  -12.08\% &  3426.44 &  {\bf 38.43} &  3006.97 &  {\bf 1.14} &  {\bf 1.07} &  {\bf 1.00} &  {\bf {\color{red} 0.94}} &  {\bf {\color{red} 0.88}} &  {\bf {\color{red} 0.83}} \\ 
\hline 
 2069 &  66 &  67 ans 6 mois &  -7.50\% &  3768.24 &  {\bf 41.72} &  3046.06 &  {\bf 1.24} &  {\bf 1.17} &  {\bf 1.10} &  {\bf 1.03} &  {\bf {\color{red} 0.97}} &  {\bf {\color{red} 0.91}} \\ 
\hline 
 2070 &  67 &  67 ans 7 mois &  -2.92\% &  4130.03 &  {\bf 45.13} &  3085.66 &  {\bf 1.34} &  {\bf 1.29} &  {\bf 1.21} &  {\bf 1.13} &  {\bf 1.06} &  {\bf {\color{red} 0.99}} \\ 
\hline 
\hline 
\end{tabular} 
\end{center} } 

 \begin{center}\includegraphics[width=0.9\textwidth]{fig/CR_2003_25_dest_retraite.pdf}\end{center} \label{fig/CR_2003_25_dest_retraite.pdf} 

\newpage 
 
\chapter{Professeur d'Université (thèse, MCF, PR2 puis PR1)} 

\begin{minipage}{0.55\linewidth}\includegraphics[width=0.7\textwidth]{fig/grille_PR.pdf}\end{minipage} 
\begin{minipage}{0.3\linewidth} 
 \begin{center} 

\begin{tabular}[htb]{|c|c|} 
\hline 
 Indice majoré &  Durée (années) \\ 
\hline \hline 
 430 &  3.00 \\ 
\hline 
 474 &  1.00 \\ 
\hline 
 510 &  2.00 \\ 
\hline 
 560 &  2.25 \\ 
\hline 
 600 &  2.50 \\ 
\hline 
 643 &  2.50 \\ 
\hline 
 667 &  1.25 \\ 
\hline 
 705 &  1.25 \\ 
\hline 
 743 &  1.25 \\ 
\hline 
 830 &  1.25 \\ 
\hline 
 830 &  3.00 \\ 
\hline 
 972 &  3.00 \\ 
\hline 
 1013 &   \\ 
\hline 
\hline 
\end{tabular} 
\end{center} 
 \end{minipage} 


 \addto{\captionsenglish}{ \renewcommand{\mtctitle}{}} \setcounter{minitocdepth}{2} 
 \minitoc \newpage 

\section{Début de carrière à 25 ans} 

\subsection{Génération 1975 (début en 2000)} 

\paragraph{Retraites possibles et ratios Revenu/SMIC à 70, 75, 80, 85, 90 ans avec le modèle \emph{Gouvernement truqué (âge-pivot bloqué à 65 ans)}}  
 
{ \scriptsize \begin{center} 
\begin{tabular}[htb]{|c|c||c|c||c|c||c||c|c|c|c|c|c|} 
\hline 
 Retraite en &  Âge &  Âge pivot &  Décote/Surcote &  Retraite (\euro{} 2019) &  Tx Rempl(\%) &  SMIC (\euro{} 2019) &  Retraite/SMIC &  Rev70/SMIC &  Rev75/SMIC &  Rev80/SMIC &  Rev85/SMIC &  Rev90/SMIC \\ 
\hline \hline 
 2037 &  62 &  64 ans 10 mois &  -14.17\% &  1859.55 &  {\bf 36.22} &  2143.00 &  {\bf {\color{red} 0.87}} &  {\bf {\color{red} 0.78}} &  {\bf {\color{red} 0.73}} &  {\bf {\color{red} 0.69}} &  {\bf {\color{red} 0.64}} &  {\bf {\color{red} 0.60}} \\ 
\hline 
 2038 &  63 &  64 ans 11 mois &  -9.58\% &  2038.54 &  {\bf 39.62} &  2170.86 &  {\bf {\color{red} 0.94}} &  {\bf {\color{red} 0.86}} &  {\bf {\color{red} 0.80}} &  {\bf {\color{red} 0.75}} &  {\bf {\color{red} 0.71}} &  {\bf {\color{red} 0.66}} \\ 
\hline 
 2039 &  64 &  65 ans 0 mois &  -5.00\% &  2228.05 &  {\bf 43.21} &  2199.08 &  {\bf 1.01} &  {\bf {\color{red} 0.94}} &  {\bf {\color{red} 0.88}} &  {\bf {\color{red} 0.82}} &  {\bf {\color{red} 0.77}} &  {\bf {\color{red} 0.72}} \\ 
\hline 
 2040 &  65 &  65 ans 0 mois &  0.00\% &  2438.73 &  {\bf 47.19} &  2227.67 &  {\bf 1.09} &  {\bf 1.03} &  {\bf {\color{red} 0.96}} &  {\bf {\color{red} 0.90}} &  {\bf {\color{red} 0.85}} &  {\bf {\color{red} 0.79}} \\ 
\hline 
 2041 &  66 &  65 ans 0 mois &  5.00\% &  2661.78 &  {\bf 51.39} &  2256.63 &  {\bf 1.18} &  {\bf 1.12} &  {\bf 1.05} &  {\bf {\color{red} 0.98}} &  {\bf {\color{red} 0.92}} &  {\bf {\color{red} 0.87}} \\ 
\hline 
 2042 &  67 &  65 ans 0 mois &  10.00\% &  2897.82 &  {\bf 55.83} &  2285.97 &  {\bf 1.27} &  {\bf 1.22} &  {\bf 1.14} &  {\bf 1.07} &  {\bf 1.00} &  {\bf {\color{red} 0.94}} \\ 
\hline 
\hline 
\end{tabular} 
\end{center} } 
\paragraph{Retraites possibles et ratios Revenu/SMIC à 70, 75, 80, 85, 90 ans avec le modèle \emph{Gouvernement corrigé (âge-pivot glissant)}}  
 
{ \scriptsize \begin{center} 
\begin{tabular}[htb]{|c|c||c|c||c|c||c||c|c|c|c|c|c|} 
\hline 
 Retraite en &  Âge &  Âge pivot &  Décote/Surcote &  Retraite (\euro{} 2019) &  Tx Rempl(\%) &  SMIC (\euro{} 2019) &  Retraite/SMIC &  Rev70/SMIC &  Rev75/SMIC &  Rev80/SMIC &  Rev85/SMIC &  Rev90/SMIC \\ 
\hline \hline 
 2037 &  62 &  64 ans 10 mois &  -14.17\% &  1859.55 &  {\bf 36.22} &  2143.00 &  {\bf {\color{red} 0.87}} &  {\bf {\color{red} 0.78}} &  {\bf {\color{red} 0.73}} &  {\bf {\color{red} 0.69}} &  {\bf {\color{red} 0.64}} &  {\bf {\color{red} 0.60}} \\ 
\hline 
 2038 &  63 &  64 ans 11 mois &  -9.58\% &  2038.54 &  {\bf 39.62} &  2170.86 &  {\bf {\color{red} 0.94}} &  {\bf {\color{red} 0.86}} &  {\bf {\color{red} 0.80}} &  {\bf {\color{red} 0.75}} &  {\bf {\color{red} 0.71}} &  {\bf {\color{red} 0.66}} \\ 
\hline 
 2039 &  64 &  65 ans 0 mois &  -5.00\% &  2228.05 &  {\bf 43.21} &  2199.08 &  {\bf 1.01} &  {\bf {\color{red} 0.94}} &  {\bf {\color{red} 0.88}} &  {\bf {\color{red} 0.82}} &  {\bf {\color{red} 0.77}} &  {\bf {\color{red} 0.72}} \\ 
\hline 
 2040 &  65 &  65 ans 1 mois &  -0.42\% &  2428.57 &  {\bf 46.99} &  2227.67 &  {\bf 1.09} &  {\bf 1.02} &  {\bf {\color{red} 0.96}} &  {\bf {\color{red} 0.90}} &  {\bf {\color{red} 0.84}} &  {\bf {\color{red} 0.79}} \\ 
\hline 
 2041 &  66 &  65 ans 2 mois &  4.17\% &  2640.65 &  {\bf 50.99} &  2256.63 &  {\bf 1.17} &  {\bf 1.11} &  {\bf 1.04} &  {\bf {\color{red} 0.98}} &  {\bf {\color{red} 0.92}} &  {\bf {\color{red} 0.86}} \\ 
\hline 
 2042 &  67 &  65 ans 3 mois &  8.75\% &  2864.89 &  {\bf 55.20} &  2285.97 &  {\bf 1.25} &  {\bf 1.21} &  {\bf 1.13} &  {\bf 1.06} &  {\bf {\color{red} 0.99}} &  {\bf {\color{red} 0.93}} \\ 
\hline 
\hline 
\end{tabular} 
\end{center} } 
\paragraph{Retraites possibles et ratios Revenu/SMIC à 70, 75, 80, 85, 90 ans avec le modèle \emph{Destinie2 (revalorisation de la fonction publique)}}  
 
{ \scriptsize \begin{center} 
\begin{tabular}[htb]{|c|c||c|c||c|c||c||c|c|c|c|c|c|} 
\hline 
 Retraite en &  Âge &  Âge pivot &  Décote/Surcote &  Retraite (\euro{} 2019) &  Tx Rempl(\%) &  SMIC (\euro{} 2019) &  Retraite/SMIC &  Rev70/SMIC &  Rev75/SMIC &  Rev80/SMIC &  Rev85/SMIC &  Rev90/SMIC \\ 
\hline \hline 
 2037 &  62 &  64 ans 10 mois &  -14.17\% &  2077.71 &  {\bf 32.21} &  2014.82 &  {\bf 1.03} &  {\bf {\color{red} 0.93}} &  {\bf {\color{red} 0.87}} &  {\bf {\color{red} 0.82}} &  {\bf {\color{red} 0.77}} &  {\bf {\color{red} 0.72}} \\ 
\hline 
 2038 &  63 &  64 ans 11 mois &  -9.58\% &  2287.57 &  {\bf 35.01} &  2041.01 &  {\bf 1.12} &  {\bf 1.02} &  {\bf {\color{red} 0.96}} &  {\bf {\color{red} 0.90}} &  {\bf {\color{red} 0.84}} &  {\bf {\color{red} 0.79}} \\ 
\hline 
 2039 &  64 &  65 ans 0 mois &  -5.00\% &  2511.23 &  {\bf 37.94} &  2067.55 &  {\bf 1.21} &  {\bf 1.12} &  {\bf 1.05} &  {\bf {\color{red} 0.99}} &  {\bf {\color{red} 0.93}} &  {\bf {\color{red} 0.87}} \\ 
\hline 
 2040 &  65 &  65 ans 1 mois &  -0.42\% &  2749.43 &  {\bf 41.01} &  2094.43 &  {\bf 1.31} &  {\bf 1.23} &  {\bf 1.15} &  {\bf 1.08} &  {\bf 1.01} &  {\bf {\color{red} 0.95}} \\ 
\hline 
 2041 &  66 &  65 ans 2 mois &  4.17\% &  3003.00 &  {\bf 44.21} &  2121.65 &  {\bf 1.42} &  {\bf 1.34} &  {\bf 1.26} &  {\bf 1.18} &  {\bf 1.11} &  {\bf 1.04} \\ 
\hline 
 2042 &  67 &  65 ans 3 mois &  8.75\% &  3272.80 &  {\bf 47.57} &  2149.23 &  {\bf 1.52} &  {\bf 1.46} &  {\bf 1.37} &  {\bf 1.29} &  {\bf 1.21} &  {\bf 1.13} \\ 
\hline 
\hline 
\end{tabular} 
\end{center} } 

 \begin{center}\includegraphics[width=0.9\textwidth]{fig/PR_1975_25_dest_retraite.pdf}\end{center} \label{fig/PR_1975_25_dest_retraite.pdf} 

\newpage 
 
\subsection{Génération 1980 (début en 2005)} 

\paragraph{Retraites possibles et ratios Revenu/SMIC à 70, 75, 80, 85, 90 ans avec le modèle \emph{Gouvernement truqué (âge-pivot bloqué à 65 ans)}}  
 
{ \scriptsize \begin{center} 
\begin{tabular}[htb]{|c|c||c|c||c|c||c||c|c|c|c|c|c|} 
\hline 
 Retraite en &  Âge &  Âge pivot &  Décote/Surcote &  Retraite (\euro{} 2019) &  Tx Rempl(\%) &  SMIC (\euro{} 2019) &  Retraite/SMIC &  Rev70/SMIC &  Rev75/SMIC &  Rev80/SMIC &  Rev85/SMIC &  Rev90/SMIC \\ 
\hline \hline 
 2042 &  62 &  65 ans 0 mois &  -15.00\% &  1886.33 &  {\bf 36.74} &  2285.97 &  {\bf {\color{red} 0.83}} &  {\bf {\color{red} 0.74}} &  {\bf {\color{red} 0.70}} &  {\bf {\color{red} 0.65}} &  {\bf {\color{red} 0.61}} &  {\bf {\color{red} 0.57}} \\ 
\hline 
 2043 &  63 &  65 ans 0 mois &  -10.00\% &  2084.64 &  {\bf 40.51} &  2315.68 &  {\bf {\color{red} 0.90}} &  {\bf {\color{red} 0.82}} &  {\bf {\color{red} 0.77}} &  {\bf {\color{red} 0.72}} &  {\bf {\color{red} 0.68}} &  {\bf {\color{red} 0.64}} \\ 
\hline 
 2044 &  64 &  65 ans 0 mois &  -5.00\% &  2295.56 &  {\bf 44.51} &  2345.79 &  {\bf {\color{red} 0.98}} &  {\bf {\color{red} 0.91}} &  {\bf {\color{red} 0.85}} &  {\bf {\color{red} 0.80}} &  {\bf {\color{red} 0.75}} &  {\bf {\color{red} 0.70}} \\ 
\hline 
 2045 &  65 &  65 ans 0 mois &  0.00\% &  2519.73 &  {\bf 48.76} &  2376.28 &  {\bf 1.06} &  {\bf {\color{red} 0.99}} &  {\bf {\color{red} 0.93}} &  {\bf {\color{red} 0.87}} &  {\bf {\color{red} 0.82}} &  {\bf {\color{red} 0.77}} \\ 
\hline 
 2046 &  66 &  65 ans 0 mois &  5.00\% &  2755.81 &  {\bf 53.21} &  2407.18 &  {\bf 1.14} &  {\bf 1.09} &  {\bf 1.02} &  {\bf {\color{red} 0.96}} &  {\bf {\color{red} 0.90}} &  {\bf {\color{red} 0.84}} \\ 
\hline 
 2047 &  67 &  65 ans 0 mois &  10.00\% &  3004.04 &  {\bf 57.88} &  2438.47 &  {\bf 1.23} &  {\bf 1.19} &  {\bf 1.11} &  {\bf 1.04} &  {\bf {\color{red} 0.98}} &  {\bf {\color{red} 0.92}} \\ 
\hline 
\hline 
\end{tabular} 
\end{center} } 
\paragraph{Retraites possibles et ratios Revenu/SMIC à 70, 75, 80, 85, 90 ans avec le modèle \emph{Gouvernement corrigé (âge-pivot glissant)}}  
 
{ \scriptsize \begin{center} 
\begin{tabular}[htb]{|c|c||c|c||c|c||c||c|c|c|c|c|c|} 
\hline 
 Retraite en &  Âge &  Âge pivot &  Décote/Surcote &  Retraite (\euro{} 2019) &  Tx Rempl(\%) &  SMIC (\euro{} 2019) &  Retraite/SMIC &  Rev70/SMIC &  Rev75/SMIC &  Rev80/SMIC &  Rev85/SMIC &  Rev90/SMIC \\ 
\hline \hline 
 2042 &  62 &  65 ans 3 mois &  -16.25\% &  1858.59 &  {\bf 36.20} &  2285.97 &  {\bf {\color{red} 0.81}} &  {\bf {\color{red} 0.73}} &  {\bf {\color{red} 0.69}} &  {\bf {\color{red} 0.64}} &  {\bf {\color{red} 0.60}} &  {\bf {\color{red} 0.57}} \\ 
\hline 
 2043 &  63 &  65 ans 4 mois &  -11.67\% &  2046.04 &  {\bf 39.76} &  2315.68 &  {\bf {\color{red} 0.88}} &  {\bf {\color{red} 0.81}} &  {\bf {\color{red} 0.76}} &  {\bf {\color{red} 0.71}} &  {\bf {\color{red} 0.67}} &  {\bf {\color{red} 0.62}} \\ 
\hline 
 2044 &  64 &  65 ans 5 mois &  -7.08\% &  2245.22 &  {\bf 43.54} &  2345.79 &  {\bf {\color{red} 0.96}} &  {\bf {\color{red} 0.89}} &  {\bf {\color{red} 0.83}} &  {\bf {\color{red} 0.78}} &  {\bf {\color{red} 0.73}} &  {\bf {\color{red} 0.68}} \\ 
\hline 
 2045 &  65 &  65 ans 6 mois &  -2.50\% &  2456.74 &  {\bf 47.54} &  2376.28 &  {\bf 1.03} &  {\bf {\color{red} 0.97}} &  {\bf {\color{red} 0.91}} &  {\bf {\color{red} 0.85}} &  {\bf {\color{red} 0.80}} &  {\bf {\color{red} 0.75}} \\ 
\hline 
 2046 &  66 &  65 ans 7 mois &  2.08\% &  2679.26 &  {\bf 51.73} &  2407.18 &  {\bf 1.11} &  {\bf 1.06} &  {\bf {\color{red} 0.99}} &  {\bf {\color{red} 0.93}} &  {\bf {\color{red} 0.87}} &  {\bf {\color{red} 0.82}} \\ 
\hline 
 2047 &  67 &  65 ans 8 mois &  6.67\% &  2913.01 &  {\bf 56.12} &  2438.47 &  {\bf 1.19} &  {\bf 1.15} &  {\bf 1.08} &  {\bf 1.01} &  {\bf {\color{red} 0.95}} &  {\bf {\color{red} 0.89}} \\ 
\hline 
\hline 
\end{tabular} 
\end{center} } 
\paragraph{Retraites possibles et ratios Revenu/SMIC à 70, 75, 80, 85, 90 ans avec le modèle \emph{Destinie2 (revalorisation de la fonction publique)}}  
 
{ \scriptsize \begin{center} 
\begin{tabular}[htb]{|c|c||c|c||c|c||c||c|c|c|c|c|c|} 
\hline 
 Retraite en &  Âge &  Âge pivot &  Décote/Surcote &  Retraite (\euro{} 2019) &  Tx Rempl(\%) &  SMIC (\euro{} 2019) &  Retraite/SMIC &  Rev70/SMIC &  Rev75/SMIC &  Rev80/SMIC &  Rev85/SMIC &  Rev90/SMIC \\ 
\hline \hline 
 2042 &  62 &  65 ans 3 mois &  -16.25\% &  2165.61 &  {\bf 31.48} &  2149.23 &  {\bf 1.01} &  {\bf {\color{red} 0.91}} &  {\bf {\color{red} 0.85}} &  {\bf {\color{red} 0.80}} &  {\bf {\color{red} 0.75}} &  {\bf {\color{red} 0.70}} \\ 
\hline 
 2043 &  63 &  65 ans 4 mois &  -11.67\% &  2396.00 &  {\bf 34.38} &  2177.17 &  {\bf 1.10} &  {\bf 1.01} &  {\bf {\color{red} 0.94}} &  {\bf {\color{red} 0.88}} &  {\bf {\color{red} 0.83}} &  {\bf {\color{red} 0.78}} \\ 
\hline 
 2044 &  64 &  65 ans 5 mois &  -7.08\% &  2642.46 &  {\bf 37.43} &  2205.48 &  {\bf 1.20} &  {\bf 1.11} &  {\bf 1.04} &  {\bf {\color{red} 0.97}} &  {\bf {\color{red} 0.91}} &  {\bf {\color{red} 0.86}} \\ 
\hline 
 2045 &  65 &  65 ans 6 mois &  -2.50\% &  2905.91 &  {\bf 40.63} &  2234.15 &  {\bf 1.30} &  {\bf 1.22} &  {\bf 1.14} &  {\bf 1.07} &  {\bf 1.00} &  {\bf {\color{red} 0.94}} \\ 
\hline 
 2046 &  66 &  65 ans 7 mois &  2.08\% &  3185.01 &  {\bf 43.96} &  2263.19 &  {\bf 1.41} &  {\bf 1.34} &  {\bf 1.25} &  {\bf 1.17} &  {\bf 1.10} &  {\bf 1.03} \\ 
\hline 
 2047 &  67 &  65 ans 8 mois &  6.67\% &  3480.24 &  {\bf 47.42} &  2292.61 &  {\bf 1.52} &  {\bf 1.46} &  {\bf 1.37} &  {\bf 1.28} &  {\bf 1.20} &  {\bf 1.13} \\ 
\hline 
\hline 
\end{tabular} 
\end{center} } 

 \begin{center}\includegraphics[width=0.9\textwidth]{fig/PR_1980_25_dest_retraite.pdf}\end{center} \label{fig/PR_1980_25_dest_retraite.pdf} 

\newpage 
 
\subsection{Génération 1990 (début en 2015)} 

\paragraph{Retraites possibles et ratios Revenu/SMIC à 70, 75, 80, 85, 90 ans avec le modèle \emph{Gouvernement truqué (âge-pivot bloqué à 65 ans)}}  
 
{ \scriptsize \begin{center} 
\begin{tabular}[htb]{|c|c||c|c||c|c||c||c|c|c|c|c|c|} 
\hline 
 Retraite en &  Âge &  Âge pivot &  Décote/Surcote &  Retraite (\euro{} 2019) &  Tx Rempl(\%) &  SMIC (\euro{} 2019) &  Retraite/SMIC &  Rev70/SMIC &  Rev75/SMIC &  Rev80/SMIC &  Rev85/SMIC &  Rev90/SMIC \\ 
\hline \hline 
 2052 &  62 &  65 ans 0 mois &  -15.00\% &  2024.10 &  {\bf 39.42} &  2601.14 &  {\bf {\color{red} 0.78}} &  {\bf {\color{red} 0.70}} &  {\bf {\color{red} 0.66}} &  {\bf {\color{red} 0.62}} &  {\bf {\color{red} 0.58}} &  {\bf {\color{red} 0.54}} \\ 
\hline 
 2053 &  63 &  65 ans 0 mois &  -10.00\% &  2235.48 &  {\bf 43.44} &  2634.96 &  {\bf {\color{red} 0.85}} &  {\bf {\color{red} 0.78}} &  {\bf {\color{red} 0.73}} &  {\bf {\color{red} 0.68}} &  {\bf {\color{red} 0.64}} &  {\bf {\color{red} 0.60}} \\ 
\hline 
 2054 &  64 &  65 ans 0 mois &  -5.00\% &  2458.54 &  {\bf 47.67} &  2669.21 &  {\bf {\color{red} 0.92}} &  {\bf {\color{red} 0.85}} &  {\bf {\color{red} 0.80}} &  {\bf {\color{red} 0.75}} &  {\bf {\color{red} 0.70}} &  {\bf {\color{red} 0.66}} \\ 
\hline 
 2055 &  65 &  65 ans 0 mois &  0.00\% &  2693.52 &  {\bf 52.12} &  2703.91 &  {\bf {\color{red} 1.00}} &  {\bf {\color{red} 0.93}} &  {\bf {\color{red} 0.88}} &  {\bf {\color{red} 0.82}} &  {\bf {\color{red} 0.77}} &  {\bf {\color{red} 0.72}} \\ 
\hline 
 2056 &  66 &  65 ans 0 mois &  5.00\% &  2940.66 &  {\bf 56.78} &  2739.06 &  {\bf 1.07} &  {\bf 1.02} &  {\bf {\color{red} 0.96}} &  {\bf {\color{red} 0.90}} &  {\bf {\color{red} 0.84}} &  {\bf {\color{red} 0.79}} \\ 
\hline 
 2057 &  67 &  65 ans 0 mois &  10.00\% &  3200.21 &  {\bf 61.66} &  2774.67 &  {\bf 1.15} &  {\bf 1.11} &  {\bf 1.04} &  {\bf {\color{red} 0.98}} &  {\bf {\color{red} 0.91}} &  {\bf {\color{red} 0.86}} \\ 
\hline 
\hline 
\end{tabular} 
\end{center} } 
\paragraph{Retraites possibles et ratios Revenu/SMIC à 70, 75, 80, 85, 90 ans avec le modèle \emph{Gouvernement corrigé (âge-pivot glissant)}}  
 
{ \scriptsize \begin{center} 
\begin{tabular}[htb]{|c|c||c|c||c|c||c||c|c|c|c|c|c|} 
\hline 
 Retraite en &  Âge &  Âge pivot &  Décote/Surcote &  Retraite (\euro{} 2019) &  Tx Rempl(\%) &  SMIC (\euro{} 2019) &  Retraite/SMIC &  Rev70/SMIC &  Rev75/SMIC &  Rev80/SMIC &  Rev85/SMIC &  Rev90/SMIC \\ 
\hline \hline 
 2052 &  62 &  66 ans 1 mois &  -20.42\% &  1895.11 &  {\bf 36.91} &  2601.14 &  {\bf {\color{red} 0.73}} &  {\bf {\color{red} 0.66}} &  {\bf {\color{red} 0.62}} &  {\bf {\color{red} 0.58}} &  {\bf {\color{red} 0.54}} &  {\bf {\color{red} 0.51}} \\ 
\hline 
 2053 &  63 &  66 ans 2 mois &  -15.83\% &  2090.59 &  {\bf 40.63} &  2634.96 &  {\bf {\color{red} 0.79}} &  {\bf {\color{red} 0.72}} &  {\bf {\color{red} 0.68}} &  {\bf {\color{red} 0.64}} &  {\bf {\color{red} 0.60}} &  {\bf {\color{red} 0.56}} \\ 
\hline 
 2054 &  64 &  66 ans 3 mois &  -11.25\% &  2296.80 &  {\bf 44.54} &  2669.21 &  {\bf {\color{red} 0.86}} &  {\bf {\color{red} 0.80}} &  {\bf {\color{red} 0.75}} &  {\bf {\color{red} 0.70}} &  {\bf {\color{red} 0.66}} &  {\bf {\color{red} 0.62}} \\ 
\hline 
 2055 &  65 &  66 ans 4 mois &  -6.67\% &  2513.95 &  {\bf 48.64} &  2703.91 &  {\bf {\color{red} 0.93}} &  {\bf {\color{red} 0.87}} &  {\bf {\color{red} 0.82}} &  {\bf {\color{red} 0.77}} &  {\bf {\color{red} 0.72}} &  {\bf {\color{red} 0.67}} \\ 
\hline 
 2056 &  66 &  66 ans 5 mois &  -2.08\% &  2742.28 &  {\bf 52.95} &  2739.06 &  {\bf 1.00} &  {\bf {\color{red} 0.95}} &  {\bf {\color{red} 0.89}} &  {\bf {\color{red} 0.84}} &  {\bf {\color{red} 0.78}} &  {\bf {\color{red} 0.73}} \\ 
\hline 
 2057 &  67 &  66 ans 6 mois &  2.50\% &  2982.02 &  {\bf 57.45} &  2774.67 &  {\bf 1.07} &  {\bf 1.03} &  {\bf {\color{red} 0.97}} &  {\bf {\color{red} 0.91}} &  {\bf {\color{red} 0.85}} &  {\bf {\color{red} 0.80}} \\ 
\hline 
\hline 
\end{tabular} 
\end{center} } 
\paragraph{Retraites possibles et ratios Revenu/SMIC à 70, 75, 80, 85, 90 ans avec le modèle \emph{Destinie2 (revalorisation de la fonction publique)}}  
 
{ \scriptsize \begin{center} 
\begin{tabular}[htb]{|c|c||c|c||c|c||c||c|c|c|c|c|c|} 
\hline 
 Retraite en &  Âge &  Âge pivot &  Décote/Surcote &  Retraite (\euro{} 2019) &  Tx Rempl(\%) &  SMIC (\euro{} 2019) &  Retraite/SMIC &  Rev70/SMIC &  Rev75/SMIC &  Rev80/SMIC &  Rev85/SMIC &  Rev90/SMIC \\ 
\hline \hline 
 2052 &  62 &  66 ans 1 mois &  -20.42\% &  2449.16 &  {\bf 31.28} &  2445.56 &  {\bf 1.00} &  {\bf {\color{red} 0.90}} &  {\bf {\color{red} 0.85}} &  {\bf {\color{red} 0.79}} &  {\bf {\color{red} 0.74}} &  {\bf {\color{red} 0.70}} \\ 
\hline 
 2053 &  63 &  66 ans 2 mois &  -15.83\% &  2716.79 &  {\bf 34.26} &  2477.35 &  {\bf 1.10} &  {\bf 1.00} &  {\bf {\color{red} 0.94}} &  {\bf {\color{red} 0.88}} &  {\bf {\color{red} 0.83}} &  {\bf {\color{red} 0.77}} \\ 
\hline 
 2054 &  64 &  66 ans 3 mois &  -11.25\% &  3001.22 &  {\bf 37.36} &  2509.56 &  {\bf 1.20} &  {\bf 1.11} &  {\bf 1.04} &  {\bf {\color{red} 0.97}} &  {\bf {\color{red} 0.91}} &  {\bf {\color{red} 0.85}} \\ 
\hline 
 2055 &  65 &  66 ans 4 mois &  -6.67\% &  3302.97 &  {\bf 40.59} &  2542.18 &  {\bf 1.30} &  {\bf 1.22} &  {\bf 1.14} &  {\bf 1.07} &  {\bf 1.00} &  {\bf {\color{red} 0.94}} \\ 
\hline 
 2056 &  66 &  66 ans 5 mois &  -2.08\% &  3622.57 &  {\bf 43.94} &  2575.23 &  {\bf 1.41} &  {\bf 1.34} &  {\bf 1.25} &  {\bf 1.17} &  {\bf 1.10} &  {\bf 1.03} \\ 
\hline 
 2057 &  67 &  66 ans 6 mois &  2.50\% &  3960.58 &  {\bf 47.43} &  2608.71 &  {\bf 1.52} &  {\bf 1.46} &  {\bf 1.37} &  {\bf 1.28} &  {\bf 1.20} &  {\bf 1.13} \\ 
\hline 
\hline 
\end{tabular} 
\end{center} } 

 \begin{center}\includegraphics[width=0.9\textwidth]{fig/PR_1990_25_dest_retraite.pdf}\end{center} \label{fig/PR_1990_25_dest_retraite.pdf} 

\newpage 
 
\subsection{Génération 2003 (début en 2028)} 

\paragraph{Retraites possibles et ratios Revenu/SMIC à 70, 75, 80, 85, 90 ans avec le modèle \emph{Gouvernement truqué (âge-pivot bloqué à 65 ans)}}  
 
{ \scriptsize \begin{center} 
\begin{tabular}[htb]{|c|c||c|c||c|c||c||c|c|c|c|c|c|} 
\hline 
 Retraite en &  Âge &  Âge pivot &  Décote/Surcote &  Retraite (\euro{} 2019) &  Tx Rempl(\%) &  SMIC (\euro{} 2019) &  Retraite/SMIC &  Rev70/SMIC &  Rev75/SMIC &  Rev80/SMIC &  Rev85/SMIC &  Rev90/SMIC \\ 
\hline \hline 
 2065 &  62 &  65 ans 0 mois &  -15.00\% &  2134.16 &  {\bf 41.56} &  3076.71 &  {\bf {\color{red} 0.69}} &  {\bf {\color{red} 0.63}} &  {\bf {\color{red} 0.59}} &  {\bf {\color{red} 0.55}} &  {\bf {\color{red} 0.52}} &  {\bf {\color{red} 0.48}} \\ 
\hline 
 2066 &  63 &  65 ans 0 mois &  -10.00\% &  2353.54 &  {\bf 45.74} &  3116.71 &  {\bf {\color{red} 0.76}} &  {\bf {\color{red} 0.69}} &  {\bf {\color{red} 0.65}} &  {\bf {\color{red} 0.61}} &  {\bf {\color{red} 0.57}} &  {\bf {\color{red} 0.53}} \\ 
\hline 
 2067 &  64 &  65 ans 0 mois &  -5.00\% &  2584.78 &  {\bf 50.12} &  3157.23 &  {\bf {\color{red} 0.82}} &  {\bf {\color{red} 0.76}} &  {\bf {\color{red} 0.71}} &  {\bf {\color{red} 0.67}} &  {\bf {\color{red} 0.62}} &  {\bf {\color{red} 0.59}} \\ 
\hline 
 2068 &  65 &  65 ans 0 mois &  0.00\% &  2828.13 &  {\bf 54.72} &  3198.27 &  {\bf {\color{red} 0.88}} &  {\bf {\color{red} 0.83}} &  {\bf {\color{red} 0.78}} &  {\bf {\color{red} 0.73}} &  {\bf {\color{red} 0.68}} &  {\bf {\color{red} 0.64}} \\ 
\hline 
 2069 &  66 &  65 ans 0 mois &  5.00\% &  3083.84 &  {\bf 59.54} &  3239.85 &  {\bf {\color{red} 0.95}} &  {\bf {\color{red} 0.90}} &  {\bf {\color{red} 0.85}} &  {\bf {\color{red} 0.79}} &  {\bf {\color{red} 0.74}} &  {\bf {\color{red} 0.70}} \\ 
\hline 
 2070 &  67 &  65 ans 0 mois &  10.00\% &  3352.16 &  {\bf 64.58} &  3281.97 &  {\bf 1.02} &  {\bf {\color{red} 0.98}} &  {\bf {\color{red} 0.92}} &  {\bf {\color{red} 0.86}} &  {\bf {\color{red} 0.81}} &  {\bf {\color{red} 0.76}} \\ 
\hline 
\hline 
\end{tabular} 
\end{center} } 
\paragraph{Retraites possibles et ratios Revenu/SMIC à 70, 75, 80, 85, 90 ans avec le modèle \emph{Gouvernement corrigé (âge-pivot glissant)}}  
 
{ \scriptsize \begin{center} 
\begin{tabular}[htb]{|c|c||c|c||c|c||c||c|c|c|c|c|c|} 
\hline 
 Retraite en &  Âge &  Âge pivot &  Décote/Surcote &  Retraite (\euro{} 2019) &  Tx Rempl(\%) &  SMIC (\euro{} 2019) &  Retraite/SMIC &  Rev70/SMIC &  Rev75/SMIC &  Rev80/SMIC &  Rev85/SMIC &  Rev90/SMIC \\ 
\hline \hline 
 2065 &  62 &  67 ans 2 mois &  -25.83\% &  1862.16 &  {\bf 36.27} &  3076.71 &  {\bf {\color{red} 0.61}} &  {\bf {\color{red} 0.55}} &  {\bf {\color{red} 0.51}} &  {\bf {\color{red} 0.48}} &  {\bf {\color{red} 0.45}} &  {\bf {\color{red} 0.42}} \\ 
\hline 
 2066 &  63 &  67 ans 3 mois &  -21.25\% &  2059.35 &  {\bf 40.02} &  3116.71 &  {\bf {\color{red} 0.66}} &  {\bf {\color{red} 0.60}} &  {\bf {\color{red} 0.57}} &  {\bf {\color{red} 0.53}} &  {\bf {\color{red} 0.50}} &  {\bf {\color{red} 0.47}} \\ 
\hline 
 2067 &  64 &  67 ans 4 mois &  -16.67\% &  2267.35 &  {\bf 43.97} &  3157.23 &  {\bf {\color{red} 0.72}} &  {\bf {\color{red} 0.66}} &  {\bf {\color{red} 0.62}} &  {\bf {\color{red} 0.58}} &  {\bf {\color{red} 0.55}} &  {\bf {\color{red} 0.51}} \\ 
\hline 
 2068 &  65 &  67 ans 5 mois &  -12.08\% &  2486.40 &  {\bf 48.11} &  3198.27 &  {\bf {\color{red} 0.78}} &  {\bf {\color{red} 0.73}} &  {\bf {\color{red} 0.68}} &  {\bf {\color{red} 0.64}} &  {\bf {\color{red} 0.60}} &  {\bf {\color{red} 0.56}} \\ 
\hline 
 2069 &  66 &  67 ans 6 mois &  -7.50\% &  2716.71 &  {\bf 52.45} &  3239.85 &  {\bf {\color{red} 0.84}} &  {\bf {\color{red} 0.80}} &  {\bf {\color{red} 0.75}} &  {\bf {\color{red} 0.70}} &  {\bf {\color{red} 0.66}} &  {\bf {\color{red} 0.62}} \\ 
\hline 
 2070 &  67 &  67 ans 7 mois &  -2.92\% &  2958.53 &  {\bf 57.00} &  3281.97 &  {\bf {\color{red} 0.90}} &  {\bf {\color{red} 0.87}} &  {\bf {\color{red} 0.81}} &  {\bf {\color{red} 0.76}} &  {\bf {\color{red} 0.71}} &  {\bf {\color{red} 0.67}} \\ 
\hline 
\hline 
\end{tabular} 
\end{center} } 
\paragraph{Retraites possibles et ratios Revenu/SMIC à 70, 75, 80, 85, 90 ans avec le modèle \emph{Destinie2 (revalorisation de la fonction publique)}}  
 
{ \scriptsize \begin{center} 
\begin{tabular}[htb]{|c|c||c|c||c|c||c||c|c|c|c|c|c|} 
\hline 
 Retraite en &  Âge &  Âge pivot &  Décote/Surcote &  Retraite (\euro{} 2019) &  Tx Rempl(\%) &  SMIC (\euro{} 2019) &  Retraite/SMIC &  Rev70/SMIC &  Rev75/SMIC &  Rev80/SMIC &  Rev85/SMIC &  Rev90/SMIC \\ 
\hline \hline 
 2065 &  62 &  67 ans 2 mois &  -25.83\% &  2819.08 &  {\bf 30.44} &  2892.68 &  {\bf {\color{red} 0.97}} &  {\bf {\color{red} 0.88}} &  {\bf {\color{red} 0.82}} &  {\bf {\color{red} 0.77}} &  {\bf {\color{red} 0.72}} &  {\bf {\color{red} 0.68}} \\ 
\hline 
 2066 &  63 &  67 ans 3 mois &  -21.25\% &  3135.02 &  {\bf 33.42} &  2930.29 &  {\bf 1.07} &  {\bf {\color{red} 0.98}} &  {\bf {\color{red} 0.92}} &  {\bf {\color{red} 0.86}} &  {\bf {\color{red} 0.81}} &  {\bf {\color{red} 0.75}} \\ 
\hline 
 2067 &  64 &  67 ans 4 mois &  -16.67\% &  3470.84 &  {\bf 36.53} &  2968.38 &  {\bf 1.17} &  {\bf 1.08} &  {\bf 1.01} &  {\bf {\color{red} 0.95}} &  {\bf {\color{red} 0.89}} &  {\bf {\color{red} 0.84}} \\ 
\hline 
 2068 &  65 &  67 ans 5 mois &  -12.08\% &  3827.13 &  {\bf 39.76} &  3006.97 &  {\bf 1.27} &  {\bf 1.19} &  {\bf 1.12} &  {\bf 1.05} &  {\bf {\color{red} 0.98}} &  {\bf {\color{red} 0.92}} \\ 
\hline 
 2069 &  66 &  67 ans 6 mois &  -7.50\% &  4204.54 &  {\bf 43.12} &  3046.06 &  {\bf 1.38} &  {\bf 1.31} &  {\bf 1.23} &  {\bf 1.15} &  {\bf 1.08} &  {\bf 1.01} \\ 
\hline 
 2070 &  67 &  67 ans 7 mois &  -2.92\% &  4603.73 &  {\bf 46.61} &  3085.66 &  {\bf 1.49} &  {\bf 1.44} &  {\bf 1.35} &  {\bf 1.26} &  {\bf 1.18} &  {\bf 1.11} \\ 
\hline 
\hline 
\end{tabular} 
\end{center} } 

 \begin{center}\includegraphics[width=0.9\textwidth]{fig/PR_2003_25_dest_retraite.pdf}\end{center} \label{fig/PR_2003_25_dest_retraite.pdf} 

\newpage 
 
\chapter{Directeur de Recherche (thèse, CRCN, DR2 puis DR1)} 

\begin{minipage}{0.55\linewidth}\includegraphics[width=0.7\textwidth]{fig/grille_DR.pdf}\end{minipage} 
\begin{minipage}{0.3\linewidth} 
 \begin{center} 

\begin{tabular}[htb]{|c|c|} 
\hline 
 Indice majoré &  Durée (années) \\ 
\hline \hline 
 430 &  3.00 \\ 
\hline 
 474 &  1.00 \\ 
\hline 
 510 &  2.00 \\ 
\hline 
 560 &  2.25 \\ 
\hline 
 600 &  2.50 \\ 
\hline 
 643 &  2.50 \\ 
\hline 
 667 &  1.25 \\ 
\hline 
 705 &  1.25 \\ 
\hline 
 743 &  1.25 \\ 
\hline 
 830 &  1.25 \\ 
\hline 
 830 &  3.00 \\ 
\hline 
 972 &  3.00 \\ 
\hline 
 1013 &   \\ 
\hline 
\hline 
\end{tabular} 
\end{center} 
 \end{minipage} 


 \addto{\captionsenglish}{ \renewcommand{\mtctitle}{}} \setcounter{minitocdepth}{2} 
 \minitoc \newpage 

\section{Début de carrière à 25 ans} 

\subsection{Génération 1975 (début en 2000)} 

\paragraph{Retraites possibles et ratios Revenu/SMIC à 70, 75, 80, 85, 90 ans avec le modèle \emph{Gouvernement truqué (âge-pivot bloqué à 65 ans)}}  
 
{ \scriptsize \begin{center} 
\begin{tabular}[htb]{|c|c||c|c||c|c||c||c|c|c|c|c|c|} 
\hline 
 Retraite en &  Âge &  Âge pivot &  Décote/Surcote &  Retraite (\euro{} 2019) &  Tx Rempl(\%) &  SMIC (\euro{} 2019) &  Retraite/SMIC &  Rev70/SMIC &  Rev75/SMIC &  Rev80/SMIC &  Rev85/SMIC &  Rev90/SMIC \\ 
\hline \hline 
 2037 &  62 &  64 ans 10 mois &  -14.17\% &  1859.55 &  {\bf 36.22} &  2143.00 &  {\bf {\color{red} 0.87}} &  {\bf {\color{red} 0.78}} &  {\bf {\color{red} 0.73}} &  {\bf {\color{red} 0.69}} &  {\bf {\color{red} 0.64}} &  {\bf {\color{red} 0.60}} \\ 
\hline 
 2038 &  63 &  64 ans 11 mois &  -9.58\% &  2038.54 &  {\bf 39.62} &  2170.86 &  {\bf {\color{red} 0.94}} &  {\bf {\color{red} 0.86}} &  {\bf {\color{red} 0.80}} &  {\bf {\color{red} 0.75}} &  {\bf {\color{red} 0.71}} &  {\bf {\color{red} 0.66}} \\ 
\hline 
 2039 &  64 &  65 ans 0 mois &  -5.00\% &  2228.05 &  {\bf 43.21} &  2199.08 &  {\bf 1.01} &  {\bf {\color{red} 0.94}} &  {\bf {\color{red} 0.88}} &  {\bf {\color{red} 0.82}} &  {\bf {\color{red} 0.77}} &  {\bf {\color{red} 0.72}} \\ 
\hline 
 2040 &  65 &  65 ans 0 mois &  0.00\% &  2438.73 &  {\bf 47.19} &  2227.67 &  {\bf 1.09} &  {\bf 1.03} &  {\bf {\color{red} 0.96}} &  {\bf {\color{red} 0.90}} &  {\bf {\color{red} 0.85}} &  {\bf {\color{red} 0.79}} \\ 
\hline 
 2041 &  66 &  65 ans 0 mois &  5.00\% &  2661.78 &  {\bf 51.39} &  2256.63 &  {\bf 1.18} &  {\bf 1.12} &  {\bf 1.05} &  {\bf {\color{red} 0.98}} &  {\bf {\color{red} 0.92}} &  {\bf {\color{red} 0.87}} \\ 
\hline 
 2042 &  67 &  65 ans 0 mois &  10.00\% &  2897.82 &  {\bf 55.83} &  2285.97 &  {\bf 1.27} &  {\bf 1.22} &  {\bf 1.14} &  {\bf 1.07} &  {\bf 1.00} &  {\bf {\color{red} 0.94}} \\ 
\hline 
\hline 
\end{tabular} 
\end{center} } 
\paragraph{Retraites possibles et ratios Revenu/SMIC à 70, 75, 80, 85, 90 ans avec le modèle \emph{Gouvernement corrigé (âge-pivot glissant)}}  
 
{ \scriptsize \begin{center} 
\begin{tabular}[htb]{|c|c||c|c||c|c||c||c|c|c|c|c|c|} 
\hline 
 Retraite en &  Âge &  Âge pivot &  Décote/Surcote &  Retraite (\euro{} 2019) &  Tx Rempl(\%) &  SMIC (\euro{} 2019) &  Retraite/SMIC &  Rev70/SMIC &  Rev75/SMIC &  Rev80/SMIC &  Rev85/SMIC &  Rev90/SMIC \\ 
\hline \hline 
 2037 &  62 &  64 ans 10 mois &  -14.17\% &  1859.55 &  {\bf 36.22} &  2143.00 &  {\bf {\color{red} 0.87}} &  {\bf {\color{red} 0.78}} &  {\bf {\color{red} 0.73}} &  {\bf {\color{red} 0.69}} &  {\bf {\color{red} 0.64}} &  {\bf {\color{red} 0.60}} \\ 
\hline 
 2038 &  63 &  64 ans 11 mois &  -9.58\% &  2038.54 &  {\bf 39.62} &  2170.86 &  {\bf {\color{red} 0.94}} &  {\bf {\color{red} 0.86}} &  {\bf {\color{red} 0.80}} &  {\bf {\color{red} 0.75}} &  {\bf {\color{red} 0.71}} &  {\bf {\color{red} 0.66}} \\ 
\hline 
 2039 &  64 &  65 ans 0 mois &  -5.00\% &  2228.05 &  {\bf 43.21} &  2199.08 &  {\bf 1.01} &  {\bf {\color{red} 0.94}} &  {\bf {\color{red} 0.88}} &  {\bf {\color{red} 0.82}} &  {\bf {\color{red} 0.77}} &  {\bf {\color{red} 0.72}} \\ 
\hline 
 2040 &  65 &  65 ans 1 mois &  -0.42\% &  2428.57 &  {\bf 46.99} &  2227.67 &  {\bf 1.09} &  {\bf 1.02} &  {\bf {\color{red} 0.96}} &  {\bf {\color{red} 0.90}} &  {\bf {\color{red} 0.84}} &  {\bf {\color{red} 0.79}} \\ 
\hline 
 2041 &  66 &  65 ans 2 mois &  4.17\% &  2640.65 &  {\bf 50.99} &  2256.63 &  {\bf 1.17} &  {\bf 1.11} &  {\bf 1.04} &  {\bf {\color{red} 0.98}} &  {\bf {\color{red} 0.92}} &  {\bf {\color{red} 0.86}} \\ 
\hline 
 2042 &  67 &  65 ans 3 mois &  8.75\% &  2864.89 &  {\bf 55.20} &  2285.97 &  {\bf 1.25} &  {\bf 1.21} &  {\bf 1.13} &  {\bf 1.06} &  {\bf {\color{red} 0.99}} &  {\bf {\color{red} 0.93}} \\ 
\hline 
\hline 
\end{tabular} 
\end{center} } 
\paragraph{Retraites possibles et ratios Revenu/SMIC à 70, 75, 80, 85, 90 ans avec le modèle \emph{Destinie2 (revalorisation de la fonction publique)}}  
 
{ \scriptsize \begin{center} 
\begin{tabular}[htb]{|c|c||c|c||c|c||c||c|c|c|c|c|c|} 
\hline 
 Retraite en &  Âge &  Âge pivot &  Décote/Surcote &  Retraite (\euro{} 2019) &  Tx Rempl(\%) &  SMIC (\euro{} 2019) &  Retraite/SMIC &  Rev70/SMIC &  Rev75/SMIC &  Rev80/SMIC &  Rev85/SMIC &  Rev90/SMIC \\ 
\hline \hline 
 2037 &  62 &  64 ans 10 mois &  -14.17\% &  2077.71 &  {\bf 32.21} &  2014.82 &  {\bf 1.03} &  {\bf {\color{red} 0.93}} &  {\bf {\color{red} 0.87}} &  {\bf {\color{red} 0.82}} &  {\bf {\color{red} 0.77}} &  {\bf {\color{red} 0.72}} \\ 
\hline 
 2038 &  63 &  64 ans 11 mois &  -9.58\% &  2287.57 &  {\bf 35.01} &  2041.01 &  {\bf 1.12} &  {\bf 1.02} &  {\bf {\color{red} 0.96}} &  {\bf {\color{red} 0.90}} &  {\bf {\color{red} 0.84}} &  {\bf {\color{red} 0.79}} \\ 
\hline 
 2039 &  64 &  65 ans 0 mois &  -5.00\% &  2511.23 &  {\bf 37.94} &  2067.55 &  {\bf 1.21} &  {\bf 1.12} &  {\bf 1.05} &  {\bf {\color{red} 0.99}} &  {\bf {\color{red} 0.93}} &  {\bf {\color{red} 0.87}} \\ 
\hline 
 2040 &  65 &  65 ans 1 mois &  -0.42\% &  2749.43 &  {\bf 41.01} &  2094.43 &  {\bf 1.31} &  {\bf 1.23} &  {\bf 1.15} &  {\bf 1.08} &  {\bf 1.01} &  {\bf {\color{red} 0.95}} \\ 
\hline 
 2041 &  66 &  65 ans 2 mois &  4.17\% &  3003.00 &  {\bf 44.21} &  2121.65 &  {\bf 1.42} &  {\bf 1.34} &  {\bf 1.26} &  {\bf 1.18} &  {\bf 1.11} &  {\bf 1.04} \\ 
\hline 
 2042 &  67 &  65 ans 3 mois &  8.75\% &  3272.80 &  {\bf 47.57} &  2149.23 &  {\bf 1.52} &  {\bf 1.46} &  {\bf 1.37} &  {\bf 1.29} &  {\bf 1.21} &  {\bf 1.13} \\ 
\hline 
\hline 
\end{tabular} 
\end{center} } 

 \begin{center}\includegraphics[width=0.9\textwidth]{fig/DR_1975_25_dest_retraite.pdf}\end{center} \label{fig/DR_1975_25_dest_retraite.pdf} 

\newpage 
 
\subsection{Génération 1980 (début en 2005)} 

\paragraph{Retraites possibles et ratios Revenu/SMIC à 70, 75, 80, 85, 90 ans avec le modèle \emph{Gouvernement truqué (âge-pivot bloqué à 65 ans)}}  
 
{ \scriptsize \begin{center} 
\begin{tabular}[htb]{|c|c||c|c||c|c||c||c|c|c|c|c|c|} 
\hline 
 Retraite en &  Âge &  Âge pivot &  Décote/Surcote &  Retraite (\euro{} 2019) &  Tx Rempl(\%) &  SMIC (\euro{} 2019) &  Retraite/SMIC &  Rev70/SMIC &  Rev75/SMIC &  Rev80/SMIC &  Rev85/SMIC &  Rev90/SMIC \\ 
\hline \hline 
 2042 &  62 &  65 ans 0 mois &  -15.00\% &  1886.33 &  {\bf 36.74} &  2285.97 &  {\bf {\color{red} 0.83}} &  {\bf {\color{red} 0.74}} &  {\bf {\color{red} 0.70}} &  {\bf {\color{red} 0.65}} &  {\bf {\color{red} 0.61}} &  {\bf {\color{red} 0.57}} \\ 
\hline 
 2043 &  63 &  65 ans 0 mois &  -10.00\% &  2084.64 &  {\bf 40.51} &  2315.68 &  {\bf {\color{red} 0.90}} &  {\bf {\color{red} 0.82}} &  {\bf {\color{red} 0.77}} &  {\bf {\color{red} 0.72}} &  {\bf {\color{red} 0.68}} &  {\bf {\color{red} 0.64}} \\ 
\hline 
 2044 &  64 &  65 ans 0 mois &  -5.00\% &  2295.56 &  {\bf 44.51} &  2345.79 &  {\bf {\color{red} 0.98}} &  {\bf {\color{red} 0.91}} &  {\bf {\color{red} 0.85}} &  {\bf {\color{red} 0.80}} &  {\bf {\color{red} 0.75}} &  {\bf {\color{red} 0.70}} \\ 
\hline 
 2045 &  65 &  65 ans 0 mois &  0.00\% &  2519.73 &  {\bf 48.76} &  2376.28 &  {\bf 1.06} &  {\bf {\color{red} 0.99}} &  {\bf {\color{red} 0.93}} &  {\bf {\color{red} 0.87}} &  {\bf {\color{red} 0.82}} &  {\bf {\color{red} 0.77}} \\ 
\hline 
 2046 &  66 &  65 ans 0 mois &  5.00\% &  2755.81 &  {\bf 53.21} &  2407.18 &  {\bf 1.14} &  {\bf 1.09} &  {\bf 1.02} &  {\bf {\color{red} 0.96}} &  {\bf {\color{red} 0.90}} &  {\bf {\color{red} 0.84}} \\ 
\hline 
 2047 &  67 &  65 ans 0 mois &  10.00\% &  3004.04 &  {\bf 57.88} &  2438.47 &  {\bf 1.23} &  {\bf 1.19} &  {\bf 1.11} &  {\bf 1.04} &  {\bf {\color{red} 0.98}} &  {\bf {\color{red} 0.92}} \\ 
\hline 
\hline 
\end{tabular} 
\end{center} } 
\paragraph{Retraites possibles et ratios Revenu/SMIC à 70, 75, 80, 85, 90 ans avec le modèle \emph{Gouvernement corrigé (âge-pivot glissant)}}  
 
{ \scriptsize \begin{center} 
\begin{tabular}[htb]{|c|c||c|c||c|c||c||c|c|c|c|c|c|} 
\hline 
 Retraite en &  Âge &  Âge pivot &  Décote/Surcote &  Retraite (\euro{} 2019) &  Tx Rempl(\%) &  SMIC (\euro{} 2019) &  Retraite/SMIC &  Rev70/SMIC &  Rev75/SMIC &  Rev80/SMIC &  Rev85/SMIC &  Rev90/SMIC \\ 
\hline \hline 
 2042 &  62 &  65 ans 3 mois &  -16.25\% &  1858.59 &  {\bf 36.20} &  2285.97 &  {\bf {\color{red} 0.81}} &  {\bf {\color{red} 0.73}} &  {\bf {\color{red} 0.69}} &  {\bf {\color{red} 0.64}} &  {\bf {\color{red} 0.60}} &  {\bf {\color{red} 0.57}} \\ 
\hline 
 2043 &  63 &  65 ans 4 mois &  -11.67\% &  2046.04 &  {\bf 39.76} &  2315.68 &  {\bf {\color{red} 0.88}} &  {\bf {\color{red} 0.81}} &  {\bf {\color{red} 0.76}} &  {\bf {\color{red} 0.71}} &  {\bf {\color{red} 0.67}} &  {\bf {\color{red} 0.62}} \\ 
\hline 
 2044 &  64 &  65 ans 5 mois &  -7.08\% &  2245.22 &  {\bf 43.54} &  2345.79 &  {\bf {\color{red} 0.96}} &  {\bf {\color{red} 0.89}} &  {\bf {\color{red} 0.83}} &  {\bf {\color{red} 0.78}} &  {\bf {\color{red} 0.73}} &  {\bf {\color{red} 0.68}} \\ 
\hline 
 2045 &  65 &  65 ans 6 mois &  -2.50\% &  2456.74 &  {\bf 47.54} &  2376.28 &  {\bf 1.03} &  {\bf {\color{red} 0.97}} &  {\bf {\color{red} 0.91}} &  {\bf {\color{red} 0.85}} &  {\bf {\color{red} 0.80}} &  {\bf {\color{red} 0.75}} \\ 
\hline 
 2046 &  66 &  65 ans 7 mois &  2.08\% &  2679.26 &  {\bf 51.73} &  2407.18 &  {\bf 1.11} &  {\bf 1.06} &  {\bf {\color{red} 0.99}} &  {\bf {\color{red} 0.93}} &  {\bf {\color{red} 0.87}} &  {\bf {\color{red} 0.82}} \\ 
\hline 
 2047 &  67 &  65 ans 8 mois &  6.67\% &  2913.01 &  {\bf 56.12} &  2438.47 &  {\bf 1.19} &  {\bf 1.15} &  {\bf 1.08} &  {\bf 1.01} &  {\bf {\color{red} 0.95}} &  {\bf {\color{red} 0.89}} \\ 
\hline 
\hline 
\end{tabular} 
\end{center} } 
\paragraph{Retraites possibles et ratios Revenu/SMIC à 70, 75, 80, 85, 90 ans avec le modèle \emph{Destinie2 (revalorisation de la fonction publique)}}  
 
{ \scriptsize \begin{center} 
\begin{tabular}[htb]{|c|c||c|c||c|c||c||c|c|c|c|c|c|} 
\hline 
 Retraite en &  Âge &  Âge pivot &  Décote/Surcote &  Retraite (\euro{} 2019) &  Tx Rempl(\%) &  SMIC (\euro{} 2019) &  Retraite/SMIC &  Rev70/SMIC &  Rev75/SMIC &  Rev80/SMIC &  Rev85/SMIC &  Rev90/SMIC \\ 
\hline \hline 
 2042 &  62 &  65 ans 3 mois &  -16.25\% &  2165.61 &  {\bf 31.48} &  2149.23 &  {\bf 1.01} &  {\bf {\color{red} 0.91}} &  {\bf {\color{red} 0.85}} &  {\bf {\color{red} 0.80}} &  {\bf {\color{red} 0.75}} &  {\bf {\color{red} 0.70}} \\ 
\hline 
 2043 &  63 &  65 ans 4 mois &  -11.67\% &  2396.00 &  {\bf 34.38} &  2177.17 &  {\bf 1.10} &  {\bf 1.01} &  {\bf {\color{red} 0.94}} &  {\bf {\color{red} 0.88}} &  {\bf {\color{red} 0.83}} &  {\bf {\color{red} 0.78}} \\ 
\hline 
 2044 &  64 &  65 ans 5 mois &  -7.08\% &  2642.46 &  {\bf 37.43} &  2205.48 &  {\bf 1.20} &  {\bf 1.11} &  {\bf 1.04} &  {\bf {\color{red} 0.97}} &  {\bf {\color{red} 0.91}} &  {\bf {\color{red} 0.86}} \\ 
\hline 
 2045 &  65 &  65 ans 6 mois &  -2.50\% &  2905.91 &  {\bf 40.63} &  2234.15 &  {\bf 1.30} &  {\bf 1.22} &  {\bf 1.14} &  {\bf 1.07} &  {\bf 1.00} &  {\bf {\color{red} 0.94}} \\ 
\hline 
 2046 &  66 &  65 ans 7 mois &  2.08\% &  3185.01 &  {\bf 43.96} &  2263.19 &  {\bf 1.41} &  {\bf 1.34} &  {\bf 1.25} &  {\bf 1.17} &  {\bf 1.10} &  {\bf 1.03} \\ 
\hline 
 2047 &  67 &  65 ans 8 mois &  6.67\% &  3480.24 &  {\bf 47.42} &  2292.61 &  {\bf 1.52} &  {\bf 1.46} &  {\bf 1.37} &  {\bf 1.28} &  {\bf 1.20} &  {\bf 1.13} \\ 
\hline 
\hline 
\end{tabular} 
\end{center} } 

 \begin{center}\includegraphics[width=0.9\textwidth]{fig/DR_1980_25_dest_retraite.pdf}\end{center} \label{fig/DR_1980_25_dest_retraite.pdf} 

\newpage 
 
\subsection{Génération 1990 (début en 2015)} 

\paragraph{Retraites possibles et ratios Revenu/SMIC à 70, 75, 80, 85, 90 ans avec le modèle \emph{Gouvernement truqué (âge-pivot bloqué à 65 ans)}}  
 
{ \scriptsize \begin{center} 
\begin{tabular}[htb]{|c|c||c|c||c|c||c||c|c|c|c|c|c|} 
\hline 
 Retraite en &  Âge &  Âge pivot &  Décote/Surcote &  Retraite (\euro{} 2019) &  Tx Rempl(\%) &  SMIC (\euro{} 2019) &  Retraite/SMIC &  Rev70/SMIC &  Rev75/SMIC &  Rev80/SMIC &  Rev85/SMIC &  Rev90/SMIC \\ 
\hline \hline 
 2052 &  62 &  65 ans 0 mois &  -15.00\% &  2024.10 &  {\bf 39.42} &  2601.14 &  {\bf {\color{red} 0.78}} &  {\bf {\color{red} 0.70}} &  {\bf {\color{red} 0.66}} &  {\bf {\color{red} 0.62}} &  {\bf {\color{red} 0.58}} &  {\bf {\color{red} 0.54}} \\ 
\hline 
 2053 &  63 &  65 ans 0 mois &  -10.00\% &  2235.48 &  {\bf 43.44} &  2634.96 &  {\bf {\color{red} 0.85}} &  {\bf {\color{red} 0.78}} &  {\bf {\color{red} 0.73}} &  {\bf {\color{red} 0.68}} &  {\bf {\color{red} 0.64}} &  {\bf {\color{red} 0.60}} \\ 
\hline 
 2054 &  64 &  65 ans 0 mois &  -5.00\% &  2458.54 &  {\bf 47.67} &  2669.21 &  {\bf {\color{red} 0.92}} &  {\bf {\color{red} 0.85}} &  {\bf {\color{red} 0.80}} &  {\bf {\color{red} 0.75}} &  {\bf {\color{red} 0.70}} &  {\bf {\color{red} 0.66}} \\ 
\hline 
 2055 &  65 &  65 ans 0 mois &  0.00\% &  2693.52 &  {\bf 52.12} &  2703.91 &  {\bf {\color{red} 1.00}} &  {\bf {\color{red} 0.93}} &  {\bf {\color{red} 0.88}} &  {\bf {\color{red} 0.82}} &  {\bf {\color{red} 0.77}} &  {\bf {\color{red} 0.72}} \\ 
\hline 
 2056 &  66 &  65 ans 0 mois &  5.00\% &  2940.66 &  {\bf 56.78} &  2739.06 &  {\bf 1.07} &  {\bf 1.02} &  {\bf {\color{red} 0.96}} &  {\bf {\color{red} 0.90}} &  {\bf {\color{red} 0.84}} &  {\bf {\color{red} 0.79}} \\ 
\hline 
 2057 &  67 &  65 ans 0 mois &  10.00\% &  3200.21 &  {\bf 61.66} &  2774.67 &  {\bf 1.15} &  {\bf 1.11} &  {\bf 1.04} &  {\bf {\color{red} 0.98}} &  {\bf {\color{red} 0.91}} &  {\bf {\color{red} 0.86}} \\ 
\hline 
\hline 
\end{tabular} 
\end{center} } 
\paragraph{Retraites possibles et ratios Revenu/SMIC à 70, 75, 80, 85, 90 ans avec le modèle \emph{Gouvernement corrigé (âge-pivot glissant)}}  
 
{ \scriptsize \begin{center} 
\begin{tabular}[htb]{|c|c||c|c||c|c||c||c|c|c|c|c|c|} 
\hline 
 Retraite en &  Âge &  Âge pivot &  Décote/Surcote &  Retraite (\euro{} 2019) &  Tx Rempl(\%) &  SMIC (\euro{} 2019) &  Retraite/SMIC &  Rev70/SMIC &  Rev75/SMIC &  Rev80/SMIC &  Rev85/SMIC &  Rev90/SMIC \\ 
\hline \hline 
 2052 &  62 &  66 ans 1 mois &  -20.42\% &  1895.11 &  {\bf 36.91} &  2601.14 &  {\bf {\color{red} 0.73}} &  {\bf {\color{red} 0.66}} &  {\bf {\color{red} 0.62}} &  {\bf {\color{red} 0.58}} &  {\bf {\color{red} 0.54}} &  {\bf {\color{red} 0.51}} \\ 
\hline 
 2053 &  63 &  66 ans 2 mois &  -15.83\% &  2090.59 &  {\bf 40.63} &  2634.96 &  {\bf {\color{red} 0.79}} &  {\bf {\color{red} 0.72}} &  {\bf {\color{red} 0.68}} &  {\bf {\color{red} 0.64}} &  {\bf {\color{red} 0.60}} &  {\bf {\color{red} 0.56}} \\ 
\hline 
 2054 &  64 &  66 ans 3 mois &  -11.25\% &  2296.80 &  {\bf 44.54} &  2669.21 &  {\bf {\color{red} 0.86}} &  {\bf {\color{red} 0.80}} &  {\bf {\color{red} 0.75}} &  {\bf {\color{red} 0.70}} &  {\bf {\color{red} 0.66}} &  {\bf {\color{red} 0.62}} \\ 
\hline 
 2055 &  65 &  66 ans 4 mois &  -6.67\% &  2513.95 &  {\bf 48.64} &  2703.91 &  {\bf {\color{red} 0.93}} &  {\bf {\color{red} 0.87}} &  {\bf {\color{red} 0.82}} &  {\bf {\color{red} 0.77}} &  {\bf {\color{red} 0.72}} &  {\bf {\color{red} 0.67}} \\ 
\hline 
 2056 &  66 &  66 ans 5 mois &  -2.08\% &  2742.28 &  {\bf 52.95} &  2739.06 &  {\bf 1.00} &  {\bf {\color{red} 0.95}} &  {\bf {\color{red} 0.89}} &  {\bf {\color{red} 0.84}} &  {\bf {\color{red} 0.78}} &  {\bf {\color{red} 0.73}} \\ 
\hline 
 2057 &  67 &  66 ans 6 mois &  2.50\% &  2982.02 &  {\bf 57.45} &  2774.67 &  {\bf 1.07} &  {\bf 1.03} &  {\bf {\color{red} 0.97}} &  {\bf {\color{red} 0.91}} &  {\bf {\color{red} 0.85}} &  {\bf {\color{red} 0.80}} \\ 
\hline 
\hline 
\end{tabular} 
\end{center} } 
\paragraph{Retraites possibles et ratios Revenu/SMIC à 70, 75, 80, 85, 90 ans avec le modèle \emph{Destinie2 (revalorisation de la fonction publique)}}  
 
{ \scriptsize \begin{center} 
\begin{tabular}[htb]{|c|c||c|c||c|c||c||c|c|c|c|c|c|} 
\hline 
 Retraite en &  Âge &  Âge pivot &  Décote/Surcote &  Retraite (\euro{} 2019) &  Tx Rempl(\%) &  SMIC (\euro{} 2019) &  Retraite/SMIC &  Rev70/SMIC &  Rev75/SMIC &  Rev80/SMIC &  Rev85/SMIC &  Rev90/SMIC \\ 
\hline \hline 
 2052 &  62 &  66 ans 1 mois &  -20.42\% &  2449.16 &  {\bf 31.28} &  2445.56 &  {\bf 1.00} &  {\bf {\color{red} 0.90}} &  {\bf {\color{red} 0.85}} &  {\bf {\color{red} 0.79}} &  {\bf {\color{red} 0.74}} &  {\bf {\color{red} 0.70}} \\ 
\hline 
 2053 &  63 &  66 ans 2 mois &  -15.83\% &  2716.79 &  {\bf 34.26} &  2477.35 &  {\bf 1.10} &  {\bf 1.00} &  {\bf {\color{red} 0.94}} &  {\bf {\color{red} 0.88}} &  {\bf {\color{red} 0.83}} &  {\bf {\color{red} 0.77}} \\ 
\hline 
 2054 &  64 &  66 ans 3 mois &  -11.25\% &  3001.22 &  {\bf 37.36} &  2509.56 &  {\bf 1.20} &  {\bf 1.11} &  {\bf 1.04} &  {\bf {\color{red} 0.97}} &  {\bf {\color{red} 0.91}} &  {\bf {\color{red} 0.85}} \\ 
\hline 
 2055 &  65 &  66 ans 4 mois &  -6.67\% &  3302.97 &  {\bf 40.59} &  2542.18 &  {\bf 1.30} &  {\bf 1.22} &  {\bf 1.14} &  {\bf 1.07} &  {\bf 1.00} &  {\bf {\color{red} 0.94}} \\ 
\hline 
 2056 &  66 &  66 ans 5 mois &  -2.08\% &  3622.57 &  {\bf 43.94} &  2575.23 &  {\bf 1.41} &  {\bf 1.34} &  {\bf 1.25} &  {\bf 1.17} &  {\bf 1.10} &  {\bf 1.03} \\ 
\hline 
 2057 &  67 &  66 ans 6 mois &  2.50\% &  3960.58 &  {\bf 47.43} &  2608.71 &  {\bf 1.52} &  {\bf 1.46} &  {\bf 1.37} &  {\bf 1.28} &  {\bf 1.20} &  {\bf 1.13} \\ 
\hline 
\hline 
\end{tabular} 
\end{center} } 

 \begin{center}\includegraphics[width=0.9\textwidth]{fig/DR_1990_25_dest_retraite.pdf}\end{center} \label{fig/DR_1990_25_dest_retraite.pdf} 

\newpage 
 
\subsection{Génération 2003 (début en 2028)} 

\paragraph{Retraites possibles et ratios Revenu/SMIC à 70, 75, 80, 85, 90 ans avec le modèle \emph{Gouvernement truqué (âge-pivot bloqué à 65 ans)}}  
 
{ \scriptsize \begin{center} 
\begin{tabular}[htb]{|c|c||c|c||c|c||c||c|c|c|c|c|c|} 
\hline 
 Retraite en &  Âge &  Âge pivot &  Décote/Surcote &  Retraite (\euro{} 2019) &  Tx Rempl(\%) &  SMIC (\euro{} 2019) &  Retraite/SMIC &  Rev70/SMIC &  Rev75/SMIC &  Rev80/SMIC &  Rev85/SMIC &  Rev90/SMIC \\ 
\hline \hline 
 2065 &  62 &  65 ans 0 mois &  -15.00\% &  2134.16 &  {\bf 41.56} &  3076.71 &  {\bf {\color{red} 0.69}} &  {\bf {\color{red} 0.63}} &  {\bf {\color{red} 0.59}} &  {\bf {\color{red} 0.55}} &  {\bf {\color{red} 0.52}} &  {\bf {\color{red} 0.48}} \\ 
\hline 
 2066 &  63 &  65 ans 0 mois &  -10.00\% &  2353.54 &  {\bf 45.74} &  3116.71 &  {\bf {\color{red} 0.76}} &  {\bf {\color{red} 0.69}} &  {\bf {\color{red} 0.65}} &  {\bf {\color{red} 0.61}} &  {\bf {\color{red} 0.57}} &  {\bf {\color{red} 0.53}} \\ 
\hline 
 2067 &  64 &  65 ans 0 mois &  -5.00\% &  2584.78 &  {\bf 50.12} &  3157.23 &  {\bf {\color{red} 0.82}} &  {\bf {\color{red} 0.76}} &  {\bf {\color{red} 0.71}} &  {\bf {\color{red} 0.67}} &  {\bf {\color{red} 0.62}} &  {\bf {\color{red} 0.59}} \\ 
\hline 
 2068 &  65 &  65 ans 0 mois &  0.00\% &  2828.13 &  {\bf 54.72} &  3198.27 &  {\bf {\color{red} 0.88}} &  {\bf {\color{red} 0.83}} &  {\bf {\color{red} 0.78}} &  {\bf {\color{red} 0.73}} &  {\bf {\color{red} 0.68}} &  {\bf {\color{red} 0.64}} \\ 
\hline 
 2069 &  66 &  65 ans 0 mois &  5.00\% &  3083.84 &  {\bf 59.54} &  3239.85 &  {\bf {\color{red} 0.95}} &  {\bf {\color{red} 0.90}} &  {\bf {\color{red} 0.85}} &  {\bf {\color{red} 0.79}} &  {\bf {\color{red} 0.74}} &  {\bf {\color{red} 0.70}} \\ 
\hline 
 2070 &  67 &  65 ans 0 mois &  10.00\% &  3352.16 &  {\bf 64.58} &  3281.97 &  {\bf 1.02} &  {\bf {\color{red} 0.98}} &  {\bf {\color{red} 0.92}} &  {\bf {\color{red} 0.86}} &  {\bf {\color{red} 0.81}} &  {\bf {\color{red} 0.76}} \\ 
\hline 
\hline 
\end{tabular} 
\end{center} } 
\paragraph{Retraites possibles et ratios Revenu/SMIC à 70, 75, 80, 85, 90 ans avec le modèle \emph{Gouvernement corrigé (âge-pivot glissant)}}  
 
{ \scriptsize \begin{center} 
\begin{tabular}[htb]{|c|c||c|c||c|c||c||c|c|c|c|c|c|} 
\hline 
 Retraite en &  Âge &  Âge pivot &  Décote/Surcote &  Retraite (\euro{} 2019) &  Tx Rempl(\%) &  SMIC (\euro{} 2019) &  Retraite/SMIC &  Rev70/SMIC &  Rev75/SMIC &  Rev80/SMIC &  Rev85/SMIC &  Rev90/SMIC \\ 
\hline \hline 
 2065 &  62 &  67 ans 2 mois &  -25.83\% &  1862.16 &  {\bf 36.27} &  3076.71 &  {\bf {\color{red} 0.61}} &  {\bf {\color{red} 0.55}} &  {\bf {\color{red} 0.51}} &  {\bf {\color{red} 0.48}} &  {\bf {\color{red} 0.45}} &  {\bf {\color{red} 0.42}} \\ 
\hline 
 2066 &  63 &  67 ans 3 mois &  -21.25\% &  2059.35 &  {\bf 40.02} &  3116.71 &  {\bf {\color{red} 0.66}} &  {\bf {\color{red} 0.60}} &  {\bf {\color{red} 0.57}} &  {\bf {\color{red} 0.53}} &  {\bf {\color{red} 0.50}} &  {\bf {\color{red} 0.47}} \\ 
\hline 
 2067 &  64 &  67 ans 4 mois &  -16.67\% &  2267.35 &  {\bf 43.97} &  3157.23 &  {\bf {\color{red} 0.72}} &  {\bf {\color{red} 0.66}} &  {\bf {\color{red} 0.62}} &  {\bf {\color{red} 0.58}} &  {\bf {\color{red} 0.55}} &  {\bf {\color{red} 0.51}} \\ 
\hline 
 2068 &  65 &  67 ans 5 mois &  -12.08\% &  2486.40 &  {\bf 48.11} &  3198.27 &  {\bf {\color{red} 0.78}} &  {\bf {\color{red} 0.73}} &  {\bf {\color{red} 0.68}} &  {\bf {\color{red} 0.64}} &  {\bf {\color{red} 0.60}} &  {\bf {\color{red} 0.56}} \\ 
\hline 
 2069 &  66 &  67 ans 6 mois &  -7.50\% &  2716.71 &  {\bf 52.45} &  3239.85 &  {\bf {\color{red} 0.84}} &  {\bf {\color{red} 0.80}} &  {\bf {\color{red} 0.75}} &  {\bf {\color{red} 0.70}} &  {\bf {\color{red} 0.66}} &  {\bf {\color{red} 0.62}} \\ 
\hline 
 2070 &  67 &  67 ans 7 mois &  -2.92\% &  2958.53 &  {\bf 57.00} &  3281.97 &  {\bf {\color{red} 0.90}} &  {\bf {\color{red} 0.87}} &  {\bf {\color{red} 0.81}} &  {\bf {\color{red} 0.76}} &  {\bf {\color{red} 0.71}} &  {\bf {\color{red} 0.67}} \\ 
\hline 
\hline 
\end{tabular} 
\end{center} } 
\paragraph{Retraites possibles et ratios Revenu/SMIC à 70, 75, 80, 85, 90 ans avec le modèle \emph{Destinie2 (revalorisation de la fonction publique)}}  
 
{ \scriptsize \begin{center} 
\begin{tabular}[htb]{|c|c||c|c||c|c||c||c|c|c|c|c|c|} 
\hline 
 Retraite en &  Âge &  Âge pivot &  Décote/Surcote &  Retraite (\euro{} 2019) &  Tx Rempl(\%) &  SMIC (\euro{} 2019) &  Retraite/SMIC &  Rev70/SMIC &  Rev75/SMIC &  Rev80/SMIC &  Rev85/SMIC &  Rev90/SMIC \\ 
\hline \hline 
 2065 &  62 &  67 ans 2 mois &  -25.83\% &  2819.08 &  {\bf 30.44} &  2892.68 &  {\bf {\color{red} 0.97}} &  {\bf {\color{red} 0.88}} &  {\bf {\color{red} 0.82}} &  {\bf {\color{red} 0.77}} &  {\bf {\color{red} 0.72}} &  {\bf {\color{red} 0.68}} \\ 
\hline 
 2066 &  63 &  67 ans 3 mois &  -21.25\% &  3135.02 &  {\bf 33.42} &  2930.29 &  {\bf 1.07} &  {\bf {\color{red} 0.98}} &  {\bf {\color{red} 0.92}} &  {\bf {\color{red} 0.86}} &  {\bf {\color{red} 0.81}} &  {\bf {\color{red} 0.75}} \\ 
\hline 
 2067 &  64 &  67 ans 4 mois &  -16.67\% &  3470.84 &  {\bf 36.53} &  2968.38 &  {\bf 1.17} &  {\bf 1.08} &  {\bf 1.01} &  {\bf {\color{red} 0.95}} &  {\bf {\color{red} 0.89}} &  {\bf {\color{red} 0.84}} \\ 
\hline 
 2068 &  65 &  67 ans 5 mois &  -12.08\% &  3827.13 &  {\bf 39.76} &  3006.97 &  {\bf 1.27} &  {\bf 1.19} &  {\bf 1.12} &  {\bf 1.05} &  {\bf {\color{red} 0.98}} &  {\bf {\color{red} 0.92}} \\ 
\hline 
 2069 &  66 &  67 ans 6 mois &  -7.50\% &  4204.54 &  {\bf 43.12} &  3046.06 &  {\bf 1.38} &  {\bf 1.31} &  {\bf 1.23} &  {\bf 1.15} &  {\bf 1.08} &  {\bf 1.01} \\ 
\hline 
 2070 &  67 &  67 ans 7 mois &  -2.92\% &  4603.73 &  {\bf 46.61} &  3085.66 &  {\bf 1.49} &  {\bf 1.44} &  {\bf 1.35} &  {\bf 1.26} &  {\bf 1.18} &  {\bf 1.11} \\ 
\hline 
\hline 
\end{tabular} 
\end{center} } 

 \begin{center}\includegraphics[width=0.9\textwidth]{fig/DR_2003_25_dest_retraite.pdf}\end{center} \label{fig/DR_2003_25_dest_retraite.pdf} 

\newpage 
 
\chapter{Magistrat (second puis premier grade)} 

\begin{minipage}{0.55\linewidth}\includegraphics[width=0.7\textwidth]{fig/grille_Magistrat.pdf}\end{minipage} 
\begin{minipage}{0.3\linewidth} 
 \begin{center} 

\begin{tabular}[htb]{|c|c|} 
\hline 
 Indice majoré &  Durée (années) \\ 
\hline \hline 
 461 &  1.00 \\ 
\hline 
 505 &  1.00 \\ 
\hline 
 555 &  2.00 \\ 
\hline 
 591 &  2.00 \\ 
\hline 
 667 &  1.50 \\ 
\hline 
 705 &  1.50 \\ 
\hline 
 743 &  1.50 \\ 
\hline 
 792 &  1.50 \\ 
\hline 
 830 &  2.00 \\ 
\hline 
 890 &  1.00 \\ 
\hline 
 925 &  1.00 \\ 
\hline 
 972 &  1.00 \\ 
\hline 
 972 &  1.00 \\ 
\hline 
 1013 &  1.00 \\ 
\hline 
 1067 &  1.00 \\ 
\hline 
 1067 &  1.00 \\ 
\hline 
 1095 &  1.00 \\ 
\hline 
 1124 &   \\ 
\hline 
\hline 
\end{tabular} 
\end{center} 
 \end{minipage} 


 \addto{\captionsenglish}{ \renewcommand{\mtctitle}{}} \setcounter{minitocdepth}{2} 
 \minitoc \newpage 

\section{Début de carrière à 22 ans} 

\subsection{Génération 1975 (début en 1997)} 

\paragraph{Retraites possibles et ratios Revenu/SMIC à 70, 75, 80, 85, 90 ans avec le modèle \emph{Gouvernement truqué (âge-pivot bloqué à 65 ans)}}  
 
{ \scriptsize \begin{center} 
\begin{tabular}[htb]{|c|c||c|c||c|c||c||c|c|c|c|c|c|} 
\hline 
 Retraite en &  Âge &  Âge pivot &  Décote/Surcote &  Retraite (\euro{} 2019) &  Tx Rempl(\%) &  SMIC (\euro{} 2019) &  Retraite/SMIC &  Rev70/SMIC &  Rev75/SMIC &  Rev80/SMIC &  Rev85/SMIC &  Rev90/SMIC \\ 
\hline \hline 
 2037 &  62 &  64 ans 10 mois &  -14.17\% &  3515.50 &  {\bf 42.36} &  2143.00 &  {\bf 1.64} &  {\bf 1.48} &  {\bf 1.39} &  {\bf 1.30} &  {\bf 1.22} &  {\bf 1.14} \\ 
\hline 
 2038 &  63 &  64 ans 11 mois &  -9.58\% &  3836.07 &  {\bf 46.15} &  2170.86 &  {\bf 1.77} &  {\bf 1.61} &  {\bf 1.51} &  {\bf 1.42} &  {\bf 1.33} &  {\bf 1.25} \\ 
\hline 
 2039 &  64 &  65 ans 0 mois &  -5.00\% &  4174.43 &  {\bf 50.15} &  2199.08 &  {\bf 1.90} &  {\bf 1.76} &  {\bf 1.65} &  {\bf 1.54} &  {\bf 1.45} &  {\bf 1.36} \\ 
\hline 
 2040 &  65 &  65 ans 0 mois &  0.00\% &  4550.42 &  {\bf 54.58} &  2227.67 &  {\bf 2.04} &  {\bf 1.91} &  {\bf 1.80} &  {\bf 1.68} &  {\bf 1.58} &  {\bf 1.48} \\ 
\hline 
 2041 &  66 &  65 ans 0 mois &  5.00\% &  4947.39 &  {\bf 59.26} &  2256.63 &  {\bf 2.19} &  {\bf 2.08} &  {\bf 1.95} &  {\bf 1.83} &  {\bf 1.72} &  {\bf 1.61} \\ 
\hline 
 2042 &  67 &  65 ans 0 mois &  10.00\% &  5366.43 &  {\bf 64.18} &  2285.97 &  {\bf 2.35} &  {\bf 2.26} &  {\bf 2.12} &  {\bf 1.98} &  {\bf 1.86} &  {\bf 1.74} \\ 
\hline 
\hline 
\end{tabular} 
\end{center} } 
\paragraph{Retraites possibles et ratios Revenu/SMIC à 70, 75, 80, 85, 90 ans avec le modèle \emph{Gouvernement corrigé (âge-pivot glissant)}}  
 
{ \scriptsize \begin{center} 
\begin{tabular}[htb]{|c|c||c|c||c|c||c||c|c|c|c|c|c|} 
\hline 
 Retraite en &  Âge &  Âge pivot &  Décote/Surcote &  Retraite (\euro{} 2019) &  Tx Rempl(\%) &  SMIC (\euro{} 2019) &  Retraite/SMIC &  Rev70/SMIC &  Rev75/SMIC &  Rev80/SMIC &  Rev85/SMIC &  Rev90/SMIC \\ 
\hline \hline 
 2037 &  62 &  64 ans 10 mois &  -14.17\% &  3515.50 &  {\bf 42.36} &  2143.00 &  {\bf 1.64} &  {\bf 1.48} &  {\bf 1.39} &  {\bf 1.30} &  {\bf 1.22} &  {\bf 1.14} \\ 
\hline 
 2038 &  63 &  64 ans 11 mois &  -9.58\% &  3836.07 &  {\bf 46.15} &  2170.86 &  {\bf 1.77} &  {\bf 1.61} &  {\bf 1.51} &  {\bf 1.42} &  {\bf 1.33} &  {\bf 1.25} \\ 
\hline 
 2039 &  64 &  65 ans 0 mois &  -5.00\% &  4174.43 &  {\bf 50.15} &  2199.08 &  {\bf 1.90} &  {\bf 1.76} &  {\bf 1.65} &  {\bf 1.54} &  {\bf 1.45} &  {\bf 1.36} \\ 
\hline 
 2040 &  65 &  65 ans 1 mois &  -0.42\% &  4531.46 &  {\bf 54.36} &  2227.67 &  {\bf 2.03} &  {\bf 1.91} &  {\bf 1.79} &  {\bf 1.68} &  {\bf 1.57} &  {\bf 1.47} \\ 
\hline 
 2041 &  66 &  65 ans 2 mois &  4.17\% &  4908.13 &  {\bf 58.79} &  2256.63 &  {\bf 2.17} &  {\bf 2.07} &  {\bf 1.94} &  {\bf 1.82} &  {\bf 1.70} &  {\bf 1.60} \\ 
\hline 
 2042 &  67 &  65 ans 3 mois &  8.75\% &  5305.45 &  {\bf 63.45} &  2285.97 &  {\bf 2.32} &  {\bf 2.23} &  {\bf 2.09} &  {\bf 1.96} &  {\bf 1.84} &  {\bf 1.72} \\ 
\hline 
\hline 
\end{tabular} 
\end{center} } 
\paragraph{Retraites possibles et ratios Revenu/SMIC à 70, 75, 80, 85, 90 ans avec le modèle \emph{Destinie2 (revalorisation de la fonction publique)}}  
 
{ \scriptsize \begin{center} 
\begin{tabular}[htb]{|c|c||c|c||c|c||c||c|c|c|c|c|c|} 
\hline 
 Retraite en &  Âge &  Âge pivot &  Décote/Surcote &  Retraite (\euro{} 2019) &  Tx Rempl(\%) &  SMIC (\euro{} 2019) &  Retraite/SMIC &  Rev70/SMIC &  Rev75/SMIC &  Rev80/SMIC &  Rev85/SMIC &  Rev90/SMIC \\ 
\hline \hline 
 2037 &  62 &  64 ans 10 mois &  -14.17\% &  3840.68 &  {\bf 37.15} &  2014.82 &  {\bf 1.91} &  {\bf 1.72} &  {\bf 1.61} &  {\bf 1.51} &  {\bf 1.42} &  {\bf 1.33} \\ 
\hline 
 2038 &  63 &  64 ans 11 mois &  -9.58\% &  4208.41 &  {\bf 40.19} &  2041.01 &  {\bf 2.06} &  {\bf 1.88} &  {\bf 1.77} &  {\bf 1.66} &  {\bf 1.55} &  {\bf 1.45} \\ 
\hline 
 2039 &  64 &  65 ans 0 mois &  -5.00\% &  4599.12 &  {\bf 43.36} &  2067.55 &  {\bf 2.22} &  {\bf 2.06} &  {\bf 1.93} &  {\bf 1.81} &  {\bf 1.70} &  {\bf 1.59} \\ 
\hline 
 2040 &  65 &  65 ans 1 mois &  -0.42\% &  5014.11 &  {\bf 46.66} &  2094.43 &  {\bf 2.39} &  {\bf 2.24} &  {\bf 2.10} &  {\bf 1.97} &  {\bf 1.85} &  {\bf 1.73} \\ 
\hline 
 2041 &  66 &  65 ans 2 mois &  4.17\% &  5454.78 &  {\bf 50.11} &  2121.65 &  {\bf 2.57} &  {\bf 2.44} &  {\bf 2.29} &  {\bf 2.15} &  {\bf 2.01} &  {\bf 1.89} \\ 
\hline 
 2042 &  67 &  65 ans 3 mois &  8.75\% &  5922.60 &  {\bf 53.71} &  2149.23 &  {\bf 2.76} &  {\bf 2.65} &  {\bf 2.49} &  {\bf 2.33} &  {\bf 2.18} &  {\bf 2.05} \\ 
\hline 
\hline 
\end{tabular} 
\end{center} } 

 \begin{center}\includegraphics[width=0.9\textwidth]{fig/Magistrat_1975_22_dest_retraite.pdf}\end{center} \label{fig/Magistrat_1975_22_dest_retraite.pdf} 

\newpage 
 
\subsection{Génération 1980 (début en 2002)} 

\paragraph{Retraites possibles et ratios Revenu/SMIC à 70, 75, 80, 85, 90 ans avec le modèle \emph{Gouvernement truqué (âge-pivot bloqué à 65 ans)}}  
 
{ \scriptsize \begin{center} 
\begin{tabular}[htb]{|c|c||c|c||c|c||c||c|c|c|c|c|c|} 
\hline 
 Retraite en &  Âge &  Âge pivot &  Décote/Surcote &  Retraite (\euro{} 2019) &  Tx Rempl(\%) &  SMIC (\euro{} 2019) &  Retraite/SMIC &  Rev70/SMIC &  Rev75/SMIC &  Rev80/SMIC &  Rev85/SMIC &  Rev90/SMIC \\ 
\hline \hline 
 2042 &  62 &  65 ans 0 mois &  -15.00\% &  3563.00 &  {\bf 42.93} &  2285.97 &  {\bf 1.56} &  {\bf 1.41} &  {\bf 1.32} &  {\bf 1.24} &  {\bf 1.16} &  {\bf 1.09} \\ 
\hline 
 2043 &  63 &  65 ans 0 mois &  -10.00\% &  3919.96 &  {\bf 47.16} &  2315.68 &  {\bf 1.69} &  {\bf 1.55} &  {\bf 1.45} &  {\bf 1.36} &  {\bf 1.27} &  {\bf 1.19} \\ 
\hline 
 2044 &  64 &  65 ans 0 mois &  -5.00\% &  4298.41 &  {\bf 51.64} &  2345.79 &  {\bf 1.83} &  {\bf 1.70} &  {\bf 1.59} &  {\bf 1.49} &  {\bf 1.40} &  {\bf 1.31} \\ 
\hline 
 2045 &  65 &  65 ans 0 mois &  0.00\% &  4699.51 &  {\bf 56.37} &  2376.28 &  {\bf 1.98} &  {\bf 1.85} &  {\bf 1.74} &  {\bf 1.63} &  {\bf 1.53} &  {\bf 1.43} \\ 
\hline 
 2046 &  66 &  65 ans 0 mois &  5.00\% &  5120.66 &  {\bf 61.33} &  2407.18 &  {\bf 2.13} &  {\bf 2.02} &  {\bf 1.89} &  {\bf 1.78} &  {\bf 1.66} &  {\bf 1.56} \\ 
\hline 
 2047 &  67 &  65 ans 0 mois &  10.00\% &  5562.26 &  {\bf 66.52} &  2438.47 &  {\bf 2.28} &  {\bf 2.19} &  {\bf 2.06} &  {\bf 1.93} &  {\bf 1.81} &  {\bf 1.69} \\ 
\hline 
\hline 
\end{tabular} 
\end{center} } 
\paragraph{Retraites possibles et ratios Revenu/SMIC à 70, 75, 80, 85, 90 ans avec le modèle \emph{Gouvernement corrigé (âge-pivot glissant)}}  
 
{ \scriptsize \begin{center} 
\begin{tabular}[htb]{|c|c||c|c||c|c||c||c|c|c|c|c|c|} 
\hline 
 Retraite en &  Âge &  Âge pivot &  Décote/Surcote &  Retraite (\euro{} 2019) &  Tx Rempl(\%) &  SMIC (\euro{} 2019) &  Retraite/SMIC &  Rev70/SMIC &  Rev75/SMIC &  Rev80/SMIC &  Rev85/SMIC &  Rev90/SMIC \\ 
\hline \hline 
 2042 &  62 &  65 ans 3 mois &  -16.25\% &  3510.60 &  {\bf 42.30} &  2285.97 &  {\bf 1.54} &  {\bf 1.38} &  {\bf 1.30} &  {\bf 1.22} &  {\bf 1.14} &  {\bf 1.07} \\ 
\hline 
 2043 &  63 &  65 ans 4 mois &  -11.67\% &  3847.36 &  {\bf 46.29} &  2315.68 &  {\bf 1.66} &  {\bf 1.52} &  {\bf 1.42} &  {\bf 1.33} &  {\bf 1.25} &  {\bf 1.17} \\ 
\hline 
 2044 &  64 &  65 ans 5 mois &  -7.08\% &  4204.15 &  {\bf 50.50} &  2345.79 &  {\bf 1.79} &  {\bf 1.66} &  {\bf 1.55} &  {\bf 1.46} &  {\bf 1.37} &  {\bf 1.28} \\ 
\hline 
 2045 &  65 &  65 ans 6 mois &  -2.50\% &  4582.02 &  {\bf 54.96} &  2376.28 &  {\bf 1.93} &  {\bf 1.81} &  {\bf 1.69} &  {\bf 1.59} &  {\bf 1.49} &  {\bf 1.40} \\ 
\hline 
 2046 &  66 &  65 ans 7 mois &  2.08\% &  4978.42 &  {\bf 59.63} &  2407.18 &  {\bf 2.07} &  {\bf 1.96} &  {\bf 1.84} &  {\bf 1.73} &  {\bf 1.62} &  {\bf 1.52} \\ 
\hline 
 2047 &  67 &  65 ans 8 mois &  6.67\% &  5393.71 &  {\bf 64.51} &  2438.47 &  {\bf 2.21} &  {\bf 2.13} &  {\bf 1.99} &  {\bf 1.87} &  {\bf 1.75} &  {\bf 1.64} \\ 
\hline 
\hline 
\end{tabular} 
\end{center} } 
\paragraph{Retraites possibles et ratios Revenu/SMIC à 70, 75, 80, 85, 90 ans avec le modèle \emph{Destinie2 (revalorisation de la fonction publique)}}  
 
{ \scriptsize \begin{center} 
\begin{tabular}[htb]{|c|c||c|c||c|c||c||c|c|c|c|c|c|} 
\hline 
 Retraite en &  Âge &  Âge pivot &  Décote/Surcote &  Retraite (\euro{} 2019) &  Tx Rempl(\%) &  SMIC (\euro{} 2019) &  Retraite/SMIC &  Rev70/SMIC &  Rev75/SMIC &  Rev80/SMIC &  Rev85/SMIC &  Rev90/SMIC \\ 
\hline \hline 
 2042 &  62 &  65 ans 3 mois &  -16.25\% &  3982.56 &  {\bf 36.12} &  2149.23 &  {\bf 1.85} &  {\bf 1.67} &  {\bf 1.57} &  {\bf 1.47} &  {\bf 1.38} &  {\bf 1.29} \\ 
\hline 
 2043 &  63 &  65 ans 4 mois &  -11.67\% &  4386.00 &  {\bf 39.27} &  2177.17 &  {\bf 2.01} &  {\bf 1.84} &  {\bf 1.73} &  {\bf 1.62} &  {\bf 1.52} &  {\bf 1.42} \\ 
\hline 
 2044 &  64 &  65 ans 5 mois &  -7.08\% &  4816.35 &  {\bf 42.56} &  2205.48 &  {\bf 2.18} &  {\bf 2.02} &  {\bf 1.89} &  {\bf 1.78} &  {\bf 1.67} &  {\bf 1.56} \\ 
\hline 
 2045 &  65 &  65 ans 6 mois &  -2.50\% &  5275.19 &  {\bf 46.02} &  2234.15 &  {\bf 2.36} &  {\bf 2.21} &  {\bf 2.08} &  {\bf 1.95} &  {\bf 1.82} &  {\bf 1.71} \\ 
\hline 
 2046 &  66 &  65 ans 7 mois &  2.08\% &  5759.96 &  {\bf 49.61} &  2263.19 &  {\bf 2.55} &  {\bf 2.42} &  {\bf 2.27} &  {\bf 2.12} &  {\bf 1.99} &  {\bf 1.87} \\ 
\hline 
 2047 &  67 &  65 ans 8 mois &  6.67\% &  6271.46 &  {\bf 53.32} &  2292.61 &  {\bf 2.74} &  {\bf 2.63} &  {\bf 2.47} &  {\bf 2.31} &  {\bf 2.17} &  {\bf 2.03} \\ 
\hline 
\hline 
\end{tabular} 
\end{center} } 

 \begin{center}\includegraphics[width=0.9\textwidth]{fig/Magistrat_1980_22_dest_retraite.pdf}\end{center} \label{fig/Magistrat_1980_22_dest_retraite.pdf} 

\newpage 
 
\subsection{Génération 1990 (début en 2012)} 

\paragraph{Retraites possibles et ratios Revenu/SMIC à 70, 75, 80, 85, 90 ans avec le modèle \emph{Gouvernement truqué (âge-pivot bloqué à 65 ans)}}  
 
{ \scriptsize \begin{center} 
\begin{tabular}[htb]{|c|c||c|c||c|c||c||c|c|c|c|c|c|} 
\hline 
 Retraite en &  Âge &  Âge pivot &  Décote/Surcote &  Retraite (\euro{} 2019) &  Tx Rempl(\%) &  SMIC (\euro{} 2019) &  Retraite/SMIC &  Rev70/SMIC &  Rev75/SMIC &  Rev80/SMIC &  Rev85/SMIC &  Rev90/SMIC \\ 
\hline \hline 
 2052 &  62 &  65 ans 0 mois &  -15.00\% &  3828.87 &  {\bf 46.13} &  2601.14 &  {\bf 1.47} &  {\bf 1.33} &  {\bf 1.24} &  {\bf 1.17} &  {\bf 1.09} &  {\bf 1.03} \\ 
\hline 
 2053 &  63 &  65 ans 0 mois &  -10.00\% &  4210.92 &  {\bf 50.66} &  2634.96 &  {\bf 1.60} &  {\bf 1.46} &  {\bf 1.37} &  {\bf 1.28} &  {\bf 1.20} &  {\bf 1.13} \\ 
\hline 
 2054 &  64 &  65 ans 0 mois &  -5.00\% &  4612.72 &  {\bf 55.41} &  2669.21 &  {\bf 1.73} &  {\bf 1.60} &  {\bf 1.50} &  {\bf 1.41} &  {\bf 1.32} &  {\bf 1.24} \\ 
\hline 
 2055 &  65 &  65 ans 0 mois &  0.00\% &  5034.66 &  {\bf 60.39} &  2703.91 &  {\bf 1.86} &  {\bf 1.75} &  {\bf 1.64} &  {\bf 1.53} &  {\bf 1.44} &  {\bf 1.35} \\ 
\hline 
 2056 &  66 &  65 ans 0 mois &  5.00\% &  5477.14 &  {\bf 65.60} &  2739.06 &  {\bf 2.00} &  {\bf 1.90} &  {\bf 1.78} &  {\bf 1.67} &  {\bf 1.56} &  {\bf 1.47} \\ 
\hline 
 2057 &  67 &  65 ans 0 mois &  10.00\% &  5940.58 &  {\bf 71.05} &  2774.67 &  {\bf 2.14} &  {\bf 2.06} &  {\bf 1.93} &  {\bf 1.81} &  {\bf 1.70} &  {\bf 1.59} \\ 
\hline 
\hline 
\end{tabular} 
\end{center} } 
\paragraph{Retraites possibles et ratios Revenu/SMIC à 70, 75, 80, 85, 90 ans avec le modèle \emph{Gouvernement corrigé (âge-pivot glissant)}}  
 
{ \scriptsize \begin{center} 
\begin{tabular}[htb]{|c|c||c|c||c|c||c||c|c|c|c|c|c|} 
\hline 
 Retraite en &  Âge &  Âge pivot &  Décote/Surcote &  Retraite (\euro{} 2019) &  Tx Rempl(\%) &  SMIC (\euro{} 2019) &  Retraite/SMIC &  Rev70/SMIC &  Rev75/SMIC &  Rev80/SMIC &  Rev85/SMIC &  Rev90/SMIC \\ 
\hline \hline 
 2052 &  62 &  66 ans 1 mois &  -20.42\% &  3584.87 &  {\bf 43.19} &  2601.14 &  {\bf 1.38} &  {\bf 1.24} &  {\bf 1.17} &  {\bf 1.09} &  {\bf 1.02} &  {\bf {\color{red} 0.96}} \\ 
\hline 
 2053 &  63 &  66 ans 2 mois &  -15.83\% &  3937.99 &  {\bf 47.38} &  2634.96 &  {\bf 1.49} &  {\bf 1.37} &  {\bf 1.28} &  {\bf 1.20} &  {\bf 1.12} &  {\bf 1.05} \\ 
\hline 
 2054 &  64 &  66 ans 3 mois &  -11.25\% &  4309.26 &  {\bf 51.77} &  2669.21 &  {\bf 1.61} &  {\bf 1.49} &  {\bf 1.40} &  {\bf 1.31} &  {\bf 1.23} &  {\bf 1.15} \\ 
\hline 
 2055 &  65 &  66 ans 4 mois &  -6.67\% &  4699.02 &  {\bf 56.37} &  2703.91 &  {\bf 1.74} &  {\bf 1.63} &  {\bf 1.53} &  {\bf 1.43} &  {\bf 1.34} &  {\bf 1.26} \\ 
\hline 
 2056 &  66 &  66 ans 5 mois &  -2.08\% &  5107.65 &  {\bf 61.18} &  2739.06 &  {\bf 1.86} &  {\bf 1.77} &  {\bf 1.66} &  {\bf 1.56} &  {\bf 1.46} &  {\bf 1.37} \\ 
\hline 
 2057 &  67 &  66 ans 6 mois &  2.50\% &  5535.54 &  {\bf 66.20} &  2774.67 &  {\bf 2.00} &  {\bf 1.92} &  {\bf 1.80} &  {\bf 1.69} &  {\bf 1.58} &  {\bf 1.48} \\ 
\hline 
\hline 
\end{tabular} 
\end{center} } 
\paragraph{Retraites possibles et ratios Revenu/SMIC à 70, 75, 80, 85, 90 ans avec le modèle \emph{Destinie2 (revalorisation de la fonction publique)}}  
 
{ \scriptsize \begin{center} 
\begin{tabular}[htb]{|c|c||c|c||c|c||c||c|c|c|c|c|c|} 
\hline 
 Retraite en &  Âge &  Âge pivot &  Décote/Surcote &  Retraite (\euro{} 2019) &  Tx Rempl(\%) &  SMIC (\euro{} 2019) &  Retraite/SMIC &  Rev70/SMIC &  Rev75/SMIC &  Rev80/SMIC &  Rev85/SMIC &  Rev90/SMIC \\ 
\hline \hline 
 2052 &  62 &  66 ans 1 mois &  -20.42\% &  4482.22 &  {\bf 35.72} &  2445.56 &  {\bf 1.83} &  {\bf 1.65} &  {\bf 1.55} &  {\bf 1.45} &  {\bf 1.36} &  {\bf 1.28} \\ 
\hline 
 2053 &  63 &  66 ans 2 mois &  -15.83\% &  4950.90 &  {\bf 38.95} &  2477.35 &  {\bf 2.00} &  {\bf 1.83} &  {\bf 1.71} &  {\bf 1.60} &  {\bf 1.50} &  {\bf 1.41} \\ 
\hline 
 2054 &  64 &  66 ans 3 mois &  -11.25\% &  5447.42 &  {\bf 42.31} &  2509.56 &  {\bf 2.17} &  {\bf 2.01} &  {\bf 1.88} &  {\bf 1.77} &  {\bf 1.65} &  {\bf 1.55} \\ 
\hline 
 2055 &  65 &  66 ans 4 mois &  -6.67\% &  5972.67 &  {\bf 45.79} &  2542.18 &  {\bf 2.35} &  {\bf 2.20} &  {\bf 2.06} &  {\bf 1.94} &  {\bf 1.81} &  {\bf 1.70} \\ 
\hline 
 2056 &  66 &  66 ans 5 mois &  -2.08\% &  6527.50 &  {\bf 49.40} &  2575.23 &  {\bf 2.53} &  {\bf 2.41} &  {\bf 2.26} &  {\bf 2.12} &  {\bf 1.98} &  {\bf 1.86} \\ 
\hline 
 2057 &  67 &  66 ans 6 mois &  2.50\% &  7112.83 &  {\bf 53.14} &  2608.71 &  {\bf 2.73} &  {\bf 2.62} &  {\bf 2.46} &  {\bf 2.31} &  {\bf 2.16} &  {\bf 2.03} \\ 
\hline 
\hline 
\end{tabular} 
\end{center} } 

 \begin{center}\includegraphics[width=0.9\textwidth]{fig/Magistrat_1990_22_dest_retraite.pdf}\end{center} \label{fig/Magistrat_1990_22_dest_retraite.pdf} 

\newpage 
 
\subsection{Génération 2003 (début en 2025)} 

\paragraph{Retraites possibles et ratios Revenu/SMIC à 70, 75, 80, 85, 90 ans avec le modèle \emph{Gouvernement truqué (âge-pivot bloqué à 65 ans)}}  
 
{ \scriptsize \begin{center} 
\begin{tabular}[htb]{|c|c||c|c||c|c||c||c|c|c|c|c|c|} 
\hline 
 Retraite en &  Âge &  Âge pivot &  Décote/Surcote &  Retraite (\euro{} 2019) &  Tx Rempl(\%) &  SMIC (\euro{} 2019) &  Retraite/SMIC &  Rev70/SMIC &  Rev75/SMIC &  Rev80/SMIC &  Rev85/SMIC &  Rev90/SMIC \\ 
\hline \hline 
 2065 &  62 &  65 ans 0 mois &  -15.00\% &  4076.14 &  {\bf 49.11} &  3076.71 &  {\bf 1.32} &  {\bf 1.19} &  {\bf 1.12} &  {\bf 1.05} &  {\bf {\color{red} 0.98}} &  {\bf {\color{red} 0.92}} \\ 
\hline 
 2066 &  63 &  65 ans 0 mois &  -10.00\% &  4476.15 &  {\bf 53.85} &  3116.71 &  {\bf 1.44} &  {\bf 1.31} &  {\bf 1.23} &  {\bf 1.15} &  {\bf 1.08} &  {\bf 1.01} \\ 
\hline 
 2067 &  64 &  65 ans 0 mois &  -5.00\% &  4896.33 &  {\bf 58.82} &  3157.23 &  {\bf 1.55} &  {\bf 1.44} &  {\bf 1.35} &  {\bf 1.26} &  {\bf 1.18} &  {\bf 1.11} \\ 
\hline 
 2068 &  65 &  65 ans 0 mois &  0.00\% &  5337.07 &  {\bf 64.02} &  3198.27 &  {\bf 1.67} &  {\bf 1.56} &  {\bf 1.47} &  {\bf 1.37} &  {\bf 1.29} &  {\bf 1.21} \\ 
\hline 
 2069 &  66 &  65 ans 0 mois &  5.00\% &  5798.80 &  {\bf 69.45} &  3239.85 &  {\bf 1.79} &  {\bf 1.70} &  {\bf 1.59} &  {\bf 1.49} &  {\bf 1.40} &  {\bf 1.31} \\ 
\hline 
 2070 &  67 &  65 ans 0 mois &  10.00\% &  6281.93 &  {\bf 75.13} &  3281.97 &  {\bf 1.91} &  {\bf 1.84} &  {\bf 1.73} &  {\bf 1.62} &  {\bf 1.52} &  {\bf 1.42} \\ 
\hline 
\hline 
\end{tabular} 
\end{center} } 
\paragraph{Retraites possibles et ratios Revenu/SMIC à 70, 75, 80, 85, 90 ans avec le modèle \emph{Gouvernement corrigé (âge-pivot glissant)}}  
 
{ \scriptsize \begin{center} 
\begin{tabular}[htb]{|c|c||c|c||c|c||c||c|c|c|c|c|c|} 
\hline 
 Retraite en &  Âge &  Âge pivot &  Décote/Surcote &  Retraite (\euro{} 2019) &  Tx Rempl(\%) &  SMIC (\euro{} 2019) &  Retraite/SMIC &  Rev70/SMIC &  Rev75/SMIC &  Rev80/SMIC &  Rev85/SMIC &  Rev90/SMIC \\ 
\hline \hline 
 2065 &  62 &  67 ans 2 mois &  -25.83\% &  3556.64 &  {\bf 42.85} &  3076.71 &  {\bf 1.16} &  {\bf 1.04} &  {\bf {\color{red} 0.98}} &  {\bf {\color{red} 0.92}} &  {\bf {\color{red} 0.86}} &  {\bf {\color{red} 0.81}} \\ 
\hline 
 2066 &  63 &  67 ans 3 mois &  -21.25\% &  3916.63 &  {\bf 47.12} &  3116.71 &  {\bf 1.26} &  {\bf 1.15} &  {\bf 1.08} &  {\bf 1.01} &  {\bf {\color{red} 0.95}} &  {\bf {\color{red} 0.89}} \\ 
\hline 
 2067 &  64 &  67 ans 4 mois &  -16.67\% &  4295.02 &  {\bf 51.60} &  3157.23 &  {\bf 1.36} &  {\bf 1.26} &  {\bf 1.18} &  {\bf 1.11} &  {\bf 1.04} &  {\bf {\color{red} 0.97}} \\ 
\hline 
 2068 &  65 &  67 ans 5 mois &  -12.08\% &  4692.17 &  {\bf 56.28} &  3198.27 &  {\bf 1.47} &  {\bf 1.38} &  {\bf 1.29} &  {\bf 1.21} &  {\bf 1.13} &  {\bf 1.06} \\ 
\hline 
 2069 &  66 &  67 ans 6 mois &  -7.50\% &  5108.47 &  {\bf 61.19} &  3239.85 &  {\bf 1.58} &  {\bf 1.50} &  {\bf 1.40} &  {\bf 1.32} &  {\bf 1.23} &  {\bf 1.16} \\ 
\hline 
 2070 &  67 &  67 ans 7 mois &  -2.92\% &  5544.28 &  {\bf 66.31} &  3281.97 &  {\bf 1.69} &  {\bf 1.63} &  {\bf 1.52} &  {\bf 1.43} &  {\bf 1.34} &  {\bf 1.26} \\ 
\hline 
\hline 
\end{tabular} 
\end{center} } 
\paragraph{Retraites possibles et ratios Revenu/SMIC à 70, 75, 80, 85, 90 ans avec le modèle \emph{Destinie2 (revalorisation de la fonction publique)}}  
 
{ \scriptsize \begin{center} 
\begin{tabular}[htb]{|c|c||c|c||c|c||c||c|c|c|c|c|c|} 
\hline 
 Retraite en &  Âge &  Âge pivot &  Décote/Surcote &  Retraite (\euro{} 2019) &  Tx Rempl(\%) &  SMIC (\euro{} 2019) &  Retraite/SMIC &  Rev70/SMIC &  Rev75/SMIC &  Rev80/SMIC &  Rev85/SMIC &  Rev90/SMIC \\ 
\hline \hline 
 2065 &  62 &  67 ans 2 mois &  -25.83\% &  5187.15 &  {\bf 34.95} &  2892.68 &  {\bf 1.79} &  {\bf 1.62} &  {\bf 1.52} &  {\bf 1.42} &  {\bf 1.33} &  {\bf 1.25} \\ 
\hline 
 2066 &  63 &  67 ans 3 mois &  -21.25\% &  5744.10 &  {\bf 38.21} &  2930.29 &  {\bf 1.96} &  {\bf 1.79} &  {\bf 1.68} &  {\bf 1.57} &  {\bf 1.48} &  {\bf 1.38} \\ 
\hline 
 2067 &  64 &  67 ans 4 mois &  -16.67\% &  6334.09 &  {\bf 41.59} &  2968.38 &  {\bf 2.13} &  {\bf 1.97} &  {\bf 1.85} &  {\bf 1.74} &  {\bf 1.63} &  {\bf 1.53} \\ 
\hline 
 2068 &  65 &  67 ans 5 mois &  -12.08\% &  6958.13 &  {\bf 45.10} &  3006.97 &  {\bf 2.31} &  {\bf 2.17} &  {\bf 2.03} &  {\bf 1.91} &  {\bf 1.79} &  {\bf 1.68} \\ 
\hline 
 2069 &  66 &  67 ans 6 mois &  -7.50\% &  7617.27 &  {\bf 48.74} &  3046.06 &  {\bf 2.50} &  {\bf 2.37} &  {\bf 2.23} &  {\bf 2.09} &  {\bf 1.96} &  {\bf 1.83} \\ 
\hline 
 2070 &  67 &  67 ans 7 mois &  -2.92\% &  8312.57 &  {\bf 52.51} &  3085.66 &  {\bf 2.69} &  {\bf 2.59} &  {\bf 2.43} &  {\bf 2.28} &  {\bf 2.14} &  {\bf 2.00} \\ 
\hline 
\hline 
\end{tabular} 
\end{center} } 

 \begin{center}\includegraphics[width=0.9\textwidth]{fig/Magistrat_2003_22_dest_retraite.pdf}\end{center} \label{fig/Magistrat_2003_22_dest_retraite.pdf} 

\newpage 
 
\chapter{Salarié privé au salaire moyen durant toute sa carrière} 


 \addto{\captionsenglish}{ \renewcommand{\mtctitle}{}} \setcounter{minitocdepth}{2} 
 \minitoc \newpage 

\section{Début de carrière à 22 ans} 

\subsection{Génération 1975 (début en 1997)} 

\paragraph{Retraites possibles et ratios Revenu/SMIC à 70, 75, 80, 85, 90 ans avec le modèle \emph{Gouvernement truqué (âge-pivot bloqué à 65 ans)}}  
 
{ \scriptsize \begin{center} 
\begin{tabular}[htb]{|c|c||c|c||c|c||c||c|c|c|c|c|c|} 
\hline 
 Retraite en &  Âge &  Âge pivot &  Décote/Surcote &  Retraite (\euro{} 2019) &  Tx Rempl(\%) &  SMIC (\euro{} 2019) &  Retraite/SMIC &  Rev70/SMIC &  Rev75/SMIC &  Rev80/SMIC &  Rev85/SMIC &  Rev90/SMIC \\ 
\hline \hline 
 2037 &  62 &  64 ans 10 mois &  -14.17\% &  1638.42 &  {\bf 40.81} &  2143.00 &  {\bf {\color{red} 0.76}} &  {\bf {\color{red} 0.69}} &  {\bf {\color{red} 0.65}} &  {\bf {\color{red} 0.61}} &  {\bf {\color{red} 0.57}} &  {\bf {\color{red} 0.53}} \\ 
\hline 
 2038 &  63 &  64 ans 11 mois &  -9.58\% &  1790.26 &  {\bf 44.02} &  2170.86 &  {\bf {\color{red} 0.82}} &  {\bf {\color{red} 0.75}} &  {\bf {\color{red} 0.71}} &  {\bf {\color{red} 0.66}} &  {\bf {\color{red} 0.62}} &  {\bf {\color{red} 0.58}} \\ 
\hline 
 2039 &  64 &  65 ans 0 mois &  -5.00\% &  1951.28 &  {\bf 47.36} &  2199.08 &  {\bf {\color{red} 0.89}} &  {\bf {\color{red} 0.82}} &  {\bf {\color{red} 0.77}} &  {\bf {\color{red} 0.72}} &  {\bf {\color{red} 0.68}} &  {\bf {\color{red} 0.63}} \\ 
\hline 
 2040 &  65 &  65 ans 0 mois &  0.00\% &  2130.88 &  {\bf 51.06} &  2227.67 &  {\bf {\color{red} 0.96}} &  {\bf {\color{red} 0.90}} &  {\bf {\color{red} 0.84}} &  {\bf {\color{red} 0.79}} &  {\bf {\color{red} 0.74}} &  {\bf {\color{red} 0.69}} \\ 
\hline 
 2041 &  66 &  65 ans 0 mois &  5.00\% &  2321.42 &  {\bf 54.91} &  2256.63 &  {\bf 1.03} &  {\bf {\color{red} 0.98}} &  {\bf {\color{red} 0.92}} &  {\bf {\color{red} 0.86}} &  {\bf {\color{red} 0.80}} &  {\bf {\color{red} 0.75}} \\ 
\hline 
 2042 &  67 &  65 ans 0 mois &  10.00\% &  2523.55 &  {\bf 58.92} &  2285.97 &  {\bf 1.10} &  {\bf 1.06} &  {\bf {\color{red} 1.00}} &  {\bf {\color{red} 0.93}} &  {\bf {\color{red} 0.87}} &  {\bf {\color{red} 0.82}} \\ 
\hline 
\hline 
\end{tabular} 
\end{center} } 
\paragraph{Retraites possibles et ratios Revenu/SMIC à 70, 75, 80, 85, 90 ans avec le modèle \emph{Gouvernement corrigé (âge-pivot glissant)}}  
 
{ \scriptsize \begin{center} 
\begin{tabular}[htb]{|c|c||c|c||c|c||c||c|c|c|c|c|c|} 
\hline 
 Retraite en &  Âge &  Âge pivot &  Décote/Surcote &  Retraite (\euro{} 2019) &  Tx Rempl(\%) &  SMIC (\euro{} 2019) &  Retraite/SMIC &  Rev70/SMIC &  Rev75/SMIC &  Rev80/SMIC &  Rev85/SMIC &  Rev90/SMIC \\ 
\hline \hline 
 2037 &  62 &  64 ans 10 mois &  -14.17\% &  1638.42 &  {\bf 40.81} &  2143.00 &  {\bf {\color{red} 0.76}} &  {\bf {\color{red} 0.69}} &  {\bf {\color{red} 0.65}} &  {\bf {\color{red} 0.61}} &  {\bf {\color{red} 0.57}} &  {\bf {\color{red} 0.53}} \\ 
\hline 
 2038 &  63 &  64 ans 11 mois &  -9.58\% &  1790.26 &  {\bf 44.02} &  2170.86 &  {\bf {\color{red} 0.82}} &  {\bf {\color{red} 0.75}} &  {\bf {\color{red} 0.71}} &  {\bf {\color{red} 0.66}} &  {\bf {\color{red} 0.62}} &  {\bf {\color{red} 0.58}} \\ 
\hline 
 2039 &  64 &  65 ans 0 mois &  -5.00\% &  1951.28 &  {\bf 47.36} &  2199.08 &  {\bf {\color{red} 0.89}} &  {\bf {\color{red} 0.82}} &  {\bf {\color{red} 0.77}} &  {\bf {\color{red} 0.72}} &  {\bf {\color{red} 0.68}} &  {\bf {\color{red} 0.63}} \\ 
\hline 
 2040 &  65 &  65 ans 1 mois &  -0.42\% &  2122.00 &  {\bf 50.85} &  2227.67 &  {\bf {\color{red} 0.95}} &  {\bf {\color{red} 0.89}} &  {\bf {\color{red} 0.84}} &  {\bf {\color{red} 0.78}} &  {\bf {\color{red} 0.74}} &  {\bf {\color{red} 0.69}} \\ 
\hline 
 2041 &  66 &  65 ans 2 mois &  4.17\% &  2303.00 &  {\bf 54.47} &  2256.63 &  {\bf 1.02} &  {\bf {\color{red} 0.97}} &  {\bf {\color{red} 0.91}} &  {\bf {\color{red} 0.85}} &  {\bf {\color{red} 0.80}} &  {\bf {\color{red} 0.75}} \\ 
\hline 
 2042 &  67 &  65 ans 3 mois &  8.75\% &  2494.87 &  {\bf 58.26} &  2285.97 &  {\bf 1.09} &  {\bf 1.05} &  {\bf {\color{red} 0.98}} &  {\bf {\color{red} 0.92}} &  {\bf {\color{red} 0.86}} &  {\bf {\color{red} 0.81}} \\ 
\hline 
\hline 
\end{tabular} 
\end{center} } 
\paragraph{Retraites possibles et ratios Revenu/SMIC à 70, 75, 80, 85, 90 ans avec le modèle \emph{Destinie2 (revalorisation de la fonction publique)}}  
 
{ \scriptsize \begin{center} 
\begin{tabular}[htb]{|c|c||c|c||c|c||c||c|c|c|c|c|c|} 
\hline 
 Retraite en &  Âge &  Âge pivot &  Décote/Surcote &  Retraite (\euro{} 2019) &  Tx Rempl(\%) &  SMIC (\euro{} 2019) &  Retraite/SMIC &  Rev70/SMIC &  Rev75/SMIC &  Rev80/SMIC &  Rev85/SMIC &  Rev90/SMIC \\ 
\hline \hline 
 2037 &  62 &  64 ans 10 mois &  -14.17\% &  1626.20 &  {\bf 41.22} &  2014.82 &  {\bf {\color{red} 0.81}} &  {\bf {\color{red} 0.73}} &  {\bf {\color{red} 0.68}} &  {\bf {\color{red} 0.64}} &  {\bf {\color{red} 0.60}} &  {\bf {\color{red} 0.56}} \\ 
\hline 
 2038 &  63 &  64 ans 11 mois &  -9.58\% &  1776.41 &  {\bf 44.45} &  2041.01 &  {\bf {\color{red} 0.87}} &  {\bf {\color{red} 0.80}} &  {\bf {\color{red} 0.75}} &  {\bf {\color{red} 0.70}} &  {\bf {\color{red} 0.66}} &  {\bf {\color{red} 0.61}} \\ 
\hline 
 2039 &  64 &  65 ans 0 mois &  -5.00\% &  1935.66 &  {\bf 47.81} &  2067.55 &  {\bf {\color{red} 0.94}} &  {\bf {\color{red} 0.87}} &  {\bf {\color{red} 0.81}} &  {\bf {\color{red} 0.76}} &  {\bf {\color{red} 0.71}} &  {\bf {\color{red} 0.67}} \\ 
\hline 
 2040 &  65 &  65 ans 1 mois &  -0.42\% &  2104.48 &  {\bf 51.31} &  2094.43 &  {\bf 1.00} &  {\bf {\color{red} 0.94}} &  {\bf {\color{red} 0.88}} &  {\bf {\color{red} 0.83}} &  {\bf {\color{red} 0.78}} &  {\bf {\color{red} 0.73}} \\ 
\hline 
 2041 &  66 &  65 ans 2 mois &  4.17\% &  2283.43 &  {\bf 54.96} &  2121.65 &  {\bf 1.08} &  {\bf 1.02} &  {\bf {\color{red} 0.96}} &  {\bf {\color{red} 0.90}} &  {\bf {\color{red} 0.84}} &  {\bf {\color{red} 0.79}} \\ 
\hline 
 2042 &  67 &  65 ans 3 mois &  8.75\% &  2473.11 &  {\bf 58.76} &  2149.23 &  {\bf 1.15} &  {\bf 1.11} &  {\bf 1.04} &  {\bf {\color{red} 0.97}} &  {\bf {\color{red} 0.91}} &  {\bf {\color{red} 0.85}} \\ 
\hline 
\hline 
\end{tabular} 
\end{center} } 

 \begin{center}\includegraphics[width=0.9\textwidth]{fig/SMPT_1975_22_dest_retraite.pdf}\end{center} \label{fig/SMPT_1975_22_dest_retraite.pdf} 

\newpage 
 
\subsection{Génération 1980 (début en 2002)} 

\paragraph{Retraites possibles et ratios Revenu/SMIC à 70, 75, 80, 85, 90 ans avec le modèle \emph{Gouvernement truqué (âge-pivot bloqué à 65 ans)}}  
 
{ \scriptsize \begin{center} 
\begin{tabular}[htb]{|c|c||c|c||c|c||c||c|c|c|c|c|c|} 
\hline 
 Retraite en &  Âge &  Âge pivot &  Décote/Surcote &  Retraite (\euro{} 2019) &  Tx Rempl(\%) &  SMIC (\euro{} 2019) &  Retraite/SMIC &  Rev70/SMIC &  Rev75/SMIC &  Rev80/SMIC &  Rev85/SMIC &  Rev90/SMIC \\ 
\hline \hline 
 2042 &  62 &  65 ans 0 mois &  -15.00\% &  1774.24 &  {\bf 41.43} &  2285.97 &  {\bf {\color{red} 0.78}} &  {\bf {\color{red} 0.70}} &  {\bf {\color{red} 0.66}} &  {\bf {\color{red} 0.62}} &  {\bf {\color{red} 0.58}} &  {\bf {\color{red} 0.54}} \\ 
\hline 
 2043 &  63 &  65 ans 0 mois &  -10.00\% &  1954.49 &  {\bf 45.05} &  2315.68 &  {\bf {\color{red} 0.84}} &  {\bf {\color{red} 0.77}} &  {\bf {\color{red} 0.72}} &  {\bf {\color{red} 0.68}} &  {\bf {\color{red} 0.64}} &  {\bf {\color{red} 0.60}} \\ 
\hline 
 2044 &  64 &  65 ans 0 mois &  -5.00\% &  2146.42 &  {\bf 48.84} &  2345.79 &  {\bf {\color{red} 0.92}} &  {\bf {\color{red} 0.85}} &  {\bf {\color{red} 0.79}} &  {\bf {\color{red} 0.74}} &  {\bf {\color{red} 0.70}} &  {\bf {\color{red} 0.65}} \\ 
\hline 
 2045 &  65 &  65 ans 0 mois &  0.00\% &  2350.73 &  {\bf 52.80} &  2376.28 &  {\bf {\color{red} 0.99}} &  {\bf {\color{red} 0.93}} &  {\bf {\color{red} 0.87}} &  {\bf {\color{red} 0.82}} &  {\bf {\color{red} 0.76}} &  {\bf {\color{red} 0.72}} \\ 
\hline 
 2046 &  66 &  65 ans 0 mois &  5.00\% &  2566.26 &  {\bf 56.90} &  2407.18 &  {\bf 1.07} &  {\bf 1.01} &  {\bf {\color{red} 0.95}} &  {\bf {\color{red} 0.89}} &  {\bf {\color{red} 0.83}} &  {\bf {\color{red} 0.78}} \\ 
\hline 
 2047 &  67 &  65 ans 0 mois &  10.00\% &  2793.36 &  {\bf 61.15} &  2438.47 &  {\bf 1.15} &  {\bf 1.10} &  {\bf 1.03} &  {\bf {\color{red} 0.97}} &  {\bf {\color{red} 0.91}} &  {\bf {\color{red} 0.85}} \\ 
\hline 
\hline 
\end{tabular} 
\end{center} } 
\paragraph{Retraites possibles et ratios Revenu/SMIC à 70, 75, 80, 85, 90 ans avec le modèle \emph{Gouvernement corrigé (âge-pivot glissant)}}  
 
{ \scriptsize \begin{center} 
\begin{tabular}[htb]{|c|c||c|c||c|c||c||c|c|c|c|c|c|} 
\hline 
 Retraite en &  Âge &  Âge pivot &  Décote/Surcote &  Retraite (\euro{} 2019) &  Tx Rempl(\%) &  SMIC (\euro{} 2019) &  Retraite/SMIC &  Rev70/SMIC &  Rev75/SMIC &  Rev80/SMIC &  Rev85/SMIC &  Rev90/SMIC \\ 
\hline \hline 
 2042 &  62 &  65 ans 3 mois &  -16.25\% &  1748.15 &  {\bf 40.82} &  2285.97 &  {\bf {\color{red} 0.76}} &  {\bf {\color{red} 0.69}} &  {\bf {\color{red} 0.65}} &  {\bf {\color{red} 0.61}} &  {\bf {\color{red} 0.57}} &  {\bf {\color{red} 0.53}} \\ 
\hline 
 2043 &  63 &  65 ans 4 mois &  -11.67\% &  1918.29 &  {\bf 44.22} &  2315.68 &  {\bf {\color{red} 0.83}} &  {\bf {\color{red} 0.76}} &  {\bf {\color{red} 0.71}} &  {\bf {\color{red} 0.67}} &  {\bf {\color{red} 0.62}} &  {\bf {\color{red} 0.58}} \\ 
\hline 
 2044 &  64 &  65 ans 5 mois &  -7.08\% &  2099.35 &  {\bf 47.77} &  2345.79 &  {\bf {\color{red} 0.89}} &  {\bf {\color{red} 0.83}} &  {\bf {\color{red} 0.78}} &  {\bf {\color{red} 0.73}} &  {\bf {\color{red} 0.68}} &  {\bf {\color{red} 0.64}} \\ 
\hline 
 2045 &  65 &  65 ans 6 mois &  -2.50\% &  2291.96 &  {\bf 51.48} &  2376.28 &  {\bf {\color{red} 0.96}} &  {\bf {\color{red} 0.90}} &  {\bf {\color{red} 0.85}} &  {\bf {\color{red} 0.79}} &  {\bf {\color{red} 0.74}} &  {\bf {\color{red} 0.70}} \\ 
\hline 
 2046 &  66 &  65 ans 7 mois &  2.08\% &  2494.98 &  {\bf 55.32} &  2407.18 &  {\bf 1.04} &  {\bf {\color{red} 0.98}} &  {\bf {\color{red} 0.92}} &  {\bf {\color{red} 0.87}} &  {\bf {\color{red} 0.81}} &  {\bf {\color{red} 0.76}} \\ 
\hline 
 2047 &  67 &  65 ans 8 mois &  6.67\% &  2708.71 &  {\bf 59.29} &  2438.47 &  {\bf 1.11} &  {\bf 1.07} &  {\bf 1.00} &  {\bf {\color{red} 0.94}} &  {\bf {\color{red} 0.88}} &  {\bf {\color{red} 0.83}} \\ 
\hline 
\hline 
\end{tabular} 
\end{center} } 
\paragraph{Retraites possibles et ratios Revenu/SMIC à 70, 75, 80, 85, 90 ans avec le modèle \emph{Destinie2 (revalorisation de la fonction publique)}}  
 
{ \scriptsize \begin{center} 
\begin{tabular}[htb]{|c|c||c|c||c|c||c||c|c|c|c|c|c|} 
\hline 
 Retraite en &  Âge &  Âge pivot &  Décote/Surcote &  Retraite (\euro{} 2019) &  Tx Rempl(\%) &  SMIC (\euro{} 2019) &  Retraite/SMIC &  Rev70/SMIC &  Rev75/SMIC &  Rev80/SMIC &  Rev85/SMIC &  Rev90/SMIC \\ 
\hline \hline 
 2042 &  62 &  65 ans 3 mois &  -16.25\% &  1731.42 &  {\bf 41.14} &  2149.23 &  {\bf {\color{red} 0.81}} &  {\bf {\color{red} 0.73}} &  {\bf {\color{red} 0.68}} &  {\bf {\color{red} 0.64}} &  {\bf {\color{red} 0.60}} &  {\bf {\color{red} 0.56}} \\ 
\hline 
 2043 &  63 &  65 ans 4 mois &  -11.67\% &  1899.53 &  {\bf 44.55} &  2177.17 &  {\bf {\color{red} 0.87}} &  {\bf {\color{red} 0.80}} &  {\bf {\color{red} 0.75}} &  {\bf {\color{red} 0.70}} &  {\bf {\color{red} 0.66}} &  {\bf {\color{red} 0.62}} \\ 
\hline 
 2044 &  64 &  65 ans 5 mois &  -7.08\% &  2078.38 &  {\bf 48.12} &  2205.48 &  {\bf {\color{red} 0.94}} &  {\bf {\color{red} 0.87}} &  {\bf {\color{red} 0.82}} &  {\bf {\color{red} 0.77}} &  {\bf {\color{red} 0.72}} &  {\bf {\color{red} 0.67}} \\ 
\hline 
 2045 &  65 &  65 ans 6 mois &  -2.50\% &  2268.63 &  {\bf 51.85} &  2234.15 &  {\bf 1.02} &  {\bf {\color{red} 0.95}} &  {\bf {\color{red} 0.89}} &  {\bf {\color{red} 0.84}} &  {\bf {\color{red} 0.78}} &  {\bf {\color{red} 0.74}} \\ 
\hline 
 2046 &  66 &  65 ans 7 mois &  2.08\% &  2469.13 &  {\bf 55.71} &  2263.19 &  {\bf 1.09} &  {\bf 1.04} &  {\bf {\color{red} 0.97}} &  {\bf {\color{red} 0.91}} &  {\bf {\color{red} 0.85}} &  {\bf {\color{red} 0.80}} \\ 
\hline 
 2047 &  67 &  65 ans 8 mois &  6.67\% &  2680.19 &  {\bf 59.70} &  2292.61 &  {\bf 1.17} &  {\bf 1.12} &  {\bf 1.05} &  {\bf {\color{red} 0.99}} &  {\bf {\color{red} 0.93}} &  {\bf {\color{red} 0.87}} \\ 
\hline 
\hline 
\end{tabular} 
\end{center} } 

 \begin{center}\includegraphics[width=0.9\textwidth]{fig/SMPT_1980_22_dest_retraite.pdf}\end{center} \label{fig/SMPT_1980_22_dest_retraite.pdf} 

\newpage 
 
\subsection{Génération 1990 (début en 2012)} 

\paragraph{Retraites possibles et ratios Revenu/SMIC à 70, 75, 80, 85, 90 ans avec le modèle \emph{Gouvernement truqué (âge-pivot bloqué à 65 ans)}}  
 
{ \scriptsize \begin{center} 
\begin{tabular}[htb]{|c|c||c|c||c|c||c||c|c|c|c|c|c|} 
\hline 
 Retraite en &  Âge &  Âge pivot &  Décote/Surcote &  Retraite (\euro{} 2019) &  Tx Rempl(\%) &  SMIC (\euro{} 2019) &  Retraite/SMIC &  Rev70/SMIC &  Rev75/SMIC &  Rev80/SMIC &  Rev85/SMIC &  Rev90/SMIC \\ 
\hline \hline 
 2052 &  62 &  65 ans 0 mois &  -15.00\% &  2157.03 &  {\bf 44.26} &  2601.14 &  {\bf {\color{red} 0.83}} &  {\bf {\color{red} 0.75}} &  {\bf {\color{red} 0.70}} &  {\bf {\color{red} 0.66}} &  {\bf {\color{red} 0.62}} &  {\bf {\color{red} 0.58}} \\ 
\hline 
 2053 &  63 &  65 ans 0 mois &  -10.00\% &  2375.44 &  {\bf 48.12} &  2634.96 &  {\bf {\color{red} 0.90}} &  {\bf {\color{red} 0.82}} &  {\bf {\color{red} 0.77}} &  {\bf {\color{red} 0.72}} &  {\bf {\color{red} 0.68}} &  {\bf {\color{red} 0.64}} \\ 
\hline 
 2054 &  64 &  65 ans 0 mois &  -5.00\% &  2606.13 &  {\bf 52.12} &  2669.21 &  {\bf {\color{red} 0.98}} &  {\bf {\color{red} 0.90}} &  {\bf {\color{red} 0.85}} &  {\bf {\color{red} 0.79}} &  {\bf {\color{red} 0.74}} &  {\bf {\color{red} 0.70}} \\ 
\hline 
 2055 &  65 &  65 ans 0 mois &  0.00\% &  2849.47 &  {\bf 56.25} &  2703.91 &  {\bf 1.05} &  {\bf {\color{red} 0.99}} &  {\bf {\color{red} 0.93}} &  {\bf {\color{red} 0.87}} &  {\bf {\color{red} 0.81}} &  {\bf {\color{red} 0.76}} \\ 
\hline 
 2056 &  66 &  65 ans 0 mois &  5.00\% &  3105.84 &  {\bf 60.52} &  2739.06 &  {\bf 1.13} &  {\bf 1.08} &  {\bf 1.01} &  {\bf {\color{red} 0.95}} &  {\bf {\color{red} 0.89}} &  {\bf {\color{red} 0.83}} \\ 
\hline 
 2057 &  67 &  65 ans 0 mois &  10.00\% &  3375.63 &  {\bf 64.94} &  2774.67 &  {\bf 1.22} &  {\bf 1.17} &  {\bf 1.10} &  {\bf 1.03} &  {\bf {\color{red} 0.96}} &  {\bf {\color{red} 0.90}} \\ 
\hline 
\hline 
\end{tabular} 
\end{center} } 
\paragraph{Retraites possibles et ratios Revenu/SMIC à 70, 75, 80, 85, 90 ans avec le modèle \emph{Gouvernement corrigé (âge-pivot glissant)}}  
 
{ \scriptsize \begin{center} 
\begin{tabular}[htb]{|c|c||c|c||c|c||c||c|c|c|c|c|c|} 
\hline 
 Retraite en &  Âge &  Âge pivot &  Décote/Surcote &  Retraite (\euro{} 2019) &  Tx Rempl(\%) &  SMIC (\euro{} 2019) &  Retraite/SMIC &  Rev70/SMIC &  Rev75/SMIC &  Rev80/SMIC &  Rev85/SMIC &  Rev90/SMIC \\ 
\hline \hline 
 2052 &  62 &  66 ans 1 mois &  -20.42\% &  2019.57 &  {\bf 41.44} &  2601.14 &  {\bf {\color{red} 0.78}} &  {\bf {\color{red} 0.70}} &  {\bf {\color{red} 0.66}} &  {\bf {\color{red} 0.62}} &  {\bf {\color{red} 0.58}} &  {\bf {\color{red} 0.54}} \\ 
\hline 
 2053 &  63 &  66 ans 2 mois &  -15.83\% &  2221.48 &  {\bf 45.00} &  2634.96 &  {\bf {\color{red} 0.84}} &  {\bf {\color{red} 0.77}} &  {\bf {\color{red} 0.72}} &  {\bf {\color{red} 0.68}} &  {\bf {\color{red} 0.63}} &  {\bf {\color{red} 0.59}} \\ 
\hline 
 2054 &  64 &  66 ans 3 mois &  -11.25\% &  2434.68 &  {\bf 48.69} &  2669.21 &  {\bf {\color{red} 0.91}} &  {\bf {\color{red} 0.84}} &  {\bf {\color{red} 0.79}} &  {\bf {\color{red} 0.74}} &  {\bf {\color{red} 0.70}} &  {\bf {\color{red} 0.65}} \\ 
\hline 
 2055 &  65 &  66 ans 4 mois &  -6.67\% &  2659.51 &  {\bf 52.50} &  2703.91 &  {\bf {\color{red} 0.98}} &  {\bf {\color{red} 0.92}} &  {\bf {\color{red} 0.86}} &  {\bf {\color{red} 0.81}} &  {\bf {\color{red} 0.76}} &  {\bf {\color{red} 0.71}} \\ 
\hline 
 2056 &  66 &  66 ans 5 mois &  -2.08\% &  2896.32 &  {\bf 56.44} &  2739.06 &  {\bf 1.06} &  {\bf 1.00} &  {\bf {\color{red} 0.94}} &  {\bf {\color{red} 0.88}} &  {\bf {\color{red} 0.83}} &  {\bf {\color{red} 0.78}} \\ 
\hline 
 2057 &  67 &  66 ans 6 mois &  2.50\% &  3145.47 &  {\bf 60.51} &  2774.67 &  {\bf 1.13} &  {\bf 1.09} &  {\bf 1.02} &  {\bf {\color{red} 0.96}} &  {\bf {\color{red} 0.90}} &  {\bf {\color{red} 0.84}} \\ 
\hline 
\hline 
\end{tabular} 
\end{center} } 
\paragraph{Retraites possibles et ratios Revenu/SMIC à 70, 75, 80, 85, 90 ans avec le modèle \emph{Destinie2 (revalorisation de la fonction publique)}}  
 
{ \scriptsize \begin{center} 
\begin{tabular}[htb]{|c|c||c|c||c|c||c||c|c|c|c|c|c|} 
\hline 
 Retraite en &  Âge &  Âge pivot &  Décote/Surcote &  Retraite (\euro{} 2019) &  Tx Rempl(\%) &  SMIC (\euro{} 2019) &  Retraite/SMIC &  Rev70/SMIC &  Rev75/SMIC &  Rev80/SMIC &  Rev85/SMIC &  Rev90/SMIC \\ 
\hline \hline 
 2052 &  62 &  66 ans 1 mois &  -20.42\% &  1992.28 &  {\bf 41.60} &  2445.56 &  {\bf {\color{red} 0.81}} &  {\bf {\color{red} 0.73}} &  {\bf {\color{red} 0.69}} &  {\bf {\color{red} 0.65}} &  {\bf {\color{red} 0.61}} &  {\bf {\color{red} 0.57}} \\ 
\hline 
 2053 &  63 &  66 ans 2 mois &  -15.83\% &  2191.25 &  {\bf 45.17} &  2477.35 &  {\bf {\color{red} 0.88}} &  {\bf {\color{red} 0.81}} &  {\bf {\color{red} 0.76}} &  {\bf {\color{red} 0.71}} &  {\bf {\color{red} 0.67}} &  {\bf {\color{red} 0.62}} \\ 
\hline 
 2054 &  64 &  66 ans 3 mois &  -11.25\% &  2401.32 &  {\bf 48.86} &  2509.56 &  {\bf {\color{red} 0.96}} &  {\bf {\color{red} 0.89}} &  {\bf {\color{red} 0.83}} &  {\bf {\color{red} 0.78}} &  {\bf {\color{red} 0.73}} &  {\bf {\color{red} 0.68}} \\ 
\hline 
 2055 &  65 &  66 ans 4 mois &  -6.67\% &  2622.83 &  {\bf 52.69} &  2542.18 &  {\bf 1.03} &  {\bf {\color{red} 0.97}} &  {\bf {\color{red} 0.91}} &  {\bf {\color{red} 0.85}} &  {\bf {\color{red} 0.80}} &  {\bf {\color{red} 0.75}} \\ 
\hline 
 2056 &  66 &  66 ans 5 mois &  -2.08\% &  2856.14 &  {\bf 56.64} &  2575.23 &  {\bf 1.11} &  {\bf 1.05} &  {\bf {\color{red} 0.99}} &  {\bf {\color{red} 0.93}} &  {\bf {\color{red} 0.87}} &  {\bf {\color{red} 0.81}} \\ 
\hline 
 2057 &  67 &  66 ans 6 mois &  2.50\% &  3101.58 &  {\bf 60.71} &  2608.71 &  {\bf 1.19} &  {\bf 1.14} &  {\bf 1.07} &  {\bf 1.01} &  {\bf {\color{red} 0.94}} &  {\bf {\color{red} 0.88}} \\ 
\hline 
\hline 
\end{tabular} 
\end{center} } 

 \begin{center}\includegraphics[width=0.9\textwidth]{fig/SMPT_1990_22_dest_retraite.pdf}\end{center} \label{fig/SMPT_1990_22_dest_retraite.pdf} 

\newpage 
 
\subsection{Génération 2003 (début en 2025)} 

\paragraph{Retraites possibles et ratios Revenu/SMIC à 70, 75, 80, 85, 90 ans avec le modèle \emph{Gouvernement truqué (âge-pivot bloqué à 65 ans)}}  
 
{ \scriptsize \begin{center} 
\begin{tabular}[htb]{|c|c||c|c||c|c||c||c|c|c|c|c|c|} 
\hline 
 Retraite en &  Âge &  Âge pivot &  Décote/Surcote &  Retraite (\euro{} 2019) &  Tx Rempl(\%) &  SMIC (\euro{} 2019) &  Retraite/SMIC &  Rev70/SMIC &  Rev75/SMIC &  Rev80/SMIC &  Rev85/SMIC &  Rev90/SMIC \\ 
\hline \hline 
 2065 &  62 &  65 ans 0 mois &  -15.00\% &  2730.40 &  {\bf 47.37} &  3076.71 &  {\bf {\color{red} 0.89}} &  {\bf {\color{red} 0.80}} &  {\bf {\color{red} 0.75}} &  {\bf {\color{red} 0.70}} &  {\bf {\color{red} 0.66}} &  {\bf {\color{red} 0.62}} \\ 
\hline 
 2066 &  63 &  65 ans 0 mois &  -10.00\% &  3001.74 &  {\bf 51.41} &  3116.71 &  {\bf {\color{red} 0.96}} &  {\bf {\color{red} 0.88}} &  {\bf {\color{red} 0.82}} &  {\bf {\color{red} 0.77}} &  {\bf {\color{red} 0.72}} &  {\bf {\color{red} 0.68}} \\ 
\hline 
 2067 &  64 &  65 ans 0 mois &  -5.00\% &  3287.91 &  {\bf 55.59} &  3157.23 &  {\bf 1.04} &  {\bf {\color{red} 0.96}} &  {\bf {\color{red} 0.90}} &  {\bf {\color{red} 0.85}} &  {\bf {\color{red} 0.79}} &  {\bf {\color{red} 0.74}} \\ 
\hline 
 2068 &  65 &  65 ans 0 mois &  0.00\% &  3589.35 &  {\bf 59.90} &  3198.27 &  {\bf 1.12} &  {\bf 1.05} &  {\bf {\color{red} 0.99}} &  {\bf {\color{red} 0.92}} &  {\bf {\color{red} 0.87}} &  {\bf {\color{red} 0.81}} \\ 
\hline 
 2069 &  66 &  65 ans 0 mois &  5.00\% &  3906.53 &  {\bf 64.36} &  3239.85 &  {\bf 1.21} &  {\bf 1.15} &  {\bf 1.07} &  {\bf 1.01} &  {\bf {\color{red} 0.94}} &  {\bf {\color{red} 0.88}} \\ 
\hline 
 2070 &  67 &  65 ans 0 mois &  10.00\% &  4239.90 &  {\bf 68.96} &  3281.97 &  {\bf 1.29} &  {\bf 1.24} &  {\bf 1.17} &  {\bf 1.09} &  {\bf 1.02} &  {\bf {\color{red} 0.96}} \\ 
\hline 
\hline 
\end{tabular} 
\end{center} } 
\paragraph{Retraites possibles et ratios Revenu/SMIC à 70, 75, 80, 85, 90 ans avec le modèle \emph{Gouvernement corrigé (âge-pivot glissant)}}  
 
{ \scriptsize \begin{center} 
\begin{tabular}[htb]{|c|c||c|c||c|c||c||c|c|c|c|c|c|} 
\hline 
 Retraite en &  Âge &  Âge pivot &  Décote/Surcote &  Retraite (\euro{} 2019) &  Tx Rempl(\%) &  SMIC (\euro{} 2019) &  Retraite/SMIC &  Rev70/SMIC &  Rev75/SMIC &  Rev80/SMIC &  Rev85/SMIC &  Rev90/SMIC \\ 
\hline \hline 
 2065 &  62 &  67 ans 2 mois &  -25.83\% &  2382.41 &  {\bf 41.33} &  3076.71 &  {\bf {\color{red} 0.77}} &  {\bf {\color{red} 0.70}} &  {\bf {\color{red} 0.65}} &  {\bf {\color{red} 0.61}} &  {\bf {\color{red} 0.58}} &  {\bf {\color{red} 0.54}} \\ 
\hline 
 2066 &  63 &  67 ans 3 mois &  -21.25\% &  2626.52 &  {\bf 44.98} &  3116.71 &  {\bf {\color{red} 0.84}} &  {\bf {\color{red} 0.77}} &  {\bf {\color{red} 0.72}} &  {\bf {\color{red} 0.68}} &  {\bf {\color{red} 0.63}} &  {\bf {\color{red} 0.59}} \\ 
\hline 
 2067 &  64 &  67 ans 4 mois &  -16.67\% &  2884.13 &  {\bf 48.76} &  3157.23 &  {\bf {\color{red} 0.91}} &  {\bf {\color{red} 0.85}} &  {\bf {\color{red} 0.79}} &  {\bf {\color{red} 0.74}} &  {\bf {\color{red} 0.70}} &  {\bf {\color{red} 0.65}} \\ 
\hline 
 2068 &  65 &  67 ans 5 mois &  -12.08\% &  3155.64 &  {\bf 52.67} &  3198.27 &  {\bf {\color{red} 0.99}} &  {\bf {\color{red} 0.92}} &  {\bf {\color{red} 0.87}} &  {\bf {\color{red} 0.81}} &  {\bf {\color{red} 0.76}} &  {\bf {\color{red} 0.71}} \\ 
\hline 
 2069 &  66 &  67 ans 6 mois &  -7.50\% &  3441.46 &  {\bf 56.70} &  3239.85 &  {\bf 1.06} &  {\bf 1.01} &  {\bf {\color{red} 0.95}} &  {\bf {\color{red} 0.89}} &  {\bf {\color{red} 0.83}} &  {\bf {\color{red} 0.78}} \\ 
\hline 
 2070 &  67 &  67 ans 7 mois &  -2.92\% &  3742.03 &  {\bf 60.86} &  3281.97 &  {\bf 1.14} &  {\bf 1.10} &  {\bf 1.03} &  {\bf {\color{red} 0.96}} &  {\bf {\color{red} 0.90}} &  {\bf {\color{red} 0.85}} \\ 
\hline 
\hline 
\end{tabular} 
\end{center} } 
\paragraph{Retraites possibles et ratios Revenu/SMIC à 70, 75, 80, 85, 90 ans avec le modèle \emph{Destinie2 (revalorisation de la fonction publique)}}  
 
{ \scriptsize \begin{center} 
\begin{tabular}[htb]{|c|c||c|c||c|c||c||c|c|c|c|c|c|} 
\hline 
 Retraite en &  Âge &  Âge pivot &  Décote/Surcote &  Retraite (\euro{} 2019) &  Tx Rempl(\%) &  SMIC (\euro{} 2019) &  Retraite/SMIC &  Rev70/SMIC &  Rev75/SMIC &  Rev80/SMIC &  Rev85/SMIC &  Rev90/SMIC \\ 
\hline \hline 
 2065 &  62 &  67 ans 2 mois &  -25.83\% &  2341.72 &  {\bf 41.34} &  2892.68 &  {\bf {\color{red} 0.81}} &  {\bf {\color{red} 0.73}} &  {\bf {\color{red} 0.68}} &  {\bf {\color{red} 0.64}} &  {\bf {\color{red} 0.60}} &  {\bf {\color{red} 0.56}} \\ 
\hline 
 2066 &  63 &  67 ans 3 mois &  -21.25\% &  2581.66 &  {\bf 44.99} &  2930.29 &  {\bf {\color{red} 0.88}} &  {\bf {\color{red} 0.80}} &  {\bf {\color{red} 0.75}} &  {\bf {\color{red} 0.71}} &  {\bf {\color{red} 0.66}} &  {\bf {\color{red} 0.62}} \\ 
\hline 
 2067 &  64 &  67 ans 4 mois &  -16.67\% &  2834.86 &  {\bf 48.77} &  2968.38 &  {\bf {\color{red} 0.96}} &  {\bf {\color{red} 0.88}} &  {\bf {\color{red} 0.83}} &  {\bf {\color{red} 0.78}} &  {\bf {\color{red} 0.73}} &  {\bf {\color{red} 0.68}} \\ 
\hline 
 2068 &  65 &  67 ans 5 mois &  -12.08\% &  3101.72 &  {\bf 52.67} &  3006.97 &  {\bf 1.03} &  {\bf {\color{red} 0.97}} &  {\bf {\color{red} 0.91}} &  {\bf {\color{red} 0.85}} &  {\bf {\color{red} 0.80}} &  {\bf {\color{red} 0.75}} \\ 
\hline 
 2069 &  66 &  67 ans 6 mois &  -7.50\% &  3382.64 &  {\bf 56.71} &  3046.06 &  {\bf 1.11} &  {\bf 1.05} &  {\bf {\color{red} 0.99}} &  {\bf {\color{red} 0.93}} &  {\bf {\color{red} 0.87}} &  {\bf {\color{red} 0.81}} \\ 
\hline 
 2070 &  67 &  67 ans 7 mois &  -2.92\% &  3678.06 &  {\bf 60.87} &  3085.66 &  {\bf 1.19} &  {\bf 1.15} &  {\bf 1.07} &  {\bf 1.01} &  {\bf {\color{red} 0.94}} &  {\bf {\color{red} 0.89}} \\ 
\hline 
\hline 
\end{tabular} 
\end{center} } 

 \begin{center}\includegraphics[width=0.9\textwidth]{fig/SMPT_2003_22_dest_retraite.pdf}\end{center} \label{fig/SMPT_2003_22_dest_retraite.pdf} 

\newpage 
 
\chapter{Salarié privé au SMIC durant toute sa carrière} 


 \addto{\captionsenglish}{ \renewcommand{\mtctitle}{}} \setcounter{minitocdepth}{2} 
 \minitoc \newpage 

\section{Début de carrière à 22 ans} 

\subsection{Génération 1975 (début en 1997)} 

\paragraph{Retraites possibles et ratios Revenu/SMIC à 70, 75, 80, 85, 90 ans avec le modèle \emph{Gouvernement truqué (âge-pivot bloqué à 65 ans)}}  
 
{ \scriptsize \begin{center} 
\begin{tabular}[htb]{|c|c||c|c||c|c||c||c|c|c|c|c|c|} 
\hline 
 Retraite en &  Âge &  Âge pivot &  Décote/Surcote &  Retraite (\euro{} 2019) &  Tx Rempl(\%) &  SMIC (\euro{} 2019) &  Retraite/SMIC &  Rev70/SMIC &  Rev75/SMIC &  Rev80/SMIC &  Rev85/SMIC &  Rev90/SMIC \\ 
\hline \hline 
 2037 &  62 &  64 ans 10 mois &  -14.17\% &  874.67 &  {\bf 40.82} &  2143.00 &  {\bf {\color{red} 0.41}} &  {\bf {\color{red} 0.37}} &  {\bf {\color{red} 0.35}} &  {\bf {\color{red} 0.32}} &  {\bf {\color{red} 0.30}} &  {\bf {\color{red} 0.28}} \\ 
\hline 
 2038 &  63 &  64 ans 11 mois &  -9.58\% &  955.72 &  {\bf 44.03} &  2170.86 &  {\bf {\color{red} 0.44}} &  {\bf {\color{red} 0.40}} &  {\bf {\color{red} 0.38}} &  {\bf {\color{red} 0.35}} &  {\bf {\color{red} 0.33}} &  {\bf {\color{red} 0.31}} \\ 
\hline 
 2039 &  64 &  65 ans 0 mois &  -5.00\% &  1041.68 &  {\bf 47.37} &  2199.08 &  {\bf {\color{red} 0.47}} &  {\bf {\color{red} 0.44}} &  {\bf {\color{red} 0.41}} &  {\bf {\color{red} 0.39}} &  {\bf {\color{red} 0.36}} &  {\bf {\color{red} 0.34}} \\ 
\hline 
 2040 &  65 &  65 ans 0 mois &  0.00\% &  1893.52 &  {\bf 85.00} &  2227.67 &  {\bf {\color{red} 0.85}} &  {\bf {\color{red} 0.80}} &  {\bf {\color{red} 0.75}} &  {\bf {\color{red} 0.70}} &  {\bf {\color{red} 0.66}} &  {\bf {\color{red} 0.62}} \\ 
\hline 
 2041 &  66 &  65 ans 0 mois &  5.00\% &  1918.14 &  {\bf 85.00} &  2256.63 &  {\bf {\color{red} 0.85}} &  {\bf {\color{red} 0.81}} &  {\bf {\color{red} 0.76}} &  {\bf {\color{red} 0.71}} &  {\bf {\color{red} 0.67}} &  {\bf {\color{red} 0.62}} \\ 
\hline 
 2042 &  67 &  65 ans 0 mois &  10.00\% &  1943.07 &  {\bf 85.00} &  2285.97 &  {\bf {\color{red} 0.85}} &  {\bf {\color{red} 0.82}} &  {\bf {\color{red} 0.77}} &  {\bf {\color{red} 0.72}} &  {\bf {\color{red} 0.67}} &  {\bf {\color{red} 0.63}} \\ 
\hline 
\hline 
\end{tabular} 
\end{center} } 
\paragraph{Retraites possibles et ratios Revenu/SMIC à 70, 75, 80, 85, 90 ans avec le modèle \emph{Gouvernement corrigé (âge-pivot glissant)}}  
 
{ \scriptsize \begin{center} 
\begin{tabular}[htb]{|c|c||c|c||c|c||c||c|c|c|c|c|c|} 
\hline 
 Retraite en &  Âge &  Âge pivot &  Décote/Surcote &  Retraite (\euro{} 2019) &  Tx Rempl(\%) &  SMIC (\euro{} 2019) &  Retraite/SMIC &  Rev70/SMIC &  Rev75/SMIC &  Rev80/SMIC &  Rev85/SMIC &  Rev90/SMIC \\ 
\hline \hline 
 2037 &  62 &  64 ans 10 mois &  -14.17\% &  874.67 &  {\bf 40.82} &  2143.00 &  {\bf {\color{red} 0.41}} &  {\bf {\color{red} 0.37}} &  {\bf {\color{red} 0.35}} &  {\bf {\color{red} 0.32}} &  {\bf {\color{red} 0.30}} &  {\bf {\color{red} 0.28}} \\ 
\hline 
 2038 &  63 &  64 ans 11 mois &  -9.58\% &  955.72 &  {\bf 44.03} &  2170.86 &  {\bf {\color{red} 0.44}} &  {\bf {\color{red} 0.40}} &  {\bf {\color{red} 0.38}} &  {\bf {\color{red} 0.35}} &  {\bf {\color{red} 0.33}} &  {\bf {\color{red} 0.31}} \\ 
\hline 
 2039 &  64 &  65 ans 0 mois &  -5.00\% &  1041.68 &  {\bf 47.37} &  2199.08 &  {\bf {\color{red} 0.47}} &  {\bf {\color{red} 0.44}} &  {\bf {\color{red} 0.41}} &  {\bf {\color{red} 0.39}} &  {\bf {\color{red} 0.36}} &  {\bf {\color{red} 0.34}} \\ 
\hline 
 2040 &  65 &  65 ans 1 mois &  -0.42\% &  1893.52 &  {\bf 85.00} &  2227.67 &  {\bf {\color{red} 0.85}} &  {\bf {\color{red} 0.80}} &  {\bf {\color{red} 0.75}} &  {\bf {\color{red} 0.70}} &  {\bf {\color{red} 0.66}} &  {\bf {\color{red} 0.62}} \\ 
\hline 
 2041 &  66 &  65 ans 2 mois &  4.17\% &  1918.14 &  {\bf 85.00} &  2256.63 &  {\bf {\color{red} 0.85}} &  {\bf {\color{red} 0.81}} &  {\bf {\color{red} 0.76}} &  {\bf {\color{red} 0.71}} &  {\bf {\color{red} 0.67}} &  {\bf {\color{red} 0.62}} \\ 
\hline 
 2042 &  67 &  65 ans 3 mois &  8.75\% &  1943.07 &  {\bf 85.00} &  2285.97 &  {\bf {\color{red} 0.85}} &  {\bf {\color{red} 0.82}} &  {\bf {\color{red} 0.77}} &  {\bf {\color{red} 0.72}} &  {\bf {\color{red} 0.67}} &  {\bf {\color{red} 0.63}} \\ 
\hline 
\hline 
\end{tabular} 
\end{center} } 
\paragraph{Retraites possibles et ratios Revenu/SMIC à 70, 75, 80, 85, 90 ans avec le modèle \emph{Destinie2 (revalorisation de la fonction publique)}}  
 
{ \scriptsize \begin{center} 
\begin{tabular}[htb]{|c|c||c|c||c|c||c||c|c|c|c|c|c|} 
\hline 
 Retraite en &  Âge &  Âge pivot &  Décote/Surcote &  Retraite (\euro{} 2019) &  Tx Rempl(\%) &  SMIC (\euro{} 2019) &  Retraite/SMIC &  Rev70/SMIC &  Rev75/SMIC &  Rev80/SMIC &  Rev85/SMIC &  Rev90/SMIC \\ 
\hline \hline 
 2037 &  62 &  64 ans 10 mois &  -14.17\% &  853.26 &  {\bf 42.35} &  2014.82 &  {\bf {\color{red} 0.42}} &  {\bf {\color{red} 0.38}} &  {\bf {\color{red} 0.36}} &  {\bf {\color{red} 0.34}} &  {\bf {\color{red} 0.31}} &  {\bf {\color{red} 0.29}} \\ 
\hline 
 2038 &  63 &  64 ans 11 mois &  -9.58\% &  931.37 &  {\bf 45.63} &  2041.01 &  {\bf {\color{red} 0.46}} &  {\bf {\color{red} 0.42}} &  {\bf {\color{red} 0.39}} &  {\bf {\color{red} 0.37}} &  {\bf {\color{red} 0.34}} &  {\bf {\color{red} 0.32}} \\ 
\hline 
 2039 &  64 &  65 ans 0 mois &  -5.00\% &  1014.13 &  {\bf 49.05} &  2067.55 &  {\bf {\color{red} 0.49}} &  {\bf {\color{red} 0.45}} &  {\bf {\color{red} 0.43}} &  {\bf {\color{red} 0.40}} &  {\bf {\color{red} 0.37}} &  {\bf {\color{red} 0.35}} \\ 
\hline 
 2040 &  65 &  65 ans 1 mois &  -0.42\% &  1780.26 &  {\bf 85.00} &  2094.43 &  {\bf {\color{red} 0.85}} &  {\bf {\color{red} 0.80}} &  {\bf {\color{red} 0.75}} &  {\bf {\color{red} 0.70}} &  {\bf {\color{red} 0.66}} &  {\bf {\color{red} 0.62}} \\ 
\hline 
 2041 &  66 &  65 ans 2 mois &  4.17\% &  1803.40 &  {\bf 85.00} &  2121.65 &  {\bf {\color{red} 0.85}} &  {\bf {\color{red} 0.81}} &  {\bf {\color{red} 0.76}} &  {\bf {\color{red} 0.71}} &  {\bf {\color{red} 0.67}} &  {\bf {\color{red} 0.62}} \\ 
\hline 
 2042 &  67 &  65 ans 3 mois &  8.75\% &  1826.85 &  {\bf 85.00} &  2149.23 &  {\bf {\color{red} 0.85}} &  {\bf {\color{red} 0.82}} &  {\bf {\color{red} 0.77}} &  {\bf {\color{red} 0.72}} &  {\bf {\color{red} 0.67}} &  {\bf {\color{red} 0.63}} \\ 
\hline 
\hline 
\end{tabular} 
\end{center} } 

 \begin{center}\includegraphics[width=0.9\textwidth]{fig/SMIC_1975_22_dest_retraite.pdf}\end{center} \label{fig/SMIC_1975_22_dest_retraite.pdf} 

\newpage 
 
\subsection{Génération 1980 (début en 2002)} 

\paragraph{Retraites possibles et ratios Revenu/SMIC à 70, 75, 80, 85, 90 ans avec le modèle \emph{Gouvernement truqué (âge-pivot bloqué à 65 ans)}}  
 
{ \scriptsize \begin{center} 
\begin{tabular}[htb]{|c|c||c|c||c|c||c||c|c|c|c|c|c|} 
\hline 
 Retraite en &  Âge &  Âge pivot &  Décote/Surcote &  Retraite (\euro{} 2019) &  Tx Rempl(\%) &  SMIC (\euro{} 2019) &  Retraite/SMIC &  Rev70/SMIC &  Rev75/SMIC &  Rev80/SMIC &  Rev85/SMIC &  Rev90/SMIC \\ 
\hline \hline 
 2042 &  62 &  65 ans 0 mois &  -15.00\% &  951.68 &  {\bf 41.63} &  2285.97 &  {\bf {\color{red} 0.42}} &  {\bf {\color{red} 0.38}} &  {\bf {\color{red} 0.35}} &  {\bf {\color{red} 0.33}} &  {\bf {\color{red} 0.31}} &  {\bf {\color{red} 0.29}} \\ 
\hline 
 2043 &  63 &  65 ans 0 mois &  -10.00\% &  1048.22 &  {\bf 45.27} &  2315.68 &  {\bf {\color{red} 0.45}} &  {\bf {\color{red} 0.41}} &  {\bf {\color{red} 0.39}} &  {\bf {\color{red} 0.36}} &  {\bf {\color{red} 0.34}} &  {\bf {\color{red} 0.32}} \\ 
\hline 
 2044 &  64 &  65 ans 0 mois &  -5.00\% &  1151.01 &  {\bf 49.07} &  2345.79 &  {\bf {\color{red} 0.49}} &  {\bf {\color{red} 0.45}} &  {\bf {\color{red} 0.43}} &  {\bf {\color{red} 0.40}} &  {\bf {\color{red} 0.37}} &  {\bf {\color{red} 0.35}} \\ 
\hline 
 2045 &  65 &  65 ans 0 mois &  0.00\% &  2019.84 &  {\bf 85.00} &  2376.28 &  {\bf {\color{red} 0.85}} &  {\bf {\color{red} 0.80}} &  {\bf {\color{red} 0.75}} &  {\bf {\color{red} 0.70}} &  {\bf {\color{red} 0.66}} &  {\bf {\color{red} 0.62}} \\ 
\hline 
 2046 &  66 &  65 ans 0 mois &  5.00\% &  2046.10 &  {\bf 85.00} &  2407.18 &  {\bf {\color{red} 0.85}} &  {\bf {\color{red} 0.81}} &  {\bf {\color{red} 0.76}} &  {\bf {\color{red} 0.71}} &  {\bf {\color{red} 0.67}} &  {\bf {\color{red} 0.62}} \\ 
\hline 
 2047 &  67 &  65 ans 0 mois &  10.00\% &  2072.70 &  {\bf 85.00} &  2438.47 &  {\bf {\color{red} 0.85}} &  {\bf {\color{red} 0.82}} &  {\bf {\color{red} 0.77}} &  {\bf {\color{red} 0.72}} &  {\bf {\color{red} 0.67}} &  {\bf {\color{red} 0.63}} \\ 
\hline 
\hline 
\end{tabular} 
\end{center} } 
\paragraph{Retraites possibles et ratios Revenu/SMIC à 70, 75, 80, 85, 90 ans avec le modèle \emph{Gouvernement corrigé (âge-pivot glissant)}}  
 
{ \scriptsize \begin{center} 
\begin{tabular}[htb]{|c|c||c|c||c|c||c||c|c|c|c|c|c|} 
\hline 
 Retraite en &  Âge &  Âge pivot &  Décote/Surcote &  Retraite (\euro{} 2019) &  Tx Rempl(\%) &  SMIC (\euro{} 2019) &  Retraite/SMIC &  Rev70/SMIC &  Rev75/SMIC &  Rev80/SMIC &  Rev85/SMIC &  Rev90/SMIC \\ 
\hline \hline 
 2042 &  62 &  65 ans 3 mois &  -16.25\% &  937.69 &  {\bf 41.02} &  2285.97 &  {\bf {\color{red} 0.41}} &  {\bf {\color{red} 0.37}} &  {\bf {\color{red} 0.35}} &  {\bf {\color{red} 0.33}} &  {\bf {\color{red} 0.30}} &  {\bf {\color{red} 0.29}} \\ 
\hline 
 2043 &  63 &  65 ans 4 mois &  -11.67\% &  1028.81 &  {\bf 44.43} &  2315.68 &  {\bf {\color{red} 0.44}} &  {\bf {\color{red} 0.41}} &  {\bf {\color{red} 0.38}} &  {\bf {\color{red} 0.36}} &  {\bf {\color{red} 0.33}} &  {\bf {\color{red} 0.31}} \\ 
\hline 
 2044 &  64 &  65 ans 5 mois &  -7.08\% &  1125.77 &  {\bf 47.99} &  2345.79 &  {\bf {\color{red} 0.48}} &  {\bf {\color{red} 0.44}} &  {\bf {\color{red} 0.42}} &  {\bf {\color{red} 0.39}} &  {\bf {\color{red} 0.37}} &  {\bf {\color{red} 0.34}} \\ 
\hline 
 2045 &  65 &  65 ans 6 mois &  -2.50\% &  2019.84 &  {\bf 85.00} &  2376.28 &  {\bf {\color{red} 0.85}} &  {\bf {\color{red} 0.80}} &  {\bf {\color{red} 0.75}} &  {\bf {\color{red} 0.70}} &  {\bf {\color{red} 0.66}} &  {\bf {\color{red} 0.62}} \\ 
\hline 
 2046 &  66 &  65 ans 7 mois &  2.08\% &  2046.10 &  {\bf 85.00} &  2407.18 &  {\bf {\color{red} 0.85}} &  {\bf {\color{red} 0.81}} &  {\bf {\color{red} 0.76}} &  {\bf {\color{red} 0.71}} &  {\bf {\color{red} 0.67}} &  {\bf {\color{red} 0.62}} \\ 
\hline 
 2047 &  67 &  65 ans 8 mois &  6.67\% &  2072.70 &  {\bf 85.00} &  2438.47 &  {\bf {\color{red} 0.85}} &  {\bf {\color{red} 0.82}} &  {\bf {\color{red} 0.77}} &  {\bf {\color{red} 0.72}} &  {\bf {\color{red} 0.67}} &  {\bf {\color{red} 0.63}} \\ 
\hline 
\hline 
\end{tabular} 
\end{center} } 
\paragraph{Retraites possibles et ratios Revenu/SMIC à 70, 75, 80, 85, 90 ans avec le modèle \emph{Destinie2 (revalorisation de la fonction publique)}}  
 
{ \scriptsize \begin{center} 
\begin{tabular}[htb]{|c|c||c|c||c|c||c||c|c|c|c|c|c|} 
\hline 
 Retraite en &  Âge &  Âge pivot &  Décote/Surcote &  Retraite (\euro{} 2019) &  Tx Rempl(\%) &  SMIC (\euro{} 2019) &  Retraite/SMIC &  Rev70/SMIC &  Rev75/SMIC &  Rev80/SMIC &  Rev85/SMIC &  Rev90/SMIC \\ 
\hline \hline 
 2042 &  62 &  65 ans 3 mois &  -16.25\% &  907.92 &  {\bf 42.24} &  2149.23 &  {\bf {\color{red} 0.42}} &  {\bf {\color{red} 0.38}} &  {\bf {\color{red} 0.36}} &  {\bf {\color{red} 0.33}} &  {\bf {\color{red} 0.31}} &  {\bf {\color{red} 0.29}} \\ 
\hline 
 2043 &  63 &  65 ans 4 mois &  -11.67\% &  995.35 &  {\bf 45.72} &  2177.17 &  {\bf {\color{red} 0.46}} &  {\bf {\color{red} 0.42}} &  {\bf {\color{red} 0.39}} &  {\bf {\color{red} 0.37}} &  {\bf {\color{red} 0.34}} &  {\bf {\color{red} 0.32}} \\ 
\hline 
 2044 &  64 &  65 ans 5 mois &  -7.08\% &  1088.32 &  {\bf 49.35} &  2205.48 &  {\bf {\color{red} 0.49}} &  {\bf {\color{red} 0.46}} &  {\bf {\color{red} 0.43}} &  {\bf {\color{red} 0.40}} &  {\bf {\color{red} 0.38}} &  {\bf {\color{red} 0.35}} \\ 
\hline 
 2045 &  65 &  65 ans 6 mois &  -2.50\% &  1899.03 &  {\bf 85.00} &  2234.15 &  {\bf {\color{red} 0.85}} &  {\bf {\color{red} 0.80}} &  {\bf {\color{red} 0.75}} &  {\bf {\color{red} 0.70}} &  {\bf {\color{red} 0.66}} &  {\bf {\color{red} 0.62}} \\ 
\hline 
 2046 &  66 &  65 ans 7 mois &  2.08\% &  1923.71 &  {\bf 85.00} &  2263.19 &  {\bf {\color{red} 0.85}} &  {\bf {\color{red} 0.81}} &  {\bf {\color{red} 0.76}} &  {\bf {\color{red} 0.71}} &  {\bf {\color{red} 0.67}} &  {\bf {\color{red} 0.62}} \\ 
\hline 
 2047 &  67 &  65 ans 8 mois &  6.67\% &  1948.72 &  {\bf 85.00} &  2292.61 &  {\bf {\color{red} 0.85}} &  {\bf {\color{red} 0.82}} &  {\bf {\color{red} 0.77}} &  {\bf {\color{red} 0.72}} &  {\bf {\color{red} 0.67}} &  {\bf {\color{red} 0.63}} \\ 
\hline 
\hline 
\end{tabular} 
\end{center} } 

 \begin{center}\includegraphics[width=0.9\textwidth]{fig/SMIC_1980_22_dest_retraite.pdf}\end{center} \label{fig/SMIC_1980_22_dest_retraite.pdf} 

\newpage 
 
\subsection{Génération 1990 (début en 2012)} 

\paragraph{Retraites possibles et ratios Revenu/SMIC à 70, 75, 80, 85, 90 ans avec le modèle \emph{Gouvernement truqué (âge-pivot bloqué à 65 ans)}}  
 
{ \scriptsize \begin{center} 
\begin{tabular}[htb]{|c|c||c|c||c|c||c||c|c|c|c|c|c|} 
\hline 
 Retraite en &  Âge &  Âge pivot &  Décote/Surcote &  Retraite (\euro{} 2019) &  Tx Rempl(\%) &  SMIC (\euro{} 2019) &  Retraite/SMIC &  Rev70/SMIC &  Rev75/SMIC &  Rev80/SMIC &  Rev85/SMIC &  Rev90/SMIC \\ 
\hline \hline 
 2052 &  62 &  65 ans 0 mois &  -15.00\% &  1156.48 &  {\bf 44.46} &  2601.14 &  {\bf {\color{red} 0.44}} &  {\bf {\color{red} 0.40}} &  {\bf {\color{red} 0.38}} &  {\bf {\color{red} 0.35}} &  {\bf {\color{red} 0.33}} &  {\bf {\color{red} 0.31}} \\ 
\hline 
 2053 &  63 &  65 ans 0 mois &  -10.00\% &  1273.44 &  {\bf 48.33} &  2634.96 &  {\bf {\color{red} 0.48}} &  {\bf {\color{red} 0.44}} &  {\bf {\color{red} 0.41}} &  {\bf {\color{red} 0.39}} &  {\bf {\color{red} 0.36}} &  {\bf {\color{red} 0.34}} \\ 
\hline 
 2054 &  64 &  65 ans 0 mois &  -5.00\% &  1396.95 &  {\bf 52.34} &  2669.21 &  {\bf {\color{red} 0.52}} &  {\bf {\color{red} 0.48}} &  {\bf {\color{red} 0.45}} &  {\bf {\color{red} 0.43}} &  {\bf {\color{red} 0.40}} &  {\bf {\color{red} 0.37}} \\ 
\hline 
 2055 &  65 &  65 ans 0 mois &  0.00\% &  2298.33 &  {\bf 85.00} &  2703.91 &  {\bf {\color{red} 0.85}} &  {\bf {\color{red} 0.80}} &  {\bf {\color{red} 0.75}} &  {\bf {\color{red} 0.70}} &  {\bf {\color{red} 0.66}} &  {\bf {\color{red} 0.62}} \\ 
\hline 
 2056 &  66 &  65 ans 0 mois &  5.00\% &  2328.20 &  {\bf 85.00} &  2739.06 &  {\bf {\color{red} 0.85}} &  {\bf {\color{red} 0.81}} &  {\bf {\color{red} 0.76}} &  {\bf {\color{red} 0.71}} &  {\bf {\color{red} 0.67}} &  {\bf {\color{red} 0.62}} \\ 
\hline 
 2057 &  67 &  65 ans 0 mois &  10.00\% &  2358.47 &  {\bf 85.00} &  2774.67 &  {\bf {\color{red} 0.85}} &  {\bf {\color{red} 0.82}} &  {\bf {\color{red} 0.77}} &  {\bf {\color{red} 0.72}} &  {\bf {\color{red} 0.67}} &  {\bf {\color{red} 0.63}} \\ 
\hline 
\hline 
\end{tabular} 
\end{center} } 
\paragraph{Retraites possibles et ratios Revenu/SMIC à 70, 75, 80, 85, 90 ans avec le modèle \emph{Gouvernement corrigé (âge-pivot glissant)}}  
 
{ \scriptsize \begin{center} 
\begin{tabular}[htb]{|c|c||c|c||c|c||c||c|c|c|c|c|c|} 
\hline 
 Retraite en &  Âge &  Âge pivot &  Décote/Surcote &  Retraite (\euro{} 2019) &  Tx Rempl(\%) &  SMIC (\euro{} 2019) &  Retraite/SMIC &  Rev70/SMIC &  Rev75/SMIC &  Rev80/SMIC &  Rev85/SMIC &  Rev90/SMIC \\ 
\hline \hline 
 2052 &  62 &  66 ans 1 mois &  -20.42\% &  1082.78 &  {\bf 41.63} &  2601.14 &  {\bf {\color{red} 0.42}} &  {\bf {\color{red} 0.38}} &  {\bf {\color{red} 0.35}} &  {\bf {\color{red} 0.33}} &  {\bf {\color{red} 0.31}} &  {\bf {\color{red} 0.29}} \\ 
\hline 
 2053 &  63 &  66 ans 2 mois &  -15.83\% &  1190.90 &  {\bf 45.20} &  2634.96 &  {\bf {\color{red} 0.45}} &  {\bf {\color{red} 0.41}} &  {\bf {\color{red} 0.39}} &  {\bf {\color{red} 0.36}} &  {\bf {\color{red} 0.34}} &  {\bf {\color{red} 0.32}} \\ 
\hline 
 2054 &  64 &  66 ans 3 mois &  -11.25\% &  1305.05 &  {\bf 48.89} &  2669.21 &  {\bf {\color{red} 0.49}} &  {\bf {\color{red} 0.45}} &  {\bf {\color{red} 0.42}} &  {\bf {\color{red} 0.40}} &  {\bf {\color{red} 0.37}} &  {\bf {\color{red} 0.35}} \\ 
\hline 
 2055 &  65 &  66 ans 4 mois &  -6.67\% &  2298.33 &  {\bf 85.00} &  2703.91 &  {\bf {\color{red} 0.85}} &  {\bf {\color{red} 0.80}} &  {\bf {\color{red} 0.75}} &  {\bf {\color{red} 0.70}} &  {\bf {\color{red} 0.66}} &  {\bf {\color{red} 0.62}} \\ 
\hline 
 2056 &  66 &  66 ans 5 mois &  -2.08\% &  2328.20 &  {\bf 85.00} &  2739.06 &  {\bf {\color{red} 0.85}} &  {\bf {\color{red} 0.81}} &  {\bf {\color{red} 0.76}} &  {\bf {\color{red} 0.71}} &  {\bf {\color{red} 0.67}} &  {\bf {\color{red} 0.62}} \\ 
\hline 
 2057 &  67 &  66 ans 6 mois &  2.50\% &  2358.47 &  {\bf 85.00} &  2774.67 &  {\bf {\color{red} 0.85}} &  {\bf {\color{red} 0.82}} &  {\bf {\color{red} 0.77}} &  {\bf {\color{red} 0.72}} &  {\bf {\color{red} 0.67}} &  {\bf {\color{red} 0.63}} \\ 
\hline 
\hline 
\end{tabular} 
\end{center} } 
\paragraph{Retraites possibles et ratios Revenu/SMIC à 70, 75, 80, 85, 90 ans avec le modèle \emph{Destinie2 (revalorisation de la fonction publique)}}  
 
{ \scriptsize \begin{center} 
\begin{tabular}[htb]{|c|c||c|c||c|c||c||c|c|c|c|c|c|} 
\hline 
 Retraite en &  Âge &  Âge pivot &  Décote/Surcote &  Retraite (\euro{} 2019) &  Tx Rempl(\%) &  SMIC (\euro{} 2019) &  Retraite/SMIC &  Rev70/SMIC &  Rev75/SMIC &  Rev80/SMIC &  Rev85/SMIC &  Rev90/SMIC \\ 
\hline \hline 
 2052 &  62 &  66 ans 1 mois &  -20.42\% &  1033.47 &  {\bf 42.26} &  2445.56 &  {\bf {\color{red} 0.42}} &  {\bf {\color{red} 0.38}} &  {\bf {\color{red} 0.36}} &  {\bf {\color{red} 0.33}} &  {\bf {\color{red} 0.31}} &  {\bf {\color{red} 0.29}} \\ 
\hline 
 2053 &  63 &  66 ans 2 mois &  -15.83\% &  1136.22 &  {\bf 45.86} &  2477.35 &  {\bf {\color{red} 0.46}} &  {\bf {\color{red} 0.42}} &  {\bf {\color{red} 0.39}} &  {\bf {\color{red} 0.37}} &  {\bf {\color{red} 0.35}} &  {\bf {\color{red} 0.32}} \\ 
\hline 
 2054 &  64 &  66 ans 3 mois &  -11.25\% &  1244.67 &  {\bf 49.60} &  2509.56 &  {\bf {\color{red} 0.50}} &  {\bf {\color{red} 0.46}} &  {\bf {\color{red} 0.43}} &  {\bf {\color{red} 0.40}} &  {\bf {\color{red} 0.38}} &  {\bf {\color{red} 0.35}} \\ 
\hline 
 2055 &  65 &  66 ans 4 mois &  -6.67\% &  2160.85 &  {\bf 85.00} &  2542.18 &  {\bf {\color{red} 0.85}} &  {\bf {\color{red} 0.80}} &  {\bf {\color{red} 0.75}} &  {\bf {\color{red} 0.70}} &  {\bf {\color{red} 0.66}} &  {\bf {\color{red} 0.62}} \\ 
\hline 
 2056 &  66 &  66 ans 5 mois &  -2.08\% &  2188.95 &  {\bf 85.00} &  2575.23 &  {\bf {\color{red} 0.85}} &  {\bf {\color{red} 0.81}} &  {\bf {\color{red} 0.76}} &  {\bf {\color{red} 0.71}} &  {\bf {\color{red} 0.67}} &  {\bf {\color{red} 0.62}} \\ 
\hline 
 2057 &  67 &  66 ans 6 mois &  2.50\% &  2217.40 &  {\bf 85.00} &  2608.71 &  {\bf {\color{red} 0.85}} &  {\bf {\color{red} 0.82}} &  {\bf {\color{red} 0.77}} &  {\bf {\color{red} 0.72}} &  {\bf {\color{red} 0.67}} &  {\bf {\color{red} 0.63}} \\ 
\hline 
\hline 
\end{tabular} 
\end{center} } 

 \begin{center}\includegraphics[width=0.9\textwidth]{fig/SMIC_1990_22_dest_retraite.pdf}\end{center} \label{fig/SMIC_1990_22_dest_retraite.pdf} 

\newpage 
 
\subsection{Génération 2003 (début en 2025)} 

\paragraph{Retraites possibles et ratios Revenu/SMIC à 70, 75, 80, 85, 90 ans avec le modèle \emph{Gouvernement truqué (âge-pivot bloqué à 65 ans)}}  
 
{ \scriptsize \begin{center} 
\begin{tabular}[htb]{|c|c||c|c||c|c||c||c|c|c|c|c|c|} 
\hline 
 Retraite en &  Âge &  Âge pivot &  Décote/Surcote &  Retraite (\euro{} 2019) &  Tx Rempl(\%) &  SMIC (\euro{} 2019) &  Retraite/SMIC &  Rev70/SMIC &  Rev75/SMIC &  Rev80/SMIC &  Rev85/SMIC &  Rev90/SMIC \\ 
\hline \hline 
 2065 &  62 &  65 ans 0 mois &  -15.00\% &  1457.41 &  {\bf 47.37} &  3076.71 &  {\bf {\color{red} 0.47}} &  {\bf {\color{red} 0.43}} &  {\bf {\color{red} 0.40}} &  {\bf {\color{red} 0.38}} &  {\bf {\color{red} 0.35}} &  {\bf {\color{red} 0.33}} \\ 
\hline 
 2066 &  63 &  65 ans 0 mois &  -10.00\% &  1602.25 &  {\bf 51.41} &  3116.71 &  {\bf {\color{red} 0.51}} &  {\bf {\color{red} 0.47}} &  {\bf {\color{red} 0.44}} &  {\bf {\color{red} 0.41}} &  {\bf {\color{red} 0.39}} &  {\bf {\color{red} 0.36}} \\ 
\hline 
 2067 &  64 &  65 ans 0 mois &  -5.00\% &  1755.00 &  {\bf 55.59} &  3157.23 &  {\bf {\color{red} 0.56}} &  {\bf {\color{red} 0.51}} &  {\bf {\color{red} 0.48}} &  {\bf {\color{red} 0.45}} &  {\bf {\color{red} 0.42}} &  {\bf {\color{red} 0.40}} \\ 
\hline 
 2068 &  65 &  65 ans 0 mois &  0.00\% &  2718.53 &  {\bf 85.00} &  3198.27 &  {\bf {\color{red} 0.85}} &  {\bf {\color{red} 0.80}} &  {\bf {\color{red} 0.75}} &  {\bf {\color{red} 0.70}} &  {\bf {\color{red} 0.66}} &  {\bf {\color{red} 0.62}} \\ 
\hline 
 2069 &  66 &  65 ans 0 mois &  5.00\% &  2753.87 &  {\bf 85.00} &  3239.85 &  {\bf {\color{red} 0.85}} &  {\bf {\color{red} 0.81}} &  {\bf {\color{red} 0.76}} &  {\bf {\color{red} 0.71}} &  {\bf {\color{red} 0.67}} &  {\bf {\color{red} 0.62}} \\ 
\hline 
 2070 &  67 &  65 ans 0 mois &  10.00\% &  2789.67 &  {\bf 85.00} &  3281.97 &  {\bf {\color{red} 0.85}} &  {\bf {\color{red} 0.82}} &  {\bf {\color{red} 0.77}} &  {\bf {\color{red} 0.72}} &  {\bf {\color{red} 0.67}} &  {\bf {\color{red} 0.63}} \\ 
\hline 
\hline 
\end{tabular} 
\end{center} } 
\paragraph{Retraites possibles et ratios Revenu/SMIC à 70, 75, 80, 85, 90 ans avec le modèle \emph{Gouvernement corrigé (âge-pivot glissant)}}  
 
{ \scriptsize \begin{center} 
\begin{tabular}[htb]{|c|c||c|c||c|c||c||c|c|c|c|c|c|} 
\hline 
 Retraite en &  Âge &  Âge pivot &  Décote/Surcote &  Retraite (\euro{} 2019) &  Tx Rempl(\%) &  SMIC (\euro{} 2019) &  Retraite/SMIC &  Rev70/SMIC &  Rev75/SMIC &  Rev80/SMIC &  Rev85/SMIC &  Rev90/SMIC \\ 
\hline \hline 
 2065 &  62 &  67 ans 2 mois &  -25.83\% &  1271.67 &  {\bf 41.33} &  3076.71 &  {\bf {\color{red} 0.41}} &  {\bf {\color{red} 0.37}} &  {\bf {\color{red} 0.35}} &  {\bf {\color{red} 0.33}} &  {\bf {\color{red} 0.31}} &  {\bf {\color{red} 0.29}} \\ 
\hline 
 2066 &  63 &  67 ans 3 mois &  -21.25\% &  1401.97 &  {\bf 44.98} &  3116.71 &  {\bf {\color{red} 0.45}} &  {\bf {\color{red} 0.41}} &  {\bf {\color{red} 0.39}} &  {\bf {\color{red} 0.36}} &  {\bf {\color{red} 0.34}} &  {\bf {\color{red} 0.32}} \\ 
\hline 
 2067 &  64 &  67 ans 4 mois &  -16.67\% &  1539.47 &  {\bf 48.76} &  3157.23 &  {\bf {\color{red} 0.49}} &  {\bf {\color{red} 0.45}} &  {\bf {\color{red} 0.42}} &  {\bf {\color{red} 0.40}} &  {\bf {\color{red} 0.37}} &  {\bf {\color{red} 0.35}} \\ 
\hline 
 2068 &  65 &  67 ans 5 mois &  -12.08\% &  2718.53 &  {\bf 85.00} &  3198.27 &  {\bf {\color{red} 0.85}} &  {\bf {\color{red} 0.80}} &  {\bf {\color{red} 0.75}} &  {\bf {\color{red} 0.70}} &  {\bf {\color{red} 0.66}} &  {\bf {\color{red} 0.62}} \\ 
\hline 
 2069 &  66 &  67 ans 6 mois &  -7.50\% &  2753.87 &  {\bf 85.00} &  3239.85 &  {\bf {\color{red} 0.85}} &  {\bf {\color{red} 0.81}} &  {\bf {\color{red} 0.76}} &  {\bf {\color{red} 0.71}} &  {\bf {\color{red} 0.67}} &  {\bf {\color{red} 0.62}} \\ 
\hline 
 2070 &  67 &  67 ans 7 mois &  -2.92\% &  2789.67 &  {\bf 85.00} &  3281.97 &  {\bf {\color{red} 0.85}} &  {\bf {\color{red} 0.82}} &  {\bf {\color{red} 0.77}} &  {\bf {\color{red} 0.72}} &  {\bf {\color{red} 0.67}} &  {\bf {\color{red} 0.63}} \\ 
\hline 
\hline 
\end{tabular} 
\end{center} } 
\paragraph{Retraites possibles et ratios Revenu/SMIC à 70, 75, 80, 85, 90 ans avec le modèle \emph{Destinie2 (revalorisation de la fonction publique)}}  
 
{ \scriptsize \begin{center} 
\begin{tabular}[htb]{|c|c||c|c||c|c||c||c|c|c|c|c|c|} 
\hline 
 Retraite en &  Âge &  Âge pivot &  Décote/Surcote &  Retraite (\euro{} 2019) &  Tx Rempl(\%) &  SMIC (\euro{} 2019) &  Retraite/SMIC &  Rev70/SMIC &  Rev75/SMIC &  Rev80/SMIC &  Rev85/SMIC &  Rev90/SMIC \\ 
\hline \hline 
 2065 &  62 &  67 ans 2 mois &  -25.83\% &  1196.48 &  {\bf 41.36} &  2892.68 &  {\bf {\color{red} 0.41}} &  {\bf {\color{red} 0.37}} &  {\bf {\color{red} 0.35}} &  {\bf {\color{red} 0.33}} &  {\bf {\color{red} 0.31}} &  {\bf {\color{red} 0.29}} \\ 
\hline 
 2066 &  63 &  67 ans 3 mois &  -21.25\% &  1319.05 &  {\bf 45.01} &  2930.29 &  {\bf {\color{red} 0.45}} &  {\bf {\color{red} 0.41}} &  {\bf {\color{red} 0.39}} &  {\bf {\color{red} 0.36}} &  {\bf {\color{red} 0.34}} &  {\bf {\color{red} 0.32}} \\ 
\hline 
 2067 &  64 &  67 ans 4 mois &  -16.67\% &  1448.40 &  {\bf 48.79} &  2968.38 &  {\bf {\color{red} 0.49}} &  {\bf {\color{red} 0.45}} &  {\bf {\color{red} 0.42}} &  {\bf {\color{red} 0.40}} &  {\bf {\color{red} 0.37}} &  {\bf {\color{red} 0.35}} \\ 
\hline 
 2068 &  65 &  67 ans 5 mois &  -12.08\% &  2555.93 &  {\bf 85.00} &  3006.97 &  {\bf {\color{red} 0.85}} &  {\bf {\color{red} 0.80}} &  {\bf {\color{red} 0.75}} &  {\bf {\color{red} 0.70}} &  {\bf {\color{red} 0.66}} &  {\bf {\color{red} 0.62}} \\ 
\hline 
 2069 &  66 &  67 ans 6 mois &  -7.50\% &  2589.15 &  {\bf 85.00} &  3046.06 &  {\bf {\color{red} 0.85}} &  {\bf {\color{red} 0.81}} &  {\bf {\color{red} 0.76}} &  {\bf {\color{red} 0.71}} &  {\bf {\color{red} 0.67}} &  {\bf {\color{red} 0.62}} \\ 
\hline 
 2070 &  67 &  67 ans 7 mois &  -2.92\% &  2622.81 &  {\bf 85.00} &  3085.66 &  {\bf {\color{red} 0.85}} &  {\bf {\color{red} 0.82}} &  {\bf {\color{red} 0.77}} &  {\bf {\color{red} 0.72}} &  {\bf {\color{red} 0.67}} &  {\bf {\color{red} 0.63}} \\ 
\hline 
\hline 
\end{tabular} 
\end{center} } 

 \begin{center}\includegraphics[width=0.9\textwidth]{fig/SMIC_2003_22_dest_retraite.pdf}\end{center} \label{fig/SMIC_2003_22_dest_retraite.pdf} 

\newpage 
 
\chapter{Salarié privé évoluant du SMIC à 2*SMIC} 


 \addto{\captionsenglish}{ \renewcommand{\mtctitle}{}} \setcounter{minitocdepth}{2} 
 \minitoc \newpage 

\section{Début de carrière à 22 ans} 

\subsection{Génération 1975 (début en 1997)} 

\paragraph{Retraites possibles et ratios Revenu/SMIC à 70, 75, 80, 85, 90 ans avec le modèle \emph{Gouvernement truqué (âge-pivot bloqué à 65 ans)}}  
 
{ \scriptsize \begin{center} 
\begin{tabular}[htb]{|c|c||c|c||c|c||c||c|c|c|c|c|c|} 
\hline 
 Retraite en &  Âge &  Âge pivot &  Décote/Surcote &  Retraite (\euro{} 2019) &  Tx Rempl(\%) &  SMIC (\euro{} 2019) &  Retraite/SMIC &  Rev70/SMIC &  Rev75/SMIC &  Rev80/SMIC &  Rev85/SMIC &  Rev90/SMIC \\ 
\hline \hline 
 2037 &  62 &  64 ans 10 mois &  -14.17\% &  1308.79 &  {\bf 31.64} &  2143.00 &  {\bf {\color{red} 0.61}} &  {\bf {\color{red} 0.55}} &  {\bf {\color{red} 0.52}} &  {\bf {\color{red} 0.48}} &  {\bf {\color{red} 0.45}} &  {\bf {\color{red} 0.43}} \\ 
\hline 
 2038 &  63 &  64 ans 11 mois &  -9.58\% &  1442.56 &  {\bf 34.02} &  2170.86 &  {\bf {\color{red} 0.66}} &  {\bf {\color{red} 0.61}} &  {\bf {\color{red} 0.57}} &  {\bf {\color{red} 0.53}} &  {\bf {\color{red} 0.50}} &  {\bf {\color{red} 0.47}} \\ 
\hline 
 2039 &  64 &  65 ans 0 mois &  -5.00\% &  1585.89 &  {\bf 36.48} &  2199.08 &  {\bf {\color{red} 0.72}} &  {\bf {\color{red} 0.67}} &  {\bf {\color{red} 0.63}} &  {\bf {\color{red} 0.59}} &  {\bf {\color{red} 0.55}} &  {\bf {\color{red} 0.52}} \\ 
\hline 
 2040 &  65 &  65 ans 0 mois &  0.00\% &  1893.52 &  {\bf 42.50} &  2227.67 &  {\bf {\color{red} 0.85}} &  {\bf {\color{red} 0.80}} &  {\bf {\color{red} 0.75}} &  {\bf {\color{red} 0.70}} &  {\bf {\color{red} 0.66}} &  {\bf {\color{red} 0.62}} \\ 
\hline 
 2041 &  66 &  65 ans 0 mois &  5.00\% &  1918.93 &  {\bf 42.03} &  2256.63 &  {\bf {\color{red} 0.85}} &  {\bf {\color{red} 0.81}} &  {\bf {\color{red} 0.76}} &  {\bf {\color{red} 0.71}} &  {\bf {\color{red} 0.67}} &  {\bf {\color{red} 0.62}} \\ 
\hline 
 2042 &  67 &  65 ans 0 mois &  10.00\% &  2103.43 &  {\bf 44.96} &  2285.97 &  {\bf {\color{red} 0.92}} &  {\bf {\color{red} 0.89}} &  {\bf {\color{red} 0.83}} &  {\bf {\color{red} 0.78}} &  {\bf {\color{red} 0.73}} &  {\bf {\color{red} 0.68}} \\ 
\hline 
\hline 
\end{tabular} 
\end{center} } 
\paragraph{Retraites possibles et ratios Revenu/SMIC à 70, 75, 80, 85, 90 ans avec le modèle \emph{Gouvernement corrigé (âge-pivot glissant)}}  
 
{ \scriptsize \begin{center} 
\begin{tabular}[htb]{|c|c||c|c||c|c||c||c|c|c|c|c|c|} 
\hline 
 Retraite en &  Âge &  Âge pivot &  Décote/Surcote &  Retraite (\euro{} 2019) &  Tx Rempl(\%) &  SMIC (\euro{} 2019) &  Retraite/SMIC &  Rev70/SMIC &  Rev75/SMIC &  Rev80/SMIC &  Rev85/SMIC &  Rev90/SMIC \\ 
\hline \hline 
 2037 &  62 &  64 ans 10 mois &  -14.17\% &  1308.79 &  {\bf 31.64} &  2143.00 &  {\bf {\color{red} 0.61}} &  {\bf {\color{red} 0.55}} &  {\bf {\color{red} 0.52}} &  {\bf {\color{red} 0.48}} &  {\bf {\color{red} 0.45}} &  {\bf {\color{red} 0.43}} \\ 
\hline 
 2038 &  63 &  64 ans 11 mois &  -9.58\% &  1442.56 &  {\bf 34.02} &  2170.86 &  {\bf {\color{red} 0.66}} &  {\bf {\color{red} 0.61}} &  {\bf {\color{red} 0.57}} &  {\bf {\color{red} 0.53}} &  {\bf {\color{red} 0.50}} &  {\bf {\color{red} 0.47}} \\ 
\hline 
 2039 &  64 &  65 ans 0 mois &  -5.00\% &  1585.89 &  {\bf 36.48} &  2199.08 &  {\bf {\color{red} 0.72}} &  {\bf {\color{red} 0.67}} &  {\bf {\color{red} 0.63}} &  {\bf {\color{red} 0.59}} &  {\bf {\color{red} 0.55}} &  {\bf {\color{red} 0.52}} \\ 
\hline 
 2040 &  65 &  65 ans 1 mois &  -0.42\% &  1893.52 &  {\bf 42.50} &  2227.67 &  {\bf {\color{red} 0.85}} &  {\bf {\color{red} 0.80}} &  {\bf {\color{red} 0.75}} &  {\bf {\color{red} 0.70}} &  {\bf {\color{red} 0.66}} &  {\bf {\color{red} 0.62}} \\ 
\hline 
 2041 &  66 &  65 ans 2 mois &  4.17\% &  1918.14 &  {\bf 42.01} &  2256.63 &  {\bf {\color{red} 0.85}} &  {\bf {\color{red} 0.81}} &  {\bf {\color{red} 0.76}} &  {\bf {\color{red} 0.71}} &  {\bf {\color{red} 0.67}} &  {\bf {\color{red} 0.62}} \\ 
\hline 
 2042 &  67 &  65 ans 3 mois &  8.75\% &  2079.53 &  {\bf 44.45} &  2285.97 &  {\bf {\color{red} 0.91}} &  {\bf {\color{red} 0.88}} &  {\bf {\color{red} 0.82}} &  {\bf {\color{red} 0.77}} &  {\bf {\color{red} 0.72}} &  {\bf {\color{red} 0.68}} \\ 
\hline 
\hline 
\end{tabular} 
\end{center} } 
\paragraph{Retraites possibles et ratios Revenu/SMIC à 70, 75, 80, 85, 90 ans avec le modèle \emph{Destinie2 (revalorisation de la fonction publique)}}  
 
{ \scriptsize \begin{center} 
\begin{tabular}[htb]{|c|c||c|c||c|c||c||c|c|c|c|c|c|} 
\hline 
 Retraite en &  Âge &  Âge pivot &  Décote/Surcote &  Retraite (\euro{} 2019) &  Tx Rempl(\%) &  SMIC (\euro{} 2019) &  Retraite/SMIC &  Rev70/SMIC &  Rev75/SMIC &  Rev80/SMIC &  Rev85/SMIC &  Rev90/SMIC \\ 
\hline \hline 
 2037 &  62 &  64 ans 10 mois &  -14.17\% &  1270.97 &  {\bf 32.68} &  2014.82 &  {\bf {\color{red} 0.63}} &  {\bf {\color{red} 0.57}} &  {\bf {\color{red} 0.53}} &  {\bf {\color{red} 0.50}} &  {\bf {\color{red} 0.47}} &  {\bf {\color{red} 0.44}} \\ 
\hline 
 2038 &  63 &  64 ans 11 mois &  -9.58\% &  1399.23 &  {\bf 35.09} &  2041.01 &  {\bf {\color{red} 0.69}} &  {\bf {\color{red} 0.63}} &  {\bf {\color{red} 0.59}} &  {\bf {\color{red} 0.55}} &  {\bf {\color{red} 0.52}} &  {\bf {\color{red} 0.48}} \\ 
\hline 
 2039 &  64 &  65 ans 0 mois &  -5.00\% &  1536.54 &  {\bf 37.60} &  2067.55 &  {\bf {\color{red} 0.74}} &  {\bf {\color{red} 0.69}} &  {\bf {\color{red} 0.64}} &  {\bf {\color{red} 0.60}} &  {\bf {\color{red} 0.57}} &  {\bf {\color{red} 0.53}} \\ 
\hline 
 2040 &  65 &  65 ans 1 mois &  -0.42\% &  1780.26 &  {\bf 42.50} &  2094.43 &  {\bf {\color{red} 0.85}} &  {\bf {\color{red} 0.80}} &  {\bf {\color{red} 0.75}} &  {\bf {\color{red} 0.70}} &  {\bf {\color{red} 0.66}} &  {\bf {\color{red} 0.62}} \\ 
\hline 
 2041 &  66 &  65 ans 2 mois &  4.17\% &  1840.69 &  {\bf 42.88} &  2121.65 &  {\bf {\color{red} 0.87}} &  {\bf {\color{red} 0.82}} &  {\bf {\color{red} 0.77}} &  {\bf {\color{red} 0.72}} &  {\bf {\color{red} 0.68}} &  {\bf {\color{red} 0.64}} \\ 
\hline 
 2042 &  67 &  65 ans 3 mois &  8.75\% &  2008.81 &  {\bf 45.67} &  2149.23 &  {\bf {\color{red} 0.93}} &  {\bf {\color{red} 0.90}} &  {\bf {\color{red} 0.84}} &  {\bf {\color{red} 0.79}} &  {\bf {\color{red} 0.74}} &  {\bf {\color{red} 0.69}} \\ 
\hline 
\hline 
\end{tabular} 
\end{center} } 

 \begin{center}\includegraphics[width=0.9\textwidth]{fig/Ascendant12_1975_22_dest_retraite.pdf}\end{center} \label{fig/Ascendant12_1975_22_dest_retraite.pdf} 

\newpage 
 
\subsection{Génération 1980 (début en 2002)} 

\paragraph{Retraites possibles et ratios Revenu/SMIC à 70, 75, 80, 85, 90 ans avec le modèle \emph{Gouvernement truqué (âge-pivot bloqué à 65 ans)}}  
 
{ \scriptsize \begin{center} 
\begin{tabular}[htb]{|c|c||c|c||c|c||c||c|c|c|c|c|c|} 
\hline 
 Retraite en &  Âge &  Âge pivot &  Décote/Surcote &  Retraite (\euro{} 2019) &  Tx Rempl(\%) &  SMIC (\euro{} 2019) &  Retraite/SMIC &  Rev70/SMIC &  Rev75/SMIC &  Rev80/SMIC &  Rev85/SMIC &  Rev90/SMIC \\ 
\hline \hline 
 2042 &  62 &  65 ans 0 mois &  -15.00\% &  1421.20 &  {\bf 32.21} &  2285.97 &  {\bf {\color{red} 0.62}} &  {\bf {\color{red} 0.56}} &  {\bf {\color{red} 0.53}} &  {\bf {\color{red} 0.49}} &  {\bf {\color{red} 0.46}} &  {\bf {\color{red} 0.43}} \\ 
\hline 
 2043 &  63 &  65 ans 0 mois &  -10.00\% &  1578.72 &  {\bf 34.90} &  2315.68 &  {\bf {\color{red} 0.68}} &  {\bf {\color{red} 0.62}} &  {\bf {\color{red} 0.58}} &  {\bf {\color{red} 0.55}} &  {\bf {\color{red} 0.51}} &  {\bf {\color{red} 0.48}} \\ 
\hline 
 2044 &  64 &  65 ans 0 mois &  -5.00\% &  1748.12 &  {\bf 37.70} &  2345.79 &  {\bf {\color{red} 0.75}} &  {\bf {\color{red} 0.69}} &  {\bf {\color{red} 0.65}} &  {\bf {\color{red} 0.61}} &  {\bf {\color{red} 0.57}} &  {\bf {\color{red} 0.53}} \\ 
\hline 
 2045 &  65 &  65 ans 0 mois &  0.00\% &  2019.84 &  {\bf 42.50} &  2376.28 &  {\bf {\color{red} 0.85}} &  {\bf {\color{red} 0.80}} &  {\bf {\color{red} 0.75}} &  {\bf {\color{red} 0.70}} &  {\bf {\color{red} 0.66}} &  {\bf {\color{red} 0.62}} \\ 
\hline 
 2046 &  66 &  65 ans 0 mois &  5.00\% &  2124.24 &  {\bf 43.62} &  2407.18 &  {\bf {\color{red} 0.88}} &  {\bf {\color{red} 0.84}} &  {\bf {\color{red} 0.79}} &  {\bf {\color{red} 0.74}} &  {\bf {\color{red} 0.69}} &  {\bf {\color{red} 0.65}} \\ 
\hline 
 2047 &  67 &  65 ans 0 mois &  10.00\% &  2330.73 &  {\bf 46.70} &  2438.47 &  {\bf {\color{red} 0.96}} &  {\bf {\color{red} 0.92}} &  {\bf {\color{red} 0.86}} &  {\bf {\color{red} 0.81}} &  {\bf {\color{red} 0.76}} &  {\bf {\color{red} 0.71}} \\ 
\hline 
\hline 
\end{tabular} 
\end{center} } 
\paragraph{Retraites possibles et ratios Revenu/SMIC à 70, 75, 80, 85, 90 ans avec le modèle \emph{Gouvernement corrigé (âge-pivot glissant)}}  
 
{ \scriptsize \begin{center} 
\begin{tabular}[htb]{|c|c||c|c||c|c||c||c|c|c|c|c|c|} 
\hline 
 Retraite en &  Âge &  Âge pivot &  Décote/Surcote &  Retraite (\euro{} 2019) &  Tx Rempl(\%) &  SMIC (\euro{} 2019) &  Retraite/SMIC &  Rev70/SMIC &  Rev75/SMIC &  Rev80/SMIC &  Rev85/SMIC &  Rev90/SMIC \\ 
\hline \hline 
 2042 &  62 &  65 ans 3 mois &  -16.25\% &  1400.30 &  {\bf 31.74} &  2285.97 &  {\bf {\color{red} 0.61}} &  {\bf {\color{red} 0.55}} &  {\bf {\color{red} 0.52}} &  {\bf {\color{red} 0.49}} &  {\bf {\color{red} 0.46}} &  {\bf {\color{red} 0.43}} \\ 
\hline 
 2043 &  63 &  65 ans 4 mois &  -11.67\% &  1549.48 &  {\bf 34.25} &  2315.68 &  {\bf {\color{red} 0.67}} &  {\bf {\color{red} 0.61}} &  {\bf {\color{red} 0.57}} &  {\bf {\color{red} 0.54}} &  {\bf {\color{red} 0.50}} &  {\bf {\color{red} 0.47}} \\ 
\hline 
 2044 &  64 &  65 ans 5 mois &  -7.08\% &  1709.78 &  {\bf 36.87} &  2345.79 &  {\bf {\color{red} 0.73}} &  {\bf {\color{red} 0.67}} &  {\bf {\color{red} 0.63}} &  {\bf {\color{red} 0.59}} &  {\bf {\color{red} 0.56}} &  {\bf {\color{red} 0.52}} \\ 
\hline 
 2045 &  65 &  65 ans 6 mois &  -2.50\% &  2019.84 &  {\bf 42.50} &  2376.28 &  {\bf {\color{red} 0.85}} &  {\bf {\color{red} 0.80}} &  {\bf {\color{red} 0.75}} &  {\bf {\color{red} 0.70}} &  {\bf {\color{red} 0.66}} &  {\bf {\color{red} 0.62}} \\ 
\hline 
 2046 &  66 &  65 ans 7 mois &  2.08\% &  2065.23 &  {\bf 42.40} &  2407.18 &  {\bf {\color{red} 0.86}} &  {\bf {\color{red} 0.81}} &  {\bf {\color{red} 0.76}} &  {\bf {\color{red} 0.72}} &  {\bf {\color{red} 0.67}} &  {\bf {\color{red} 0.63}} \\ 
\hline 
 2047 &  67 &  65 ans 8 mois &  6.67\% &  2260.11 &  {\bf 45.29} &  2438.47 &  {\bf {\color{red} 0.93}} &  {\bf {\color{red} 0.89}} &  {\bf {\color{red} 0.84}} &  {\bf {\color{red} 0.78}} &  {\bf {\color{red} 0.73}} &  {\bf {\color{red} 0.69}} \\ 
\hline 
\hline 
\end{tabular} 
\end{center} } 
\paragraph{Retraites possibles et ratios Revenu/SMIC à 70, 75, 80, 85, 90 ans avec le modèle \emph{Destinie2 (revalorisation de la fonction publique)}}  
 
{ \scriptsize \begin{center} 
\begin{tabular}[htb]{|c|c||c|c||c|c||c||c|c|c|c|c|c|} 
\hline 
 Retraite en &  Âge &  Âge pivot &  Décote/Surcote &  Retraite (\euro{} 2019) &  Tx Rempl(\%) &  SMIC (\euro{} 2019) &  Retraite/SMIC &  Rev70/SMIC &  Rev75/SMIC &  Rev80/SMIC &  Rev85/SMIC &  Rev90/SMIC \\ 
\hline \hline 
 2042 &  62 &  65 ans 3 mois &  -16.25\% &  1349.31 &  {\bf 32.53} &  2149.23 &  {\bf {\color{red} 0.63}} &  {\bf {\color{red} 0.57}} &  {\bf {\color{red} 0.53}} &  {\bf {\color{red} 0.50}} &  {\bf {\color{red} 0.47}} &  {\bf {\color{red} 0.44}} \\ 
\hline 
 2043 &  63 &  65 ans 4 mois &  -11.67\% &  1491.76 &  {\bf 35.07} &  2177.17 &  {\bf {\color{red} 0.69}} &  {\bf {\color{red} 0.63}} &  {\bf {\color{red} 0.59}} &  {\bf {\color{red} 0.55}} &  {\bf {\color{red} 0.52}} &  {\bf {\color{red} 0.48}} \\ 
\hline 
 2044 &  64 &  65 ans 5 mois &  -7.08\% &  1644.74 &  {\bf 37.73} &  2205.48 &  {\bf {\color{red} 0.75}} &  {\bf {\color{red} 0.69}} &  {\bf {\color{red} 0.65}} &  {\bf {\color{red} 0.61}} &  {\bf {\color{red} 0.57}} &  {\bf {\color{red} 0.53}} \\ 
\hline 
 2045 &  65 &  65 ans 6 mois &  -2.50\% &  1899.03 &  {\bf 42.50} &  2234.15 &  {\bf {\color{red} 0.85}} &  {\bf {\color{red} 0.80}} &  {\bf {\color{red} 0.75}} &  {\bf {\color{red} 0.70}} &  {\bf {\color{red} 0.66}} &  {\bf {\color{red} 0.62}} \\ 
\hline 
 2046 &  66 &  65 ans 7 mois &  2.08\% &  1983.67 &  {\bf 43.32} &  2263.19 &  {\bf {\color{red} 0.88}} &  {\bf {\color{red} 0.83}} &  {\bf {\color{red} 0.78}} &  {\bf {\color{red} 0.73}} &  {\bf {\color{red} 0.69}} &  {\bf {\color{red} 0.64}} \\ 
\hline 
 2047 &  67 &  65 ans 8 mois &  6.67\% &  2169.34 &  {\bf 46.24} &  2292.61 &  {\bf {\color{red} 0.95}} &  {\bf {\color{red} 0.91}} &  {\bf {\color{red} 0.85}} &  {\bf {\color{red} 0.80}} &  {\bf {\color{red} 0.75}} &  {\bf {\color{red} 0.70}} \\ 
\hline 
\hline 
\end{tabular} 
\end{center} } 

 \begin{center}\includegraphics[width=0.9\textwidth]{fig/Ascendant12_1980_22_dest_retraite.pdf}\end{center} \label{fig/Ascendant12_1980_22_dest_retraite.pdf} 

\newpage 
 
\subsection{Génération 1990 (début en 2012)} 

\paragraph{Retraites possibles et ratios Revenu/SMIC à 70, 75, 80, 85, 90 ans avec le modèle \emph{Gouvernement truqué (âge-pivot bloqué à 65 ans)}}  
 
{ \scriptsize \begin{center} 
\begin{tabular}[htb]{|c|c||c|c||c|c||c||c|c|c|c|c|c|} 
\hline 
 Retraite en &  Âge &  Âge pivot &  Décote/Surcote &  Retraite (\euro{} 2019) &  Tx Rempl(\%) &  SMIC (\euro{} 2019) &  Retraite/SMIC &  Rev70/SMIC &  Rev75/SMIC &  Rev80/SMIC &  Rev85/SMIC &  Rev90/SMIC \\ 
\hline \hline 
 2052 &  62 &  65 ans 0 mois &  -15.00\% &  1721.37 &  {\bf 34.28} &  2601.14 &  {\bf {\color{red} 0.66}} &  {\bf {\color{red} 0.60}} &  {\bf {\color{red} 0.56}} &  {\bf {\color{red} 0.52}} &  {\bf {\color{red} 0.49}} &  {\bf {\color{red} 0.46}} \\ 
\hline 
 2053 &  63 &  65 ans 0 mois &  -10.00\% &  1910.81 &  {\bf 37.12} &  2634.96 &  {\bf {\color{red} 0.73}} &  {\bf {\color{red} 0.66}} &  {\bf {\color{red} 0.62}} &  {\bf {\color{red} 0.58}} &  {\bf {\color{red} 0.55}} &  {\bf {\color{red} 0.51}} \\ 
\hline 
 2054 &  64 &  65 ans 0 mois &  -5.00\% &  2112.95 &  {\bf 40.05} &  2669.21 &  {\bf {\color{red} 0.79}} &  {\bf {\color{red} 0.73}} &  {\bf {\color{red} 0.69}} &  {\bf {\color{red} 0.64}} &  {\bf {\color{red} 0.60}} &  {\bf {\color{red} 0.57}} \\ 
\hline 
 2055 &  65 &  65 ans 0 mois &  0.00\% &  2328.35 &  {\bf 43.06} &  2703.91 &  {\bf {\color{red} 0.86}} &  {\bf {\color{red} 0.81}} &  {\bf {\color{red} 0.76}} &  {\bf {\color{red} 0.71}} &  {\bf {\color{red} 0.67}} &  {\bf {\color{red} 0.62}} \\ 
\hline 
 2056 &  66 &  65 ans 0 mois &  5.00\% &  2557.55 &  {\bf 46.15} &  2739.06 &  {\bf {\color{red} 0.93}} &  {\bf {\color{red} 0.89}} &  {\bf {\color{red} 0.83}} &  {\bf {\color{red} 0.78}} &  {\bf {\color{red} 0.73}} &  {\bf {\color{red} 0.68}} \\ 
\hline 
 2057 &  67 &  65 ans 0 mois &  10.00\% &  2801.11 &  {\bf 49.33} &  2774.67 &  {\bf 1.01} &  {\bf {\color{red} 0.97}} &  {\bf {\color{red} 0.91}} &  {\bf {\color{red} 0.85}} &  {\bf {\color{red} 0.80}} &  {\bf {\color{red} 0.75}} \\ 
\hline 
\hline 
\end{tabular} 
\end{center} } 
\paragraph{Retraites possibles et ratios Revenu/SMIC à 70, 75, 80, 85, 90 ans avec le modèle \emph{Gouvernement corrigé (âge-pivot glissant)}}  
 
{ \scriptsize \begin{center} 
\begin{tabular}[htb]{|c|c||c|c||c|c||c||c|c|c|c|c|c|} 
\hline 
 Retraite en &  Âge &  Âge pivot &  Décote/Surcote &  Retraite (\euro{} 2019) &  Tx Rempl(\%) &  SMIC (\euro{} 2019) &  Retraite/SMIC &  Rev70/SMIC &  Rev75/SMIC &  Rev80/SMIC &  Rev85/SMIC &  Rev90/SMIC \\ 
\hline \hline 
 2052 &  62 &  66 ans 1 mois &  -20.42\% &  1611.68 &  {\bf 32.10} &  2601.14 &  {\bf {\color{red} 0.62}} &  {\bf {\color{red} 0.56}} &  {\bf {\color{red} 0.52}} &  {\bf {\color{red} 0.49}} &  {\bf {\color{red} 0.46}} &  {\bf {\color{red} 0.43}} \\ 
\hline 
 2053 &  63 &  66 ans 2 mois &  -15.83\% &  1786.96 &  {\bf 34.72} &  2634.96 &  {\bf {\color{red} 0.68}} &  {\bf {\color{red} 0.62}} &  {\bf {\color{red} 0.58}} &  {\bf {\color{red} 0.54}} &  {\bf {\color{red} 0.51}} &  {\bf {\color{red} 0.48}} \\ 
\hline 
 2054 &  64 &  66 ans 3 mois &  -11.25\% &  1973.94 &  {\bf 37.41} &  2669.21 &  {\bf {\color{red} 0.74}} &  {\bf {\color{red} 0.68}} &  {\bf {\color{red} 0.64}} &  {\bf {\color{red} 0.60}} &  {\bf {\color{red} 0.56}} &  {\bf {\color{red} 0.53}} \\ 
\hline 
 2055 &  65 &  66 ans 4 mois &  -6.67\% &  2298.33 &  {\bf 42.50} &  2703.91 &  {\bf {\color{red} 0.85}} &  {\bf {\color{red} 0.80}} &  {\bf {\color{red} 0.75}} &  {\bf {\color{red} 0.70}} &  {\bf {\color{red} 0.66}} &  {\bf {\color{red} 0.62}} \\ 
\hline 
 2056 &  66 &  66 ans 5 mois &  -2.08\% &  2385.01 &  {\bf 43.04} &  2739.06 &  {\bf {\color{red} 0.87}} &  {\bf {\color{red} 0.83}} &  {\bf {\color{red} 0.78}} &  {\bf {\color{red} 0.73}} &  {\bf {\color{red} 0.68}} &  {\bf {\color{red} 0.64}} \\ 
\hline 
 2057 &  67 &  66 ans 6 mois &  2.50\% &  2610.12 &  {\bf 45.97} &  2774.67 &  {\bf {\color{red} 0.94}} &  {\bf {\color{red} 0.90}} &  {\bf {\color{red} 0.85}} &  {\bf {\color{red} 0.80}} &  {\bf {\color{red} 0.75}} &  {\bf {\color{red} 0.70}} \\ 
\hline 
\hline 
\end{tabular} 
\end{center} } 
\paragraph{Retraites possibles et ratios Revenu/SMIC à 70, 75, 80, 85, 90 ans avec le modèle \emph{Destinie2 (revalorisation de la fonction publique)}}  
 
{ \scriptsize \begin{center} 
\begin{tabular}[htb]{|c|c||c|c||c|c||c||c|c|c|c|c|c|} 
\hline 
 Retraite en &  Âge &  Âge pivot &  Décote/Surcote &  Retraite (\euro{} 2019) &  Tx Rempl(\%) &  SMIC (\euro{} 2019) &  Retraite/SMIC &  Rev70/SMIC &  Rev75/SMIC &  Rev80/SMIC &  Rev85/SMIC &  Rev90/SMIC \\ 
\hline \hline 
 2052 &  62 &  66 ans 1 mois &  -20.42\% &  1532.71 &  {\bf 32.47} &  2445.56 &  {\bf {\color{red} 0.63}} &  {\bf {\color{red} 0.57}} &  {\bf {\color{red} 0.53}} &  {\bf {\color{red} 0.50}} &  {\bf {\color{red} 0.47}} &  {\bf {\color{red} 0.44}} \\ 
\hline 
 2053 &  63 &  66 ans 2 mois &  -15.83\% &  1698.75 &  {\bf 35.10} &  2477.35 &  {\bf {\color{red} 0.69}} &  {\bf {\color{red} 0.63}} &  {\bf {\color{red} 0.59}} &  {\bf {\color{red} 0.55}} &  {\bf {\color{red} 0.52}} &  {\bf {\color{red} 0.48}} \\ 
\hline 
 2054 &  64 &  66 ans 3 mois &  -11.25\% &  1875.83 &  {\bf 37.81} &  2509.56 &  {\bf {\color{red} 0.75}} &  {\bf {\color{red} 0.69}} &  {\bf {\color{red} 0.65}} &  {\bf {\color{red} 0.61}} &  {\bf {\color{red} 0.57}} &  {\bf {\color{red} 0.53}} \\ 
\hline 
 2055 &  65 &  66 ans 4 mois &  -6.67\% &  2160.85 &  {\bf 42.50} &  2542.18 &  {\bf {\color{red} 0.85}} &  {\bf {\color{red} 0.80}} &  {\bf {\color{red} 0.75}} &  {\bf {\color{red} 0.70}} &  {\bf {\color{red} 0.66}} &  {\bf {\color{red} 0.62}} \\ 
\hline 
 2056 &  66 &  66 ans 5 mois &  -2.08\% &  2264.94 &  {\bf 43.47} &  2575.23 &  {\bf {\color{red} 0.88}} &  {\bf {\color{red} 0.84}} &  {\bf {\color{red} 0.78}} &  {\bf {\color{red} 0.73}} &  {\bf {\color{red} 0.69}} &  {\bf {\color{red} 0.65}} \\ 
\hline 
 2057 &  67 &  66 ans 6 mois &  2.50\% &  2477.95 &  {\bf 46.41} &  2608.71 &  {\bf {\color{red} 0.95}} &  {\bf {\color{red} 0.91}} &  {\bf {\color{red} 0.86}} &  {\bf {\color{red} 0.80}} &  {\bf {\color{red} 0.75}} &  {\bf {\color{red} 0.71}} \\ 
\hline 
\hline 
\end{tabular} 
\end{center} } 

 \begin{center}\includegraphics[width=0.9\textwidth]{fig/Ascendant12_1990_22_dest_retraite.pdf}\end{center} \label{fig/Ascendant12_1990_22_dest_retraite.pdf} 

\newpage 
 
\subsection{Génération 2003 (début en 2025)} 

\paragraph{Retraites possibles et ratios Revenu/SMIC à 70, 75, 80, 85, 90 ans avec le modèle \emph{Gouvernement truqué (âge-pivot bloqué à 65 ans)}}  
 
{ \scriptsize \begin{center} 
\begin{tabular}[htb]{|c|c||c|c||c|c||c||c|c|c|c|c|c|} 
\hline 
 Retraite en &  Âge &  Âge pivot &  Décote/Surcote &  Retraite (\euro{} 2019) &  Tx Rempl(\%) &  SMIC (\euro{} 2019) &  Retraite/SMIC &  Rev70/SMIC &  Rev75/SMIC &  Rev80/SMIC &  Rev85/SMIC &  Rev90/SMIC \\ 
\hline \hline 
 2065 &  62 &  65 ans 0 mois &  -15.00\% &  2147.85 &  {\bf 36.17} &  3076.71 &  {\bf {\color{red} 0.70}} &  {\bf {\color{red} 0.63}} &  {\bf {\color{red} 0.59}} &  {\bf {\color{red} 0.55}} &  {\bf {\color{red} 0.52}} &  {\bf {\color{red} 0.49}} \\ 
\hline 
 2066 &  63 &  65 ans 0 mois &  -10.00\% &  2380.03 &  {\bf 39.09} &  3116.71 &  {\bf {\color{red} 0.76}} &  {\bf {\color{red} 0.70}} &  {\bf {\color{red} 0.65}} &  {\bf {\color{red} 0.61}} &  {\bf {\color{red} 0.57}} &  {\bf {\color{red} 0.54}} \\ 
\hline 
 2067 &  64 &  65 ans 0 mois &  -5.00\% &  2627.45 &  {\bf 42.10} &  3157.23 &  {\bf {\color{red} 0.83}} &  {\bf {\color{red} 0.77}} &  {\bf {\color{red} 0.72}} &  {\bf {\color{red} 0.68}} &  {\bf {\color{red} 0.63}} &  {\bf {\color{red} 0.59}} \\ 
\hline 
 2068 &  65 &  65 ans 0 mois &  0.00\% &  2890.72 &  {\bf 45.19} &  3198.27 &  {\bf {\color{red} 0.90}} &  {\bf {\color{red} 0.85}} &  {\bf {\color{red} 0.79}} &  {\bf {\color{red} 0.74}} &  {\bf {\color{red} 0.70}} &  {\bf {\color{red} 0.65}} \\ 
\hline 
 2069 &  66 &  65 ans 0 mois &  5.00\% &  3170.52 &  {\bf 48.37} &  3239.85 &  {\bf {\color{red} 0.98}} &  {\bf {\color{red} 0.93}} &  {\bf {\color{red} 0.87}} &  {\bf {\color{red} 0.82}} &  {\bf {\color{red} 0.77}} &  {\bf {\color{red} 0.72}} \\ 
\hline 
 2070 &  67 &  65 ans 0 mois &  10.00\% &  3467.52 &  {\bf 51.63} &  3281.97 &  {\bf 1.06} &  {\bf 1.02} &  {\bf {\color{red} 0.95}} &  {\bf {\color{red} 0.89}} &  {\bf {\color{red} 0.84}} &  {\bf {\color{red} 0.78}} \\ 
\hline 
\hline 
\end{tabular} 
\end{center} } 
\paragraph{Retraites possibles et ratios Revenu/SMIC à 70, 75, 80, 85, 90 ans avec le modèle \emph{Gouvernement corrigé (âge-pivot glissant)}}  
 
{ \scriptsize \begin{center} 
\begin{tabular}[htb]{|c|c||c|c||c|c||c||c|c|c|c|c|c|} 
\hline 
 Retraite en &  Âge &  Âge pivot &  Décote/Surcote &  Retraite (\euro{} 2019) &  Tx Rempl(\%) &  SMIC (\euro{} 2019) &  Retraite/SMIC &  Rev70/SMIC &  Rev75/SMIC &  Rev80/SMIC &  Rev85/SMIC &  Rev90/SMIC \\ 
\hline \hline 
 2065 &  62 &  67 ans 2 mois &  -25.83\% &  1874.11 &  {\bf 31.56} &  3076.71 &  {\bf {\color{red} 0.61}} &  {\bf {\color{red} 0.55}} &  {\bf {\color{red} 0.51}} &  {\bf {\color{red} 0.48}} &  {\bf {\color{red} 0.45}} &  {\bf {\color{red} 0.42}} \\ 
\hline 
 2066 &  63 &  67 ans 3 mois &  -21.25\% &  2082.53 &  {\bf 34.20} &  3116.71 &  {\bf {\color{red} 0.67}} &  {\bf {\color{red} 0.61}} &  {\bf {\color{red} 0.57}} &  {\bf {\color{red} 0.54}} &  {\bf {\color{red} 0.50}} &  {\bf {\color{red} 0.47}} \\ 
\hline 
 2067 &  64 &  67 ans 4 mois &  -16.67\% &  2304.78 &  {\bf 36.93} &  3157.23 &  {\bf {\color{red} 0.73}} &  {\bf {\color{red} 0.68}} &  {\bf {\color{red} 0.63}} &  {\bf {\color{red} 0.59}} &  {\bf {\color{red} 0.56}} &  {\bf {\color{red} 0.52}} \\ 
\hline 
 2068 &  65 &  67 ans 5 mois &  -12.08\% &  2718.53 &  {\bf 42.50} &  3198.27 &  {\bf {\color{red} 0.85}} &  {\bf {\color{red} 0.80}} &  {\bf {\color{red} 0.75}} &  {\bf {\color{red} 0.70}} &  {\bf {\color{red} 0.66}} &  {\bf {\color{red} 0.62}} \\ 
\hline 
 2069 &  66 &  67 ans 6 mois &  -7.50\% &  2793.08 &  {\bf 42.61} &  3239.85 &  {\bf {\color{red} 0.86}} &  {\bf {\color{red} 0.82}} &  {\bf {\color{red} 0.77}} &  {\bf {\color{red} 0.72}} &  {\bf {\color{red} 0.67}} &  {\bf {\color{red} 0.63}} \\ 
\hline 
 2070 &  67 &  67 ans 7 mois &  -2.92\% &  3060.35 &  {\bf 45.56} &  3281.97 &  {\bf {\color{red} 0.93}} &  {\bf {\color{red} 0.90}} &  {\bf {\color{red} 0.84}} &  {\bf {\color{red} 0.79}} &  {\bf {\color{red} 0.74}} &  {\bf {\color{red} 0.69}} \\ 
\hline 
\hline 
\end{tabular} 
\end{center} } 
\paragraph{Retraites possibles et ratios Revenu/SMIC à 70, 75, 80, 85, 90 ans avec le modèle \emph{Destinie2 (revalorisation de la fonction publique)}}  
 
{ \scriptsize \begin{center} 
\begin{tabular}[htb]{|c|c||c|c||c|c||c||c|c|c|c|c|c|} 
\hline 
 Retraite en &  Âge &  Âge pivot &  Décote/Surcote &  Retraite (\euro{} 2019) &  Tx Rempl(\%) &  SMIC (\euro{} 2019) &  Retraite/SMIC &  Rev70/SMIC &  Rev75/SMIC &  Rev80/SMIC &  Rev85/SMIC &  Rev90/SMIC \\ 
\hline \hline 
 2065 &  62 &  67 ans 2 mois &  -25.83\% &  1762.91 &  {\bf 31.57} &  2892.68 &  {\bf {\color{red} 0.61}} &  {\bf {\color{red} 0.55}} &  {\bf {\color{red} 0.52}} &  {\bf {\color{red} 0.48}} &  {\bf {\color{red} 0.45}} &  {\bf {\color{red} 0.42}} \\ 
\hline 
 2066 &  63 &  67 ans 3 mois &  -21.25\% &  1958.93 &  {\bf 34.22} &  2930.29 &  {\bf {\color{red} 0.67}} &  {\bf {\color{red} 0.61}} &  {\bf {\color{red} 0.57}} &  {\bf {\color{red} 0.54}} &  {\bf {\color{red} 0.50}} &  {\bf {\color{red} 0.47}} \\ 
\hline 
 2067 &  64 &  67 ans 4 mois &  -16.67\% &  2167.96 &  {\bf 36.95} &  2968.38 &  {\bf {\color{red} 0.73}} &  {\bf {\color{red} 0.68}} &  {\bf {\color{red} 0.63}} &  {\bf {\color{red} 0.59}} &  {\bf {\color{red} 0.56}} &  {\bf {\color{red} 0.52}} \\ 
\hline 
 2068 &  65 &  67 ans 5 mois &  -12.08\% &  2555.93 &  {\bf 42.50} &  3006.97 &  {\bf {\color{red} 0.85}} &  {\bf {\color{red} 0.80}} &  {\bf {\color{red} 0.75}} &  {\bf {\color{red} 0.70}} &  {\bf {\color{red} 0.66}} &  {\bf {\color{red} 0.62}} \\ 
\hline 
 2069 &  66 &  67 ans 6 mois &  -7.50\% &  2627.19 &  {\bf 42.63} &  3046.06 &  {\bf {\color{red} 0.86}} &  {\bf {\color{red} 0.82}} &  {\bf {\color{red} 0.77}} &  {\bf {\color{red} 0.72}} &  {\bf {\color{red} 0.67}} &  {\bf {\color{red} 0.63}} \\ 
\hline 
 2070 &  67 &  67 ans 7 mois &  -2.92\% &  2878.55 &  {\bf 45.58} &  3085.66 &  {\bf {\color{red} 0.93}} &  {\bf {\color{red} 0.90}} &  {\bf {\color{red} 0.84}} &  {\bf {\color{red} 0.79}} &  {\bf {\color{red} 0.74}} &  {\bf {\color{red} 0.69}} \\ 
\hline 
\hline 
\end{tabular} 
\end{center} } 

 \begin{center}\includegraphics[width=0.9\textwidth]{fig/Ascendant12_2003_22_dest_retraite.pdf}\end{center} \label{fig/Ascendant12_2003_22_dest_retraite.pdf} 

\newpage 
 
\chapter{Salarié privé évoluant de 1.5*SMIC à 2.5*SMIC} 


 \addto{\captionsenglish}{ \renewcommand{\mtctitle}{}} \setcounter{minitocdepth}{2} 
 \minitoc \newpage 

\section{Début de carrière à 22 ans} 

\subsection{Génération 1975 (début en 1997)} 

\paragraph{Retraites possibles et ratios Revenu/SMIC à 70, 75, 80, 85, 90 ans avec le modèle \emph{Gouvernement truqué (âge-pivot bloqué à 65 ans)}}  
 
{ \scriptsize \begin{center} 
\begin{tabular}[htb]{|c|c||c|c||c|c||c||c|c|c|c|c|c|} 
\hline 
 Retraite en &  Âge &  Âge pivot &  Décote/Surcote &  Retraite (\euro{} 2019) &  Tx Rempl(\%) &  SMIC (\euro{} 2019) &  Retraite/SMIC &  Rev70/SMIC &  Rev75/SMIC &  Rev80/SMIC &  Rev85/SMIC &  Rev90/SMIC \\ 
\hline \hline 
 2037 &  62 &  64 ans 10 mois &  -14.17\% &  1746.12 &  {\bf 33.53} &  2143.00 &  {\bf {\color{red} 0.81}} &  {\bf {\color{red} 0.73}} &  {\bf {\color{red} 0.69}} &  {\bf {\color{red} 0.65}} &  {\bf {\color{red} 0.61}} &  {\bf {\color{red} 0.57}} \\ 
\hline 
 2038 &  63 &  64 ans 11 mois &  -9.58\% &  1920.42 &  {\bf 36.06} &  2170.86 &  {\bf {\color{red} 0.88}} &  {\bf {\color{red} 0.81}} &  {\bf {\color{red} 0.76}} &  {\bf {\color{red} 0.71}} &  {\bf {\color{red} 0.67}} &  {\bf {\color{red} 0.62}} \\ 
\hline 
 2039 &  64 &  65 ans 0 mois &  -5.00\% &  2106.73 &  {\bf 38.68} &  2199.08 &  {\bf {\color{red} 0.96}} &  {\bf {\color{red} 0.89}} &  {\bf {\color{red} 0.83}} &  {\bf {\color{red} 0.78}} &  {\bf {\color{red} 0.73}} &  {\bf {\color{red} 0.68}} \\ 
\hline 
 2040 &  65 &  65 ans 0 mois &  0.00\% &  2315.44 &  {\bf 41.58} &  2227.67 &  {\bf 1.04} &  {\bf {\color{red} 0.97}} &  {\bf {\color{red} 0.91}} &  {\bf {\color{red} 0.86}} &  {\bf {\color{red} 0.80}} &  {\bf {\color{red} 0.75}} \\ 
\hline 
 2041 &  66 &  65 ans 0 mois &  5.00\% &  2538.56 &  {\bf 44.58} &  2256.63 &  {\bf 1.12} &  {\bf 1.07} &  {\bf 1.00} &  {\bf {\color{red} 0.94}} &  {\bf {\color{red} 0.88}} &  {\bf {\color{red} 0.83}} \\ 
\hline 
 2042 &  67 &  65 ans 0 mois &  10.00\% &  2777.02 &  {\bf 47.70} &  2285.97 &  {\bf 1.21} &  {\bf 1.17} &  {\bf 1.10} &  {\bf 1.03} &  {\bf {\color{red} 0.96}} &  {\bf {\color{red} 0.90}} \\ 
\hline 
\hline 
\end{tabular} 
\end{center} } 
\paragraph{Retraites possibles et ratios Revenu/SMIC à 70, 75, 80, 85, 90 ans avec le modèle \emph{Gouvernement corrigé (âge-pivot glissant)}}  
 
{ \scriptsize \begin{center} 
\begin{tabular}[htb]{|c|c||c|c||c|c||c||c|c|c|c|c|c|} 
\hline 
 Retraite en &  Âge &  Âge pivot &  Décote/Surcote &  Retraite (\euro{} 2019) &  Tx Rempl(\%) &  SMIC (\euro{} 2019) &  Retraite/SMIC &  Rev70/SMIC &  Rev75/SMIC &  Rev80/SMIC &  Rev85/SMIC &  Rev90/SMIC \\ 
\hline \hline 
 2037 &  62 &  64 ans 10 mois &  -14.17\% &  1746.12 &  {\bf 33.53} &  2143.00 &  {\bf {\color{red} 0.81}} &  {\bf {\color{red} 0.73}} &  {\bf {\color{red} 0.69}} &  {\bf {\color{red} 0.65}} &  {\bf {\color{red} 0.61}} &  {\bf {\color{red} 0.57}} \\ 
\hline 
 2038 &  63 &  64 ans 11 mois &  -9.58\% &  1920.42 &  {\bf 36.06} &  2170.86 &  {\bf {\color{red} 0.88}} &  {\bf {\color{red} 0.81}} &  {\bf {\color{red} 0.76}} &  {\bf {\color{red} 0.71}} &  {\bf {\color{red} 0.67}} &  {\bf {\color{red} 0.62}} \\ 
\hline 
 2039 &  64 &  65 ans 0 mois &  -5.00\% &  2106.73 &  {\bf 38.68} &  2199.08 &  {\bf {\color{red} 0.96}} &  {\bf {\color{red} 0.89}} &  {\bf {\color{red} 0.83}} &  {\bf {\color{red} 0.78}} &  {\bf {\color{red} 0.73}} &  {\bf {\color{red} 0.68}} \\ 
\hline 
 2040 &  65 &  65 ans 1 mois &  -0.42\% &  2305.79 &  {\bf 41.40} &  2227.67 &  {\bf 1.04} &  {\bf {\color{red} 0.97}} &  {\bf {\color{red} 0.91}} &  {\bf {\color{red} 0.85}} &  {\bf {\color{red} 0.80}} &  {\bf {\color{red} 0.75}} \\ 
\hline 
 2041 &  66 &  65 ans 2 mois &  4.17\% &  2518.42 &  {\bf 44.23} &  2256.63 &  {\bf 1.12} &  {\bf 1.06} &  {\bf {\color{red} 0.99}} &  {\bf {\color{red} 0.93}} &  {\bf {\color{red} 0.87}} &  {\bf {\color{red} 0.82}} \\ 
\hline 
 2042 &  67 &  65 ans 3 mois &  8.75\% &  2745.46 &  {\bf 47.16} &  2285.97 &  {\bf 1.20} &  {\bf 1.16} &  {\bf 1.08} &  {\bf 1.02} &  {\bf {\color{red} 0.95}} &  {\bf {\color{red} 0.89}} \\ 
\hline 
\hline 
\end{tabular} 
\end{center} } 
\paragraph{Retraites possibles et ratios Revenu/SMIC à 70, 75, 80, 85, 90 ans avec le modèle \emph{Destinie2 (revalorisation de la fonction publique)}}  
 
{ \scriptsize \begin{center} 
\begin{tabular}[htb]{|c|c||c|c||c|c||c||c|c|c|c|c|c|} 
\hline 
 Retraite en &  Âge &  Âge pivot &  Décote/Surcote &  Retraite (\euro{} 2019) &  Tx Rempl(\%) &  SMIC (\euro{} 2019) &  Retraite/SMIC &  Rev70/SMIC &  Rev75/SMIC &  Rev80/SMIC &  Rev85/SMIC &  Rev90/SMIC \\ 
\hline \hline 
 2037 &  62 &  64 ans 10 mois &  -14.17\% &  1697.60 &  {\bf 34.67} &  2014.82 &  {\bf {\color{red} 0.84}} &  {\bf {\color{red} 0.76}} &  {\bf {\color{red} 0.71}} &  {\bf {\color{red} 0.67}} &  {\bf {\color{red} 0.63}} &  {\bf {\color{red} 0.59}} \\ 
\hline 
 2038 &  63 &  64 ans 11 mois &  -9.58\% &  1864.91 &  {\bf 37.24} &  2041.01 &  {\bf {\color{red} 0.91}} &  {\bf {\color{red} 0.83}} &  {\bf {\color{red} 0.78}} &  {\bf {\color{red} 0.73}} &  {\bf {\color{red} 0.69}} &  {\bf {\color{red} 0.64}} \\ 
\hline 
 2039 &  64 &  65 ans 0 mois &  -5.00\% &  2043.60 &  {\bf 39.91} &  2067.55 &  {\bf {\color{red} 0.99}} &  {\bf {\color{red} 0.91}} &  {\bf {\color{red} 0.86}} &  {\bf {\color{red} 0.80}} &  {\bf {\color{red} 0.75}} &  {\bf {\color{red} 0.71}} \\ 
\hline 
 2040 &  65 &  65 ans 1 mois &  -0.42\% &  2234.40 &  {\bf 42.67} &  2094.43 &  {\bf 1.07} &  {\bf 1.00} &  {\bf {\color{red} 0.94}} &  {\bf {\color{red} 0.88}} &  {\bf {\color{red} 0.82}} &  {\bf {\color{red} 0.77}} \\ 
\hline 
 2041 &  66 &  65 ans 2 mois &  4.17\% &  2438.06 &  {\bf 45.54} &  2121.65 &  {\bf 1.15} &  {\bf 1.09} &  {\bf 1.02} &  {\bf {\color{red} 0.96}} &  {\bf {\color{red} 0.90}} &  {\bf {\color{red} 0.84}} \\ 
\hline 
 2042 &  67 &  65 ans 3 mois &  8.75\% &  2655.40 &  {\bf 48.52} &  2149.23 &  {\bf 1.24} &  {\bf 1.19} &  {\bf 1.11} &  {\bf 1.04} &  {\bf {\color{red} 0.98}} &  {\bf {\color{red} 0.92}} \\ 
\hline 
\hline 
\end{tabular} 
\end{center} } 

 \begin{center}\includegraphics[width=0.9\textwidth]{fig/Ascendant1525_1975_22_dest_retraite.pdf}\end{center} \label{fig/Ascendant1525_1975_22_dest_retraite.pdf} 

\newpage 
 
\subsection{Génération 1980 (début en 2002)} 

\paragraph{Retraites possibles et ratios Revenu/SMIC à 70, 75, 80, 85, 90 ans avec le modèle \emph{Gouvernement truqué (âge-pivot bloqué à 65 ans)}}  
 
{ \scriptsize \begin{center} 
\begin{tabular}[htb]{|c|c||c|c||c|c||c||c|c|c|c|c|c|} 
\hline 
 Retraite en &  Âge &  Âge pivot &  Décote/Surcote &  Retraite (\euro{} 2019) &  Tx Rempl(\%) &  SMIC (\euro{} 2019) &  Retraite/SMIC &  Rev70/SMIC &  Rev75/SMIC &  Rev80/SMIC &  Rev85/SMIC &  Rev90/SMIC \\ 
\hline \hline 
 2042 &  62 &  65 ans 0 mois &  -15.00\% &  1897.04 &  {\bf 34.15} &  2285.97 &  {\bf {\color{red} 0.83}} &  {\bf {\color{red} 0.75}} &  {\bf {\color{red} 0.70}} &  {\bf {\color{red} 0.66}} &  {\bf {\color{red} 0.62}} &  {\bf {\color{red} 0.58}} \\ 
\hline 
 2043 &  63 &  65 ans 0 mois &  -10.00\% &  2102.83 &  {\bf 37.01} &  2315.68 &  {\bf {\color{red} 0.91}} &  {\bf {\color{red} 0.83}} &  {\bf {\color{red} 0.78}} &  {\bf {\color{red} 0.73}} &  {\bf {\color{red} 0.68}} &  {\bf {\color{red} 0.64}} \\ 
\hline 
 2044 &  64 &  65 ans 0 mois &  -5.00\% &  2323.62 &  {\bf 39.99} &  2345.79 &  {\bf {\color{red} 0.99}} &  {\bf {\color{red} 0.92}} &  {\bf {\color{red} 0.86}} &  {\bf {\color{red} 0.81}} &  {\bf {\color{red} 0.76}} &  {\bf {\color{red} 0.71}} \\ 
\hline 
 2045 &  65 &  65 ans 0 mois &  0.00\% &  2560.41 &  {\bf 43.10} &  2376.28 &  {\bf 1.08} &  {\bf 1.01} &  {\bf {\color{red} 0.95}} &  {\bf {\color{red} 0.89}} &  {\bf {\color{red} 0.83}} &  {\bf {\color{red} 0.78}} \\ 
\hline 
 2046 &  66 &  65 ans 0 mois &  5.00\% &  2812.15 &  {\bf 46.30} &  2407.18 &  {\bf 1.17} &  {\bf 1.11} &  {\bf 1.04} &  {\bf {\color{red} 0.97}} &  {\bf {\color{red} 0.91}} &  {\bf {\color{red} 0.86}} \\ 
\hline 
 2047 &  67 &  65 ans 0 mois &  10.00\% &  3079.44 &  {\bf 49.59} &  2438.47 &  {\bf 1.26} &  {\bf 1.21} &  {\bf 1.14} &  {\bf 1.07} &  {\bf 1.00} &  {\bf {\color{red} 0.94}} \\ 
\hline 
\hline 
\end{tabular} 
\end{center} } 
\paragraph{Retraites possibles et ratios Revenu/SMIC à 70, 75, 80, 85, 90 ans avec le modèle \emph{Gouvernement corrigé (âge-pivot glissant)}}  
 
{ \scriptsize \begin{center} 
\begin{tabular}[htb]{|c|c||c|c||c|c||c||c|c|c|c|c|c|} 
\hline 
 Retraite en &  Âge &  Âge pivot &  Décote/Surcote &  Retraite (\euro{} 2019) &  Tx Rempl(\%) &  SMIC (\euro{} 2019) &  Retraite/SMIC &  Rev70/SMIC &  Rev75/SMIC &  Rev80/SMIC &  Rev85/SMIC &  Rev90/SMIC \\ 
\hline \hline 
 2042 &  62 &  65 ans 3 mois &  -16.25\% &  1869.14 &  {\bf 33.65} &  2285.97 &  {\bf {\color{red} 0.82}} &  {\bf {\color{red} 0.74}} &  {\bf {\color{red} 0.69}} &  {\bf {\color{red} 0.65}} &  {\bf {\color{red} 0.61}} &  {\bf {\color{red} 0.57}} \\ 
\hline 
 2043 &  63 &  65 ans 4 mois &  -11.67\% &  2063.89 &  {\bf 36.33} &  2315.68 &  {\bf {\color{red} 0.89}} &  {\bf {\color{red} 0.81}} &  {\bf {\color{red} 0.76}} &  {\bf {\color{red} 0.72}} &  {\bf {\color{red} 0.67}} &  {\bf {\color{red} 0.63}} \\ 
\hline 
 2044 &  64 &  65 ans 5 mois &  -7.08\% &  2272.67 &  {\bf 39.12} &  2345.79 &  {\bf {\color{red} 0.97}} &  {\bf {\color{red} 0.90}} &  {\bf {\color{red} 0.84}} &  {\bf {\color{red} 0.79}} &  {\bf {\color{red} 0.74}} &  {\bf {\color{red} 0.69}} \\ 
\hline 
 2045 &  65 &  65 ans 6 mois &  -2.50\% &  2496.40 &  {\bf 42.02} &  2376.28 &  {\bf 1.05} &  {\bf {\color{red} 0.98}} &  {\bf {\color{red} 0.92}} &  {\bf {\color{red} 0.87}} &  {\bf {\color{red} 0.81}} &  {\bf {\color{red} 0.76}} \\ 
\hline 
 2046 &  66 &  65 ans 7 mois &  2.08\% &  2734.04 &  {\bf 45.01} &  2407.18 &  {\bf 1.14} &  {\bf 1.08} &  {\bf 1.01} &  {\bf {\color{red} 0.95}} &  {\bf {\color{red} 0.89}} &  {\bf {\color{red} 0.83}} \\ 
\hline 
 2047 &  67 &  65 ans 8 mois &  6.67\% &  2986.12 &  {\bf 48.09} &  2438.47 &  {\bf 1.22} &  {\bf 1.18} &  {\bf 1.10} &  {\bf 1.04} &  {\bf {\color{red} 0.97}} &  {\bf {\color{red} 0.91}} \\ 
\hline 
\hline 
\end{tabular} 
\end{center} } 
\paragraph{Retraites possibles et ratios Revenu/SMIC à 70, 75, 80, 85, 90 ans avec le modèle \emph{Destinie2 (revalorisation de la fonction publique)}}  
 
{ \scriptsize \begin{center} 
\begin{tabular}[htb]{|c|c||c|c||c|c||c||c|c|c|c|c|c|} 
\hline 
 Retraite en &  Âge &  Âge pivot &  Décote/Surcote &  Retraite (\euro{} 2019) &  Tx Rempl(\%) &  SMIC (\euro{} 2019) &  Retraite/SMIC &  Rev70/SMIC &  Rev75/SMIC &  Rev80/SMIC &  Rev85/SMIC &  Rev90/SMIC \\ 
\hline \hline 
 2042 &  62 &  65 ans 3 mois &  -16.25\% &  1803.27 &  {\bf 34.52} &  2149.23 &  {\bf {\color{red} 0.84}} &  {\bf {\color{red} 0.76}} &  {\bf {\color{red} 0.71}} &  {\bf {\color{red} 0.66}} &  {\bf {\color{red} 0.62}} &  {\bf {\color{red} 0.58}} \\ 
\hline 
 2043 &  63 &  65 ans 4 mois &  -11.67\% &  1989.43 &  {\bf 37.24} &  2177.17 &  {\bf {\color{red} 0.91}} &  {\bf {\color{red} 0.83}} &  {\bf {\color{red} 0.78}} &  {\bf {\color{red} 0.73}} &  {\bf {\color{red} 0.69}} &  {\bf {\color{red} 0.64}} \\ 
\hline 
 2044 &  64 &  65 ans 5 mois &  -7.08\% &  2188.90 &  {\bf 40.07} &  2205.48 &  {\bf {\color{red} 0.99}} &  {\bf {\color{red} 0.92}} &  {\bf {\color{red} 0.86}} &  {\bf {\color{red} 0.81}} &  {\bf {\color{red} 0.76}} &  {\bf {\color{red} 0.71}} \\ 
\hline 
 2045 &  65 &  65 ans 6 mois &  -2.50\% &  2402.53 &  {\bf 43.01} &  2234.15 &  {\bf 1.08} &  {\bf 1.01} &  {\bf {\color{red} 0.95}} &  {\bf {\color{red} 0.89}} &  {\bf {\color{red} 0.83}} &  {\bf {\color{red} 0.78}} \\ 
\hline 
 2046 &  66 &  65 ans 7 mois &  2.08\% &  2629.32 &  {\bf 46.04} &  2263.19 &  {\bf 1.16} &  {\bf 1.10} &  {\bf 1.03} &  {\bf {\color{red} 0.97}} &  {\bf {\color{red} 0.91}} &  {\bf {\color{red} 0.85}} \\ 
\hline 
 2047 &  67 &  65 ans 8 mois &  6.67\% &  2869.77 &  {\bf 49.16} &  2292.61 &  {\bf 1.25} &  {\bf 1.20} &  {\bf 1.13} &  {\bf 1.06} &  {\bf {\color{red} 0.99}} &  {\bf {\color{red} 0.93}} \\ 
\hline 
\hline 
\end{tabular} 
\end{center} } 

 \begin{center}\includegraphics[width=0.9\textwidth]{fig/Ascendant1525_1980_22_dest_retraite.pdf}\end{center} \label{fig/Ascendant1525_1980_22_dest_retraite.pdf} 

\newpage 
 
\subsection{Génération 1990 (début en 2012)} 

\paragraph{Retraites possibles et ratios Revenu/SMIC à 70, 75, 80, 85, 90 ans avec le modèle \emph{Gouvernement truqué (âge-pivot bloqué à 65 ans)}}  
 
{ \scriptsize \begin{center} 
\begin{tabular}[htb]{|c|c||c|c||c|c||c||c|c|c|c|c|c|} 
\hline 
 Retraite en &  Âge &  Âge pivot &  Décote/Surcote &  Retraite (\euro{} 2019) &  Tx Rempl(\%) &  SMIC (\euro{} 2019) &  Retraite/SMIC &  Rev70/SMIC &  Rev75/SMIC &  Rev80/SMIC &  Rev85/SMIC &  Rev90/SMIC \\ 
\hline \hline 
 2052 &  62 &  65 ans 0 mois &  -15.00\% &  2299.61 &  {\bf 36.38} &  2601.14 &  {\bf {\color{red} 0.88}} &  {\bf {\color{red} 0.80}} &  {\bf {\color{red} 0.75}} &  {\bf {\color{red} 0.70}} &  {\bf {\color{red} 0.66}} &  {\bf {\color{red} 0.62}} \\ 
\hline 
 2053 &  63 &  65 ans 0 mois &  -10.00\% &  2547.53 &  {\bf 39.41} &  2634.96 &  {\bf {\color{red} 0.97}} &  {\bf {\color{red} 0.88}} &  {\bf {\color{red} 0.83}} &  {\bf {\color{red} 0.78}} &  {\bf {\color{red} 0.73}} &  {\bf {\color{red} 0.68}} \\ 
\hline 
 2054 &  64 &  65 ans 0 mois &  -5.00\% &  2811.43 &  {\bf 42.53} &  2669.21 &  {\bf 1.05} &  {\bf {\color{red} 0.97}} &  {\bf {\color{red} 0.91}} &  {\bf {\color{red} 0.86}} &  {\bf {\color{red} 0.80}} &  {\bf {\color{red} 0.75}} \\ 
\hline 
 2055 &  65 &  65 ans 0 mois &  0.00\% &  3091.97 &  {\bf 45.74} &  2703.91 &  {\bf 1.14} &  {\bf 1.07} &  {\bf 1.00} &  {\bf {\color{red} 0.94}} &  {\bf {\color{red} 0.88}} &  {\bf {\color{red} 0.83}} \\ 
\hline 
 2056 &  66 &  65 ans 0 mois &  5.00\% &  3389.78 &  {\bf 49.05} &  2739.06 &  {\bf 1.24} &  {\bf 1.18} &  {\bf 1.10} &  {\bf 1.03} &  {\bf {\color{red} 0.97}} &  {\bf {\color{red} 0.91}} \\ 
\hline 
 2057 &  67 &  65 ans 0 mois &  10.00\% &  3705.55 &  {\bf 52.44} &  2774.67 &  {\bf 1.34} &  {\bf 1.28} &  {\bf 1.20} &  {\bf 1.13} &  {\bf 1.06} &  {\bf {\color{red} 0.99}} \\ 
\hline 
\hline 
\end{tabular} 
\end{center} } 
\paragraph{Retraites possibles et ratios Revenu/SMIC à 70, 75, 80, 85, 90 ans avec le modèle \emph{Gouvernement corrigé (âge-pivot glissant)}}  
 
{ \scriptsize \begin{center} 
\begin{tabular}[htb]{|c|c||c|c||c|c||c||c|c|c|c|c|c|} 
\hline 
 Retraite en &  Âge &  Âge pivot &  Décote/Surcote &  Retraite (\euro{} 2019) &  Tx Rempl(\%) &  SMIC (\euro{} 2019) &  Retraite/SMIC &  Rev70/SMIC &  Rev75/SMIC &  Rev80/SMIC &  Rev85/SMIC &  Rev90/SMIC \\ 
\hline \hline 
 2052 &  62 &  66 ans 1 mois &  -20.42\% &  2153.07 &  {\bf 34.06} &  2601.14 &  {\bf {\color{red} 0.83}} &  {\bf {\color{red} 0.75}} &  {\bf {\color{red} 0.70}} &  {\bf {\color{red} 0.66}} &  {\bf {\color{red} 0.62}} &  {\bf {\color{red} 0.58}} \\ 
\hline 
 2053 &  63 &  66 ans 2 mois &  -15.83\% &  2382.41 &  {\bf 36.85} &  2634.96 &  {\bf {\color{red} 0.90}} &  {\bf {\color{red} 0.83}} &  {\bf {\color{red} 0.77}} &  {\bf {\color{red} 0.73}} &  {\bf {\color{red} 0.68}} &  {\bf {\color{red} 0.64}} \\ 
\hline 
 2054 &  64 &  66 ans 3 mois &  -11.25\% &  2626.47 &  {\bf 39.73} &  2669.21 &  {\bf {\color{red} 0.98}} &  {\bf {\color{red} 0.91}} &  {\bf {\color{red} 0.85}} &  {\bf {\color{red} 0.80}} &  {\bf {\color{red} 0.75}} &  {\bf {\color{red} 0.70}} \\ 
\hline 
 2055 &  65 &  66 ans 4 mois &  -6.67\% &  2885.84 &  {\bf 42.69} &  2703.91 &  {\bf 1.07} &  {\bf 1.00} &  {\bf {\color{red} 0.94}} &  {\bf {\color{red} 0.88}} &  {\bf {\color{red} 0.82}} &  {\bf {\color{red} 0.77}} \\ 
\hline 
 2056 &  66 &  66 ans 5 mois &  -2.08\% &  3161.11 &  {\bf 45.74} &  2739.06 &  {\bf 1.15} &  {\bf 1.10} &  {\bf 1.03} &  {\bf {\color{red} 0.96}} &  {\bf {\color{red} 0.90}} &  {\bf {\color{red} 0.85}} \\ 
\hline 
 2057 &  67 &  66 ans 6 mois &  2.50\% &  3452.90 &  {\bf 48.87} &  2774.67 &  {\bf 1.24} &  {\bf 1.20} &  {\bf 1.12} &  {\bf 1.05} &  {\bf {\color{red} 0.99}} &  {\bf {\color{red} 0.92}} \\ 
\hline 
\hline 
\end{tabular} 
\end{center} } 
\paragraph{Retraites possibles et ratios Revenu/SMIC à 70, 75, 80, 85, 90 ans avec le modèle \emph{Destinie2 (revalorisation de la fonction publique)}}  
 
{ \scriptsize \begin{center} 
\begin{tabular}[htb]{|c|c||c|c||c|c||c||c|c|c|c|c|c|} 
\hline 
 Retraite en &  Âge &  Âge pivot &  Décote/Surcote &  Retraite (\euro{} 2019) &  Tx Rempl(\%) &  SMIC (\euro{} 2019) &  Retraite/SMIC &  Rev70/SMIC &  Rev75/SMIC &  Rev80/SMIC &  Rev85/SMIC &  Rev90/SMIC \\ 
\hline \hline 
 2052 &  62 &  66 ans 1 mois &  -20.42\% &  2049.45 &  {\bf 34.48} &  2445.56 &  {\bf {\color{red} 0.84}} &  {\bf {\color{red} 0.76}} &  {\bf {\color{red} 0.71}} &  {\bf {\color{red} 0.66}} &  {\bf {\color{red} 0.62}} &  {\bf {\color{red} 0.58}} \\ 
\hline 
 2053 &  63 &  66 ans 2 mois &  -15.83\% &  2266.86 &  {\bf 37.30} &  2477.35 &  {\bf {\color{red} 0.92}} &  {\bf {\color{red} 0.84}} &  {\bf {\color{red} 0.78}} &  {\bf {\color{red} 0.73}} &  {\bf {\color{red} 0.69}} &  {\bf {\color{red} 0.65}} \\ 
\hline 
 2054 &  64 &  66 ans 3 mois &  -11.25\% &  2498.16 &  {\bf 40.19} &  2509.56 &  {\bf {\color{red} 1.00}} &  {\bf {\color{red} 0.92}} &  {\bf {\color{red} 0.86}} &  {\bf {\color{red} 0.81}} &  {\bf {\color{red} 0.76}} &  {\bf {\color{red} 0.71}} \\ 
\hline 
 2055 &  65 &  66 ans 4 mois &  -6.67\% &  2743.90 &  {\bf 43.17} &  2542.18 &  {\bf 1.08} &  {\bf 1.01} &  {\bf {\color{red} 0.95}} &  {\bf {\color{red} 0.89}} &  {\bf {\color{red} 0.83}} &  {\bf {\color{red} 0.78}} \\ 
\hline 
 2056 &  66 &  66 ans 5 mois &  -2.08\% &  3004.63 &  {\bf 46.24} &  2575.23 &  {\bf 1.17} &  {\bf 1.11} &  {\bf 1.04} &  {\bf {\color{red} 0.97}} &  {\bf {\color{red} 0.91}} &  {\bf {\color{red} 0.86}} \\ 
\hline 
 2057 &  67 &  66 ans 6 mois &  2.50\% &  3280.94 &  {\bf 49.39} &  2608.71 &  {\bf 1.26} &  {\bf 1.21} &  {\bf 1.13} &  {\bf 1.06} &  {\bf {\color{red} 1.00}} &  {\bf {\color{red} 0.93}} \\ 
\hline 
\hline 
\end{tabular} 
\end{center} } 

 \begin{center}\includegraphics[width=0.9\textwidth]{fig/Ascendant1525_1990_22_dest_retraite.pdf}\end{center} \label{fig/Ascendant1525_1990_22_dest_retraite.pdf} 

\newpage 
 
\subsection{Génération 2003 (début en 2025)} 

\paragraph{Retraites possibles et ratios Revenu/SMIC à 70, 75, 80, 85, 90 ans avec le modèle \emph{Gouvernement truqué (âge-pivot bloqué à 65 ans)}}  
 
{ \scriptsize \begin{center} 
\begin{tabular}[htb]{|c|c||c|c||c|c||c||c|c|c|c|c|c|} 
\hline 
 Retraite en &  Âge &  Âge pivot &  Décote/Surcote &  Retraite (\euro{} 2019) &  Tx Rempl(\%) &  SMIC (\euro{} 2019) &  Retraite/SMIC &  Rev70/SMIC &  Rev75/SMIC &  Rev80/SMIC &  Rev85/SMIC &  Rev90/SMIC \\ 
\hline \hline 
 2065 &  62 &  65 ans 0 mois &  -15.00\% &  2876.56 &  {\bf 38.47} &  3076.71 &  {\bf {\color{red} 0.93}} &  {\bf {\color{red} 0.84}} &  {\bf {\color{red} 0.79}} &  {\bf {\color{red} 0.74}} &  {\bf {\color{red} 0.69}} &  {\bf {\color{red} 0.65}} \\ 
\hline 
 2066 &  63 &  65 ans 0 mois &  -10.00\% &  3181.16 &  {\bf 41.60} &  3116.71 &  {\bf 1.02} &  {\bf {\color{red} 0.93}} &  {\bf {\color{red} 0.87}} &  {\bf {\color{red} 0.82}} &  {\bf {\color{red} 0.77}} &  {\bf {\color{red} 0.72}} \\ 
\hline 
 2067 &  64 &  65 ans 0 mois &  -5.00\% &  3504.95 &  {\bf 44.82} &  3157.23 &  {\bf 1.11} &  {\bf 1.03} &  {\bf {\color{red} 0.96}} &  {\bf {\color{red} 0.90}} &  {\bf {\color{red} 0.85}} &  {\bf {\color{red} 0.79}} \\ 
\hline 
 2068 &  65 &  65 ans 0 mois &  0.00\% &  3848.67 &  {\bf 48.13} &  3198.27 &  {\bf 1.20} &  {\bf 1.13} &  {\bf 1.06} &  {\bf {\color{red} 0.99}} &  {\bf {\color{red} 0.93}} &  {\bf {\color{red} 0.87}} \\ 
\hline 
 2069 &  66 &  65 ans 0 mois &  5.00\% &  4213.12 &  {\bf 51.54} &  3239.85 &  {\bf 1.30} &  {\bf 1.23} &  {\bf 1.16} &  {\bf 1.09} &  {\bf 1.02} &  {\bf {\color{red} 0.95}} \\ 
\hline 
 2070 &  67 &  65 ans 0 mois &  10.00\% &  4599.09 &  {\bf 55.03} &  3281.97 &  {\bf 1.40} &  {\bf 1.35} &  {\bf 1.26} &  {\bf 1.18} &  {\bf 1.11} &  {\bf 1.04} \\ 
\hline 
\hline 
\end{tabular} 
\end{center} } 
\paragraph{Retraites possibles et ratios Revenu/SMIC à 70, 75, 80, 85, 90 ans avec le modèle \emph{Gouvernement corrigé (âge-pivot glissant)}}  
 
{ \scriptsize \begin{center} 
\begin{tabular}[htb]{|c|c||c|c||c|c||c||c|c|c|c|c|c|} 
\hline 
 Retraite en &  Âge &  Âge pivot &  Décote/Surcote &  Retraite (\euro{} 2019) &  Tx Rempl(\%) &  SMIC (\euro{} 2019) &  Retraite/SMIC &  Rev70/SMIC &  Rev75/SMIC &  Rev80/SMIC &  Rev85/SMIC &  Rev90/SMIC \\ 
\hline \hline 
 2065 &  62 &  67 ans 2 mois &  -25.83\% &  2509.94 &  {\bf 33.57} &  3076.71 &  {\bf {\color{red} 0.82}} &  {\bf {\color{red} 0.74}} &  {\bf {\color{red} 0.69}} &  {\bf {\color{red} 0.65}} &  {\bf {\color{red} 0.61}} &  {\bf {\color{red} 0.57}} \\ 
\hline 
 2066 &  63 &  67 ans 3 mois &  -21.25\% &  2783.51 &  {\bf 36.40} &  3116.71 &  {\bf {\color{red} 0.89}} &  {\bf {\color{red} 0.82}} &  {\bf {\color{red} 0.76}} &  {\bf {\color{red} 0.72}} &  {\bf {\color{red} 0.67}} &  {\bf {\color{red} 0.63}} \\ 
\hline 
 2067 &  64 &  67 ans 4 mois &  -16.67\% &  3074.51 &  {\bf 39.32} &  3157.23 &  {\bf {\color{red} 0.97}} &  {\bf {\color{red} 0.90}} &  {\bf {\color{red} 0.84}} &  {\bf {\color{red} 0.79}} &  {\bf {\color{red} 0.74}} &  {\bf {\color{red} 0.70}} \\ 
\hline 
 2068 &  65 &  67 ans 5 mois &  -12.08\% &  3383.63 &  {\bf 42.32} &  3198.27 &  {\bf 1.06} &  {\bf {\color{red} 0.99}} &  {\bf {\color{red} 0.93}} &  {\bf {\color{red} 0.87}} &  {\bf {\color{red} 0.82}} &  {\bf {\color{red} 0.77}} \\ 
\hline 
 2069 &  66 &  67 ans 6 mois &  -7.50\% &  3711.56 &  {\bf 45.40} &  3239.85 &  {\bf 1.15} &  {\bf 1.09} &  {\bf 1.02} &  {\bf {\color{red} 0.96}} &  {\bf {\color{red} 0.90}} &  {\bf {\color{red} 0.84}} \\ 
\hline 
 2070 &  67 &  67 ans 7 mois &  -2.92\% &  4059.05 &  {\bf 48.57} &  3281.97 &  {\bf 1.24} &  {\bf 1.19} &  {\bf 1.12} &  {\bf 1.05} &  {\bf {\color{red} 0.98}} &  {\bf {\color{red} 0.92}} \\ 
\hline 
\hline 
\end{tabular} 
\end{center} } 
\paragraph{Retraites possibles et ratios Revenu/SMIC à 70, 75, 80, 85, 90 ans avec le modèle \emph{Destinie2 (revalorisation de la fonction publique)}}  
 
{ \scriptsize \begin{center} 
\begin{tabular}[htb]{|c|c||c|c||c|c||c||c|c|c|c|c|c|} 
\hline 
 Retraite en &  Âge &  Âge pivot &  Décote/Surcote &  Retraite (\euro{} 2019) &  Tx Rempl(\%) &  SMIC (\euro{} 2019) &  Retraite/SMIC &  Rev70/SMIC &  Rev75/SMIC &  Rev80/SMIC &  Rev85/SMIC &  Rev90/SMIC \\ 
\hline \hline 
 2065 &  62 &  67 ans 2 mois &  -25.83\% &  2361.15 &  {\bf 33.59} &  2892.68 &  {\bf {\color{red} 0.82}} &  {\bf {\color{red} 0.74}} &  {\bf {\color{red} 0.69}} &  {\bf {\color{red} 0.65}} &  {\bf {\color{red} 0.61}} &  {\bf {\color{red} 0.57}} \\ 
\hline 
 2066 &  63 &  67 ans 3 mois &  -21.25\% &  2618.46 &  {\bf 36.42} &  2930.29 &  {\bf {\color{red} 0.89}} &  {\bf {\color{red} 0.82}} &  {\bf {\color{red} 0.77}} &  {\bf {\color{red} 0.72}} &  {\bf {\color{red} 0.67}} &  {\bf {\color{red} 0.63}} \\ 
\hline 
 2067 &  64 &  67 ans 4 mois &  -16.67\% &  2892.16 &  {\bf 39.34} &  2968.38 &  {\bf {\color{red} 0.97}} &  {\bf {\color{red} 0.90}} &  {\bf {\color{red} 0.85}} &  {\bf {\color{red} 0.79}} &  {\bf {\color{red} 0.74}} &  {\bf {\color{red} 0.70}} \\ 
\hline 
 2068 &  65 &  67 ans 5 mois &  -12.08\% &  3182.89 &  {\bf 42.34} &  3006.97 &  {\bf 1.06} &  {\bf {\color{red} 0.99}} &  {\bf {\color{red} 0.93}} &  {\bf {\color{red} 0.87}} &  {\bf {\color{red} 0.82}} &  {\bf {\color{red} 0.77}} \\ 
\hline 
 2069 &  66 &  67 ans 6 mois &  -7.50\% &  3491.31 &  {\bf 45.42} &  3046.06 &  {\bf 1.15} &  {\bf 1.09} &  {\bf 1.02} &  {\bf {\color{red} 0.96}} &  {\bf {\color{red} 0.90}} &  {\bf {\color{red} 0.84}} \\ 
\hline 
 2070 &  67 &  67 ans 7 mois &  -2.92\% &  3818.13 &  {\bf 48.59} &  3085.66 &  {\bf 1.24} &  {\bf 1.19} &  {\bf 1.12} &  {\bf 1.05} &  {\bf {\color{red} 0.98}} &  {\bf {\color{red} 0.92}} \\ 
\hline 
\hline 
\end{tabular} 
\end{center} } 

 \begin{center}\includegraphics[width=0.9\textwidth]{fig/Ascendant1525_2003_22_dest_retraite.pdf}\end{center} \label{fig/Ascendant1525_2003_22_dest_retraite.pdf} 

\newpage 
 
\chapter{Salarié privé évoluant de 2*SMIC à 3*SMIC} 


 \addto{\captionsenglish}{ \renewcommand{\mtctitle}{}} \setcounter{minitocdepth}{2} 
 \minitoc \newpage 

\section{Début de carrière à 22 ans} 

\subsection{Génération 1975 (début en 1997)} 

\paragraph{Retraites possibles et ratios Revenu/SMIC à 70, 75, 80, 85, 90 ans avec le modèle \emph{Gouvernement truqué (âge-pivot bloqué à 65 ans)}}  
 
{ \scriptsize \begin{center} 
\begin{tabular}[htb]{|c|c||c|c||c|c||c||c|c|c|c|c|c|} 
\hline 
 Retraite en &  Âge &  Âge pivot &  Décote/Surcote &  Retraite (\euro{} 2019) &  Tx Rempl(\%) &  SMIC (\euro{} 2019) &  Retraite/SMIC &  Rev70/SMIC &  Rev75/SMIC &  Rev80/SMIC &  Rev85/SMIC &  Rev90/SMIC \\ 
\hline \hline 
 2037 &  62 &  64 ans 10 mois &  -14.17\% &  2183.46 &  {\bf 34.77} &  2143.00 &  {\bf 1.02} &  {\bf {\color{red} 0.92}} &  {\bf {\color{red} 0.86}} &  {\bf {\color{red} 0.81}} &  {\bf {\color{red} 0.76}} &  {\bf {\color{red} 0.71}} \\ 
\hline 
 2038 &  63 &  64 ans 11 mois &  -9.58\% &  2398.28 &  {\bf 37.41} &  2170.86 &  {\bf 1.10} &  {\bf 1.01} &  {\bf {\color{red} 0.95}} &  {\bf {\color{red} 0.89}} &  {\bf {\color{red} 0.83}} &  {\bf {\color{red} 0.78}} \\ 
\hline 
 2039 &  64 &  65 ans 0 mois &  -5.00\% &  2627.57 &  {\bf 40.14} &  2199.08 &  {\bf 1.19} &  {\bf 1.11} &  {\bf 1.04} &  {\bf {\color{red} 0.97}} &  {\bf {\color{red} 0.91}} &  {\bf {\color{red} 0.85}} \\ 
\hline 
 2040 &  65 &  65 ans 0 mois &  0.00\% &  2884.22 &  {\bf 43.16} &  2227.67 &  {\bf 1.29} &  {\bf 1.21} &  {\bf 1.14} &  {\bf 1.07} &  {\bf {\color{red} 1.00}} &  {\bf {\color{red} 0.94}} \\ 
\hline 
 2041 &  66 &  65 ans 0 mois &  5.00\% &  3158.20 &  {\bf 46.29} &  2256.63 &  {\bf 1.40} &  {\bf 1.33} &  {\bf 1.25} &  {\bf 1.17} &  {\bf 1.09} &  {\bf 1.03} \\ 
\hline 
 2042 &  67 &  65 ans 0 mois &  10.00\% &  3450.60 &  {\bf 49.55} &  2285.97 &  {\bf 1.51} &  {\bf 1.45} &  {\bf 1.36} &  {\bf 1.28} &  {\bf 1.20} &  {\bf 1.12} \\ 
\hline 
\hline 
\end{tabular} 
\end{center} } 
\paragraph{Retraites possibles et ratios Revenu/SMIC à 70, 75, 80, 85, 90 ans avec le modèle \emph{Gouvernement corrigé (âge-pivot glissant)}}  
 
{ \scriptsize \begin{center} 
\begin{tabular}[htb]{|c|c||c|c||c|c||c||c|c|c|c|c|c|} 
\hline 
 Retraite en &  Âge &  Âge pivot &  Décote/Surcote &  Retraite (\euro{} 2019) &  Tx Rempl(\%) &  SMIC (\euro{} 2019) &  Retraite/SMIC &  Rev70/SMIC &  Rev75/SMIC &  Rev80/SMIC &  Rev85/SMIC &  Rev90/SMIC \\ 
\hline \hline 
 2037 &  62 &  64 ans 10 mois &  -14.17\% &  2183.46 &  {\bf 34.77} &  2143.00 &  {\bf 1.02} &  {\bf {\color{red} 0.92}} &  {\bf {\color{red} 0.86}} &  {\bf {\color{red} 0.81}} &  {\bf {\color{red} 0.76}} &  {\bf {\color{red} 0.71}} \\ 
\hline 
 2038 &  63 &  64 ans 11 mois &  -9.58\% &  2398.28 &  {\bf 37.41} &  2170.86 &  {\bf 1.10} &  {\bf 1.01} &  {\bf {\color{red} 0.95}} &  {\bf {\color{red} 0.89}} &  {\bf {\color{red} 0.83}} &  {\bf {\color{red} 0.78}} \\ 
\hline 
 2039 &  64 &  65 ans 0 mois &  -5.00\% &  2627.57 &  {\bf 40.14} &  2199.08 &  {\bf 1.19} &  {\bf 1.11} &  {\bf 1.04} &  {\bf {\color{red} 0.97}} &  {\bf {\color{red} 0.91}} &  {\bf {\color{red} 0.85}} \\ 
\hline 
 2040 &  65 &  65 ans 1 mois &  -0.42\% &  2872.20 &  {\bf 42.98} &  2227.67 &  {\bf 1.29} &  {\bf 1.21} &  {\bf 1.13} &  {\bf 1.06} &  {\bf {\color{red} 1.00}} &  {\bf {\color{red} 0.93}} \\ 
\hline 
 2041 &  66 &  65 ans 2 mois &  4.17\% &  3133.13 &  {\bf 45.92} &  2256.63 &  {\bf 1.39} &  {\bf 1.32} &  {\bf 1.24} &  {\bf 1.16} &  {\bf 1.09} &  {\bf 1.02} \\ 
\hline 
 2042 &  67 &  65 ans 3 mois &  8.75\% &  3411.39 &  {\bf 48.98} &  2285.97 &  {\bf 1.49} &  {\bf 1.44} &  {\bf 1.35} &  {\bf 1.26} &  {\bf 1.18} &  {\bf 1.11} \\ 
\hline 
\hline 
\end{tabular} 
\end{center} } 
\paragraph{Retraites possibles et ratios Revenu/SMIC à 70, 75, 80, 85, 90 ans avec le modèle \emph{Destinie2 (revalorisation de la fonction publique)}}  
 
{ \scriptsize \begin{center} 
\begin{tabular}[htb]{|c|c||c|c||c|c||c||c|c|c|c|c|c|} 
\hline 
 Retraite en &  Âge &  Âge pivot &  Décote/Surcote &  Retraite (\euro{} 2019) &  Tx Rempl(\%) &  SMIC (\euro{} 2019) &  Retraite/SMIC &  Rev70/SMIC &  Rev75/SMIC &  Rev80/SMIC &  Rev85/SMIC &  Rev90/SMIC \\ 
\hline \hline 
 2037 &  62 &  64 ans 10 mois &  -14.17\% &  2124.23 &  {\bf 35.98} &  2014.82 &  {\bf 1.05} &  {\bf {\color{red} 0.95}} &  {\bf {\color{red} 0.89}} &  {\bf {\color{red} 0.84}} &  {\bf {\color{red} 0.78}} &  {\bf {\color{red} 0.73}} \\ 
\hline 
 2038 &  63 &  64 ans 11 mois &  -9.58\% &  2330.59 &  {\bf 38.66} &  2041.01 &  {\bf 1.14} &  {\bf 1.04} &  {\bf {\color{red} 0.98}} &  {\bf {\color{red} 0.92}} &  {\bf {\color{red} 0.86}} &  {\bf {\color{red} 0.81}} \\ 
\hline 
 2039 &  64 &  65 ans 0 mois &  -5.00\% &  2550.67 &  {\bf 41.44} &  2067.55 &  {\bf 1.23} &  {\bf 1.14} &  {\bf 1.07} &  {\bf 1.00} &  {\bf {\color{red} 0.94}} &  {\bf {\color{red} 0.88}} \\ 
\hline 
 2040 &  65 &  65 ans 1 mois &  -0.42\% &  2785.31 &  {\bf 44.33} &  2094.43 &  {\bf 1.33} &  {\bf 1.25} &  {\bf 1.17} &  {\bf 1.10} &  {\bf 1.03} &  {\bf {\color{red} 0.96}} \\ 
\hline 
 2041 &  66 &  65 ans 2 mois &  4.17\% &  3035.43 &  {\bf 47.32} &  2121.65 &  {\bf 1.43} &  {\bf 1.36} &  {\bf 1.27} &  {\bf 1.19} &  {\bf 1.12} &  {\bf 1.05} \\ 
\hline 
 2042 &  67 &  65 ans 3 mois &  8.75\% &  3301.99 &  {\bf 50.43} &  2149.23 &  {\bf 1.54} &  {\bf 1.48} &  {\bf 1.39} &  {\bf 1.30} &  {\bf 1.22} &  {\bf 1.14} \\ 
\hline 
\hline 
\end{tabular} 
\end{center} } 

 \begin{center}\includegraphics[width=0.9\textwidth]{fig/Ascendant23_1975_22_dest_retraite.pdf}\end{center} \label{fig/Ascendant23_1975_22_dest_retraite.pdf} 

\newpage 
 
\subsection{Génération 1980 (début en 2002)} 

\paragraph{Retraites possibles et ratios Revenu/SMIC à 70, 75, 80, 85, 90 ans avec le modèle \emph{Gouvernement truqué (âge-pivot bloqué à 65 ans)}}  
 
{ \scriptsize \begin{center} 
\begin{tabular}[htb]{|c|c||c|c||c|c||c||c|c|c|c|c|c|} 
\hline 
 Retraite en &  Âge &  Âge pivot &  Décote/Surcote &  Retraite (\euro{} 2019) &  Tx Rempl(\%) &  SMIC (\euro{} 2019) &  Retraite/SMIC &  Rev70/SMIC &  Rev75/SMIC &  Rev80/SMIC &  Rev85/SMIC &  Rev90/SMIC \\ 
\hline \hline 
 2042 &  62 &  65 ans 0 mois &  -15.00\% &  2372.88 &  {\bf 35.42} &  2285.97 &  {\bf 1.04} &  {\bf {\color{red} 0.94}} &  {\bf {\color{red} 0.88}} &  {\bf {\color{red} 0.82}} &  {\bf {\color{red} 0.77}} &  {\bf {\color{red} 0.72}} \\ 
\hline 
 2043 &  63 &  65 ans 0 mois &  -10.00\% &  2626.94 &  {\bf 38.41} &  2315.68 &  {\bf 1.13} &  {\bf 1.04} &  {\bf {\color{red} 0.97}} &  {\bf {\color{red} 0.91}} &  {\bf {\color{red} 0.85}} &  {\bf {\color{red} 0.80}} \\ 
\hline 
 2044 &  64 &  65 ans 0 mois &  -5.00\% &  2899.13 &  {\bf 41.52} &  2345.79 &  {\bf 1.24} &  {\bf 1.14} &  {\bf 1.07} &  {\bf 1.01} &  {\bf {\color{red} 0.94}} &  {\bf {\color{red} 0.88}} \\ 
\hline 
 2045 &  65 &  65 ans 0 mois &  0.00\% &  3190.62 &  {\bf 44.76} &  2376.28 &  {\bf 1.34} &  {\bf 1.26} &  {\bf 1.18} &  {\bf 1.11} &  {\bf 1.04} &  {\bf {\color{red} 0.97}} \\ 
\hline 
 2046 &  66 &  65 ans 0 mois &  5.00\% &  3500.06 &  {\bf 48.09} &  2407.18 &  {\bf 1.45} &  {\bf 1.38} &  {\bf 1.29} &  {\bf 1.21} &  {\bf 1.14} &  {\bf 1.07} \\ 
\hline 
 2047 &  67 &  65 ans 0 mois &  10.00\% &  3828.15 &  {\bf 51.53} &  2438.47 &  {\bf 1.57} &  {\bf 1.51} &  {\bf 1.42} &  {\bf 1.33} &  {\bf 1.24} &  {\bf 1.17} \\ 
\hline 
\hline 
\end{tabular} 
\end{center} } 
\paragraph{Retraites possibles et ratios Revenu/SMIC à 70, 75, 80, 85, 90 ans avec le modèle \emph{Gouvernement corrigé (âge-pivot glissant)}}  
 
{ \scriptsize \begin{center} 
\begin{tabular}[htb]{|c|c||c|c||c|c||c||c|c|c|c|c|c|} 
\hline 
 Retraite en &  Âge &  Âge pivot &  Décote/Surcote &  Retraite (\euro{} 2019) &  Tx Rempl(\%) &  SMIC (\euro{} 2019) &  Retraite/SMIC &  Rev70/SMIC &  Rev75/SMIC &  Rev80/SMIC &  Rev85/SMIC &  Rev90/SMIC \\ 
\hline \hline 
 2042 &  62 &  65 ans 3 mois &  -16.25\% &  2337.98 &  {\bf 34.90} &  2285.97 &  {\bf 1.02} &  {\bf {\color{red} 0.92}} &  {\bf {\color{red} 0.86}} &  {\bf {\color{red} 0.81}} &  {\bf {\color{red} 0.76}} &  {\bf {\color{red} 0.71}} \\ 
\hline 
 2043 &  63 &  65 ans 4 mois &  -11.67\% &  2578.29 &  {\bf 37.70} &  2315.68 &  {\bf 1.11} &  {\bf 1.02} &  {\bf {\color{red} 0.95}} &  {\bf {\color{red} 0.89}} &  {\bf {\color{red} 0.84}} &  {\bf {\color{red} 0.79}} \\ 
\hline 
 2044 &  64 &  65 ans 5 mois &  -7.08\% &  2835.55 &  {\bf 40.61} &  2345.79 &  {\bf 1.21} &  {\bf 1.12} &  {\bf 1.05} &  {\bf {\color{red} 0.98}} &  {\bf {\color{red} 0.92}} &  {\bf {\color{red} 0.86}} \\ 
\hline 
 2045 &  65 &  65 ans 6 mois &  -2.50\% &  3110.85 &  {\bf 43.64} &  2376.28 &  {\bf 1.31} &  {\bf 1.23} &  {\bf 1.15} &  {\bf 1.08} &  {\bf 1.01} &  {\bf {\color{red} 0.95}} \\ 
\hline 
 2046 &  66 &  65 ans 7 mois &  2.08\% &  3402.84 &  {\bf 46.76} &  2407.18 &  {\bf 1.41} &  {\bf 1.34} &  {\bf 1.26} &  {\bf 1.18} &  {\bf 1.11} &  {\bf 1.04} \\ 
\hline 
 2047 &  67 &  65 ans 8 mois &  6.67\% &  3712.14 &  {\bf 49.97} &  2438.47 &  {\bf 1.52} &  {\bf 1.46} &  {\bf 1.37} &  {\bf 1.29} &  {\bf 1.21} &  {\bf 1.13} \\ 
\hline 
\hline 
\end{tabular} 
\end{center} } 
\paragraph{Retraites possibles et ratios Revenu/SMIC à 70, 75, 80, 85, 90 ans avec le modèle \emph{Destinie2 (revalorisation de la fonction publique)}}  
 
{ \scriptsize \begin{center} 
\begin{tabular}[htb]{|c|c||c|c||c|c||c||c|c|c|c|c|c|} 
\hline 
 Retraite en &  Âge &  Âge pivot &  Décote/Surcote &  Retraite (\euro{} 2019) &  Tx Rempl(\%) &  SMIC (\euro{} 2019) &  Retraite/SMIC &  Rev70/SMIC &  Rev75/SMIC &  Rev80/SMIC &  Rev85/SMIC &  Rev90/SMIC \\ 
\hline \hline 
 2042 &  62 &  65 ans 3 mois &  -16.25\% &  2257.23 &  {\bf 35.84} &  2149.23 &  {\bf 1.05} &  {\bf {\color{red} 0.95}} &  {\bf {\color{red} 0.89}} &  {\bf {\color{red} 0.83}} &  {\bf {\color{red} 0.78}} &  {\bf {\color{red} 0.73}} \\ 
\hline 
 2043 &  63 &  65 ans 4 mois &  -11.67\% &  2487.11 &  {\bf 38.68} &  2177.17 &  {\bf 1.14} &  {\bf 1.04} &  {\bf {\color{red} 0.98}} &  {\bf {\color{red} 0.92}} &  {\bf {\color{red} 0.86}} &  {\bf {\color{red} 0.81}} \\ 
\hline 
 2044 &  64 &  65 ans 5 mois &  -7.08\% &  2733.06 &  {\bf 41.63} &  2205.48 &  {\bf 1.24} &  {\bf 1.15} &  {\bf 1.08} &  {\bf 1.01} &  {\bf {\color{red} 0.94}} &  {\bf {\color{red} 0.89}} \\ 
\hline 
 2045 &  65 &  65 ans 6 mois &  -2.50\% &  2996.12 &  {\bf 44.70} &  2234.15 &  {\bf 1.34} &  {\bf 1.26} &  {\bf 1.18} &  {\bf 1.10} &  {\bf 1.04} &  {\bf {\color{red} 0.97}} \\ 
\hline 
 2046 &  66 &  65 ans 7 mois &  2.08\% &  3274.97 &  {\bf 47.86} &  2263.19 &  {\bf 1.45} &  {\bf 1.37} &  {\bf 1.29} &  {\bf 1.21} &  {\bf 1.13} &  {\bf 1.06} \\ 
\hline 
 2047 &  67 &  65 ans 8 mois &  6.67\% &  3570.19 &  {\bf 51.12} &  2292.61 &  {\bf 1.56} &  {\bf 1.50} &  {\bf 1.40} &  {\bf 1.32} &  {\bf 1.23} &  {\bf 1.16} \\ 
\hline 
\hline 
\end{tabular} 
\end{center} } 

 \begin{center}\includegraphics[width=0.9\textwidth]{fig/Ascendant23_1980_22_dest_retraite.pdf}\end{center} \label{fig/Ascendant23_1980_22_dest_retraite.pdf} 

\newpage 
 
\subsection{Génération 1990 (début en 2012)} 

\paragraph{Retraites possibles et ratios Revenu/SMIC à 70, 75, 80, 85, 90 ans avec le modèle \emph{Gouvernement truqué (âge-pivot bloqué à 65 ans)}}  
 
{ \scriptsize \begin{center} 
\begin{tabular}[htb]{|c|c||c|c||c|c||c||c|c|c|c|c|c|} 
\hline 
 Retraite en &  Âge &  Âge pivot &  Décote/Surcote &  Retraite (\euro{} 2019) &  Tx Rempl(\%) &  SMIC (\euro{} 2019) &  Retraite/SMIC &  Rev70/SMIC &  Rev75/SMIC &  Rev80/SMIC &  Rev85/SMIC &  Rev90/SMIC \\ 
\hline \hline 
 2052 &  62 &  65 ans 0 mois &  -15.00\% &  2877.85 &  {\bf 37.76} &  2601.14 &  {\bf 1.11} &  {\bf {\color{red} 1.00}} &  {\bf {\color{red} 0.94}} &  {\bf {\color{red} 0.88}} &  {\bf {\color{red} 0.82}} &  {\bf {\color{red} 0.77}} \\ 
\hline 
 2053 &  63 &  65 ans 0 mois &  -10.00\% &  3184.24 &  {\bf 40.92} &  2634.96 &  {\bf 1.21} &  {\bf 1.10} &  {\bf 1.03} &  {\bf {\color{red} 0.97}} &  {\bf {\color{red} 0.91}} &  {\bf {\color{red} 0.85}} \\ 
\hline 
 2054 &  64 &  65 ans 0 mois &  -5.00\% &  3509.91 &  {\bf 44.17} &  2669.21 &  {\bf 1.31} &  {\bf 1.22} &  {\bf 1.14} &  {\bf 1.07} &  {\bf 1.00} &  {\bf {\color{red} 0.94}} \\ 
\hline 
 2055 &  65 &  65 ans 0 mois &  0.00\% &  3855.58 &  {\bf 47.53} &  2703.91 &  {\bf 1.43} &  {\bf 1.34} &  {\bf 1.25} &  {\bf 1.17} &  {\bf 1.10} &  {\bf 1.03} \\ 
\hline 
 2056 &  66 &  65 ans 0 mois &  5.00\% &  4222.02 &  {\bf 50.99} &  2739.06 &  {\bf 1.54} &  {\bf 1.46} &  {\bf 1.37} &  {\bf 1.29} &  {\bf 1.21} &  {\bf 1.13} \\ 
\hline 
 2057 &  67 &  65 ans 0 mois &  10.00\% &  4609.99 &  {\bf 54.54} &  2774.67 &  {\bf 1.66} &  {\bf 1.60} &  {\bf 1.50} &  {\bf 1.40} &  {\bf 1.32} &  {\bf 1.23} \\ 
\hline 
\hline 
\end{tabular} 
\end{center} } 
\paragraph{Retraites possibles et ratios Revenu/SMIC à 70, 75, 80, 85, 90 ans avec le modèle \emph{Gouvernement corrigé (âge-pivot glissant)}}  
 
{ \scriptsize \begin{center} 
\begin{tabular}[htb]{|c|c||c|c||c|c||c||c|c|c|c|c|c|} 
\hline 
 Retraite en &  Âge &  Âge pivot &  Décote/Surcote &  Retraite (\euro{} 2019) &  Tx Rempl(\%) &  SMIC (\euro{} 2019) &  Retraite/SMIC &  Rev70/SMIC &  Rev75/SMIC &  Rev80/SMIC &  Rev85/SMIC &  Rev90/SMIC \\ 
\hline \hline 
 2052 &  62 &  66 ans 1 mois &  -20.42\% &  2694.46 &  {\bf 35.35} &  2601.14 &  {\bf 1.04} &  {\bf {\color{red} 0.93}} &  {\bf {\color{red} 0.88}} &  {\bf {\color{red} 0.82}} &  {\bf {\color{red} 0.77}} &  {\bf {\color{red} 0.72}} \\ 
\hline 
 2053 &  63 &  66 ans 2 mois &  -15.83\% &  2977.86 &  {\bf 38.26} &  2634.96 &  {\bf 1.13} &  {\bf 1.03} &  {\bf {\color{red} 0.97}} &  {\bf {\color{red} 0.91}} &  {\bf {\color{red} 0.85}} &  {\bf {\color{red} 0.80}} \\ 
\hline 
 2054 &  64 &  66 ans 3 mois &  -11.25\% &  3278.99 &  {\bf 41.27} &  2669.21 &  {\bf 1.23} &  {\bf 1.14} &  {\bf 1.07} &  {\bf {\color{red} 1.00}} &  {\bf {\color{red} 0.94}} &  {\bf {\color{red} 0.88}} \\ 
\hline 
 2055 &  65 &  66 ans 4 mois &  -6.67\% &  3598.54 &  {\bf 44.36} &  2703.91 &  {\bf 1.33} &  {\bf 1.25} &  {\bf 1.17} &  {\bf 1.10} &  {\bf 1.03} &  {\bf {\color{red} 0.96}} \\ 
\hline 
 2056 &  66 &  66 ans 5 mois &  -2.08\% &  3937.20 &  {\bf 47.55} &  2739.06 &  {\bf 1.44} &  {\bf 1.37} &  {\bf 1.28} &  {\bf 1.20} &  {\bf 1.12} &  {\bf 1.05} \\ 
\hline 
 2057 &  67 &  66 ans 6 mois &  2.50\% &  4295.68 &  {\bf 50.82} &  2774.67 &  {\bf 1.55} &  {\bf 1.49} &  {\bf 1.40} &  {\bf 1.31} &  {\bf 1.23} &  {\bf 1.15} \\ 
\hline 
\hline 
\end{tabular} 
\end{center} } 
\paragraph{Retraites possibles et ratios Revenu/SMIC à 70, 75, 80, 85, 90 ans avec le modèle \emph{Destinie2 (revalorisation de la fonction publique)}}  
 
{ \scriptsize \begin{center} 
\begin{tabular}[htb]{|c|c||c|c||c|c||c||c|c|c|c|c|c|} 
\hline 
 Retraite en &  Âge &  Âge pivot &  Décote/Surcote &  Retraite (\euro{} 2019) &  Tx Rempl(\%) &  SMIC (\euro{} 2019) &  Retraite/SMIC &  Rev70/SMIC &  Rev75/SMIC &  Rev80/SMIC &  Rev85/SMIC &  Rev90/SMIC \\ 
\hline \hline 
 2052 &  62 &  66 ans 1 mois &  -20.42\% &  2566.18 &  {\bf 35.81} &  2445.56 &  {\bf 1.05} &  {\bf {\color{red} 0.95}} &  {\bf {\color{red} 0.89}} &  {\bf {\color{red} 0.83}} &  {\bf {\color{red} 0.78}} &  {\bf {\color{red} 0.73}} \\ 
\hline 
 2053 &  63 &  66 ans 2 mois &  -15.83\% &  2834.98 &  {\bf 38.75} &  2477.35 &  {\bf 1.14} &  {\bf 1.05} &  {\bf {\color{red} 0.98}} &  {\bf {\color{red} 0.92}} &  {\bf {\color{red} 0.86}} &  {\bf {\color{red} 0.81}} \\ 
\hline 
 2054 &  64 &  66 ans 3 mois &  -11.25\% &  3120.50 &  {\bf 41.77} &  2509.56 &  {\bf 1.24} &  {\bf 1.15} &  {\bf 1.08} &  {\bf 1.01} &  {\bf {\color{red} 0.95}} &  {\bf {\color{red} 0.89}} \\ 
\hline 
 2055 &  65 &  66 ans 4 mois &  -6.67\% &  3423.39 &  {\bf 44.89} &  2542.18 &  {\bf 1.35} &  {\bf 1.26} &  {\bf 1.18} &  {\bf 1.11} &  {\bf 1.04} &  {\bf {\color{red} 0.98}} \\ 
\hline 
 2056 &  66 &  66 ans 5 mois &  -2.08\% &  3744.31 &  {\bf 48.09} &  2575.23 &  {\bf 1.45} &  {\bf 1.38} &  {\bf 1.29} &  {\bf 1.21} &  {\bf 1.14} &  {\bf 1.07} \\ 
\hline 
 2057 &  67 &  66 ans 6 mois &  2.50\% &  4083.92 &  {\bf 51.39} &  2608.71 &  {\bf 1.57} &  {\bf 1.51} &  {\bf 1.41} &  {\bf 1.32} &  {\bf 1.24} &  {\bf 1.16} \\ 
\hline 
\hline 
\end{tabular} 
\end{center} } 

 \begin{center}\includegraphics[width=0.9\textwidth]{fig/Ascendant23_1990_22_dest_retraite.pdf}\end{center} \label{fig/Ascendant23_1990_22_dest_retraite.pdf} 

\newpage 
 
\subsection{Génération 2003 (début en 2025)} 

\paragraph{Retraites possibles et ratios Revenu/SMIC à 70, 75, 80, 85, 90 ans avec le modèle \emph{Gouvernement truqué (âge-pivot bloqué à 65 ans)}}  
 
{ \scriptsize \begin{center} 
\begin{tabular}[htb]{|c|c||c|c||c|c||c||c|c|c|c|c|c|} 
\hline 
 Retraite en &  Âge &  Âge pivot &  Décote/Surcote &  Retraite (\euro{} 2019) &  Tx Rempl(\%) &  SMIC (\euro{} 2019) &  Retraite/SMIC &  Rev70/SMIC &  Rev75/SMIC &  Rev80/SMIC &  Rev85/SMIC &  Rev90/SMIC \\ 
\hline \hline 
 2065 &  62 &  65 ans 0 mois &  -15.00\% &  3605.27 &  {\bf 39.99} &  3076.71 &  {\bf 1.17} &  {\bf 1.06} &  {\bf {\color{red} 0.99}} &  {\bf {\color{red} 0.93}} &  {\bf {\color{red} 0.87}} &  {\bf {\color{red} 0.82}} \\ 
\hline 
 2066 &  63 &  65 ans 0 mois &  -10.00\% &  3982.28 &  {\bf 43.26} &  3116.71 &  {\bf 1.28} &  {\bf 1.17} &  {\bf 1.09} &  {\bf 1.03} &  {\bf {\color{red} 0.96}} &  {\bf {\color{red} 0.90}} \\ 
\hline 
 2067 &  64 &  65 ans 0 mois &  -5.00\% &  4382.45 &  {\bf 46.63} &  3157.23 &  {\bf 1.39} &  {\bf 1.28} &  {\bf 1.20} &  {\bf 1.13} &  {\bf 1.06} &  {\bf {\color{red} 0.99}} \\ 
\hline 
 2068 &  65 &  65 ans 0 mois &  0.00\% &  4806.63 &  {\bf 50.10} &  3198.27 &  {\bf 1.50} &  {\bf 1.41} &  {\bf 1.32} &  {\bf 1.24} &  {\bf 1.16} &  {\bf 1.09} \\ 
\hline 
 2069 &  66 &  65 ans 0 mois &  5.00\% &  5255.72 &  {\bf 53.66} &  3239.85 &  {\bf 1.62} &  {\bf 1.54} &  {\bf 1.44} &  {\bf 1.35} &  {\bf 1.27} &  {\bf 1.19} \\ 
\hline 
 2070 &  67 &  65 ans 0 mois &  10.00\% &  5730.66 &  {\bf 57.31} &  3281.97 &  {\bf 1.75} &  {\bf 1.68} &  {\bf 1.57} &  {\bf 1.48} &  {\bf 1.38} &  {\bf 1.30} \\ 
\hline 
\hline 
\end{tabular} 
\end{center} } 
\paragraph{Retraites possibles et ratios Revenu/SMIC à 70, 75, 80, 85, 90 ans avec le modèle \emph{Gouvernement corrigé (âge-pivot glissant)}}  
 
{ \scriptsize \begin{center} 
\begin{tabular}[htb]{|c|c||c|c||c|c||c||c|c|c|c|c|c|} 
\hline 
 Retraite en &  Âge &  Âge pivot &  Décote/Surcote &  Retraite (\euro{} 2019) &  Tx Rempl(\%) &  SMIC (\euro{} 2019) &  Retraite/SMIC &  Rev70/SMIC &  Rev75/SMIC &  Rev80/SMIC &  Rev85/SMIC &  Rev90/SMIC \\ 
\hline \hline 
 2065 &  62 &  67 ans 2 mois &  -25.83\% &  3145.77 &  {\bf 34.89} &  3076.71 &  {\bf 1.02} &  {\bf {\color{red} 0.92}} &  {\bf {\color{red} 0.86}} &  {\bf {\color{red} 0.81}} &  {\bf {\color{red} 0.76}} &  {\bf {\color{red} 0.71}} \\ 
\hline 
 2066 &  63 &  67 ans 3 mois &  -21.25\% &  3484.50 &  {\bf 37.85} &  3116.71 &  {\bf 1.12} &  {\bf 1.02} &  {\bf {\color{red} 0.96}} &  {\bf {\color{red} 0.90}} &  {\bf {\color{red} 0.84}} &  {\bf {\color{red} 0.79}} \\ 
\hline 
 2067 &  64 &  67 ans 4 mois &  -16.67\% &  3844.25 &  {\bf 40.90} &  3157.23 &  {\bf 1.22} &  {\bf 1.13} &  {\bf 1.06} &  {\bf {\color{red} 0.99}} &  {\bf {\color{red} 0.93}} &  {\bf {\color{red} 0.87}} \\ 
\hline 
 2068 &  65 &  67 ans 5 mois &  -12.08\% &  4225.82 &  {\bf 44.04} &  3198.27 &  {\bf 1.32} &  {\bf 1.24} &  {\bf 1.16} &  {\bf 1.09} &  {\bf 1.02} &  {\bf {\color{red} 0.96}} \\ 
\hline 
 2069 &  66 &  67 ans 6 mois &  -7.50\% &  4630.04 &  {\bf 47.27} &  3239.85 &  {\bf 1.43} &  {\bf 1.36} &  {\bf 1.27} &  {\bf 1.19} &  {\bf 1.12} &  {\bf 1.05} \\ 
\hline 
 2070 &  67 &  67 ans 7 mois &  -2.92\% &  5057.75 &  {\bf 50.58} &  3281.97 &  {\bf 1.54} &  {\bf 1.48} &  {\bf 1.39} &  {\bf 1.30} &  {\bf 1.22} &  {\bf 1.14} \\ 
\hline 
\hline 
\end{tabular} 
\end{center} } 
\paragraph{Retraites possibles et ratios Revenu/SMIC à 70, 75, 80, 85, 90 ans avec le modèle \emph{Destinie2 (revalorisation de la fonction publique)}}  
 
{ \scriptsize \begin{center} 
\begin{tabular}[htb]{|c|c||c|c||c|c||c||c|c|c|c|c|c|} 
\hline 
 Retraite en &  Âge &  Âge pivot &  Décote/Surcote &  Retraite (\euro{} 2019) &  Tx Rempl(\%) &  SMIC (\euro{} 2019) &  Retraite/SMIC &  Rev70/SMIC &  Rev75/SMIC &  Rev80/SMIC &  Rev85/SMIC &  Rev90/SMIC \\ 
\hline \hline 
 2065 &  62 &  67 ans 2 mois &  -25.83\% &  2959.39 &  {\bf 34.91} &  2892.68 &  {\bf 1.02} &  {\bf {\color{red} 0.92}} &  {\bf {\color{red} 0.86}} &  {\bf {\color{red} 0.81}} &  {\bf {\color{red} 0.76}} &  {\bf {\color{red} 0.71}} \\ 
\hline 
 2066 &  63 &  67 ans 3 mois &  -21.25\% &  3277.99 &  {\bf 37.88} &  2930.29 &  {\bf 1.12} &  {\bf 1.02} &  {\bf {\color{red} 0.96}} &  {\bf {\color{red} 0.90}} &  {\bf {\color{red} 0.84}} &  {\bf {\color{red} 0.79}} \\ 
\hline 
 2067 &  64 &  67 ans 4 mois &  -16.67\% &  3616.36 &  {\bf 40.93} &  2968.38 &  {\bf 1.22} &  {\bf 1.13} &  {\bf 1.06} &  {\bf {\color{red} 0.99}} &  {\bf {\color{red} 0.93}} &  {\bf {\color{red} 0.87}} \\ 
\hline 
 2068 &  65 &  67 ans 5 mois &  -12.08\% &  3975.25 &  {\bf 44.07} &  3006.97 &  {\bf 1.32} &  {\bf 1.24} &  {\bf 1.16} &  {\bf 1.09} &  {\bf 1.02} &  {\bf {\color{red} 0.96}} \\ 
\hline 
 2069 &  66 &  67 ans 6 mois &  -7.50\% &  4355.43 &  {\bf 47.30} &  3046.06 &  {\bf 1.43} &  {\bf 1.36} &  {\bf 1.27} &  {\bf 1.19} &  {\bf 1.12} &  {\bf 1.05} \\ 
\hline 
 2070 &  67 &  67 ans 7 mois &  -2.92\% &  4757.70 &  {\bf 50.61} &  3085.66 &  {\bf 1.54} &  {\bf 1.48} &  {\bf 1.39} &  {\bf 1.30} &  {\bf 1.22} &  {\bf 1.15} \\ 
\hline 
\hline 
\end{tabular} 
\end{center} } 

 \begin{center}\includegraphics[width=0.9\textwidth]{fig/Ascendant23_2003_22_dest_retraite.pdf}\end{center} \label{fig/Ascendant23_2003_22_dest_retraite.pdf} 

\newpage 
 
\chapter{Salarié privé évoluant du 3*SMIC à 4*SMIC} 


 \addto{\captionsenglish}{ \renewcommand{\mtctitle}{}} \setcounter{minitocdepth}{2} 
 \minitoc \newpage 

\section{Début de carrière à 22 ans} 

\subsection{Génération 1975 (début en 1997)} 

\paragraph{Retraites possibles et ratios Revenu/SMIC à 70, 75, 80, 85, 90 ans avec le modèle \emph{Gouvernement truqué (âge-pivot bloqué à 65 ans)}}  
 
{ \scriptsize \begin{center} 
\begin{tabular}[htb]{|c|c||c|c||c|c||c||c|c|c|c|c|c|} 
\hline 
 Retraite en &  Âge &  Âge pivot &  Décote/Surcote &  Retraite (\euro{} 2019) &  Tx Rempl(\%) &  SMIC (\euro{} 2019) &  Retraite/SMIC &  Rev70/SMIC &  Rev75/SMIC &  Rev80/SMIC &  Rev85/SMIC &  Rev90/SMIC \\ 
\hline \hline 
 2037 &  62 &  64 ans 10 mois &  -14.17\% &  3058.13 &  {\bf 36.31} &  2143.00 &  {\bf 1.43} &  {\bf 1.29} &  {\bf 1.21} &  {\bf 1.13} &  {\bf 1.06} &  {\bf {\color{red} 0.99}} \\ 
\hline 
 2038 &  63 &  64 ans 11 mois &  -9.58\% &  3354.01 &  {\bf 39.08} &  2170.86 &  {\bf 1.55} &  {\bf 1.41} &  {\bf 1.32} &  {\bf 1.24} &  {\bf 1.16} &  {\bf 1.09} \\ 
\hline 
 2039 &  64 &  65 ans 0 mois &  -5.00\% &  3669.25 &  {\bf 41.96} &  2199.08 &  {\bf 1.67} &  {\bf 1.54} &  {\bf 1.45} &  {\bf 1.36} &  {\bf 1.27} &  {\bf 1.19} \\ 
\hline 
 2040 &  65 &  65 ans 0 mois &  0.00\% &  4021.77 &  {\bf 45.13} &  2227.67 &  {\bf 1.81} &  {\bf 1.69} &  {\bf 1.59} &  {\bf 1.49} &  {\bf 1.39} &  {\bf 1.31} \\ 
\hline 
 2041 &  66 &  65 ans 0 mois &  5.00\% &  4397.47 &  {\bf 48.44} &  2256.63 &  {\bf 1.95} &  {\bf 1.85} &  {\bf 1.73} &  {\bf 1.63} &  {\bf 1.52} &  {\bf 1.43} \\ 
\hline 
 2042 &  67 &  65 ans 0 mois &  10.00\% &  4797.77 &  {\bf 51.87} &  2285.97 &  {\bf 2.10} &  {\bf 2.02} &  {\bf 1.89} &  {\bf 1.77} &  {\bf 1.66} &  {\bf 1.56} \\ 
\hline 
\hline 
\end{tabular} 
\end{center} } 
\paragraph{Retraites possibles et ratios Revenu/SMIC à 70, 75, 80, 85, 90 ans avec le modèle \emph{Gouvernement corrigé (âge-pivot glissant)}}  
 
{ \scriptsize \begin{center} 
\begin{tabular}[htb]{|c|c||c|c||c|c||c||c|c|c|c|c|c|} 
\hline 
 Retraite en &  Âge &  Âge pivot &  Décote/Surcote &  Retraite (\euro{} 2019) &  Tx Rempl(\%) &  SMIC (\euro{} 2019) &  Retraite/SMIC &  Rev70/SMIC &  Rev75/SMIC &  Rev80/SMIC &  Rev85/SMIC &  Rev90/SMIC \\ 
\hline \hline 
 2037 &  62 &  64 ans 10 mois &  -14.17\% &  3058.13 &  {\bf 36.31} &  2143.00 &  {\bf 1.43} &  {\bf 1.29} &  {\bf 1.21} &  {\bf 1.13} &  {\bf 1.06} &  {\bf {\color{red} 0.99}} \\ 
\hline 
 2038 &  63 &  64 ans 11 mois &  -9.58\% &  3354.01 &  {\bf 39.08} &  2170.86 &  {\bf 1.55} &  {\bf 1.41} &  {\bf 1.32} &  {\bf 1.24} &  {\bf 1.16} &  {\bf 1.09} \\ 
\hline 
 2039 &  64 &  65 ans 0 mois &  -5.00\% &  3669.25 &  {\bf 41.96} &  2199.08 &  {\bf 1.67} &  {\bf 1.54} &  {\bf 1.45} &  {\bf 1.36} &  {\bf 1.27} &  {\bf 1.19} \\ 
\hline 
 2040 &  65 &  65 ans 1 mois &  -0.42\% &  4005.01 &  {\bf 44.95} &  2227.67 &  {\bf 1.80} &  {\bf 1.69} &  {\bf 1.58} &  {\bf 1.48} &  {\bf 1.39} &  {\bf 1.30} \\ 
\hline 
 2041 &  66 &  65 ans 2 mois &  4.17\% &  4362.57 &  {\bf 48.05} &  2256.63 &  {\bf 1.93} &  {\bf 1.84} &  {\bf 1.72} &  {\bf 1.61} &  {\bf 1.51} &  {\bf 1.42} \\ 
\hline 
 2042 &  67 &  65 ans 3 mois &  8.75\% &  4743.25 &  {\bf 51.28} &  2285.97 &  {\bf 2.07} &  {\bf 2.00} &  {\bf 1.87} &  {\bf 1.75} &  {\bf 1.64} &  {\bf 1.54} \\ 
\hline 
\hline 
\end{tabular} 
\end{center} } 
\paragraph{Retraites possibles et ratios Revenu/SMIC à 70, 75, 80, 85, 90 ans avec le modèle \emph{Destinie2 (revalorisation de la fonction publique)}}  
 
{ \scriptsize \begin{center} 
\begin{tabular}[htb]{|c|c||c|c||c|c||c||c|c|c|c|c|c|} 
\hline 
 Retraite en &  Âge &  Âge pivot &  Décote/Surcote &  Retraite (\euro{} 2019) &  Tx Rempl(\%) &  SMIC (\euro{} 2019) &  Retraite/SMIC &  Rev70/SMIC &  Rev75/SMIC &  Rev80/SMIC &  Rev85/SMIC &  Rev90/SMIC \\ 
\hline \hline 
 2037 &  62 &  64 ans 10 mois &  -14.17\% &  2977.49 &  {\bf 37.60} &  2014.82 &  {\bf 1.48} &  {\bf 1.33} &  {\bf 1.25} &  {\bf 1.17} &  {\bf 1.10} &  {\bf 1.03} \\ 
\hline 
 2038 &  63 &  64 ans 11 mois &  -9.58\% &  3261.96 &  {\bf 40.43} &  2041.01 &  {\bf 1.60} &  {\bf 1.46} &  {\bf 1.37} &  {\bf 1.28} &  {\bf 1.20} &  {\bf 1.13} \\ 
\hline 
 2039 &  64 &  65 ans 0 mois &  -5.00\% &  3564.80 &  {\bf 43.36} &  2067.55 &  {\bf 1.72} &  {\bf 1.60} &  {\bf 1.50} &  {\bf 1.40} &  {\bf 1.31} &  {\bf 1.23} \\ 
\hline 
 2040 &  65 &  65 ans 1 mois &  -0.42\% &  3887.14 &  {\bf 46.40} &  2094.43 &  {\bf 1.86} &  {\bf 1.74} &  {\bf 1.63} &  {\bf 1.53} &  {\bf 1.43} &  {\bf 1.34} \\ 
\hline 
 2041 &  66 &  65 ans 2 mois &  4.17\% &  4230.17 &  {\bf 49.56} &  2121.65 &  {\bf 1.99} &  {\bf 1.89} &  {\bf 1.78} &  {\bf 1.66} &  {\bf 1.56} &  {\bf 1.46} \\ 
\hline 
 2042 &  67 &  65 ans 3 mois &  8.75\% &  4595.17 &  {\bf 52.84} &  2149.23 &  {\bf 2.14} &  {\bf 2.06} &  {\bf 1.93} &  {\bf 1.81} &  {\bf 1.69} &  {\bf 1.59} \\ 
\hline 
\hline 
\end{tabular} 
\end{center} } 

 \begin{center}\includegraphics[width=0.9\textwidth]{fig/Ascendant34_1975_22_dest_retraite.pdf}\end{center} \label{fig/Ascendant34_1975_22_dest_retraite.pdf} 

\newpage 
 
\subsection{Génération 1980 (début en 2002)} 

\paragraph{Retraites possibles et ratios Revenu/SMIC à 70, 75, 80, 85, 90 ans avec le modèle \emph{Gouvernement truqué (âge-pivot bloqué à 65 ans)}}  
 
{ \scriptsize \begin{center} 
\begin{tabular}[htb]{|c|c||c|c||c|c||c||c|c|c|c|c|c|} 
\hline 
 Retraite en &  Âge &  Âge pivot &  Décote/Surcote &  Retraite (\euro{} 2019) &  Tx Rempl(\%) &  SMIC (\euro{} 2019) &  Retraite/SMIC &  Rev70/SMIC &  Rev75/SMIC &  Rev80/SMIC &  Rev85/SMIC &  Rev90/SMIC \\ 
\hline \hline 
 2042 &  62 &  65 ans 0 mois &  -15.00\% &  3324.56 &  {\bf 37.00} &  2285.97 &  {\bf 1.45} &  {\bf 1.31} &  {\bf 1.23} &  {\bf 1.15} &  {\bf 1.08} &  {\bf 1.01} \\ 
\hline 
 2043 &  63 &  65 ans 0 mois &  -10.00\% &  3675.16 &  {\bf 40.14} &  2315.68 &  {\bf 1.59} &  {\bf 1.45} &  {\bf 1.36} &  {\bf 1.27} &  {\bf 1.19} &  {\bf 1.12} \\ 
\hline 
 2044 &  64 &  65 ans 0 mois &  -5.00\% &  4050.14 &  {\bf 43.42} &  2345.79 &  {\bf 1.73} &  {\bf 1.60} &  {\bf 1.50} &  {\bf 1.40} &  {\bf 1.32} &  {\bf 1.23} \\ 
\hline 
 2045 &  65 &  65 ans 0 mois &  0.00\% &  4451.03 &  {\bf 46.83} &  2376.28 &  {\bf 1.87} &  {\bf 1.76} &  {\bf 1.65} &  {\bf 1.54} &  {\bf 1.45} &  {\bf 1.36} \\ 
\hline 
 2046 &  66 &  65 ans 0 mois &  5.00\% &  4875.89 &  {\bf 50.35} &  2407.18 &  {\bf 2.03} &  {\bf 1.92} &  {\bf 1.80} &  {\bf 1.69} &  {\bf 1.58} &  {\bf 1.49} \\ 
\hline 
 2047 &  67 &  65 ans 0 mois &  10.00\% &  5325.56 &  {\bf 53.97} &  2438.47 &  {\bf 2.18} &  {\bf 2.10} &  {\bf 1.97} &  {\bf 1.85} &  {\bf 1.73} &  {\bf 1.62} \\ 
\hline 
\hline 
\end{tabular} 
\end{center} } 
\paragraph{Retraites possibles et ratios Revenu/SMIC à 70, 75, 80, 85, 90 ans avec le modèle \emph{Gouvernement corrigé (âge-pivot glissant)}}  
 
{ \scriptsize \begin{center} 
\begin{tabular}[htb]{|c|c||c|c||c|c||c||c|c|c|c|c|c|} 
\hline 
 Retraite en &  Âge &  Âge pivot &  Décote/Surcote &  Retraite (\euro{} 2019) &  Tx Rempl(\%) &  SMIC (\euro{} 2019) &  Retraite/SMIC &  Rev70/SMIC &  Rev75/SMIC &  Rev80/SMIC &  Rev85/SMIC &  Rev90/SMIC \\ 
\hline \hline 
 2042 &  62 &  65 ans 3 mois &  -16.25\% &  3275.67 &  {\bf 36.46} &  2285.97 &  {\bf 1.43} &  {\bf 1.29} &  {\bf 1.21} &  {\bf 1.14} &  {\bf 1.06} &  {\bf {\color{red} 1.00}} \\ 
\hline 
 2043 &  63 &  65 ans 4 mois &  -11.67\% &  3607.10 &  {\bf 39.40} &  2315.68 &  {\bf 1.56} &  {\bf 1.42} &  {\bf 1.33} &  {\bf 1.25} &  {\bf 1.17} &  {\bf 1.10} \\ 
\hline 
 2044 &  64 &  65 ans 5 mois &  -7.08\% &  3961.32 &  {\bf 42.46} &  2345.79 &  {\bf 1.69} &  {\bf 1.56} &  {\bf 1.47} &  {\bf 1.37} &  {\bf 1.29} &  {\bf 1.21} \\ 
\hline 
 2045 &  65 &  65 ans 6 mois &  -2.50\% &  4339.76 &  {\bf 45.66} &  2376.28 &  {\bf 1.83} &  {\bf 1.71} &  {\bf 1.60} &  {\bf 1.50} &  {\bf 1.41} &  {\bf 1.32} \\ 
\hline 
 2046 &  66 &  65 ans 7 mois &  2.08\% &  4740.44 &  {\bf 48.95} &  2407.18 &  {\bf 1.97} &  {\bf 1.87} &  {\bf 1.75} &  {\bf 1.64} &  {\bf 1.54} &  {\bf 1.44} \\ 
\hline 
 2047 &  67 &  65 ans 8 mois &  6.67\% &  5164.18 &  {\bf 52.34} &  2438.47 &  {\bf 2.12} &  {\bf 2.04} &  {\bf 1.91} &  {\bf 1.79} &  {\bf 1.68} &  {\bf 1.57} \\ 
\hline 
\hline 
\end{tabular} 
\end{center} } 
\paragraph{Retraites possibles et ratios Revenu/SMIC à 70, 75, 80, 85, 90 ans avec le modèle \emph{Destinie2 (revalorisation de la fonction publique)}}  
 
{ \scriptsize \begin{center} 
\begin{tabular}[htb]{|c|c||c|c||c|c||c||c|c|c|c|c|c|} 
\hline 
 Retraite en &  Âge &  Âge pivot &  Décote/Surcote &  Retraite (\euro{} 2019) &  Tx Rempl(\%) &  SMIC (\euro{} 2019) &  Retraite/SMIC &  Rev70/SMIC &  Rev75/SMIC &  Rev80/SMIC &  Rev85/SMIC &  Rev90/SMIC \\ 
\hline \hline 
 2042 &  62 &  65 ans 3 mois &  -16.25\% &  3165.14 &  {\bf 37.47} &  2149.23 &  {\bf 1.47} &  {\bf 1.33} &  {\bf 1.25} &  {\bf 1.17} &  {\bf 1.09} &  {\bf 1.03} \\ 
\hline 
 2043 &  63 &  65 ans 4 mois &  -11.67\% &  3482.46 &  {\bf 40.46} &  2177.17 &  {\bf 1.60} &  {\bf 1.46} &  {\bf 1.37} &  {\bf 1.28} &  {\bf 1.20} &  {\bf 1.13} \\ 
\hline 
 2044 &  64 &  65 ans 5 mois &  -7.08\% &  3821.39 &  {\bf 43.57} &  2205.48 &  {\bf 1.73} &  {\bf 1.60} &  {\bf 1.50} &  {\bf 1.41} &  {\bf 1.32} &  {\bf 1.24} \\ 
\hline 
 2045 &  65 &  65 ans 6 mois &  -2.50\% &  4183.30 &  {\bf 46.81} &  2234.15 &  {\bf 1.87} &  {\bf 1.76} &  {\bf 1.65} &  {\bf 1.54} &  {\bf 1.45} &  {\bf 1.36} \\ 
\hline 
 2046 &  66 &  65 ans 7 mois &  2.08\% &  4566.27 &  {\bf 50.15} &  2263.19 &  {\bf 2.02} &  {\bf 1.92} &  {\bf 1.80} &  {\bf 1.68} &  {\bf 1.58} &  {\bf 1.48} \\ 
\hline 
 2047 &  67 &  65 ans 8 mois &  6.67\% &  4971.05 &  {\bf 53.58} &  2292.61 &  {\bf 2.17} &  {\bf 2.09} &  {\bf 1.96} &  {\bf 1.83} &  {\bf 1.72} &  {\bf 1.61} \\ 
\hline 
\hline 
\end{tabular} 
\end{center} } 

 \begin{center}\includegraphics[width=0.9\textwidth]{fig/Ascendant34_1980_22_dest_retraite.pdf}\end{center} \label{fig/Ascendant34_1980_22_dest_retraite.pdf} 

\newpage 
 
\subsection{Génération 1990 (début en 2012)} 

\paragraph{Retraites possibles et ratios Revenu/SMIC à 70, 75, 80, 85, 90 ans avec le modèle \emph{Gouvernement truqué (âge-pivot bloqué à 65 ans)}}  
 
{ \scriptsize \begin{center} 
\begin{tabular}[htb]{|c|c||c|c||c|c||c||c|c|c|c|c|c|} 
\hline 
 Retraite en &  Âge &  Âge pivot &  Décote/Surcote &  Retraite (\euro{} 2019) &  Tx Rempl(\%) &  SMIC (\euro{} 2019) &  Retraite/SMIC &  Rev70/SMIC &  Rev75/SMIC &  Rev80/SMIC &  Rev85/SMIC &  Rev90/SMIC \\ 
\hline \hline 
 2052 &  62 &  65 ans 0 mois &  -15.00\% &  4034.33 &  {\bf 39.46} &  2601.14 &  {\bf 1.55} &  {\bf 1.40} &  {\bf 1.31} &  {\bf 1.23} &  {\bf 1.15} &  {\bf 1.08} \\ 
\hline 
 2053 &  63 &  65 ans 0 mois &  -10.00\% &  4457.68 &  {\bf 42.79} &  2634.96 &  {\bf 1.69} &  {\bf 1.55} &  {\bf 1.45} &  {\bf 1.36} &  {\bf 1.27} &  {\bf 1.19} \\ 
\hline 
 2054 &  64 &  65 ans 0 mois &  -5.00\% &  4906.86 &  {\bf 46.23} &  2669.21 &  {\bf 1.84} &  {\bf 1.70} &  {\bf 1.59} &  {\bf 1.50} &  {\bf 1.40} &  {\bf 1.31} \\ 
\hline 
 2055 &  65 &  65 ans 0 mois &  0.00\% &  5382.81 &  {\bf 49.77} &  2703.91 &  {\bf 1.99} &  {\bf 1.87} &  {\bf 1.75} &  {\bf 1.64} &  {\bf 1.54} &  {\bf 1.44} \\ 
\hline 
 2056 &  66 &  65 ans 0 mois &  5.00\% &  5886.49 &  {\bf 53.42} &  2739.06 &  {\bf 2.15} &  {\bf 2.04} &  {\bf 1.91} &  {\bf 1.79} &  {\bf 1.68} &  {\bf 1.58} \\ 
\hline 
 2057 &  67 &  65 ans 0 mois &  10.00\% &  6418.88 &  {\bf 57.17} &  2774.67 &  {\bf 2.31} &  {\bf 2.23} &  {\bf 2.09} &  {\bf 1.96} &  {\bf 1.83} &  {\bf 1.72} \\ 
\hline 
\hline 
\end{tabular} 
\end{center} } 
\paragraph{Retraites possibles et ratios Revenu/SMIC à 70, 75, 80, 85, 90 ans avec le modèle \emph{Gouvernement corrigé (âge-pivot glissant)}}  
 
{ \scriptsize \begin{center} 
\begin{tabular}[htb]{|c|c||c|c||c|c||c||c|c|c|c|c|c|} 
\hline 
 Retraite en &  Âge &  Âge pivot &  Décote/Surcote &  Retraite (\euro{} 2019) &  Tx Rempl(\%) &  SMIC (\euro{} 2019) &  Retraite/SMIC &  Rev70/SMIC &  Rev75/SMIC &  Rev80/SMIC &  Rev85/SMIC &  Rev90/SMIC \\ 
\hline \hline 
 2052 &  62 &  66 ans 1 mois &  -20.42\% &  3777.24 &  {\bf 36.95} &  2601.14 &  {\bf 1.45} &  {\bf 1.31} &  {\bf 1.23} &  {\bf 1.15} &  {\bf 1.08} &  {\bf 1.01} \\ 
\hline 
 2053 &  63 &  66 ans 2 mois &  -15.83\% &  4168.76 &  {\bf 40.02} &  2634.96 &  {\bf 1.58} &  {\bf 1.45} &  {\bf 1.35} &  {\bf 1.27} &  {\bf 1.19} &  {\bf 1.12} \\ 
\hline 
 2054 &  64 &  66 ans 3 mois &  -11.25\% &  4584.04 &  {\bf 43.19} &  2669.21 &  {\bf 1.72} &  {\bf 1.59} &  {\bf 1.49} &  {\bf 1.40} &  {\bf 1.31} &  {\bf 1.23} \\ 
\hline 
 2055 &  65 &  66 ans 4 mois &  -6.67\% &  5023.96 &  {\bf 46.45} &  2703.91 &  {\bf 1.86} &  {\bf 1.74} &  {\bf 1.63} &  {\bf 1.53} &  {\bf 1.44} &  {\bf 1.35} \\ 
\hline 
 2056 &  66 &  66 ans 5 mois &  -2.08\% &  5489.39 &  {\bf 49.81} &  2739.06 &  {\bf 2.00} &  {\bf 1.90} &  {\bf 1.78} &  {\bf 1.67} &  {\bf 1.57} &  {\bf 1.47} \\ 
\hline 
 2057 &  67 &  66 ans 6 mois &  2.50\% &  5981.23 &  {\bf 53.27} &  2774.67 &  {\bf 2.16} &  {\bf 2.07} &  {\bf 1.94} &  {\bf 1.82} &  {\bf 1.71} &  {\bf 1.60} \\ 
\hline 
\hline 
\end{tabular} 
\end{center} } 
\paragraph{Retraites possibles et ratios Revenu/SMIC à 70, 75, 80, 85, 90 ans avec le modèle \emph{Destinie2 (revalorisation de la fonction publique)}}  
 
{ \scriptsize \begin{center} 
\begin{tabular}[htb]{|c|c||c|c||c|c||c||c|c|c|c|c|c|} 
\hline 
 Retraite en &  Âge &  Âge pivot &  Décote/Surcote &  Retraite (\euro{} 2019) &  Tx Rempl(\%) &  SMIC (\euro{} 2019) &  Retraite/SMIC &  Rev70/SMIC &  Rev75/SMIC &  Rev80/SMIC &  Rev85/SMIC &  Rev90/SMIC \\ 
\hline \hline 
 2052 &  62 &  66 ans 1 mois &  -20.42\% &  3599.66 &  {\bf 37.45} &  2445.56 &  {\bf 1.47} &  {\bf 1.33} &  {\bf 1.24} &  {\bf 1.17} &  {\bf 1.09} &  {\bf 1.03} \\ 
\hline 
 2053 &  63 &  66 ans 2 mois &  -15.83\% &  3971.20 &  {\bf 40.55} &  2477.35 &  {\bf 1.60} &  {\bf 1.46} &  {\bf 1.37} &  {\bf 1.29} &  {\bf 1.21} &  {\bf 1.13} \\ 
\hline 
 2054 &  64 &  66 ans 3 mois &  -11.25\% &  4365.18 &  {\bf 43.74} &  2509.56 &  {\bf 1.74} &  {\bf 1.61} &  {\bf 1.51} &  {\bf 1.41} &  {\bf 1.33} &  {\bf 1.24} \\ 
\hline 
 2055 &  65 &  66 ans 4 mois &  -6.67\% &  4782.39 &  {\bf 47.03} &  2542.18 &  {\bf 1.88} &  {\bf 1.76} &  {\bf 1.65} &  {\bf 1.55} &  {\bf 1.45} &  {\bf 1.36} \\ 
\hline 
 2056 &  66 &  66 ans 5 mois &  -2.08\% &  5223.68 &  {\bf 50.42} &  2575.23 &  {\bf 2.03} &  {\bf 1.93} &  {\bf 1.81} &  {\bf 1.69} &  {\bf 1.59} &  {\bf 1.49} \\ 
\hline 
 2057 &  67 &  66 ans 6 mois &  2.50\% &  5689.88 &  {\bf 53.90} &  2608.71 &  {\bf 2.18} &  {\bf 2.10} &  {\bf 1.97} &  {\bf 1.84} &  {\bf 1.73} &  {\bf 1.62} \\ 
\hline 
\hline 
\end{tabular} 
\end{center} } 

 \begin{center}\includegraphics[width=0.9\textwidth]{fig/Ascendant34_1990_22_dest_retraite.pdf}\end{center} \label{fig/Ascendant34_1990_22_dest_retraite.pdf} 

\newpage 
 
\subsection{Génération 2003 (début en 2025)} 

\paragraph{Retraites possibles et ratios Revenu/SMIC à 70, 75, 80, 85, 90 ans avec le modèle \emph{Gouvernement truqué (âge-pivot bloqué à 65 ans)}}  
 
{ \scriptsize \begin{center} 
\begin{tabular}[htb]{|c|c||c|c||c|c||c||c|c|c|c|c|c|} 
\hline 
 Retraite en &  Âge &  Âge pivot &  Décote/Surcote &  Retraite (\euro{} 2019) &  Tx Rempl(\%) &  SMIC (\euro{} 2019) &  Retraite/SMIC &  Rev70/SMIC &  Rev75/SMIC &  Rev80/SMIC &  Rev85/SMIC &  Rev90/SMIC \\ 
\hline \hline 
 2065 &  62 &  65 ans 0 mois &  -15.00\% &  5062.68 &  {\bf 41.87} &  3076.71 &  {\bf 1.65} &  {\bf 1.48} &  {\bf 1.39} &  {\bf 1.30} &  {\bf 1.22} &  {\bf 1.15} \\ 
\hline 
 2066 &  63 &  65 ans 0 mois &  -10.00\% &  5584.53 &  {\bf 45.32} &  3116.71 &  {\bf 1.79} &  {\bf 1.64} &  {\bf 1.53} &  {\bf 1.44} &  {\bf 1.35} &  {\bf 1.26} \\ 
\hline 
 2067 &  64 &  65 ans 0 mois &  -5.00\% &  6137.45 &  {\bf 48.88} &  3157.23 &  {\bf 1.94} &  {\bf 1.80} &  {\bf 1.69} &  {\bf 1.58} &  {\bf 1.48} &  {\bf 1.39} \\ 
\hline 
 2068 &  65 &  65 ans 0 mois &  0.00\% &  6722.53 &  {\bf 52.55} &  3198.27 &  {\bf 2.10} &  {\bf 1.97} &  {\bf 1.85} &  {\bf 1.73} &  {\bf 1.62} &  {\bf 1.52} \\ 
\hline 
 2069 &  66 &  65 ans 0 mois &  5.00\% &  7340.92 &  {\bf 56.32} &  3239.85 &  {\bf 2.27} &  {\bf 2.15} &  {\bf 2.02} &  {\bf 1.89} &  {\bf 1.77} &  {\bf 1.66} \\ 
\hline 
 2070 &  67 &  65 ans 0 mois &  10.00\% &  7993.81 &  {\bf 60.19} &  3281.97 &  {\bf 2.44} &  {\bf 2.34} &  {\bf 2.20} &  {\bf 2.06} &  {\bf 1.93} &  {\bf 1.81} \\ 
\hline 
\hline 
\end{tabular} 
\end{center} } 
\paragraph{Retraites possibles et ratios Revenu/SMIC à 70, 75, 80, 85, 90 ans avec le modèle \emph{Gouvernement corrigé (âge-pivot glissant)}}  
 
{ \scriptsize \begin{center} 
\begin{tabular}[htb]{|c|c||c|c||c|c||c||c|c|c|c|c|c|} 
\hline 
 Retraite en &  Âge &  Âge pivot &  Décote/Surcote &  Retraite (\euro{} 2019) &  Tx Rempl(\%) &  SMIC (\euro{} 2019) &  Retraite/SMIC &  Rev70/SMIC &  Rev75/SMIC &  Rev80/SMIC &  Rev85/SMIC &  Rev90/SMIC \\ 
\hline \hline 
 2065 &  62 &  67 ans 2 mois &  -25.83\% &  4417.44 &  {\bf 36.53} &  3076.71 &  {\bf 1.44} &  {\bf 1.29} &  {\bf 1.21} &  {\bf 1.14} &  {\bf 1.07} &  {\bf 1.00} \\ 
\hline 
 2066 &  63 &  67 ans 3 mois &  -21.25\% &  4886.47 &  {\bf 39.66} &  3116.71 &  {\bf 1.57} &  {\bf 1.43} &  {\bf 1.34} &  {\bf 1.26} &  {\bf 1.18} &  {\bf 1.11} \\ 
\hline 
 2067 &  64 &  67 ans 4 mois &  -16.67\% &  5383.72 &  {\bf 42.88} &  3157.23 &  {\bf 1.71} &  {\bf 1.58} &  {\bf 1.48} &  {\bf 1.39} &  {\bf 1.30} &  {\bf 1.22} \\ 
\hline 
 2068 &  65 &  67 ans 5 mois &  -12.08\% &  5910.22 &  {\bf 46.20} &  3198.27 &  {\bf 1.85} &  {\bf 1.73} &  {\bf 1.62} &  {\bf 1.52} &  {\bf 1.43} &  {\bf 1.34} \\ 
\hline 
 2069 &  66 &  67 ans 6 mois &  -7.50\% &  6467.00 &  {\bf 49.61} &  3239.85 &  {\bf 2.00} &  {\bf 1.90} &  {\bf 1.78} &  {\bf 1.67} &  {\bf 1.56} &  {\bf 1.46} \\ 
\hline 
 2070 &  67 &  67 ans 7 mois &  -2.92\% &  7055.14 &  {\bf 53.12} &  3281.97 &  {\bf 2.15} &  {\bf 2.07} &  {\bf 1.94} &  {\bf 1.82} &  {\bf 1.70} &  {\bf 1.60} \\ 
\hline 
\hline 
\end{tabular} 
\end{center} } 
\paragraph{Retraites possibles et ratios Revenu/SMIC à 70, 75, 80, 85, 90 ans avec le modèle \emph{Destinie2 (revalorisation de la fonction publique)}}  
 
{ \scriptsize \begin{center} 
\begin{tabular}[htb]{|c|c||c|c||c|c||c||c|c|c|c|c|c|} 
\hline 
 Retraite en &  Âge &  Âge pivot &  Décote/Surcote &  Retraite (\euro{} 2019) &  Tx Rempl(\%) &  SMIC (\euro{} 2019) &  Retraite/SMIC &  Rev70/SMIC &  Rev75/SMIC &  Rev80/SMIC &  Rev85/SMIC &  Rev90/SMIC \\ 
\hline \hline 
 2065 &  62 &  67 ans 2 mois &  -25.83\% &  4155.86 &  {\bf 36.55} &  2892.68 &  {\bf 1.44} &  {\bf 1.30} &  {\bf 1.21} &  {\bf 1.14} &  {\bf 1.07} &  {\bf 1.00} \\ 
\hline 
 2066 &  63 &  67 ans 3 mois &  -21.25\% &  4597.04 &  {\bf 39.68} &  2930.29 &  {\bf 1.57} &  {\bf 1.43} &  {\bf 1.34} &  {\bf 1.26} &  {\bf 1.18} &  {\bf 1.11} \\ 
\hline 
 2067 &  64 &  67 ans 4 mois &  -16.67\% &  5064.76 &  {\bf 42.91} &  2968.38 &  {\bf 1.71} &  {\bf 1.58} &  {\bf 1.48} &  {\bf 1.39} &  {\bf 1.30} &  {\bf 1.22} \\ 
\hline 
 2068 &  65 &  67 ans 5 mois &  -12.08\% &  5559.97 &  {\bf 46.23} &  3006.97 &  {\bf 1.85} &  {\bf 1.73} &  {\bf 1.62} &  {\bf 1.52} &  {\bf 1.43} &  {\bf 1.34} \\ 
\hline 
 2069 &  66 &  67 ans 6 mois &  -7.50\% &  6083.67 &  {\bf 49.64} &  3046.06 &  {\bf 2.00} &  {\bf 1.90} &  {\bf 1.78} &  {\bf 1.67} &  {\bf 1.56} &  {\bf 1.46} \\ 
\hline 
 2070 &  67 &  67 ans 7 mois &  -2.92\% &  6636.85 &  {\bf 53.15} &  3085.66 &  {\bf 2.15} &  {\bf 2.07} &  {\bf 1.94} &  {\bf 1.82} &  {\bf 1.70} &  {\bf 1.60} \\ 
\hline 
\hline 
\end{tabular} 
\end{center} } 

 \begin{center}\includegraphics[width=0.9\textwidth]{fig/Ascendant34_2003_22_dest_retraite.pdf}\end{center} \label{fig/Ascendant34_2003_22_dest_retraite.pdf} 

\newpage 
 
\chapter{Salarié privé évoluant du 4*SMIC à 5*SMIC} 


 \addto{\captionsenglish}{ \renewcommand{\mtctitle}{}} \setcounter{minitocdepth}{2} 
 \minitoc \newpage 

\section{Début de carrière à 22 ans} 

\subsection{Génération 1975 (début en 1997)} 

\paragraph{Retraites possibles et ratios Revenu/SMIC à 70, 75, 80, 85, 90 ans avec le modèle \emph{Gouvernement truqué (âge-pivot bloqué à 65 ans)}}  
 
{ \scriptsize \begin{center} 
\begin{tabular}[htb]{|c|c||c|c||c|c||c||c|c|c|c|c|c|} 
\hline 
 Retraite en &  Âge &  Âge pivot &  Décote/Surcote &  Retraite (\euro{} 2019) &  Tx Rempl(\%) &  SMIC (\euro{} 2019) &  Retraite/SMIC &  Rev70/SMIC &  Rev75/SMIC &  Rev80/SMIC &  Rev85/SMIC &  Rev90/SMIC \\ 
\hline \hline 
 2037 &  62 &  64 ans 10 mois &  -14.17\% &  3932.80 &  {\bf 37.22} &  2143.00 &  {\bf 1.84} &  {\bf 1.66} &  {\bf 1.55} &  {\bf 1.45} &  {\bf 1.36} &  {\bf 1.28} \\ 
\hline 
 2038 &  63 &  64 ans 11 mois &  -9.58\% &  4309.73 &  {\bf 40.08} &  2170.86 &  {\bf 1.99} &  {\bf 1.81} &  {\bf 1.70} &  {\bf 1.59} &  {\bf 1.49} &  {\bf 1.40} \\ 
\hline 
 2039 &  64 &  65 ans 0 mois &  -5.00\% &  4710.92 &  {\bf 43.04} &  2199.08 &  {\bf 2.14} &  {\bf 1.98} &  {\bf 1.86} &  {\bf 1.74} &  {\bf 1.63} &  {\bf 1.53} \\ 
\hline 
 2040 &  65 &  65 ans 0 mois &  0.00\% &  5159.32 &  {\bf 46.32} &  2227.67 &  {\bf 2.32} &  {\bf 2.17} &  {\bf 2.04} &  {\bf 1.91} &  {\bf 1.79} &  {\bf 1.68} \\ 
\hline 
 2041 &  66 &  65 ans 0 mois &  5.00\% &  5636.74 &  {\bf 49.73} &  2256.63 &  {\bf 2.50} &  {\bf 2.37} &  {\bf 2.22} &  {\bf 2.08} &  {\bf 1.95} &  {\bf 1.83} \\ 
\hline 
 2042 &  67 &  65 ans 0 mois &  10.00\% &  6144.94 &  {\bf 53.27} &  2285.97 &  {\bf 2.69} &  {\bf 2.59} &  {\bf 2.42} &  {\bf 2.27} &  {\bf 2.13} &  {\bf 2.00} \\ 
\hline 
\hline 
\end{tabular} 
\end{center} } 
\paragraph{Retraites possibles et ratios Revenu/SMIC à 70, 75, 80, 85, 90 ans avec le modèle \emph{Gouvernement corrigé (âge-pivot glissant)}}  
 
{ \scriptsize \begin{center} 
\begin{tabular}[htb]{|c|c||c|c||c|c||c||c|c|c|c|c|c|} 
\hline 
 Retraite en &  Âge &  Âge pivot &  Décote/Surcote &  Retraite (\euro{} 2019) &  Tx Rempl(\%) &  SMIC (\euro{} 2019) &  Retraite/SMIC &  Rev70/SMIC &  Rev75/SMIC &  Rev80/SMIC &  Rev85/SMIC &  Rev90/SMIC \\ 
\hline \hline 
 2037 &  62 &  64 ans 10 mois &  -14.17\% &  3932.80 &  {\bf 37.22} &  2143.00 &  {\bf 1.84} &  {\bf 1.66} &  {\bf 1.55} &  {\bf 1.45} &  {\bf 1.36} &  {\bf 1.28} \\ 
\hline 
 2038 &  63 &  64 ans 11 mois &  -9.58\% &  4309.73 &  {\bf 40.08} &  2170.86 &  {\bf 1.99} &  {\bf 1.81} &  {\bf 1.70} &  {\bf 1.59} &  {\bf 1.49} &  {\bf 1.40} \\ 
\hline 
 2039 &  64 &  65 ans 0 mois &  -5.00\% &  4710.92 &  {\bf 43.04} &  2199.08 &  {\bf 2.14} &  {\bf 1.98} &  {\bf 1.86} &  {\bf 1.74} &  {\bf 1.63} &  {\bf 1.53} \\ 
\hline 
 2040 &  65 &  65 ans 1 mois &  -0.42\% &  5137.83 &  {\bf 46.13} &  2227.67 &  {\bf 2.31} &  {\bf 2.16} &  {\bf 2.03} &  {\bf 1.90} &  {\bf 1.78} &  {\bf 1.67} \\ 
\hline 
 2041 &  66 &  65 ans 2 mois &  4.17\% &  5592.00 &  {\bf 49.33} &  2256.63 &  {\bf 2.48} &  {\bf 2.35} &  {\bf 2.21} &  {\bf 2.07} &  {\bf 1.94} &  {\bf 1.82} \\ 
\hline 
 2042 &  67 &  65 ans 3 mois &  8.75\% &  6075.11 &  {\bf 52.66} &  2285.97 &  {\bf 2.66} &  {\bf 2.56} &  {\bf 2.40} &  {\bf 2.25} &  {\bf 2.11} &  {\bf 1.97} \\ 
\hline 
\hline 
\end{tabular} 
\end{center} } 
\paragraph{Retraites possibles et ratios Revenu/SMIC à 70, 75, 80, 85, 90 ans avec le modèle \emph{Destinie2 (revalorisation de la fonction publique)}}  
 
{ \scriptsize \begin{center} 
\begin{tabular}[htb]{|c|c||c|c||c|c||c||c|c|c|c|c|c|} 
\hline 
 Retraite en &  Âge &  Âge pivot &  Décote/Surcote &  Retraite (\euro{} 2019) &  Tx Rempl(\%) &  SMIC (\euro{} 2019) &  Retraite/SMIC &  Rev70/SMIC &  Rev75/SMIC &  Rev80/SMIC &  Rev85/SMIC &  Rev90/SMIC \\ 
\hline \hline 
 2037 &  62 &  64 ans 10 mois &  -14.17\% &  3830.75 &  {\bf 38.56} &  2014.82 &  {\bf 1.90} &  {\bf 1.71} &  {\bf 1.61} &  {\bf 1.51} &  {\bf 1.41} &  {\bf 1.32} \\ 
\hline 
 2038 &  63 &  64 ans 11 mois &  -9.58\% &  4193.33 &  {\bf 41.48} &  2041.01 &  {\bf 2.05} &  {\bf 1.88} &  {\bf 1.76} &  {\bf 1.65} &  {\bf 1.55} &  {\bf 1.45} \\ 
\hline 
 2039 &  64 &  65 ans 0 mois &  -5.00\% &  4578.93 &  {\bf 44.50} &  2067.55 &  {\bf 2.21} &  {\bf 2.05} &  {\bf 1.92} &  {\bf 1.80} &  {\bf 1.69} &  {\bf 1.58} \\ 
\hline 
 2040 &  65 &  65 ans 1 mois &  -0.42\% &  4988.96 &  {\bf 47.64} &  2094.43 &  {\bf 2.38} &  {\bf 2.23} &  {\bf 2.09} &  {\bf 1.96} &  {\bf 1.84} &  {\bf 1.72} \\ 
\hline 
 2041 &  66 &  65 ans 2 mois &  4.17\% &  5424.91 &  {\bf 50.90} &  2121.65 &  {\bf 2.56} &  {\bf 2.43} &  {\bf 2.28} &  {\bf 2.13} &  {\bf 2.00} &  {\bf 1.88} \\ 
\hline 
 2042 &  67 &  65 ans 3 mois &  8.75\% &  5888.35 &  {\bf 54.29} &  2149.23 &  {\bf 2.74} &  {\bf 2.64} &  {\bf 2.47} &  {\bf 2.32} &  {\bf 2.17} &  {\bf 2.04} \\ 
\hline 
\hline 
\end{tabular} 
\end{center} } 

 \begin{center}\includegraphics[width=0.9\textwidth]{fig/Ascendant45_1975_22_dest_retraite.pdf}\end{center} \label{fig/Ascendant45_1975_22_dest_retraite.pdf} 

\newpage 
 
\subsection{Génération 1980 (début en 2002)} 

\paragraph{Retraites possibles et ratios Revenu/SMIC à 70, 75, 80, 85, 90 ans avec le modèle \emph{Gouvernement truqué (âge-pivot bloqué à 65 ans)}}  
 
{ \scriptsize \begin{center} 
\begin{tabular}[htb]{|c|c||c|c||c|c||c||c|c|c|c|c|c|} 
\hline 
 Retraite en &  Âge &  Âge pivot &  Décote/Surcote &  Retraite (\euro{} 2019) &  Tx Rempl(\%) &  SMIC (\euro{} 2019) &  Retraite/SMIC &  Rev70/SMIC &  Rev75/SMIC &  Rev80/SMIC &  Rev85/SMIC &  Rev90/SMIC \\ 
\hline \hline 
 2042 &  62 &  65 ans 0 mois &  -15.00\% &  4276.24 &  {\bf 37.94} &  2285.97 &  {\bf 1.87} &  {\bf 1.69} &  {\bf 1.58} &  {\bf 1.48} &  {\bf 1.39} &  {\bf 1.30} \\ 
\hline 
 2043 &  63 &  65 ans 0 mois &  -10.00\% &  4723.38 &  {\bf 41.18} &  2315.68 &  {\bf 2.04} &  {\bf 1.86} &  {\bf 1.75} &  {\bf 1.64} &  {\bf 1.54} &  {\bf 1.44} \\ 
\hline 
 2044 &  64 &  65 ans 0 mois &  -5.00\% &  5201.15 &  {\bf 44.55} &  2345.79 &  {\bf 2.22} &  {\bf 2.05} &  {\bf 1.92} &  {\bf 1.80} &  {\bf 1.69} &  {\bf 1.58} \\ 
\hline 
 2045 &  65 &  65 ans 0 mois &  0.00\% &  5711.45 &  {\bf 48.07} &  2376.28 &  {\bf 2.40} &  {\bf 2.25} &  {\bf 2.11} &  {\bf 1.98} &  {\bf 1.86} &  {\bf 1.74} \\ 
\hline 
 2046 &  66 &  65 ans 0 mois &  5.00\% &  6251.71 &  {\bf 51.70} &  2407.18 &  {\bf 2.60} &  {\bf 2.47} &  {\bf 2.31} &  {\bf 2.17} &  {\bf 2.03} &  {\bf 1.90} \\ 
\hline 
 2047 &  67 &  65 ans 0 mois &  10.00\% &  6822.97 &  {\bf 55.45} &  2438.47 &  {\bf 2.80} &  {\bf 2.69} &  {\bf 2.52} &  {\bf 2.37} &  {\bf 2.22} &  {\bf 2.08} \\ 
\hline 
\hline 
\end{tabular} 
\end{center} } 
\paragraph{Retraites possibles et ratios Revenu/SMIC à 70, 75, 80, 85, 90 ans avec le modèle \emph{Gouvernement corrigé (âge-pivot glissant)}}  
 
{ \scriptsize \begin{center} 
\begin{tabular}[htb]{|c|c||c|c||c|c||c||c|c|c|c|c|c|} 
\hline 
 Retraite en &  Âge &  Âge pivot &  Décote/Surcote &  Retraite (\euro{} 2019) &  Tx Rempl(\%) &  SMIC (\euro{} 2019) &  Retraite/SMIC &  Rev70/SMIC &  Rev75/SMIC &  Rev80/SMIC &  Rev85/SMIC &  Rev90/SMIC \\ 
\hline \hline 
 2042 &  62 &  65 ans 3 mois &  -16.25\% &  4213.36 &  {\bf 37.38} &  2285.97 &  {\bf 1.84} &  {\bf 1.66} &  {\bf 1.56} &  {\bf 1.46} &  {\bf 1.37} &  {\bf 1.28} \\ 
\hline 
 2043 &  63 &  65 ans 4 mois &  -11.67\% &  4635.91 &  {\bf 40.42} &  2315.68 &  {\bf 2.00} &  {\bf 1.83} &  {\bf 1.71} &  {\bf 1.61} &  {\bf 1.51} &  {\bf 1.41} \\ 
\hline 
 2044 &  64 &  65 ans 5 mois &  -7.08\% &  5087.09 &  {\bf 43.57} &  2345.79 &  {\bf 2.17} &  {\bf 2.01} &  {\bf 1.88} &  {\bf 1.76} &  {\bf 1.65} &  {\bf 1.55} \\ 
\hline 
 2045 &  65 &  65 ans 6 mois &  -2.50\% &  5568.66 &  {\bf 46.87} &  2376.28 &  {\bf 2.34} &  {\bf 2.20} &  {\bf 2.06} &  {\bf 1.93} &  {\bf 1.81} &  {\bf 1.70} \\ 
\hline 
 2046 &  66 &  65 ans 7 mois &  2.08\% &  6078.05 &  {\bf 50.27} &  2407.18 &  {\bf 2.52} &  {\bf 2.40} &  {\bf 2.25} &  {\bf 2.11} &  {\bf 1.98} &  {\bf 1.85} \\ 
\hline 
 2047 &  67 &  65 ans 8 mois &  6.67\% &  6616.21 &  {\bf 53.77} &  2438.47 &  {\bf 2.71} &  {\bf 2.61} &  {\bf 2.45} &  {\bf 2.29} &  {\bf 2.15} &  {\bf 2.02} \\ 
\hline 
\hline 
\end{tabular} 
\end{center} } 
\paragraph{Retraites possibles et ratios Revenu/SMIC à 70, 75, 80, 85, 90 ans avec le modèle \emph{Destinie2 (revalorisation de la fonction publique)}}  
 
{ \scriptsize \begin{center} 
\begin{tabular}[htb]{|c|c||c|c||c|c||c||c|c|c|c|c|c|} 
\hline 
 Retraite en &  Âge &  Âge pivot &  Décote/Surcote &  Retraite (\euro{} 2019) &  Tx Rempl(\%) &  SMIC (\euro{} 2019) &  Retraite/SMIC &  Rev70/SMIC &  Rev75/SMIC &  Rev80/SMIC &  Rev85/SMIC &  Rev90/SMIC \\ 
\hline \hline 
 2042 &  62 &  65 ans 3 mois &  -16.25\% &  4073.06 &  {\bf 38.44} &  2149.23 &  {\bf 1.90} &  {\bf 1.71} &  {\bf 1.60} &  {\bf 1.50} &  {\bf 1.41} &  {\bf 1.32} \\ 
\hline 
 2043 &  63 &  65 ans 4 mois &  -11.67\% &  4477.80 &  {\bf 41.52} &  2177.17 &  {\bf 2.06} &  {\bf 1.88} &  {\bf 1.76} &  {\bf 1.65} &  {\bf 1.55} &  {\bf 1.45} \\ 
\hline 
 2044 &  64 &  65 ans 5 mois &  -7.08\% &  4909.71 &  {\bf 44.73} &  2205.48 &  {\bf 2.23} &  {\bf 2.06} &  {\bf 1.93} &  {\bf 1.81} &  {\bf 1.70} &  {\bf 1.59} \\ 
\hline 
 2045 &  65 &  65 ans 6 mois &  -2.50\% &  5370.47 &  {\bf 48.08} &  2234.15 &  {\bf 2.40} &  {\bf 2.25} &  {\bf 2.11} &  {\bf 1.98} &  {\bf 1.86} &  {\bf 1.74} \\ 
\hline 
 2046 &  66 &  65 ans 7 mois &  2.08\% &  5857.57 &  {\bf 51.52} &  2263.19 &  {\bf 2.59} &  {\bf 2.46} &  {\bf 2.30} &  {\bf 2.16} &  {\bf 2.02} &  {\bf 1.90} \\ 
\hline 
 2047 &  67 &  65 ans 8 mois &  6.67\% &  6371.91 &  {\bf 55.07} &  2292.61 &  {\bf 2.78} &  {\bf 2.67} &  {\bf 2.51} &  {\bf 2.35} &  {\bf 2.20} &  {\bf 2.07} \\ 
\hline 
\hline 
\end{tabular} 
\end{center} } 

 \begin{center}\includegraphics[width=0.9\textwidth]{fig/Ascendant45_1980_22_dest_retraite.pdf}\end{center} \label{fig/Ascendant45_1980_22_dest_retraite.pdf} 

\newpage 
 
\subsection{Génération 1990 (début en 2012)} 

\paragraph{Retraites possibles et ratios Revenu/SMIC à 70, 75, 80, 85, 90 ans avec le modèle \emph{Gouvernement truqué (âge-pivot bloqué à 65 ans)}}  
 
{ \scriptsize \begin{center} 
\begin{tabular}[htb]{|c|c||c|c||c|c||c||c|c|c|c|c|c|} 
\hline 
 Retraite en &  Âge &  Âge pivot &  Décote/Surcote &  Retraite (\euro{} 2019) &  Tx Rempl(\%) &  SMIC (\euro{} 2019) &  Retraite/SMIC &  Rev70/SMIC &  Rev75/SMIC &  Rev80/SMIC &  Rev85/SMIC &  Rev90/SMIC \\ 
\hline \hline 
 2052 &  62 &  65 ans 0 mois &  -15.00\% &  5190.82 &  {\bf 40.48} &  2601.14 &  {\bf 2.00} &  {\bf 1.80} &  {\bf 1.69} &  {\bf 1.58} &  {\bf 1.48} &  {\bf 1.39} \\ 
\hline 
 2053 &  63 &  65 ans 0 mois &  -10.00\% &  5731.12 &  {\bf 43.91} &  2634.96 &  {\bf 2.18} &  {\bf 1.99} &  {\bf 1.86} &  {\bf 1.75} &  {\bf 1.64} &  {\bf 1.53} \\ 
\hline 
 2054 &  64 &  65 ans 0 mois &  -5.00\% &  6303.82 &  {\bf 47.45} &  2669.21 &  {\bf 2.36} &  {\bf 2.19} &  {\bf 2.05} &  {\bf 1.92} &  {\bf 1.80} &  {\bf 1.69} \\ 
\hline 
 2055 &  65 &  65 ans 0 mois &  0.00\% &  6910.05 &  {\bf 51.11} &  2703.91 &  {\bf 2.56} &  {\bf 2.40} &  {\bf 2.25} &  {\bf 2.11} &  {\bf 1.97} &  {\bf 1.85} \\ 
\hline 
 2056 &  66 &  65 ans 0 mois &  5.00\% &  7550.96 &  {\bf 54.88} &  2739.06 &  {\bf 2.76} &  {\bf 2.62} &  {\bf 2.45} &  {\bf 2.30} &  {\bf 2.16} &  {\bf 2.02} \\ 
\hline 
 2057 &  67 &  65 ans 0 mois &  10.00\% &  8227.77 &  {\bf 58.76} &  2774.67 &  {\bf 2.97} &  {\bf 2.85} &  {\bf 2.67} &  {\bf 2.51} &  {\bf 2.35} &  {\bf 2.20} \\ 
\hline 
\hline 
\end{tabular} 
\end{center} } 
\paragraph{Retraites possibles et ratios Revenu/SMIC à 70, 75, 80, 85, 90 ans avec le modèle \emph{Gouvernement corrigé (âge-pivot glissant)}}  
 
{ \scriptsize \begin{center} 
\begin{tabular}[htb]{|c|c||c|c||c|c||c||c|c|c|c|c|c|} 
\hline 
 Retraite en &  Âge &  Âge pivot &  Décote/Surcote &  Retraite (\euro{} 2019) &  Tx Rempl(\%) &  SMIC (\euro{} 2019) &  Retraite/SMIC &  Rev70/SMIC &  Rev75/SMIC &  Rev80/SMIC &  Rev85/SMIC &  Rev90/SMIC \\ 
\hline \hline 
 2052 &  62 &  66 ans 1 mois &  -20.42\% &  4860.03 &  {\bf 37.90} &  2601.14 &  {\bf 1.87} &  {\bf 1.68} &  {\bf 1.58} &  {\bf 1.48} &  {\bf 1.39} &  {\bf 1.30} \\ 
\hline 
 2053 &  63 &  66 ans 2 mois &  -15.83\% &  5359.66 &  {\bf 41.06} &  2634.96 &  {\bf 2.03} &  {\bf 1.86} &  {\bf 1.74} &  {\bf 1.63} &  {\bf 1.53} &  {\bf 1.44} \\ 
\hline 
 2054 &  64 &  66 ans 3 mois &  -11.25\% &  5889.09 &  {\bf 44.33} &  2669.21 &  {\bf 2.21} &  {\bf 2.04} &  {\bf 1.91} &  {\bf 1.79} &  {\bf 1.68} &  {\bf 1.58} \\ 
\hline 
 2055 &  65 &  66 ans 4 mois &  -6.67\% &  6449.38 &  {\bf 47.70} &  2703.91 &  {\bf 2.39} &  {\bf 2.24} &  {\bf 2.10} &  {\bf 1.97} &  {\bf 1.84} &  {\bf 1.73} \\ 
\hline 
 2056 &  66 &  66 ans 5 mois &  -2.08\% &  7041.57 &  {\bf 51.18} &  2739.06 &  {\bf 2.57} &  {\bf 2.44} &  {\bf 2.29} &  {\bf 2.15} &  {\bf 2.01} &  {\bf 1.89} \\ 
\hline 
 2057 &  67 &  66 ans 6 mois &  2.50\% &  7666.78 &  {\bf 54.75} &  2774.67 &  {\bf 2.76} &  {\bf 2.66} &  {\bf 2.49} &  {\bf 2.34} &  {\bf 2.19} &  {\bf 2.05} \\ 
\hline 
\hline 
\end{tabular} 
\end{center} } 
\paragraph{Retraites possibles et ratios Revenu/SMIC à 70, 75, 80, 85, 90 ans avec le modèle \emph{Destinie2 (revalorisation de la fonction publique)}}  
 
{ \scriptsize \begin{center} 
\begin{tabular}[htb]{|c|c||c|c||c|c||c||c|c|c|c|c|c|} 
\hline 
 Retraite en &  Âge &  Âge pivot &  Décote/Surcote &  Retraite (\euro{} 2019) &  Tx Rempl(\%) &  SMIC (\euro{} 2019) &  Retraite/SMIC &  Rev70/SMIC &  Rev75/SMIC &  Rev80/SMIC &  Rev85/SMIC &  Rev90/SMIC \\ 
\hline \hline 
 2052 &  62 &  66 ans 1 mois &  -20.42\% &  4633.13 &  {\bf 38.43} &  2445.56 &  {\bf 1.89} &  {\bf 1.71} &  {\bf 1.60} &  {\bf 1.50} &  {\bf 1.41} &  {\bf 1.32} \\ 
\hline 
 2053 &  63 &  66 ans 2 mois &  -15.83\% &  5107.42 &  {\bf 41.62} &  2477.35 &  {\bf 2.06} &  {\bf 1.88} &  {\bf 1.77} &  {\bf 1.66} &  {\bf 1.55} &  {\bf 1.45} \\ 
\hline 
 2054 &  64 &  66 ans 3 mois &  -11.25\% &  5609.85 &  {\bf 44.92} &  2509.56 &  {\bf 2.24} &  {\bf 2.07} &  {\bf 1.94} &  {\bf 1.82} &  {\bf 1.70} &  {\bf 1.60} \\ 
\hline 
 2055 &  65 &  66 ans 4 mois &  -6.67\% &  6141.39 &  {\bf 48.32} &  2542.18 &  {\bf 2.42} &  {\bf 2.26} &  {\bf 2.12} &  {\bf 1.99} &  {\bf 1.87} &  {\bf 1.75} \\ 
\hline 
 2056 &  66 &  66 ans 5 mois &  -2.08\% &  6703.04 &  {\bf 51.82} &  2575.23 &  {\bf 2.60} &  {\bf 2.47} &  {\bf 2.32} &  {\bf 2.17} &  {\bf 2.04} &  {\bf 1.91} \\ 
\hline 
 2057 &  67 &  66 ans 6 mois &  2.50\% &  7295.85 &  {\bf 55.42} &  2608.71 &  {\bf 2.80} &  {\bf 2.69} &  {\bf 2.52} &  {\bf 2.36} &  {\bf 2.22} &  {\bf 2.08} \\ 
\hline 
\hline 
\end{tabular} 
\end{center} } 

 \begin{center}\includegraphics[width=0.9\textwidth]{fig/Ascendant45_1990_22_dest_retraite.pdf}\end{center} \label{fig/Ascendant45_1990_22_dest_retraite.pdf} 

\newpage 
 
\subsection{Génération 2003 (début en 2025)} 

\paragraph{Retraites possibles et ratios Revenu/SMIC à 70, 75, 80, 85, 90 ans avec le modèle \emph{Gouvernement truqué (âge-pivot bloqué à 65 ans)}}  
 
{ \scriptsize \begin{center} 
\begin{tabular}[htb]{|c|c||c|c||c|c||c||c|c|c|c|c|c|} 
\hline 
 Retraite en &  Âge &  Âge pivot &  Décote/Surcote &  Retraite (\euro{} 2019) &  Tx Rempl(\%) &  SMIC (\euro{} 2019) &  Retraite/SMIC &  Rev70/SMIC &  Rev75/SMIC &  Rev80/SMIC &  Rev85/SMIC &  Rev90/SMIC \\ 
\hline \hline 
 2065 &  62 &  65 ans 0 mois &  -15.00\% &  6520.10 &  {\bf 42.98} &  3076.71 &  {\bf 2.12} &  {\bf 1.91} &  {\bf 1.79} &  {\bf 1.68} &  {\bf 1.57} &  {\bf 1.48} \\ 
\hline 
 2066 &  63 &  65 ans 0 mois &  -10.00\% &  7186.78 &  {\bf 46.55} &  3116.71 &  {\bf 2.31} &  {\bf 2.11} &  {\bf 1.97} &  {\bf 1.85} &  {\bf 1.74} &  {\bf 1.63} \\ 
\hline 
 2067 &  64 &  65 ans 0 mois &  -5.00\% &  7892.44 &  {\bf 50.23} &  3157.23 &  {\bf 2.50} &  {\bf 2.31} &  {\bf 2.17} &  {\bf 2.03} &  {\bf 1.91} &  {\bf 1.79} \\ 
\hline 
 2068 &  65 &  65 ans 0 mois &  0.00\% &  8638.43 &  {\bf 54.02} &  3198.27 &  {\bf 2.70} &  {\bf 2.53} &  {\bf 2.37} &  {\bf 2.23} &  {\bf 2.09} &  {\bf 1.96} \\ 
\hline 
 2069 &  66 &  65 ans 0 mois &  5.00\% &  9426.12 &  {\bf 57.92} &  3239.85 &  {\bf 2.91} &  {\bf 2.76} &  {\bf 2.59} &  {\bf 2.43} &  {\bf 2.28} &  {\bf 2.13} \\ 
\hline 
 2070 &  67 &  65 ans 0 mois &  10.00\% &  10256.96 &  {\bf 61.93} &  3281.97 &  {\bf 3.13} &  {\bf 3.01} &  {\bf 2.82} &  {\bf 2.64} &  {\bf 2.48} &  {\bf 2.32} \\ 
\hline 
\hline 
\end{tabular} 
\end{center} } 
\paragraph{Retraites possibles et ratios Revenu/SMIC à 70, 75, 80, 85, 90 ans avec le modèle \emph{Gouvernement corrigé (âge-pivot glissant)}}  
 
{ \scriptsize \begin{center} 
\begin{tabular}[htb]{|c|c||c|c||c|c||c||c|c|c|c|c|c|} 
\hline 
 Retraite en &  Âge &  Âge pivot &  Décote/Surcote &  Retraite (\euro{} 2019) &  Tx Rempl(\%) &  SMIC (\euro{} 2019) &  Retraite/SMIC &  Rev70/SMIC &  Rev75/SMIC &  Rev80/SMIC &  Rev85/SMIC &  Rev90/SMIC \\ 
\hline \hline 
 2065 &  62 &  67 ans 2 mois &  -25.83\% &  5689.10 &  {\bf 37.50} &  3076.71 &  {\bf 1.85} &  {\bf 1.67} &  {\bf 1.56} &  {\bf 1.47} &  {\bf 1.37} &  {\bf 1.29} \\ 
\hline 
 2066 &  63 &  67 ans 3 mois &  -21.25\% &  6288.44 &  {\bf 40.73} &  3116.71 &  {\bf 2.02} &  {\bf 1.84} &  {\bf 1.73} &  {\bf 1.62} &  {\bf 1.52} &  {\bf 1.42} \\ 
\hline 
 2067 &  64 &  67 ans 4 mois &  -16.67\% &  6923.20 &  {\bf 44.06} &  3157.23 &  {\bf 2.19} &  {\bf 2.03} &  {\bf 1.90} &  {\bf 1.78} &  {\bf 1.67} &  {\bf 1.57} \\ 
\hline 
 2068 &  65 &  67 ans 5 mois &  -12.08\% &  7594.62 &  {\bf 47.49} &  3198.27 &  {\bf 2.37} &  {\bf 2.23} &  {\bf 2.09} &  {\bf 1.96} &  {\bf 1.83} &  {\bf 1.72} \\ 
\hline 
 2069 &  66 &  67 ans 6 mois &  -7.50\% &  8303.97 &  {\bf 51.02} &  3239.85 &  {\bf 2.56} &  {\bf 2.43} &  {\bf 2.28} &  {\bf 2.14} &  {\bf 2.01} &  {\bf 1.88} \\ 
\hline 
 2070 &  67 &  67 ans 7 mois &  -2.92\% &  9052.54 &  {\bf 54.66} &  3281.97 &  {\bf 2.76} &  {\bf 2.65} &  {\bf 2.49} &  {\bf 2.33} &  {\bf 2.19} &  {\bf 2.05} \\ 
\hline 
\hline 
\end{tabular} 
\end{center} } 
\paragraph{Retraites possibles et ratios Revenu/SMIC à 70, 75, 80, 85, 90 ans avec le modèle \emph{Destinie2 (revalorisation de la fonction publique)}}  
 
{ \scriptsize \begin{center} 
\begin{tabular}[htb]{|c|c||c|c||c|c||c||c|c|c|c|c|c|} 
\hline 
 Retraite en &  Âge &  Âge pivot &  Décote/Surcote &  Retraite (\euro{} 2019) &  Tx Rempl(\%) &  SMIC (\euro{} 2019) &  Retraite/SMIC &  Rev70/SMIC &  Rev75/SMIC &  Rev80/SMIC &  Rev85/SMIC &  Rev90/SMIC \\ 
\hline \hline 
 2065 &  62 &  67 ans 2 mois &  -25.83\% &  5352.34 &  {\bf 37.53} &  2892.68 &  {\bf 1.85} &  {\bf 1.67} &  {\bf 1.56} &  {\bf 1.47} &  {\bf 1.37} &  {\bf 1.29} \\ 
\hline 
 2066 &  63 &  67 ans 3 mois &  -21.25\% &  5916.09 &  {\bf 40.76} &  2930.29 &  {\bf 2.02} &  {\bf 1.84} &  {\bf 1.73} &  {\bf 1.62} &  {\bf 1.52} &  {\bf 1.42} \\ 
\hline 
 2067 &  64 &  67 ans 4 mois &  -16.67\% &  6513.16 &  {\bf 44.09} &  2968.38 &  {\bf 2.19} &  {\bf 2.03} &  {\bf 1.90} &  {\bf 1.78} &  {\bf 1.67} &  {\bf 1.57} \\ 
\hline 
 2068 &  65 &  67 ans 5 mois &  -12.08\% &  7144.70 &  {\bf 47.52} &  3006.97 &  {\bf 2.38} &  {\bf 2.23} &  {\bf 2.09} &  {\bf 1.96} &  {\bf 1.84} &  {\bf 1.72} \\ 
\hline 
 2069 &  66 &  67 ans 6 mois &  -7.50\% &  7811.91 &  {\bf 51.05} &  3046.06 &  {\bf 2.56} &  {\bf 2.44} &  {\bf 2.28} &  {\bf 2.14} &  {\bf 2.01} &  {\bf 1.88} \\ 
\hline 
 2070 &  67 &  67 ans 7 mois &  -2.92\% &  8516.00 &  {\bf 54.69} &  3085.66 &  {\bf 2.76} &  {\bf 2.65} &  {\bf 2.49} &  {\bf 2.33} &  {\bf 2.19} &  {\bf 2.05} \\ 
\hline 
\hline 
\end{tabular} 
\end{center} } 

 \begin{center}\includegraphics[width=0.9\textwidth]{fig/Ascendant45_2003_22_dest_retraite.pdf}\end{center} \label{fig/Ascendant45_2003_22_dest_retraite.pdf} 

\newpage 
 
\chapter{Salarié privé à 10*SMIC durant toute sa carrière} 


 \addto{\captionsenglish}{ \renewcommand{\mtctitle}{}} \setcounter{minitocdepth}{2} 
 \minitoc \newpage 

\section{Début de carrière à 22 ans} 

\subsection{Génération 1975 (début en 1997)} 

\paragraph{Retraites possibles et ratios Revenu/SMIC à 70, 75, 80, 85, 90 ans avec le modèle \emph{Gouvernement truqué (âge-pivot bloqué à 65 ans)}}  
 
{ \scriptsize \begin{center} 
\begin{tabular}[htb]{|c|c||c|c||c|c||c||c|c|c|c|c|c|} 
\hline 
 Retraite en &  Âge &  Âge pivot &  Décote/Surcote &  Retraite (\euro{} 2019) &  Tx Rempl(\%) &  SMIC (\euro{} 2019) &  Retraite/SMIC &  Rev70/SMIC &  Rev75/SMIC &  Rev80/SMIC &  Rev85/SMIC &  Rev90/SMIC \\ 
\hline \hline 
 2037 &  62 &  64 ans 10 mois &  -14.17\% &  8746.71 &  {\bf 40.82} &  2143.00 &  {\bf 4.08} &  {\bf 3.68} &  {\bf 3.45} &  {\bf 3.23} &  {\bf 3.03} &  {\bf 2.84} \\ 
\hline 
 2038 &  63 &  64 ans 11 mois &  -9.58\% &  9557.25 &  {\bf 44.03} &  2170.86 &  {\bf 4.40} &  {\bf 4.02} &  {\bf 3.77} &  {\bf 3.53} &  {\bf 3.31} &  {\bf 3.11} \\ 
\hline 
 2039 &  64 &  65 ans 0 mois &  -5.00\% &  10416.78 &  {\bf 47.37} &  2199.08 &  {\bf 4.74} &  {\bf 4.38} &  {\bf 4.11} &  {\bf 3.85} &  {\bf 3.61} &  {\bf 3.39} \\ 
\hline 
 2040 &  65 &  65 ans 0 mois &  0.00\% &  11375.54 &  {\bf 51.06} &  2227.67 &  {\bf 5.11} &  {\bf 4.79} &  {\bf 4.49} &  {\bf 4.21} &  {\bf 3.94} &  {\bf 3.70} \\ 
\hline 
 2041 &  66 &  65 ans 0 mois &  5.00\% &  12392.70 &  {\bf 54.92} &  2256.63 &  {\bf 5.49} &  {\bf 5.22} &  {\bf 4.89} &  {\bf 4.58} &  {\bf 4.30} &  {\bf 4.03} \\ 
\hline 
 2042 &  67 &  65 ans 0 mois &  10.00\% &  13471.67 &  {\bf 58.93} &  2285.97 &  {\bf 5.89} &  {\bf 5.67} &  {\bf 5.31} &  {\bf 4.98} &  {\bf 4.67} &  {\bf 4.38} \\ 
\hline 
\hline 
\end{tabular} 
\end{center} } 
\paragraph{Retraites possibles et ratios Revenu/SMIC à 70, 75, 80, 85, 90 ans avec le modèle \emph{Gouvernement corrigé (âge-pivot glissant)}}  
 
{ \scriptsize \begin{center} 
\begin{tabular}[htb]{|c|c||c|c||c|c||c||c|c|c|c|c|c|} 
\hline 
 Retraite en &  Âge &  Âge pivot &  Décote/Surcote &  Retraite (\euro{} 2019) &  Tx Rempl(\%) &  SMIC (\euro{} 2019) &  Retraite/SMIC &  Rev70/SMIC &  Rev75/SMIC &  Rev80/SMIC &  Rev85/SMIC &  Rev90/SMIC \\ 
\hline \hline 
 2037 &  62 &  64 ans 10 mois &  -14.17\% &  8746.71 &  {\bf 40.82} &  2143.00 &  {\bf 4.08} &  {\bf 3.68} &  {\bf 3.45} &  {\bf 3.23} &  {\bf 3.03} &  {\bf 2.84} \\ 
\hline 
 2038 &  63 &  64 ans 11 mois &  -9.58\% &  9557.25 &  {\bf 44.03} &  2170.86 &  {\bf 4.40} &  {\bf 4.02} &  {\bf 3.77} &  {\bf 3.53} &  {\bf 3.31} &  {\bf 3.11} \\ 
\hline 
 2039 &  64 &  65 ans 0 mois &  -5.00\% &  10416.78 &  {\bf 47.37} &  2199.08 &  {\bf 4.74} &  {\bf 4.38} &  {\bf 4.11} &  {\bf 3.85} &  {\bf 3.61} &  {\bf 3.39} \\ 
\hline 
 2040 &  65 &  65 ans 1 mois &  -0.42\% &  11328.15 &  {\bf 50.85} &  2227.67 &  {\bf 5.09} &  {\bf 4.77} &  {\bf 4.47} &  {\bf 4.19} &  {\bf 3.93} &  {\bf 3.68} \\ 
\hline 
 2041 &  66 &  65 ans 2 mois &  4.17\% &  12294.34 &  {\bf 54.48} &  2256.63 &  {\bf 5.45} &  {\bf 5.17} &  {\bf 4.85} &  {\bf 4.55} &  {\bf 4.26} &  {\bf 4.00} \\ 
\hline 
 2042 &  67 &  65 ans 3 mois &  8.75\% &  13318.58 &  {\bf 58.26} &  2285.97 &  {\bf 5.83} &  {\bf 5.60} &  {\bf 5.25} &  {\bf 4.93} &  {\bf 4.62} &  {\bf 4.33} \\ 
\hline 
\hline 
\end{tabular} 
\end{center} } 
\paragraph{Retraites possibles et ratios Revenu/SMIC à 70, 75, 80, 85, 90 ans avec le modèle \emph{Destinie2 (revalorisation de la fonction publique)}}  
 
{ \scriptsize \begin{center} 
\begin{tabular}[htb]{|c|c||c|c||c|c||c||c|c|c|c|c|c|} 
\hline 
 Retraite en &  Âge &  Âge pivot &  Décote/Surcote &  Retraite (\euro{} 2019) &  Tx Rempl(\%) &  SMIC (\euro{} 2019) &  Retraite/SMIC &  Rev70/SMIC &  Rev75/SMIC &  Rev80/SMIC &  Rev85/SMIC &  Rev90/SMIC \\ 
\hline \hline 
 2037 &  62 &  64 ans 10 mois &  -14.17\% &  8532.62 &  {\bf 42.35} &  2014.82 &  {\bf 4.23} &  {\bf 3.82} &  {\bf 3.58} &  {\bf 3.36} &  {\bf 3.15} &  {\bf 2.95} \\ 
\hline 
 2038 &  63 &  64 ans 11 mois &  -9.58\% &  9313.66 &  {\bf 45.63} &  2041.01 &  {\bf 4.56} &  {\bf 4.17} &  {\bf 3.91} &  {\bf 3.66} &  {\bf 3.43} &  {\bf 3.22} \\ 
\hline 
 2039 &  64 &  65 ans 0 mois &  -5.00\% &  10141.31 &  {\bf 49.05} &  2067.55 &  {\bf 4.90} &  {\bf 4.54} &  {\bf 4.26} &  {\bf 3.99} &  {\bf 3.74} &  {\bf 3.51} \\ 
\hline 
 2040 &  65 &  65 ans 1 mois &  -0.42\% &  11018.27 &  {\bf 52.61} &  2094.43 &  {\bf 5.26} &  {\bf 4.93} &  {\bf 4.62} &  {\bf 4.33} &  {\bf 4.06} &  {\bf 3.81} \\ 
\hline 
 2041 &  66 &  65 ans 2 mois &  4.17\% &  11947.42 &  {\bf 56.31} &  2121.65 &  {\bf 5.63} &  {\bf 5.35} &  {\bf 5.01} &  {\bf 4.70} &  {\bf 4.41} &  {\bf 4.13} \\ 
\hline 
 2042 &  67 &  65 ans 3 mois &  8.75\% &  12931.82 &  {\bf 60.17} &  2149.23 &  {\bf 6.02} &  {\bf 5.79} &  {\bf 5.43} &  {\bf 5.09} &  {\bf 4.77} &  {\bf 4.47} \\ 
\hline 
\hline 
\end{tabular} 
\end{center} } 

 \begin{center}\includegraphics[width=0.9\textwidth]{fig/Riche_1975_22_dest_retraite.pdf}\end{center} \label{fig/Riche_1975_22_dest_retraite.pdf} 

\newpage 
 
\subsection{Génération 1980 (début en 2002)} 

\paragraph{Retraites possibles et ratios Revenu/SMIC à 70, 75, 80, 85, 90 ans avec le modèle \emph{Gouvernement truqué (âge-pivot bloqué à 65 ans)}}  
 
{ \scriptsize \begin{center} 
\begin{tabular}[htb]{|c|c||c|c||c|c||c||c|c|c|c|c|c|} 
\hline 
 Retraite en &  Âge &  Âge pivot &  Décote/Surcote &  Retraite (\euro{} 2019) &  Tx Rempl(\%) &  SMIC (\euro{} 2019) &  Retraite/SMIC &  Rev70/SMIC &  Rev75/SMIC &  Rev80/SMIC &  Rev85/SMIC &  Rev90/SMIC \\ 
\hline \hline 
 2042 &  62 &  65 ans 0 mois &  -15.00\% &  9516.82 &  {\bf 41.63} &  2285.97 &  {\bf 4.16} &  {\bf 3.75} &  {\bf 3.52} &  {\bf 3.30} &  {\bf 3.09} &  {\bf 2.90} \\ 
\hline 
 2043 &  63 &  65 ans 0 mois &  -10.00\% &  10482.23 &  {\bf 45.27} &  2315.68 &  {\bf 4.53} &  {\bf 4.14} &  {\bf 3.88} &  {\bf 3.63} &  {\bf 3.41} &  {\bf 3.19} \\ 
\hline 
 2044 &  64 &  65 ans 0 mois &  -5.00\% &  11510.09 &  {\bf 49.07} &  2345.79 &  {\bf 4.91} &  {\bf 4.54} &  {\bf 4.26} &  {\bf 3.99} &  {\bf 3.74} &  {\bf 3.51} \\ 
\hline 
 2045 &  65 &  65 ans 0 mois &  0.00\% &  12604.16 &  {\bf 53.04} &  2376.28 &  {\bf 5.30} &  {\bf 4.97} &  {\bf 4.66} &  {\bf 4.37} &  {\bf 4.10} &  {\bf 3.84} \\ 
\hline 
 2046 &  66 &  65 ans 0 mois &  5.00\% &  13758.23 &  {\bf 57.16} &  2407.18 &  {\bf 5.72} &  {\bf 5.43} &  {\bf 5.09} &  {\bf 4.77} &  {\bf 4.47} &  {\bf 4.19} \\ 
\hline 
 2047 &  67 &  65 ans 0 mois &  10.00\% &  14974.12 &  {\bf 61.41} &  2438.47 &  {\bf 6.14} &  {\bf 5.91} &  {\bf 5.54} &  {\bf 5.19} &  {\bf 4.87} &  {\bf 4.56} \\ 
\hline 
\hline 
\end{tabular} 
\end{center} } 
\paragraph{Retraites possibles et ratios Revenu/SMIC à 70, 75, 80, 85, 90 ans avec le modèle \emph{Gouvernement corrigé (âge-pivot glissant)}}  
 
{ \scriptsize \begin{center} 
\begin{tabular}[htb]{|c|c||c|c||c|c||c||c|c|c|c|c|c|} 
\hline 
 Retraite en &  Âge &  Âge pivot &  Décote/Surcote &  Retraite (\euro{} 2019) &  Tx Rempl(\%) &  SMIC (\euro{} 2019) &  Retraite/SMIC &  Rev70/SMIC &  Rev75/SMIC &  Rev80/SMIC &  Rev85/SMIC &  Rev90/SMIC \\ 
\hline \hline 
 2042 &  62 &  65 ans 3 mois &  -16.25\% &  9376.87 &  {\bf 41.02} &  2285.97 &  {\bf 4.10} &  {\bf 3.70} &  {\bf 3.47} &  {\bf 3.25} &  {\bf 3.05} &  {\bf 2.86} \\ 
\hline 
 2043 &  63 &  65 ans 4 mois &  -11.67\% &  10288.11 &  {\bf 44.43} &  2315.68 &  {\bf 4.44} &  {\bf 4.06} &  {\bf 3.80} &  {\bf 3.57} &  {\bf 3.34} &  {\bf 3.13} \\ 
\hline 
 2044 &  64 &  65 ans 5 mois &  -7.08\% &  11257.68 &  {\bf 47.99} &  2345.79 &  {\bf 4.80} &  {\bf 4.44} &  {\bf 4.16} &  {\bf 3.90} &  {\bf 3.66} &  {\bf 3.43} \\ 
\hline 
 2045 &  65 &  65 ans 6 mois &  -2.50\% &  12289.05 &  {\bf 51.72} &  2376.28 &  {\bf 5.17} &  {\bf 4.85} &  {\bf 4.54} &  {\bf 4.26} &  {\bf 3.99} &  {\bf 3.74} \\ 
\hline 
 2046 &  66 &  65 ans 7 mois &  2.08\% &  13376.06 &  {\bf 55.57} &  2407.18 &  {\bf 5.56} &  {\bf 5.28} &  {\bf 4.95} &  {\bf 4.64} &  {\bf 4.35} &  {\bf 4.08} \\ 
\hline 
 2047 &  67 &  65 ans 8 mois &  6.67\% &  14520.36 &  {\bf 59.55} &  2438.47 &  {\bf 5.95} &  {\bf 5.73} &  {\bf 5.37} &  {\bf 5.03} &  {\bf 4.72} &  {\bf 4.42} \\ 
\hline 
\hline 
\end{tabular} 
\end{center} } 
\paragraph{Retraites possibles et ratios Revenu/SMIC à 70, 75, 80, 85, 90 ans avec le modèle \emph{Destinie2 (revalorisation de la fonction publique)}}  
 
{ \scriptsize \begin{center} 
\begin{tabular}[htb]{|c|c||c|c||c|c||c||c|c|c|c|c|c|} 
\hline 
 Retraite en &  Âge &  Âge pivot &  Décote/Surcote &  Retraite (\euro{} 2019) &  Tx Rempl(\%) &  SMIC (\euro{} 2019) &  Retraite/SMIC &  Rev70/SMIC &  Rev75/SMIC &  Rev80/SMIC &  Rev85/SMIC &  Rev90/SMIC \\ 
\hline \hline 
 2042 &  62 &  65 ans 3 mois &  -16.25\% &  9079.17 &  {\bf 42.24} &  2149.23 &  {\bf 4.22} &  {\bf 3.81} &  {\bf 3.57} &  {\bf 3.35} &  {\bf 3.14} &  {\bf 2.94} \\ 
\hline 
 2043 &  63 &  65 ans 4 mois &  -11.67\% &  9953.49 &  {\bf 45.72} &  2177.17 &  {\bf 4.57} &  {\bf 4.18} &  {\bf 3.92} &  {\bf 3.67} &  {\bf 3.44} &  {\bf 3.23} \\ 
\hline 
 2044 &  64 &  65 ans 5 mois &  -7.08\% &  10883.24 &  {\bf 49.35} &  2205.48 &  {\bf 4.93} &  {\bf 4.57} &  {\bf 4.28} &  {\bf 4.01} &  {\bf 3.76} &  {\bf 3.53} \\ 
\hline 
 2045 &  65 &  65 ans 6 mois &  -2.50\% &  11871.75 &  {\bf 53.14} &  2234.15 &  {\bf 5.31} &  {\bf 4.98} &  {\bf 4.67} &  {\bf 4.38} &  {\bf 4.10} &  {\bf 3.85} \\ 
\hline 
 2046 &  66 &  65 ans 7 mois &  2.08\% &  12913.00 &  {\bf 57.06} &  2263.19 &  {\bf 5.71} &  {\bf 5.42} &  {\bf 5.08} &  {\bf 4.76} &  {\bf 4.46} &  {\bf 4.18} \\ 
\hline 
 2047 &  67 &  65 ans 8 mois &  6.67\% &  14008.57 &  {\bf 61.10} &  2292.61 &  {\bf 6.11} &  {\bf 5.88} &  {\bf 5.51} &  {\bf 5.17} &  {\bf 4.84} &  {\bf 4.54} \\ 
\hline 
\hline 
\end{tabular} 
\end{center} } 

 \begin{center}\includegraphics[width=0.9\textwidth]{fig/Riche_1980_22_dest_retraite.pdf}\end{center} \label{fig/Riche_1980_22_dest_retraite.pdf} 

\newpage 
 
\subsection{Génération 1990 (début en 2012)} 

\paragraph{Retraites possibles et ratios Revenu/SMIC à 70, 75, 80, 85, 90 ans avec le modèle \emph{Gouvernement truqué (âge-pivot bloqué à 65 ans)}}  
 
{ \scriptsize \begin{center} 
\begin{tabular}[htb]{|c|c||c|c||c|c||c||c|c|c|c|c|c|} 
\hline 
 Retraite en &  Âge &  Âge pivot &  Décote/Surcote &  Retraite (\euro{} 2019) &  Tx Rempl(\%) &  SMIC (\euro{} 2019) &  Retraite/SMIC &  Rev70/SMIC &  Rev75/SMIC &  Rev80/SMIC &  Rev85/SMIC &  Rev90/SMIC \\ 
\hline \hline 
 2052 &  62 &  65 ans 0 mois &  -15.00\% &  11564.81 &  {\bf 44.46} &  2601.14 &  {\bf 4.45} &  {\bf 4.01} &  {\bf 3.76} &  {\bf 3.52} &  {\bf 3.30} &  {\bf 3.10} \\ 
\hline 
 2053 &  63 &  65 ans 0 mois &  -10.00\% &  12734.38 &  {\bf 48.33} &  2634.96 &  {\bf 4.83} &  {\bf 4.42} &  {\bf 4.14} &  {\bf 3.88} &  {\bf 3.64} &  {\bf 3.41} \\ 
\hline 
 2054 &  64 &  65 ans 0 mois &  -5.00\% &  13969.55 &  {\bf 52.34} &  2669.21 &  {\bf 5.23} &  {\bf 4.84} &  {\bf 4.54} &  {\bf 4.26} &  {\bf 3.99} &  {\bf 3.74} \\ 
\hline 
 2055 &  65 &  65 ans 0 mois &  0.00\% &  15272.32 &  {\bf 56.48} &  2703.91 &  {\bf 5.65} &  {\bf 5.29} &  {\bf 4.96} &  {\bf 4.65} &  {\bf 4.36} &  {\bf 4.09} \\ 
\hline 
 2056 &  66 &  65 ans 0 mois &  5.00\% &  16644.73 &  {\bf 60.77} &  2739.06 &  {\bf 6.08} &  {\bf 5.77} &  {\bf 5.41} &  {\bf 5.07} &  {\bf 4.75} &  {\bf 4.46} \\ 
\hline 
 2057 &  67 &  65 ans 0 mois &  10.00\% &  18088.86 &  {\bf 65.19} &  2774.67 &  {\bf 6.52} &  {\bf 6.27} &  {\bf 5.88} &  {\bf 5.51} &  {\bf 5.17} &  {\bf 4.84} \\ 
\hline 
\hline 
\end{tabular} 
\end{center} } 
\paragraph{Retraites possibles et ratios Revenu/SMIC à 70, 75, 80, 85, 90 ans avec le modèle \emph{Gouvernement corrigé (âge-pivot glissant)}}  
 
{ \scriptsize \begin{center} 
\begin{tabular}[htb]{|c|c||c|c||c|c||c||c|c|c|c|c|c|} 
\hline 
 Retraite en &  Âge &  Âge pivot &  Décote/Surcote &  Retraite (\euro{} 2019) &  Tx Rempl(\%) &  SMIC (\euro{} 2019) &  Retraite/SMIC &  Rev70/SMIC &  Rev75/SMIC &  Rev80/SMIC &  Rev85/SMIC &  Rev90/SMIC \\ 
\hline \hline 
 2052 &  62 &  66 ans 1 mois &  -20.42\% &  10827.84 &  {\bf 41.63} &  2601.14 &  {\bf 4.16} &  {\bf 3.75} &  {\bf 3.52} &  {\bf 3.30} &  {\bf 3.09} &  {\bf 2.90} \\ 
\hline 
 2053 &  63 &  66 ans 2 mois &  -15.83\% &  11909.00 &  {\bf 45.20} &  2634.96 &  {\bf 4.52} &  {\bf 4.13} &  {\bf 3.87} &  {\bf 3.63} &  {\bf 3.40} &  {\bf 3.19} \\ 
\hline 
 2054 &  64 &  66 ans 3 mois &  -11.25\% &  13050.50 &  {\bf 48.89} &  2669.21 &  {\bf 4.89} &  {\bf 4.52} &  {\bf 4.24} &  {\bf 3.98} &  {\bf 3.73} &  {\bf 3.49} \\ 
\hline 
 2055 &  65 &  66 ans 4 mois &  -6.67\% &  14254.16 &  {\bf 52.72} &  2703.91 &  {\bf 5.27} &  {\bf 4.94} &  {\bf 4.63} &  {\bf 4.34} &  {\bf 4.07} &  {\bf 3.82} \\ 
\hline 
 2056 &  66 &  66 ans 5 mois &  -2.08\% &  15521.87 &  {\bf 56.67} &  2739.06 &  {\bf 5.67} &  {\bf 5.38} &  {\bf 5.04} &  {\bf 4.73} &  {\bf 4.43} &  {\bf 4.16} \\ 
\hline 
 2057 &  67 &  66 ans 6 mois &  2.50\% &  16855.53 &  {\bf 60.75} &  2774.67 &  {\bf 6.07} &  {\bf 5.84} &  {\bf 5.48} &  {\bf 5.14} &  {\bf 4.81} &  {\bf 4.51} \\ 
\hline 
\hline 
\end{tabular} 
\end{center} } 
\paragraph{Retraites possibles et ratios Revenu/SMIC à 70, 75, 80, 85, 90 ans avec le modèle \emph{Destinie2 (revalorisation de la fonction publique)}}  
 
{ \scriptsize \begin{center} 
\begin{tabular}[htb]{|c|c||c|c||c|c||c||c|c|c|c|c|c|} 
\hline 
 Retraite en &  Âge &  Âge pivot &  Décote/Surcote &  Retraite (\euro{} 2019) &  Tx Rempl(\%) &  SMIC (\euro{} 2019) &  Retraite/SMIC &  Rev70/SMIC &  Rev75/SMIC &  Rev80/SMIC &  Rev85/SMIC &  Rev90/SMIC \\ 
\hline \hline 
 2052 &  62 &  66 ans 1 mois &  -20.42\% &  10334.73 &  {\bf 42.26} &  2445.56 &  {\bf 4.23} &  {\bf 3.81} &  {\bf 3.57} &  {\bf 3.35} &  {\bf 3.14} &  {\bf 2.94} \\ 
\hline 
 2053 &  63 &  66 ans 2 mois &  -15.83\% &  11362.24 &  {\bf 45.86} &  2477.35 &  {\bf 4.59} &  {\bf 4.19} &  {\bf 3.93} &  {\bf 3.68} &  {\bf 3.45} &  {\bf 3.24} \\ 
\hline 
 2054 &  64 &  66 ans 3 mois &  -11.25\% &  12446.75 &  {\bf 49.60} &  2509.56 &  {\bf 4.96} &  {\bf 4.59} &  {\bf 4.30} &  {\bf 4.03} &  {\bf 3.78} &  {\bf 3.54} \\ 
\hline 
 2055 &  65 &  66 ans 4 mois &  -6.67\% &  13589.97 &  {\bf 53.46} &  2542.18 &  {\bf 5.35} &  {\bf 5.01} &  {\bf 4.70} &  {\bf 4.40} &  {\bf 4.13} &  {\bf 3.87} \\ 
\hline 
 2056 &  66 &  66 ans 5 mois &  -2.08\% &  14793.67 &  {\bf 57.45} &  2575.23 &  {\bf 5.74} &  {\bf 5.46} &  {\bf 5.11} &  {\bf 4.79} &  {\bf 4.49} &  {\bf 4.21} \\ 
\hline 
 2057 &  67 &  66 ans 6 mois &  2.50\% &  16059.65 &  {\bf 61.56} &  2608.71 &  {\bf 6.16} &  {\bf 5.92} &  {\bf 5.55} &  {\bf 5.20} &  {\bf 4.88} &  {\bf 4.57} \\ 
\hline 
\hline 
\end{tabular} 
\end{center} } 

 \begin{center}\includegraphics[width=0.9\textwidth]{fig/Riche_1990_22_dest_retraite.pdf}\end{center} \label{fig/Riche_1990_22_dest_retraite.pdf} 

\newpage 
 
\subsection{Génération 2003 (début en 2025)} 

\paragraph{Retraites possibles et ratios Revenu/SMIC à 70, 75, 80, 85, 90 ans avec le modèle \emph{Gouvernement truqué (âge-pivot bloqué à 65 ans)}}  
 
{ \scriptsize \begin{center} 
\begin{tabular}[htb]{|c|c||c|c||c|c||c||c|c|c|c|c|c|} 
\hline 
 Retraite en &  Âge &  Âge pivot &  Décote/Surcote &  Retraite (\euro{} 2019) &  Tx Rempl(\%) &  SMIC (\euro{} 2019) &  Retraite/SMIC &  Rev70/SMIC &  Rev75/SMIC &  Rev80/SMIC &  Rev85/SMIC &  Rev90/SMIC \\ 
\hline \hline 
 2065 &  62 &  65 ans 0 mois &  -15.00\% &  14574.14 &  {\bf 47.37} &  3076.71 &  {\bf 4.74} &  {\bf 4.27} &  {\bf 4.00} &  {\bf 3.75} &  {\bf 3.52} &  {\bf 3.30} \\ 
\hline 
 2066 &  63 &  65 ans 0 mois &  -10.00\% &  16022.50 &  {\bf 51.41} &  3116.71 &  {\bf 5.14} &  {\bf 4.70} &  {\bf 4.40} &  {\bf 4.13} &  {\bf 3.87} &  {\bf 3.63} \\ 
\hline 
 2067 &  64 &  65 ans 0 mois &  -5.00\% &  17549.99 &  {\bf 55.59} &  3157.23 &  {\bf 5.56} &  {\bf 5.14} &  {\bf 4.82} &  {\bf 4.52} &  {\bf 4.24} &  {\bf 3.97} \\ 
\hline 
 2068 &  65 &  65 ans 0 mois &  0.00\% &  19159.02 &  {\bf 59.90} &  3198.27 &  {\bf 5.99} &  {\bf 5.62} &  {\bf 5.26} &  {\bf 4.94} &  {\bf 4.63} &  {\bf 4.34} \\ 
\hline 
 2069 &  66 &  65 ans 0 mois &  5.00\% &  20852.00 &  {\bf 64.36} &  3239.85 &  {\bf 6.44} &  {\bf 6.11} &  {\bf 5.73} &  {\bf 5.37} &  {\bf 5.04} &  {\bf 4.72} \\ 
\hline 
 2070 &  67 &  65 ans 0 mois &  10.00\% &  22631.45 &  {\bf 68.96} &  3281.97 &  {\bf 6.90} &  {\bf 6.63} &  {\bf 6.22} &  {\bf 5.83} &  {\bf 5.47} &  {\bf 5.12} \\ 
\hline 
\hline 
\end{tabular} 
\end{center} } 
\paragraph{Retraites possibles et ratios Revenu/SMIC à 70, 75, 80, 85, 90 ans avec le modèle \emph{Gouvernement corrigé (âge-pivot glissant)}}  
 
{ \scriptsize \begin{center} 
\begin{tabular}[htb]{|c|c||c|c||c|c||c||c|c|c|c|c|c|} 
\hline 
 Retraite en &  Âge &  Âge pivot &  Décote/Surcote &  Retraite (\euro{} 2019) &  Tx Rempl(\%) &  SMIC (\euro{} 2019) &  Retraite/SMIC &  Rev70/SMIC &  Rev75/SMIC &  Rev80/SMIC &  Rev85/SMIC &  Rev90/SMIC \\ 
\hline \hline 
 2065 &  62 &  67 ans 2 mois &  -25.83\% &  12716.65 &  {\bf 41.33} &  3076.71 &  {\bf 4.13} &  {\bf 3.73} &  {\bf 3.49} &  {\bf 3.28} &  {\bf 3.07} &  {\bf 2.88} \\ 
\hline 
 2066 &  63 &  67 ans 3 mois &  -21.25\% &  14019.68 &  {\bf 44.98} &  3116.71 &  {\bf 4.50} &  {\bf 4.11} &  {\bf 3.85} &  {\bf 3.61} &  {\bf 3.39} &  {\bf 3.17} \\ 
\hline 
 2067 &  64 &  67 ans 4 mois &  -16.67\% &  15394.73 &  {\bf 48.76} &  3157.23 &  {\bf 4.88} &  {\bf 4.51} &  {\bf 4.23} &  {\bf 3.97} &  {\bf 3.72} &  {\bf 3.49} \\ 
\hline 
 2068 &  65 &  67 ans 5 mois &  -12.08\% &  16843.97 &  {\bf 52.67} &  3198.27 &  {\bf 5.27} &  {\bf 4.94} &  {\bf 4.63} &  {\bf 4.34} &  {\bf 4.07} &  {\bf 3.81} \\ 
\hline 
 2069 &  66 &  67 ans 6 mois &  -7.50\% &  18369.62 &  {\bf 56.70} &  3239.85 &  {\bf 5.67} &  {\bf 5.38} &  {\bf 5.05} &  {\bf 4.73} &  {\bf 4.44} &  {\bf 4.16} \\ 
\hline 
 2070 &  67 &  67 ans 7 mois &  -2.92\% &  19973.97 &  {\bf 60.86} &  3281.97 &  {\bf 6.09} &  {\bf 5.85} &  {\bf 5.49} &  {\bf 5.15} &  {\bf 4.82} &  {\bf 4.52} \\ 
\hline 
\hline 
\end{tabular} 
\end{center} } 
\paragraph{Retraites possibles et ratios Revenu/SMIC à 70, 75, 80, 85, 90 ans avec le modèle \emph{Destinie2 (revalorisation de la fonction publique)}}  
 
{ \scriptsize \begin{center} 
\begin{tabular}[htb]{|c|c||c|c||c|c||c||c|c|c|c|c|c|} 
\hline 
 Retraite en &  Âge &  Âge pivot &  Décote/Surcote &  Retraite (\euro{} 2019) &  Tx Rempl(\%) &  SMIC (\euro{} 2019) &  Retraite/SMIC &  Rev70/SMIC &  Rev75/SMIC &  Rev80/SMIC &  Rev85/SMIC &  Rev90/SMIC \\ 
\hline \hline 
 2065 &  62 &  67 ans 2 mois &  -25.83\% &  11964.78 &  {\bf 41.36} &  2892.68 &  {\bf 4.14} &  {\bf 3.73} &  {\bf 3.50} &  {\bf 3.28} &  {\bf 3.07} &  {\bf 2.88} \\ 
\hline 
 2066 &  63 &  67 ans 3 mois &  -21.25\% &  13190.54 &  {\bf 45.01} &  2930.29 &  {\bf 4.50} &  {\bf 4.11} &  {\bf 3.86} &  {\bf 3.61} &  {\bf 3.39} &  {\bf 3.18} \\ 
\hline 
 2067 &  64 &  67 ans 4 mois &  -16.67\% &  14484.01 &  {\bf 48.79} &  2968.38 &  {\bf 4.88} &  {\bf 4.52} &  {\bf 4.23} &  {\bf 3.97} &  {\bf 3.72} &  {\bf 3.49} \\ 
\hline 
 2068 &  65 &  67 ans 5 mois &  -12.08\% &  15847.26 &  {\bf 52.70} &  3006.97 &  {\bf 5.27} &  {\bf 4.94} &  {\bf 4.63} &  {\bf 4.34} &  {\bf 4.07} &  {\bf 3.82} \\ 
\hline 
 2069 &  66 &  67 ans 6 mois &  -7.50\% &  17282.37 &  {\bf 56.74} &  3046.06 &  {\bf 5.67} &  {\bf 5.39} &  {\bf 5.05} &  {\bf 4.74} &  {\bf 4.44} &  {\bf 4.16} \\ 
\hline 
 2070 &  67 &  67 ans 7 mois &  -2.92\% &  18791.48 &  {\bf 60.90} &  3085.66 &  {\bf 6.09} &  {\bf 5.86} &  {\bf 5.49} &  {\bf 5.15} &  {\bf 4.83} &  {\bf 4.52} \\ 
\hline 
\hline 
\end{tabular} 
\end{center} } 

 \begin{center}\includegraphics[width=0.9\textwidth]{fig/Riche_2003_22_dest_retraite.pdf}\end{center} \label{fig/Riche_2003_22_dest_retraite.pdf} 

\newpage 
 


\end{document}
